\documentclass[aps,prb,superscriptaddress,nofootinbib]{revtex4}
\usepackage{amsfonts}
\usepackage{amsmath}
\usepackage{amssymb}
\usepackage{graphbox}
\usepackage{graphicx}
\usepackage{caption}
\usepackage{bm}
\usepackage{bbm}
\usepackage{cancel}
\usepackage{color}
\usepackage{mathrsfs}
\usepackage[colorlinks,bookmarks=true,citecolor=blue,linkcolor=red,urlcolor=blue]{hyperref}
\usepackage{simpler-wick}
\usepackage{appendix}
\usepackage{float}
\usepackage{array}
\usepackage{booktabs}
\usepackage[export]{adjustbox}
\setlength{\parindent}{10 pt}
\setlength{\parskip}{2 pt}
\setcounter{MaxMatrixCols}{30}
\bibliographystyle{apsrev}
\newcommand{\RNum}[1]{\uppercase\expandafter{\romannumeral #1\relax}}
\newcommand{\normord}[1]{{:\mathrel{#1}:}}
\def\tbs{\textbackslash}
\def \tr{\operatorname{tr}}
\def \Tr{\operatorname{Tr}}


\begin{document}
\title{Anomaly}
\author{Jie Ren}



\maketitle



Most of the time, a classical symmetry is also a symmetry of the quantum theory based on the same Lagrangian.
When it is not, the symmetry is said to be \textit{anomalous}.
For classical field described by the Lagrangian $\mathcal L[\phi]$, for an infinitesimal transformation 
\begin{equation}
	\delta\phi = \epsilon(x) X \phi(x),
\end{equation}
the Lagrangian transforms as
\begin{equation}
\begin{aligned}
	\delta \mathcal L 
	&= \frac{\partial \mathcal L}{\partial(\partial_\mu \phi)} \partial_\mu[\epsilon(x) X \phi(x)] + \frac{\partial \mathcal L}{\partial \phi} \epsilon(x) X \phi(x) \\
	&= \frac{\partial \mathcal L}{\partial(\partial_\mu \phi)}X \phi (\partial_\mu\epsilon) + \left[\frac{\partial \mathcal L}{\partial \phi}X\phi + \frac{\partial \mathcal L}{\partial(\partial_\mu \phi)}\partial_\mu X \phi \right]\epsilon.
\end{aligned}
\end{equation}
For a symmetry transformation, when $\epsilon(x)$ is constant, $\delta \mathcal L = 0$, we know the expression in the bracket is zero, which is just the classical equation of motion.
We can define what is left as the current:
\begin{equation}
	\delta \mathcal L = \left[\frac{\partial \mathcal L}{\partial(\partial_\mu \phi)}X \right] \partial_\mu\epsilon(x)
	\equiv J^\mu(x) \partial_\mu \epsilon(x).
\end{equation}
The action is
\begin{equation}
	\delta S = \int d^d x \delta \mathcal L 
	= -\int d^d x\ \epsilon(x) \partial_\mu J^\mu(x),
\end{equation}
which is true for all $\epsilon(x)$ configuration, which implies
\begin{equation}
	\partial_\mu J^\mu(x) = 0.
\end{equation}
This statement is the Noether's theorem, saying that symmetries lead to conserved current.
For quantum field theory, $J^\mu(x)$ is an operator, not a number.
To see what the Noether's theorem means, consider the quantum partition functional:
\begin{equation}
	Z = \int D[\phi] e^{iS[\phi]}.
\end{equation}
After the infinitesimal axial transformation, the partition function becomes
\begin{equation}\label{eq:AN-partition-1}
	Z = \int D[\phi'] e^{iS[\phi]} \exp\left[-i\int d^d x \ \epsilon(x) \partial_\mu J^\mu (x)\right].
\end{equation}
Since the change of variable will have no effect on the functional integral, the partition function (\ref{eq:AN-partition-1}) will not change under the transformation.
If we assume the transformation does not change the measure, i.e.,
\begin{equation}\label{eq:AN-measure-eq}
	D[\phi'] = D[\phi],
\end{equation}
then, since Eq.~(\ref{eq:AN-partition-1}) stands for any $\epsilon(x)$, we must have the operator identity:
\begin{equation}
	\partial_\mu J^\mu(x) = 0.
\end{equation}
For the case of anomalies, however, Eq.~(\ref{eq:AN-measure-eq}) is no longer true.
The change of measure will break the classical symmetries.

\tableofcontents



\section{Chiral Anomaly}
The first anomaly being discovered is the chiral anomaly for the massless Dirac field Lagrangian $\mathcal L = i\bar\psi \cancel D \psi$.
It is the symmetry of of chiral (axial) rotation 
\begin{equation}
	\psi \rightarrow e^{i\epsilon \gamma^5}\psi, \quad
	\bar\psi \rightarrow \bar\psi e^{i\epsilon \gamma^5}.
\end{equation}
The Noether's theorem gives the conserved current
\begin{equation}
	J^\mu_5 = \frac{\partial \mathcal L}{\partial(\partial_\mu \psi)} i\gamma^5\psi = \bar\psi \gamma^\mu \gamma^5\psi.
\end{equation}
After the infinitesimal axial transformation, the quantum partition function becomes:
\begin{equation}\label{eq:AN-CH-partition}
\begin{aligned}
	Z &= \int D[\bar\psi,\psi] \exp\left[i\int d^d x\ \bar\psi(x) e^{i\epsilon(x)\gamma^5}\gamma^\mu(i\partial_\mu - e A_\mu) e^{i\epsilon(x)\gamma^5}\psi(x) \right] \\
	&= \int D[\bar\psi',\psi'] \exp\left\{i\int d^d x\ \bar\psi(x) \gamma^\mu\left[iD_\mu-\partial_\mu\epsilon(x)\gamma^5\right] \psi(x) \right\} \\
	&= \int D[\bar\psi,\psi] e^{iS[\bar\psi,\psi]} \mathcal J^2 \exp\left[i \int d^d x\ \epsilon(x) \partial_\mu J_5^\mu(x) \right],
\end{aligned}
\end{equation}
where $\mathcal J$ is the Jacobian:
\begin{equation}
	\mathcal J = \det\left(\frac{D[\bar\psi']}{D[\bar\psi]}\right)^{-1}=\det\left(\frac{D[\psi']}{D[\psi]}\right)^{-1}.
\end{equation}
Note that in contract to the c-number calculus, since the integration of a Grassmann variable is algebraically identical to differentiation, the Jacobian for the change of variable is the inverse of the determinant.


\subsection{Anomaly in the Measure}

We now going to evaluate the Jacobian for the measure:
\begin{equation}
	\mathcal J = \exp\left(-i \int d^d x \ \epsilon(x) \Tr[\gamma^5] \right),
\end{equation}
where $\mathcal A = \Tr[\gamma^5]$.
Note that the trace is over all eigenstates of the Hamiltonian (Lagrangian).
The orthonormal basis for the (massless) Dirac equation is:
\begin{equation}
	\gamma^\mu (\partial_\mu + ie A_\mu ) \xi_k(x) = E_k \xi_k(x), \quad
	\bar\eta_k(x) \left(\overleftarrow\partial_\mu -ieA_\mu\right) \gamma^\mu = E_k \bar\eta_k(x).
\end{equation}
Note that the subscript $k$ is the label for the eigenstate (or wave function).
It is to be differentiated from the state in the momentum space.
The trace is carried out as
\begin{equation}\label{eq:AN-tr-1}
	\Tr\left[\gamma^5\right] 
	= \int \frac{d^d k}{(2\pi)^d} \tr\left[\bar\eta_k \gamma^5 \xi_k \right],
\end{equation}
where we distinguish the small $\tr$ as the trace over the spinor indices. 
The integral (\ref{eq:AN-tr-1}) is divergent, the simplest regulation would be putting a heat-kernel regulator to the expression, i.e., by putting a gaussian regulator suppressing the large momenta:
\begin{equation}
	\exp\left(\frac{\cancel{\Pi}^2}{\Lambda^2}\right)
	= \exp\left[\frac{(\cancel{k}-e\cancel{A})^2}{\Lambda^2}\right].
\end{equation}
Such Gaussian regulator, when the functional integral is converted to Euclidean space, provide a Gaussian suppression for large momenta.
Here, to preserve the gauge invariance, the canonical momentum $\Pi$ as the eigenvalue of the covariant derivative should be used instead of mechanical momentum. 
Note that
\begin{equation}
\begin{aligned}
	(\cancel D)^2 &= \gamma^\mu \gamma^\nu {D}_\mu {D}_\nu \\
	&= \frac{1}{2} \{\gamma^\mu,\gamma^\nu\} {D}_\mu {D}_\nu + [\gamma^\mu,\gamma^\nu] {D}_\mu {D}_\nu \\
	&= \left(\partial_\mu+ie A_\mu \right)^2 + \frac{1}{2} \gamma^\mu \gamma^\nu \left[\partial_\mu+ie A_\mu, \partial_\nu+ie A_\nu \right] \\
	&= \left(\partial_\mu+ie A_\mu \right)^2 + \frac{ie}{2}\gamma^\mu \gamma^\nu F_{\mu\nu},
\end{aligned}
\end{equation}
which implies
\begin{equation}
	\cancel{\Pi}^2 = \Pi^2 - \frac{ie}{2}\gamma^\mu \gamma^\nu F_{\mu\nu}.
\end{equation}
The trace of $\gamma^5$ with the Gaussian regulator is
\begin{equation}
\begin{aligned}
	\Tr\left[\gamma^5\right] 
	&= \int \frac{d^d k}{(2\pi)^d} \tr\left[\bar\eta_k \gamma^5 e^{\cancel{\Pi}^2/\Lambda^2}\xi_k \right] \\
	&= \int \frac{d^d k}{(2\pi)^d} e^{(k-eA)^2/\Lambda^2}\tr\left[\gamma^5 \exp\left(-\frac{ie}{2\Lambda^2} \gamma^\mu \gamma^\nu F_{\mu\nu}\right)\right].
\end{aligned}
\end{equation}
Shift $k \rightarrow k + e A$, and then expand the exponential according to the order of $\Lambda^{-1}$:
\begin{equation}
\begin{aligned}
	\exp\left(-\frac{ie}{2\Lambda^2} \gamma^\mu \gamma^\nu F_{\mu\nu}\right)
	=&\ 1-\frac{ie}{2\Lambda^2} \gamma^\mu \gamma^\nu F_{\mu\nu}-\frac{e^2}{8\Lambda^4} \gamma^\mu \gamma^\nu \gamma^\rho \gamma^\sigma F_{\mu\nu} F_{\rho\sigma} + O\left(\frac{1}{\Lambda^6}\right).
\end{aligned}
\end{equation}
For $d=4$ case, the only term surviving the gamma trace is
\begin{equation}
	\mathrm{tr} \left[\gamma^5 \gamma^\mu \gamma^\nu \gamma^\rho \gamma^\sigma \right] 
	= -4i\varepsilon^{\mu\nu\rho\sigma}.
\end{equation}
The trace formula (\ref{eq:AN-tr-1}) is then
\begin{equation}
	\Tr[\gamma^5] = \lim_{\Lambda\rightarrow\infty} \frac{ie^2}{2\Lambda^4} \int \frac{d^4k}{(2\pi)^4} e^{\frac{k^2}{\Lambda^2}}  \varepsilon^{\mu\nu\rho\sigma}F_{\mu\nu} F_{\rho\sigma}.
\end{equation}
Now we do the Wick rotation $k^0 \rightarrow -k^0$. 
The trace then becomes
\begin{equation}
\begin{aligned}
	\Tr[\gamma^5] &= \lim_{\Lambda\rightarrow\infty} \frac{-e^2}{2\Lambda^4} \int \frac{\Omega_4 k^3_E dk_E}{(2\pi)^4} e^{-\frac{k_E^2}{\Lambda^2}}  \varepsilon^{\mu\nu\rho\sigma}F_{\mu\nu} F_{\rho\sigma} \\
	&= -\frac{e^2}{32\pi^2} \varepsilon^{\mu\nu\rho\sigma}F_{\mu\nu} F_{\rho\sigma}.
\end{aligned}
\end{equation}
Now return to Eq.~(\ref{eq:AN-CH-partition}), the final result is
\begin{equation}
	Z = \int D[\bar\psi,\psi] e^{iS[\bar\psi,\psi]} \exp\left[i\int d^d x \ \epsilon(x) \left(\partial_\mu J_5^\mu+ \frac{e^2}{16\pi^2}\varepsilon^{\mu\nu\rho\sigma}F_{\mu\nu} F_{\rho\sigma}\right)\right].
\end{equation}
We thus have
\begin{equation}
	\partial_\mu J^\mu_5 = -\frac{e^2}{16\pi^2}\varepsilon^{\mu\nu\rho\sigma}F_{\mu\nu} F_{\rho\sigma}.
\end{equation}

If we instead consider the quantum field theory on 2d spacetime, the trace of gamma matrices satisfies:
\begin{equation}
	\tr[\gamma^5 \gamma^\mu \gamma^\nu] = 2 \varepsilon^{\mu\nu}.
\end{equation}
The trace formula then becomes
\begin{equation}
\begin{aligned}
	\Tr[\gamma^5] 
	&= i\int \frac{d^2 k_E}{(2\pi)^2} e^{-k_E^2/\Lambda^2}\tr\left[\gamma^5 \exp\left(-\frac{ie}{2\Lambda^2} \gamma^\mu \gamma^\nu F_{\mu\nu}\right)\right] \\
	&= \frac{e}{2\Lambda^2} \int \frac{d^2 k_E}{(2\pi)^2} e^{-k_E^2/\Lambda^2}\tr\left[\gamma^5 \gamma^\mu \gamma^\nu \right]F_{\mu\nu} \\
	&= \frac{e}{4\pi}\varepsilon^{\mu\nu}F_{\mu\nu}.
\end{aligned}
\end{equation}
We thus have
\begin{equation}
	\partial_\mu J^\mu_5 = \frac{e}{2\pi} \varepsilon^{\mu\nu} F_{\mu\nu}.
\end{equation}
In general, for $d=2n$ case, 
\begin{equation}
	\tr[\gamma^5 \gamma^{\mu_1}\cdots \gamma^{\mu_{2n}}] = (-i)^{n-1} 2^n \varepsilon^{\gamma^{\mu_1}\cdots \gamma^{\mu_{2n}}}
\end{equation}
The crucial part of the calculation is
\begin{equation}
\begin{aligned}
	&\ \tr\left[\gamma^5 \exp\left(-\frac{ie}{2\Lambda^2} \gamma^\mu \gamma^\nu F_{\mu\nu}\right)\right] \\
	=&\ \frac{1}{n!}\left(-\frac{ie}{2\Lambda^2}\right)^n \tr[\gamma^5 \gamma^{\mu_1}\cdots \gamma^{\mu_{2n}}] F_{\mu_1\mu_2}\cdots F_{\mu_{2n-1}\mu_{2n}} \\
	=&\ (-1)^n i \frac{e^n}{n! \Lambda^{2n}} \varepsilon^{\gamma^{\mu_1}\cdots \gamma^{\mu_{2n}}} F_{\mu_1\mu_2}\cdots F_{\mu_{2n-1}\mu_{2n}}
\end{aligned}
\end{equation}
The trace is then
\begin{equation}
\begin{aligned}
	\Tr[\gamma^5] 
	&= (-1)^{n+1} \frac{e^n}{n! \Lambda^{2n}}\varepsilon^{\gamma^{\mu_1}\cdots \gamma^{\mu_{2n}}} F_{\mu_1\mu_2}\cdots F_{\mu_{2n-1}\mu_{2n}} \int \frac{d^d k_E}{(2\pi)^d} e^{-k_E^2/\Lambda^2} \\
	&= (-1)^{n+1} \frac{e^n}{n! (4\pi)^n} \varepsilon^{\gamma^{\mu_1}\cdots \gamma^{\mu_{2n}}} F_{\mu_1\mu_2}\cdots F_{\mu_{2n-1}\mu_{2n}}.
\end{aligned}
\end{equation}
which leads to
\begin{equation}
	\partial_\mu J^\mu_5 = (-1)^{n+1} \frac{2 e^n}{n! (4\pi)^n} \varepsilon^{\gamma^{\mu_1}\cdots \gamma^{\mu_{2n}}} F_{\mu_1\mu_2}\cdots F_{\mu_{2n-1}\mu_{2n}}.
\end{equation}






\subsection{Triangle Diagram}
The trace (\ref{eq:AN-tr-1}) can be rewritten as 
\begin{equation}
	\Tr[\gamma^5] = -\mathrm{tr}[\gamma^5 iG_F(x,x)].
\end{equation}
The Fermion propagator in the presence of the background field is
\begin{equation}
\begin{aligned}
	iG_F(x_1,x_2) =&\ iG^0_F(x_1,x_2) + iG^0_F(x_1,y)[-ie\cancel{A}(y)]iG^0_F(y,x_2) \\
	& + iG^0_F(x_1,y_1)[-ie\cancel{A}(y_1)]iG^0_F(y_1,y_2)[-ie\cancel{A}(y_2)]iG^0_F(y_2,x_2).
\end{aligned}
\end{equation}
From previous discussion, it suggests that the chiral anomaly appears in the second order perturbation theory.
We thus consider the second order one-loop diagram (sometimes known as the triangle diagram):
\begin{equation}
\begin{aligned}
	i\mathcal{M}^{\mu\nu\rho}(p_1,p_2,q)
	=&\ \includegraphics[width=0.5\linewidth,align=c]{pics/AN-Tri.png} \\
	=&\ -\int \frac{d^{4} k}{(2 \pi)^{4}} \tr\left[\frac{i}{\cancel k} \gamma^{\rho} \gamma^{5} \frac{i}{\cancel k-\cancel q} \gamma^{\nu} \frac{i}{\cancel k+\cancel p_{1}} \gamma^{\mu} \right]  \\
	&\ -\int \frac{d^{4} k}{(2 \pi)^{4}} \tr\left[\frac{i}{\cancel k + \cancel \beta} \gamma^{\rho} \gamma^{5} \frac{i}{\cancel k-\cancel q +\cancel \beta} \gamma^{\mu} \frac{i}{\cancel k+\cancel p_{2}+\cancel \beta} \gamma^{\nu}\right]
\end{aligned}
\end{equation}
To the derivative of the axial current is related to the amplitude
\begin{equation}
\begin{aligned}
	q_\rho \mathcal{M}^{\mu\nu\rho} 
	=&\ \int \frac{d^{4} k}{(2 \pi)^{4}} \tr\left[\frac{1}{\cancel k} (\cancel{q}-\cancel{k}+\cancel{k}) \gamma^{5} \frac{1}{\cancel k-\cancel q} \gamma^{\nu} \frac{1}{\cancel k+\cancel p_{1}} \gamma^\mu \right] + \\
	&\ \int \frac{d^{4} k}{(2 \pi)^{4}} \tr\left[\frac{1}{\cancel k +\cancel \beta} (\cancel{q}-\cancel{k}-\cancel \beta +\cancel{k} +\cancel\beta) \gamma^{5} \frac{1}{\cancel k-\cancel q+\cancel\beta} \gamma^{\mu} \frac{1}{\cancel k+\cancel p_{2}+\cancel \beta} \gamma^\nu \right] \\
	=&\ \int \frac{d^{4} k}{(2 \pi)^{4}} \tr\left[\gamma^{5} \frac{1}{\cancel k+\cancel p_{1}} \gamma^\mu \frac{1}{\cancel k-\cancel q} \gamma^{\nu}  -  \gamma^{5}   \frac{1}{\cancel k+\cancel p_{1}} \gamma^\mu \frac{1}{\cancel k} \gamma^{\nu} \right]+ \\
	&\ \int \frac{d^{4} k}{(2 \pi)^{4}} \tr\left[\gamma^{5} \frac{1}{\cancel k-\cancel q+\cancel\beta} \gamma^{\mu} \frac{1}{\cancel k+\cancel p_{2}+\cancel \beta} \gamma^\nu -\gamma^5\frac{1}{\cancel k+\cancel\beta}\gamma^\mu\frac{1}{\cancel k+\cancel p_2+\cancel\beta}\gamma^\nu\right] \\
	\equiv &\ \Delta^{\mu\nu}_1 + \Delta^{\mu\nu}_2,
\end{aligned}
\end{equation}
where we define
\begin{equation}\label{eq:AN-Tri-div-int}
\begin{aligned}
	\Delta^{\mu\nu}_1 &= \int \frac{d^{4} k}{(2 \pi)^{4}} \tr\left[
		\gamma^{5} \frac{1}{\cancel k+\cancel p_{1}} \gamma^\mu \frac{1}{\cancel k+\cancel p_1+\cancel p_2} \gamma^{\nu} -
		\gamma^5\frac{1}{\cancel k+\cancel\beta}\gamma^\mu\frac{1}{\cancel k+\cancel p_2+\cancel\beta}\gamma^\nu 
	\right], \\
	\Delta^{\mu\nu}_2 &= \int \frac{d^{4} k}{(2 \pi)^{4}} \tr\left[
		\gamma^{5} \frac{1}{\cancel k+\cancel p_1 +\cancel p_2+\cancel\beta} \gamma^{\mu} \frac{1}{\cancel k+\cancel p_{2}+\cancel \beta} \gamma^\nu -
		\gamma^{5}   \frac{1}{\cancel k+\cancel p_{1}} \gamma^\mu \frac{1}{\cancel k} \gamma^{\nu}
	\right].
\end{aligned}
\end{equation}
We come cross a similar situation where the expression, although naively appears to be zero, is actually (linearly) divergent and may produce a finite number when introducing a regulator.
However, usual regulators fail in this specific case.
The dimensional regularization is not suitable since the chiral fermions are a feature of four dimensions.
Pauli-Villars, which would introduce a heavy fermion, will not work either, since the fermion mass explicitly breaks the chiral symmetry we are trying to verify. 
Instead, we proceed by trying to make sense of the linearly divergent integrals directly.

Consider the one-dimensional integral
\begin{equation}
	\Delta(a) = \int^{+\infty}_{-\infty} dx \left[f(x+a)-f(x)\right],
\end{equation}
where the function $f(x)$ approaches to constants in both $x \rightarrow \pm\infty$ limit.
We can evaluate the integral by Taylor expansion:
\begin{equation}
	\Delta(a) = \int^{+\infty}_{-\infty} dx \left[af'(x)+\frac{a^2}{2}f''(x)+\cdots\right]
	= a\left[f(+\infty)-f(-\infty)\right],
\end{equation}
where the higher-derivative terms do not contribute since $f(\pm\infty)\rightarrow \text{const}$. 
In four dimensions, we can do the similar thing.
In this case, we are evaluate the integral
\begin{equation}
\begin{aligned}
	\Delta(a) 
	&= \int \frac{d^4k}{(2\pi)^4} \left[F(k+a)-F(k)\right] \\
	&= i\int \frac{d^4k_E}{(2\pi)^4} \left[\frac{\partial F}{\partial k_E^\rho} a^\rho + \frac{1}{2} \frac{\partial^2 F}{\partial k_E^\rho \partial k_E^\sigma} a^\rho a^\sigma + \cdots\right].
\end{aligned}
\end{equation}
If the integral of $F(k)$ is linear divergent, only the first term contributes, and the result is just the integral over the surface:
\begin{equation}
\begin{aligned}
	\Delta(a) &= a^\rho \lim_{k\rightarrow \infty} \int \frac{d^3 S_\rho}{(2\pi)^4} F(k_E) \\
	&= a^\rho \lim_{k\rightarrow \infty} \int \frac{k^2_E k_{E\rho} d \Omega_4}{(2\pi)^4} F(k_E).
\end{aligned}
\end{equation}
To evaluate the integrals (\ref{eq:AN-Tri-div-int}), we just need to consider the function
\begin{equation}
\begin{aligned}
	\lim_{k\rightarrow \infty} F 
	&= \lim_{k\rightarrow \infty} \tr\left[\gamma^{5} \frac{1}{\cancel k + \cancel p} \gamma^\mu \frac{1}{\cancel k+\cancel q}\gamma^{\nu}\right] \\
	&= -4i \varepsilon^{\rho\mu\sigma\nu} \lim_{k\rightarrow \infty} \frac{k_\rho q_\sigma + p_\rho k_\sigma + p_\rho q_\sigma}{(k+p)^2 k^2} \\
	&= -4i \varepsilon^{\rho\mu\sigma\nu} \frac{k_\rho q_\sigma + p_\rho k_\sigma + p_\rho q_\sigma}{k^4}.
\end{aligned}
\end{equation}
We then have
\begin{equation}\label{eq:AN-CH-int-form}
\begin{aligned}
	\Delta(a) 
	&= 4 \varepsilon^{\rho\mu\sigma\nu} a^\tau \int \frac{d \Omega_4}{(2\pi)^4} \frac{k_\tau (k_\rho q_\sigma + p_\rho k_\sigma + p_\rho q_\sigma)}{k^2} \\
	&= 4 \varepsilon^{\rho\mu\sigma\nu} a^\tau \frac{\Omega_4}{(2\pi)^4} \frac{\delta_{\tau\rho}q_\sigma + \delta_{\tau\sigma}p_\rho}{4} \\
	&= \frac{1}{8\pi^2} \varepsilon^{\mu\nu\rho\sigma} (p-q)_\rho a_\sigma.
\end{aligned}
\end{equation}
For $\Delta_{1}^{\mu\nu}$, $p=\beta$, $q=p_2+\beta$ and $a=p_1-\beta$, Eq.~(\ref{eq:AN-CH-int-form}) gives
\begin{equation}
	\Delta^{\mu\nu}_1 = \frac{1}{8\pi^2} \varepsilon^{\mu\nu\rho\sigma} (p_1-\beta)_\rho p_{2\sigma}.
\end{equation}
For $\Delta_2^{\mu\nu}$, $p=p_1$, $q=0$, and $a=p_2+\beta$, Eq.~(\ref{eq:AN-CH-int-form}) gives
\begin{equation}
	\Delta^{\mu\nu}_2 = \frac{1}{8\pi^2} \varepsilon^{\mu\nu\rho\sigma} p_{1\rho}(p_2+\beta)_\sigma.
\end{equation}
The sum of two terms is
\begin{equation}
	q_\rho \mathcal{M}^{\mu\nu\rho}(p_1,p_2,q) = \frac{1}{8\pi^2} \varepsilon^{\mu\nu\rho\sigma} \left[2 p_{1\rho} p_{2\sigma} + (p_1+p_2)_\rho \beta_\sigma \right].
\end{equation}

The result leaves us a free parameter $\beta$.
To see what is the best choice of $\beta$, we recall that the Ward identity requires
\begin{equation}
	p_{1\mu} \mathcal{M}^{\mu\nu\rho}(p_1,p_2,q) = 0.
\end{equation}
A self-consistent gauge symmetry should be anomalous-free, i.e., the Ward identity should always be satisfies, and to each order perturbation. 
The left hand side for a specific choice of $\beta$, for the one-loop diagram gives the result
\begin{equation}
\begin{aligned}
	p_{1\mu} \mathcal{M}^{\mu\nu\rho} 
	=&\ \int \frac{d^{4} k}{(2 \pi)^{4}} \left\{\tr\left[\frac{1}{\cancel k} \gamma^\rho \gamma^{5} \frac{1}{\cancel k-\cancel q} \gamma^{\nu} \frac{1}{\cancel k+\cancel p_{1}} (\cancel p_1 +\cancel k -\cancel k) \right] \right. + \\
	&\ \left.\tr\left[\frac{1}{\cancel k +\cancel \beta} \gamma^\rho \gamma^{5} \frac{1}{\cancel k-\cancel q+\cancel\beta} (\cancel k -\cancel q +\cancel \beta -\cancel k -\cancel p_2-\cancel \beta) \frac{1}{\cancel k+\cancel p_{2}+\cancel \beta} \gamma^\nu \right]\right\} \\
	=&\ \int \frac{d^{4} k}{(2 \pi)^{4}} \tr\left[\gamma^{5} \frac{1}{\cancel k - \cancel q} \gamma^\nu \frac{1}{\cancel k} \gamma^{\rho} - \gamma^{5} \frac{1}{\cancel k - \cancel q} \gamma^\nu \frac{1}{\cancel k+\cancel p_1} \gamma^{\rho}\right]+ \\
	&\ \int \frac{d^{4} k}{(2 \pi)^{4}} \tr\left[\gamma^{5} \frac{1}{\cancel k+\cancel p_2+\cancel\beta} \gamma^{\nu} \frac{1}{\cancel k+\cancel \beta} \gamma^\rho -\gamma^5\frac{1}{\cancel k-\cancel q+\cancel\beta}\gamma^\nu\frac{1}{\cancel k+\cancel\beta}\gamma^\rho\right] \\
	\equiv &\ \tilde\Delta^{\mu\nu}_1 + \tilde\Delta^{\mu\nu}_2,
\end{aligned}
\end{equation}
where we have defined
\begin{equation}
\begin{aligned}
	\tilde\Delta^{\mu\nu}_1 &= \int \frac{d^{4} k}{(2 \pi)^{4}} \tr\left[
		\gamma^{5} \frac{1}{\cancel k + \cancel p_1 +\cancel p_2} \gamma^\nu \frac{1}{\cancel k} \gamma^{\rho} -
		\gamma^5\frac{1}{\cancel k+ \cancel p_1 +\cancel p_2+\cancel\beta}\gamma^\mu\frac{1}{\cancel k+\cancel\beta}\gamma^\nu 
	\right], \\
	\tilde\Delta^{\mu\nu}_2 &= \int \frac{d^{4} k}{(2 \pi)^{4}} \tr\left[
		\gamma^{5} \frac{1}{\cancel k+\cancel p_2+\cancel\beta} \gamma^{\nu} \frac{1}{\cancel k+\cancel \beta} \gamma^\rho -
		\gamma^{5} \frac{1}{\cancel k + \cancel p_1+\cancel p_2} \gamma^\nu \frac{1}{\cancel k+\cancel p_1} \gamma^{\rho}
	\right].
\end{aligned}
\end{equation}
For $\tilde\Delta_{1}^{\mu\nu}$, $p=\beta+p_1+p_2$, $q=\beta$ and $a=-\beta$, Eq.~(\ref{eq:AN-CH-int-form}) gives
\begin{equation}
	\tilde\Delta^{\mu\nu}_1 = -\frac{1}{8\pi^2} \varepsilon^{\mu\nu\rho\sigma} (p_1+p_2)_\rho \beta_{\sigma}.
\end{equation}
For $\tilde\Delta_2^{\mu\nu}$, $p=p_1+p_2$, $q=p_1$, and $a=\beta-p_1$, Eq.~(\ref{eq:AN-CH-int-form}) gives
\begin{equation}
	\tilde\Delta^{\mu\nu}_2 = \frac{1}{8\pi^2} \varepsilon^{\mu\nu\rho\sigma} (p_1-\beta)_\rho p_{2\sigma}.
\end{equation}
The sum gives
\begin{equation}
	p_{1\mu} \mathcal{M}^{\mu\nu\rho}(p_1,p_2,q) = \frac{1}{8\pi^2} \varepsilon^{\mu\nu\rho\sigma} p_{1\rho}(p_{2}-\beta)_\sigma.
\end{equation}
Using the symmetry of the diagram, we can make the exchange ($p_1 \leftrightarrow p_2$, $\mu \leftrightarrow \nu$), which inverse the loop direction and thus change the sign of $\beta$.
We immediately get
\begin{equation}
	p_{2\nu} \mathcal{M}^{\mu\nu\rho}(p_1,p_2,q) = -\frac{1}{8\pi^2} \varepsilon^{\mu\nu\rho\sigma} p_{2\rho}(p_{1}+\beta)_\sigma.
\end{equation}
Note that the sum of three terms is in dependent of $\beta$:
\begin{equation}
	(p_{1\mu}+p_{2\nu}+q_{\rho}) \mathcal M^{\mu\nu\rho}(p_1,p_2,q) = \frac{1}{2\pi^2} \varepsilon^{\mu\nu\rho\sigma} p_{1\rho} p_{2\sigma}.
\end{equation}
To enforce Ward identity, the only choice we have is
\begin{equation}
	\beta = p_2-p_1,
\end{equation}
which lead to
\begin{equation}
	q_\rho \mathcal{M}^{\mu\nu\rho}(p_1,p_2,q) = \frac{1}{2\pi^2} \varepsilon^{\mu\nu\rho\sigma} p_{1\rho} p_{2\sigma}.
\end{equation}
We remark that in the path-integral formalism where we directly calculate the Jacobian of the measure, the heat-kernel regulator ensures the gauge invariant, so the Ward identity of the vector current is automatically ensured, at the expense of chiral symmetry.

The vertex function for the axial current and gauge field interaction is
\begin{equation}
	\Gamma_A^{\mu\nu\rho}(p_1,p_2,q) = -e^2 \mathcal{M}^{\mu\nu\rho}(p_1,p_2,q).
\end{equation}
Also, note that because of the symmetry, in the effective action, the contribution from this vertex is $\frac{1}{2}\Gamma_A^{\mu\nu\rho}$. 
The expectation of axial current derivative $\partial_\mu J_5^\mu(x)$ in the presence of back ground field $A(x)$ is then
\begin{equation}
\begin{aligned}
	\langle \partial_\mu J_5^\mu(x) \rangle 
	&= \frac{1}{2} \int \frac{d^4 q}{(2\pi)^4}\frac{d^4 p_1}{(2\pi)^4}\frac{d^4 p_2}{(2\pi)^4} e^{-i(q+p_1+p_2)x} q_\rho \Gamma^{\mu\nu\rho} \tilde A_\mu(p_1) \tilde A_\nu(p_2) \\
	&= -\frac{e^2}{4\pi^2} \int \frac{d^4 q}{(2\pi)^4}\frac{d^4 p_1}{(2\pi)^4}\frac{d^4 p_2}{(2\pi)^4} e^{-i(q+p_1+p_2)x} \varepsilon^{\mu\nu\rho\sigma} p_{1\rho} \tilde A_\mu(p_1) p_{2\sigma} \tilde A_\nu(p_2) \\
	&= -\frac{e^2}{4\pi^2} \int \frac{d^4 q}{(2\pi)^4}\frac{d^4 p_1}{(2\pi)^4}\frac{d^4 p_2}{(4\pi)^4} e^{-i(q+p_1+p_2)x} \varepsilon^{\mu\nu\rho\sigma} \left(-\frac{1}{4}\right)\tilde F_{\rho\mu}(p_1) \tilde F_{\sigma\nu}(p_2) \\
	&= -\frac{e^2}{16\pi^2} \varepsilon^{\mu\nu\rho\sigma} F_{\mu\nu}(x) F_{\rho\sigma}(x)
\end{aligned}
\end{equation}
In the above calculation, we use the Fourier transformed 
\begin{equation}
	\tilde F_{\mu\nu}(p) = \int \frac{d^4 x}{(2\pi)^4} e^{ipx} \partial_{[\mu} A_{\nu]}
	= i p_{[\mu} \tilde A_{\nu]}(p).
\end{equation}
The one-loop calculation tells us that all information of the chiral anomaly lies in the triangle diagram.
Sometimes, it is called that the chiral anomaly is one-loop exact.



\section{Gauge Anomaly}
While anomalies in global symmetry are physically interesting, anomalies in gauge symmetries render the theory mathematically inconsistent, since the gauge is not a symmetry, but a redundancy in describing the system.
If we wish to build a consistent theory, we must make sure that all gauge anomaly vanish.

\subsection{Chiral Fermions}
If the fermion field is massless, in the Dirac Lagrangian, the left- and right- handed spinors are formally decoupled.
One may wonder if we can build a QED with only left-handed massless Weyl fermions.
The answer is no, since the U(1) gauge field will be anomalous.

Now consider the triangle diagram with three QED vertices:
\begin{equation}
	i\mathcal M^{\mu\nu\rho}(p,q,r) = \includegraphics[width=0.5\linewidth,align=c]{pics/AN-Tri-2.png}.
\end{equation}
Note that in the above diagram, all fermion lines are chiral -- they involve only left-handed spinor.
We can evaluate the diagram by replacing all the chiral fermion lines with Dirac fermion lines, plus an additional projector inserted: 
\begin{equation}
	\tr\left[\frac{1-\gamma^5}{2} (\cdots)\right] = \frac{1}{2}\tr\left[(\cdots)\right] - \frac{1}{2}\tr\left[\gamma^5 (\cdots)\right].
\end{equation}
The first term is just an ordinary photon vertex in QED, which is not anomalous.
The second term, however, is exactly the triangle diagram producing the chiral anomaly.
Here the axial current is coupled to the gauge field, the result, we know from the previous calculation that
\begin{equation}
	(p_\mu + q_\nu + r_\rho)\mathcal M^{\mu\nu\rho} 
	= -\frac{ie^3}{4\pi^2}\varepsilon^{\mu\nu\rho\sigma} p_\rho q_\sigma
	= -\frac{ie^3}{4\pi^2}\varepsilon^{\mu\nu\rho\sigma} q_\rho r_\sigma
	= -\frac{ie^3}{4\pi^2}\varepsilon^{\mu\nu\rho\sigma} r_\rho p_\sigma.
\end{equation}
We conclude from this the QED with charged single-handed Weyl fermion is not consistent.
Moreover, if we consider a chiral theory, where the left- and right- handed spinors have different charges.
The triangle diagram is the sum of all Weyl spinors.
The anomalous-free condition is thus
\begin{equation}
	\sum_{i=1}^{N_L} q_{Li}^3 = \sum_{j=1}^{N_R} q_{Rj}^3.
\end{equation}
In standard model, the $\mathrm{U}(1)_{\mathrm{Y}}^3$ anomaly cancels out.
To see this, consider the first generation fermions:
\begin{equation}
\begin{tabular}{c|cc|cccc}
	\hline
	   & $Q$ & $L$  & $u_R$ & $d_R$ & $e_R$ & $\nu_{eR}$ \\ \hline
	6Y & $1$ & $-3$ & $4$   & $-2$  & $-6$  & $0$        \\
	\hline
\end{tabular}
\end{equation}
Here we write the hyperchage in the unit of $1/6$, the difference between left and right hand spinor is
\begin{equation}
\begin{aligned}
	\sum_{i=1}^{N_L} (6Y_{Li})^3 - \sum_{j=1}^{N_R} (6Y_{Rj})^3 
	&= \left[1^3 \times 6 + (-3)^3 \times 2 \right] -
	\left[4^3 \times 3 + (-2)^3 \times 3 + (-6)^3 \right] \\
	&= 0.
\end{aligned}
\end{equation}
We remark that the quark sector or lepton sector alone will lead to $\mathrm{U}(1)_{\mathrm{Y}}^3$ anomaly.
It is the combination of them that make the gauge theory consistent.




\subsection{Nonabelian Gauge}

We are now consider the nonabelian gauge anomaly.
As a convention, we absorb the coupling constant into the Lie algebra $A^a T^a_R$.
The perturbative anomaly still comes from the triangle diagram.
We have seen that the outcome 3-point vertex is symmetric under changing the external legs.
It means, when coupled to the gauge field $A = A^a T^a$, there will be an additional factor
\begin{equation}
	\tr\left[T_R^a \{T_R^b, T_R^c\}\right] = d^{abc}(R) = A(R) d^{abc}(\text{fund}).
\end{equation}
We see that the symmetric coefficients $d^{abc}$ is related to the anomaly of the nonabelian gauge anomaly.
Following the same logic, the anomaly cancellation condition is
\begin{equation}
	\sum_{i=1}^{N_L} d^{abc}(R_{Li}) = \sum_{j=1}^{N_R} d^{abc}(R_{Rj}).
\end{equation}
Note that for the abelian case, the generator is $qI$, and the symmetric coefficient is $d^{111} = q^3$, which leads to the same condition.
One important fact about the coefficients $d^{abc}$ is that it vanishes for real or pseudoreal representations.
To see this, we first note that the generators are hermitian.
The generators of the conjugate representation is
\begin{equation}
	\bar T^a = - (T^{a})^* = - (T^{a})^T.
\end{equation}
A real or pseudoreal representation means $T^a \sim \bar T^a$, which leads to
\begin{equation}
\begin{aligned}
	d^{abc}(R) 
	&= \tr\left[T^a_R \left\{T^b_R, T^c_R \right\} \right] \\
	&= \tr\left[\bar T_R^a \left\{\bar T_R^b, \bar T_R^c \right\} \right] \\
	&= -\tr\left[(T_R^a)^T \left\{(T_R^b)^T,(T_R^c)^T \right\} \right] \\
	&= -\tr\left[T_R^a \left\{T_R^b,T_R^c \right\} \right] = 0.
\end{aligned}
\end{equation}



\end{document}


