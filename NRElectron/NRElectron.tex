\documentclass[aps,prb,superscriptaddress,nofootinbib]{revtex4}
\usepackage{amsfonts}
\usepackage{amsmath}
\usepackage{amssymb}
\usepackage{graphicx}
\usepackage{caption}
\usepackage{bm}
\usepackage{bbm}
\usepackage{cancel}
\usepackage{color}
\usepackage{mathrsfs}
\usepackage[colorlinks,bookmarks=true,citecolor=blue,linkcolor=red,urlcolor=blue]{hyperref}
\usepackage{appendix}
\usepackage{float}
\usepackage{array}
\usepackage{booktabs}
\setlength{\parindent}{10 pt}
\setlength{\parskip}{2 pt}
\setcounter{MaxMatrixCols}{30}
\bibliographystyle{apsrev}

\newcommand{\normord}[1]{{:\mathrel{#1}:}}
\def\tbs{\textbackslash}
\def \tr{\operatorname{tr}}
\def \Tr{\operatorname{Tr}}


\begin{document}
\title{Nonrelativistic Electrons}
\author{Jie Ren}



\maketitle

\tableofcontents



\section{Homogeneous Electron Gas}


In this section, we consider the finite temperature electron system.
A general non-relativistic field theory is described by the action (with repeated indices automatically summed):
\begin{equation}
	S = S_0 + S_{\mathrm{int}} = \int dt \int d^d x\ \mathcal{L}_0 - \int dt\ \mathcal{V}_{\mathrm{int}}[\bar\psi,\psi],
\end{equation}
where the free theory is described by a quadratic Lagrangian $\mathcal{L}_0 = \bar\psi (i\partial_t-\hat H)\psi$.
The classical equation of motion for the free field satisfies the Schr\"{o}dinger equation:
\begin{equation}
	\partial_\mu \frac{\partial \mathcal L_0}{\partial(\partial_\mu \bar\psi_a)} - \frac{\partial \mathcal L_0}{\partial\bar{\psi}_a} 
	= - i\partial_t \psi_a + \hat H_{ab}\psi_b = 0.
\end{equation}
For the homogeneous electron gas, we will mostly work in the momentum and frequency domain.
To obtain a similar form as the relativistic field theory, we use the similar definition of the Fourier transformation:
\begin{equation}
\begin{aligned}
	\psi_a(k,\omega) &= \int dt\int_{L^d} d^dx\ e^{-i k \cdot x+i\omega t}\psi_a(x,t), \\
	\psi_a(x,t) &= \frac{1}{L^d}\sum_{k} e^{i k \cdot x} \int \frac{d\omega}{2\pi} e^{-i\omega t} \psi_a(k,\omega) 
	\sim \int\frac{d\omega}{2\pi}\int_{|k|<\Lambda}\frac{d^d k}{(2\pi)^d} e^{i k \cdot x-i\omega t} \psi_a(k,\omega).
\end{aligned}
\end{equation}
Note that in condensed matter system, we usually use the lattice regularization: we think of the system as a finite $d$-dimensional cubic with length $L = N a$.
The lattice spacing $a$ impose a natural UV cutoff $\Lambda = \pi/a$.
In the thermodynamic limit, the summation becomes the integral: 
\begin{equation}
	\frac{1}{L^d}\sum_k \longrightarrow \int_{|k|<\Lambda} \frac{d^d k}{(2\pi)^d}.
\end{equation}
In the momentum space, the free theory can be simplified:
\begin{equation}
	S_0[\bar\psi,\psi] = \int dt \int \frac{d^d k}{(2\pi)^d} \bar\psi(k) [i\partial_t-\hat H(k)]\psi_b(k)
	= \int\frac{d\omega}{2\pi}\int\frac{d^d k}{(2\pi)^d} \bar\psi(k, \omega)[\omega - \hat H(k)]\psi(k, \omega),
\end{equation}
which gives the real-time Green's function
\begin{equation}
	iG_0(k,\omega) = \frac{1}{\omega - \hat H(k)}
\end{equation}



\subsection{Finite Temperature Formalism}

The original real-time partition function is defined as\footnote{As with the relativistic case, we introduce an auxiliary source $J$, which is bosonic/fermionic if the field $\psi$ is bosonic/fermionic.
}
\begin{equation}
	Z[J] = \int D[\bar\psi,\psi] \exp\left\{i\int dt \int d^dx \left[\mathcal{L}+\bar{J}_a(x)\psi_a(x)+\bar{\psi}_a(x)J_a(x)\right]\right\}.
\end{equation}
If we make a analytic continuation of $t$ to the complex plane: $t \rightarrow -i\tau, \ \omega \rightarrow i\omega$, the free action transforms as:
\begin{equation}
	iS_0 = i \int dt dx\ \psi(\bm x, t) (i\partial_t - E) \psi(\bm x, t) \ \longrightarrow \ 
	-\int d\tau dx\ \psi(\bm x, \tau) (\partial_\tau + E) \psi(\bm x, \tau).
\end{equation}
Note that in the frequency domain, the singularities for positive frequency lies below the complex plane, as we always include an infinitesimal $-i\epsilon$ to the energy (mass) term of the theory in ensure convergence. So, the rotation of the real axis anti-clock-wisely to the imaginary axis will not cross any singularity, and thus the can be analytically extended.
The partition function can then be written as $Z[J] = \int D[\bar\psi,\psi] e^{-S_0[\bar\psi,\psi]+\bar{J}\cdot\psi+\bar{\psi}\cdot J}$, where the Euclidean free action defined as:
\begin{equation}
	S = \int d\tau \left[\int d^dx\ \bar\psi_a(\bm x,\tau) (\delta_{ab}\partial_\tau+\hat H_{ab})\psi_b(\bm x,\tau) + \mathcal{V}_\mathrm{int}\right].
\end{equation}
The Euclidean action is suitable to describe the system both in zero temperature or finite temperature.
For finite temperature case, the integral over the imaginary time $\tau$ is over $[0,\beta)$.
The Fourier transformation of the field on the imaginary time domain is defined as:\footnote{Note that the fermion field satisfies the anti-periodic boundary condition on the interval $[0,\beta)$: it change sign when crossing the boundary. The Matsubara frequencies for fermions are $\omega_n = (2n+1)\pi/\beta$ for $n \ge 0$.}
\begin{equation}
	\psi(\omega_n) = \int_0^\beta d\tau e^{i\omega_n\tau} \psi(\tau),\quad
	\psi(\tau) = \frac{1}{\beta}\sum_{\omega_n} e^{-i\omega_n\tau} \psi(\omega_n).
\end{equation}
Under such convention, in the thermodynamic limit and zero-temperature limit, the spatial-temporal Fourier transformation agrees with the relativistic case (up to a Wick rotation).
The Fourier transformation of the free field action is
\begin{equation}
	S_0 = \frac{1}{\beta}\sum_{\omega_n} \int_{\Lambda} \frac{d^dk}{(2\pi)^d}
	\bar{\psi}_a(k,\omega_n)\left[-i\omega_n + \hat{H}_{ab}(k)\right]\psi_b(k,\omega_n).
\end{equation}
The partition function with source is
\begin{equation}
	\frac{Z_0[J]}{Z_0[0]} = \exp\left[-\frac{1}{\beta}\sum_{\omega_n} \int_{\Lambda} \frac{d^dk}{(2\pi)^d}\bar J_a(k,\omega_n) G_{ab}(k,\omega_n) J_b(k,\omega_n) \right],
\end{equation}
where the Green's function is 
\begin{equation}
	G_{ab}(k,\omega_n) = \frac{1}{i\omega_n - \hat H(k)}.
\end{equation}
Unlike the relativistic case, the value of the value of partition function without source $Z_0[0]$ is related to the free energy.
We can express it formally as $Z_0[0]= \det (-G_{ab})$.
To get the correct dimensionality, we set the determinant as 
\begin{equation}
	Z_0[0] \equiv \prod_{k,\omega_n}\left\{\beta \det\left[-i\omega_n+\tilde{H}(k)\right]\right\}.
\end{equation}
The summation on Matsubara frequencies is capture by the singularities of the density function of the states:
\begin{equation}
	\rho(z) = \frac{1}{e^{\beta z}+1}.
\end{equation}
The residue on imaginary frequency $i\omega_n$ is alway $\frac{1}{\beta}$. In this way, the summation is:
\begin{equation}
	\frac{1}{\beta}\sum_{\omega_n} f(i\omega_n) 
	= \frac{1}{2\pi i} \oint \rho(z)f(z).
\end{equation}
The right-hand side can usually be evaluated using the residue theorem.



\subsection{RPA Correction to Photon Propagator}
Under the random phase approximation, the photon 
\begin{equation}
	\Pi(p,\omega) 
\end{equation}






\end{document}


