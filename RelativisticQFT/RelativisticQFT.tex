\documentclass[aps,prb,superscriptaddress,nofootinbib]{revtex4}
\usepackage{amsfonts}
\usepackage{amsmath}
\usepackage{amssymb}
\usepackage{graphicx}
\usepackage{caption}
\usepackage{bm}
\usepackage{bbm}
\usepackage{cancel}
\usepackage{color}
\usepackage{mathrsfs}
\usepackage[colorlinks,bookmarks=true,citecolor=blue,linkcolor=red,urlcolor=blue]{hyperref}
\usepackage{appendix}
\usepackage{float}
\usepackage{array}
\usepackage{booktabs}
\setlength{\parindent}{10 pt}
\setlength{\parskip}{2 pt}
\setcounter{MaxMatrixCols}{30}
\bibliographystyle{apsrev}

\newcommand{\normord}[1]{{:\mathrel{#1}:}}
\def\tbs{\textbackslash}
\def \tr{\operatorname{tr}}
\def \Tr{\operatorname{Tr}}


\begin{document}
\title{Relativistic Quantum Field Theories}
\author{Jie Ren}



\maketitle


The Lorentz invariance is the fundamental symmetry for relativistic field theory.
For generic $(d+1)$-dimensional spacetime, the Minkowski metric is $g_{\mu\nu}=g^{\mu\nu} = \operatorname{diag}(+1,-1,\cdots,-1)$.
A Lorentz transformation ${\Lambda^{\mu}}_{\nu}$ preserves the metric: ${\Lambda^{\mu}}_{\alpha}{\Lambda^{\nu}}_{\beta} g^{\alpha\beta} = g^{\mu\nu}$.
We can multiply a $g^{\gamma\alpha}$ on both sides to get: $g^{\gamma\alpha}{\Lambda^{\mu}}_{\alpha}{\Lambda^{\nu}}_{\beta} g_{\mu\nu} = g^{\gamma\alpha}g_{\alpha\beta}$, which leads to ${\Lambda_{\nu}}^{\gamma}{\Lambda^{\nu}}_{\beta} = {\delta^{\gamma}}_{\beta}$.
The inverse Lorentz transformation satisfies ${(\Lambda^{-1})^{\mu}}_{\nu} = {\Lambda_{\nu}}^{\mu}$.

Consider the infinitesimal Lorentz transformation ${\Lambda^{\mu}}_{\nu} = \delta^{\mu}_{\nu}+\delta{\omega^{\mu}}_{\nu}$, its inverse is ${(\Lambda^{-1})^\mu}_\nu = \delta^{\mu}_{\nu}-\delta{\omega^\mu}_\nu$, which means $\delta {\omega^\mu}_\nu = -\delta {\omega_\nu}^\mu$.
We can further use the metric tensor $g_{\mu\nu}$ to lower the indices and get $\delta\omega_{\alpha\beta} = -\delta\omega_{\beta\alpha}$, i.e., the infinitesimal parameter $\delta \omega_{\mu\nu}$ is anti-symmetric under the swap of indices $\mu \leftrightarrow \nu$.
In general, given a representation $M_R$ of the Lie algebra, we can exponentiate the $\delta \omega_{\mu\nu}$ to obtain the representation of the Lorentz group as $U_R(\Lambda) = \exp\left(\frac{i}{2}\omega_{\mu\nu}M_R^{\mu\nu}\right)$, where the anti-symmetric $\omega_{\mu\nu}$'s are the finite group parameters.


\tableofcontents

\section{(3+1)D Lorentz Invariance}
For the $(3+1)$D spacetime, the Lorentz group can be represented as 
\begin{equation}\label{eq:lorentz-parameter}
	\Lambda(\bm \theta,\bm \beta) = \exp\left(i \theta_i J_i +i\beta_i K_i\right), \quad
	\theta_i \equiv \frac{1}{2}\varepsilon_{ijk}\omega_{jk},\quad
	\beta_i \equiv \omega_{i0},
\end{equation}
where the Lie algebra contains 3 rotational generators $J_i \equiv \frac{1}{2}\varepsilon_{ijk}M^{jk}$ and 3 boost generators $K_i \equiv M^{i0}$. 
In the fundamental representation, the generators are represented by
\begin{equation}
\begin{aligned}
	J_1 &= \left[\begin{array}{cccc} 0 & & & \\ & 0 & & \\ & & 0 & -i \\ & & i & 0 \end{array}\right], & 
	J_2 &= \left[\begin{array}{cccc} 0 & & & \\ & 0 & & i \\ & & 0 & \\ & -i & & 0 \end{array}\right], &
	J_3 &= \left[\begin{array}{cccc} 0 & & & \\ & 0 & -i & \\ & i & 0 & \\ & & & 0 \end{array}\right], \\
	K_1 &= \left[\begin{array}{cccc} 0 & -i & & \\ -i & 0 & & \\ & & 0 & \\ & & & 0 \end{array}\right], & 
	K_2 &= \left[\begin{array}{cccc} 0 & & -i & \\ & 0 & & \\ -i & & 0 & \\ & & & 0 \end{array}\right], &
	K_3 &= \left[\begin{array}{cccc} 0 & & & -i \\ & 0 & & \\ & & 0 & \\ -i & & & 0 \end{array}\right].
\end{aligned}
\end{equation}
The Lie algebra of the Lorentz algebra can be explicitly done using the fundamental representation. 
The result is
\begin{equation}
	\left[J_i, J_j\right] = i \varepsilon_{ijk} J_k, \quad
	\left[J_i, K_j\right] = i \varepsilon_{ijk} K_k, \quad
	\left[K_i, K_j\right] = -i\varepsilon_{ijk} J_k.
\end{equation}
A special combination of those generators $N_i^{L/R} \equiv \frac{1}{2}(J_i \mp i K_i)$ will create two independent algebras:
\begin{equation}\label{eq:RFT-LG-Alg}
\begin{aligned}
	\left[N_i^L, N_j^L \right] &= i\varepsilon_{ijk}N_k^L, \quad
	\left[N_i^R, N_j^R \right] &= i\varepsilon_{ijk}N_k^R, \quad
	\left[N_i^L, N_j^R \right] &= 0.
\end{aligned}
\end{equation}
In the following, we show such generators give all irreducible representations of the Lorentz group $\mathrm{SO}(3,1)$, which are the building blocks of the relativistic field theory.




We see from (\ref{eq:RFT-LG-Alg}) that after the recombination, the Lorentz algebra breaks into two independent $\mathfrak{su}(2)$ algebra.
Mathematically it means $\mathfrak{so}(3,1) \simeq \mathfrak{su}(2) \oplus \mathfrak{su}(2)$.
That is,
\begin{equation}
	U_{j_L,j_R}(\Lambda)
	= \exp\left[i(\bm\theta+i\bm\beta)\cdot \bm N^L_{j_L} + i(\bm\theta-i\bm\beta)\cdot \bm N^R_{j_R}\right],
\end{equation}
where we see that the representation of the Lorentz algebra can be labeled by $j_L$ and $ j_R$.

In relativistic QFT, the Lorentz symmetry restricts the possible terms that can appear in the Lagrangian.
Different free fields correspond to different representations of the Lorentz algebra, and the Lagrangian should be singlet under the Lorentz transformations.
In the following, we will introduce the invariant symbols, which can be regarded as the Clebsch-Gordan coefficients, to help construct the singlets: 
\begin{equation*}
\begin{tabular}{ccc}
	\hline \hline 
	Symbol & Representation  & Invariant \\ \hline
	$1$ & $(0,0)$ & $\phi$\\
	$\varepsilon^{ab}$ & $(2,0)\otimes (2,0)$ & $\psi_L \cdot \chi_L$ \\
	$\varepsilon^{\dot a \dot b}$ & $(0,2)\otimes (0,2)$ & $\psi_R \cdot \chi_R$ \\
	$\sigma^\mu_{\dot ab}$ & $(2,2)\otimes (0,2)\otimes(2,0)$ & $a_\mu \psi_L^\dagger \sigma^\mu \chi_L$ \\
	$\bar \sigma^\mu_{a \dot b}$ & $(2,2)\otimes (2,0)\otimes(0,2)$ & $a_\mu \psi_R^\dagger \bar\sigma^\mu \chi_R$ \\
	$\varepsilon^{\mu\nu\rho\sigma}$ & $(2,2)^{\otimes 4}$ & $\varepsilon^{\mu\nu\rho\sigma}a_\mu b_\nu c_\rho d_\sigma$ \\
	\hline \hline
\end{tabular}
\end{equation*}

\subsection{Trivial representation.}
The $(j_L,j_R) = (0,0)$ representation corresponds to the scalar field denoted as $\phi(x)$.
Since the field itself is singlet, any polynomial of the field in principle can appear in the theory.
When considering the free theory, we restrict our attention to the quadratic terms, therefore the allowed free theory can only be
\begin{equation}
	\mathcal L_{\mathrm{KG}} = \frac{1}{2}\partial^\mu \phi \partial_\mu \phi -\frac{m^2}{2}\phi^2 
	\simeq -\frac{1}{2}\phi (\partial^2+m^2) \phi.
\end{equation}
Besides the field Lagrangian $\mathcal L_{\mathrm{KG}}$, there are more general Lorentz-invariant terms that can be added to the Lagrangian, which describe the interaction of the theory.
One of the simplest interacting scalar field theories is the $\phi^4$ theory $\mathcal L = \mathcal L_{\mathrm{KG}} - \frac{g}{4!}\phi^4$.

\subsection{Spinor representation.}
The spinor representations are those with $j_L=1/2$ or $j_R=1/2$. 
Specifically, we define the left-hand spinor $\psi_L = (\psi_L^1, \psi_L^2)^T$ and right-hand spinor $\psi_R = (\psi_R^1, \psi_R^2)^T$ whose transformations define $\Lambda_L$ and $\Lambda_R$:
\begin{equation}\label{eq:qft-left-right-spinor-rep}
	\Lambda_L(\bm\theta,\bm\beta) = \exp\left(\frac{i}{2}\bm\theta\cdot\bm\sigma-\frac{1}{2}\bm\beta\cdot\bm\sigma \right), \quad
	\Lambda_R(\bm\theta,\bm\beta) = \exp\left(\frac{i}{2}\bm\theta\cdot\bm\sigma+\frac{1}{2}\bm\beta\cdot\bm\sigma \right) = \Lambda_L(\bm \theta, -\bm \beta).
\end{equation}
We can create a ``scalar-like'' object by the inner product of the spinors.
Note that the $\psi_L^\dagger \psi_R$ and $\psi_R^\dagger \psi_L$ are Lorentz invariant object.
The product of two single-handed spinors like $\psi_L^\dagger \psi_L$ is not Lorentz invariant.
However, note that $\sigma^2 \psi_L^*$ transforms like a right-handed spinor:\footnote{We use the identity $\sigma^2 \cdot \bm\sigma^* \cdot\sigma^2 = -\bm\sigma$ in the second equation.}
\begin{equation}
	\sigma^2 \psi_L^*
	\rightarrow \sigma^2 \exp\left(-\frac{i}{2}\bm\theta\cdot\bm\sigma^*-\frac{1}{2}\bm\beta\cdot\bm\sigma^* \right) \sigma^2 \sigma^2 \psi_L^* 
	= \Lambda_L(\bm \theta, -\bm\beta) \sigma^2 \psi_L^*.
\end{equation}
For this reason, the left-hand and right-hand spinor can be interchanged by the action of $\sigma^2 \mathcal K$ ($\mathcal K$ is complex conjugation).
In this way, we get new invariants $\psi_L^T \sigma^2 \psi_L$ and $\psi_R^T \sigma^2 \psi_R$.
To simplify the notation here, we introduce two sets of spinor indices $a$ and $\dot a$, where the undotted index transforms as $(2,1)$, and the dotted index transforms as $(1,2)$:
\begin{equation}
	\Lambda(\bm\theta,\bm\beta) \psi^a \equiv {[\Lambda_L(\bm\theta,\bm\beta)]^a}_b \psi^b, \quad
	\Lambda(\bm\theta,\bm\beta) \psi^{\dot a} \equiv {[\Lambda_L(\bm\theta,-\bm\beta)]^{\dot a}}_{\dot b} \psi^{\dot b}.
\end{equation}
We see working on this set of indices, we no longer need to specify the chirality.
Moreover, we have seen from the above discussion that the left- and right-handed spinors are interchangeable, it is the representation that matters.
To relate to the invariants we get, we introduce two invariant symbols, corresponding to the decomposition $2 \otimes 2 = 1 \oplus 3$:
\begin{equation}
	\varepsilon^{ab} = \varepsilon^{\dot a \dot b} = i\sigma^2, \quad
	\varepsilon_{ab} = \varepsilon_{\dot a \dot b} = -i\sigma^2.
\end{equation}
The symbols are used to raise or lower the spinor indices, for example, $\psi_a = \varepsilon_{ab}\psi^b$, and $\psi^a = \varepsilon^{ab}\psi_b$.
The lower indices transform as $\psi_a \rightarrow \psi_b {[\Lambda_L(-\bm\theta,-\bm\beta)]^b}_a$.
In this way, contracting superscript and subscript ensures the Lorentz invariance.
The invariant we got can be expressed as 
\begin{equation}
	-i\psi_L^T \sigma^2 \psi_L = \varepsilon_{ab}\psi_L^a \psi_L^b \equiv \psi_L\cdot \psi_L,\quad
	-i\psi_R^T \sigma^2 \psi_R = \varepsilon_{\dot a \dot b}\psi_R^{\dot a} \psi_R^{\dot b} \equiv \psi_R\cdot \psi_R.
\end{equation}
Also, note that fact that the conjugate field $\psi^\dagger_L$ transforms like $\psi_{\dot a}^\dagger$, i.e., $\psi_{\dot a}^\dagger \rightarrow \psi_{\dot b}^\dagger {[\Lambda_L(-\bm\theta,+\bm\beta)]^{\dot b}}_{\dot a}$.
This produces $\psi^\dagger_R \psi_L$ as Lorentz invariant.
The free (quadratic) Lagrangian for spinor field can have the following terms:
\begin{equation}
	\psi_L^\dagger \bar\sigma^\mu \partial_\mu \psi_L,\ 
	\psi_R^\dagger \sigma^\mu \partial_\mu \psi_R,\ 
	\psi_L^\dagger \psi_R,\ \psi_R^\dagger \psi_L,\ 
	\psi_L \cdot \psi_L + \psi_L^\dagger \cdot \psi_L^\dagger,\ 
	\psi_R \cdot \psi_R + \psi_R^\dagger \cdot \psi_R^\dagger.
\end{equation}
The Dirac field describes the theory with both left-hand and right-hand spinors.
The Lagrangian is $\mathcal{L}_{\mathrm{Dirac}} = \bar\psi \left(i\gamma^\mu \partial_\mu - m\right)\psi$, where the Dirac spinor contains left- and right-handed Weyl spinors:
\begin{eqnarray}
	\psi = \begin{pmatrix}
		\psi_L \\ \psi_R
	\end{pmatrix},\ 
	\bar\psi = \begin{pmatrix}
		\psi_R^\dagger & \psi_L^\dagger
	\end{pmatrix},\ 
	\gamma^\mu = \begin{bmatrix}
		0 & \sigma^\mu \\
		\bar\sigma^\mu & 0
	\end{bmatrix}.
\end{eqnarray}
In addition, we could consider using the last two terms as the mass, the result theory is the \textit{Majorana field theory}:
\begin{equation}
\begin{aligned}
	\mathcal{L}_{\mathrm{Maj}}
	= i \psi_L^\dagger \bar\sigma^\mu \partial_\mu  \psi_L -m(\psi_L \cdot \psi_L + \psi_L^\dagger \cdot \psi_L^\dagger).
\end{aligned}
\end{equation} 
For the spinor basis, the Dirac Algebra is generated by $M^{\mu\nu} = \frac{i}{4}[\gamma^\mu, \gamma^\nu]$.
Using the familiar parametrization (\ref{eq:lorentz-parameter}), 
\begin{equation}
	J_i = \begin{bmatrix}
		\sigma^i & 0 \\ 0 & -\sigma^i
	\end{bmatrix}, \quad 
	K_i = \frac{i}{2}\begin{bmatrix}
		\sigma^i & 0 \\ 0 & -\sigma^i
	\end{bmatrix},
\end{equation}
which agree with the transformation property (\ref{eq:qft-left-right-spinor-rep}).


\subsection{Vector representation.}
The symbols $\sigma^\mu = (1, \bm \sigma)^\mu$ and $\bar\sigma^\mu=(1,-\bm\sigma)$ comes from the decomposition $\left(2, 1\right) \otimes \left(1,2\right) \otimes \left(2, 2\right) = \left(1, 1\right) \oplus \cdots$.
The symbol $\sigma^\mu_{\dot a b}$ transforms as:
\begin{equation}\label{eq:RFT-sigma_symbol}
	\sigma^\mu_{\dot a b} \rightarrow
	{\Lambda^\mu}_\nu(\bm\theta,\bm\beta) \sigma^\nu_{\dot c d}
	{\left[\Lambda_L(-\bm\theta, \bm\beta)\right]^{\dot c}}_{\dot a} {\left[\Lambda_L(-\bm\theta, -\bm\beta)\right]^d}_b,
\end{equation}
The spinor part cancels the transformations of the left- and right-handed spinor fields, leaving the combination $\psi_L^\dagger \sigma^\mu \psi_L$ transform like a Lorentz vector.

Now we explicitly check the symbol $\sigma^\mu$ indeed transforms like (\ref{eq:RFT-sigma_symbol}).
Here we temporarily drop the index notation and adopt the view that only the spinor field transform.
We then absorb the transformation matrix from the spinors to the $\sigma^\mu$ symbol.
Firstly, for the spatial rotation, $\Lambda_L(\bm\theta,\bm 0) = \exp(i\bm\theta\cdot \bm\sigma/2)$, the Pauli matrices transform as the defining SO(3) rotation
\begin{equation}
	\left(1-\frac{i}{2}\delta\bm\theta\cdot \bm\sigma \right)\sigma^j\left(1+\frac{i}{2}\delta\bm\theta\cdot \bm\sigma\right)
	= \sigma^j + i\delta\theta_i \left(-i \varepsilon_{ijk}\sigma^k \right).
\end{equation}
Secondly, for the boost $\Lambda_{L}(0, \bm\beta) = \exp\left(-\bm\beta\cdot \bm\sigma /2 \right)$, the Pauli matrices transform like a boost in the defining representation of the Lorentz group:
\begin{equation}
	\left(1+\frac{1}{2}\delta\bm\beta\cdot \bm\sigma\right) \sigma^\mu 
	\left(1+\frac{1}{2}\delta\bm\beta\cdot \bm\sigma\right) = \begin{cases}
		 \sigma^0 + i\delta\beta_i \cdot (-i\sigma^i), & \mu = 0 \\
		 \sigma^j + i\delta\beta_j (-i\sigma^0), & \mu = j
	\end{cases}.
\end{equation}
The story for the $\bar\sigma^\mu=(1,-\bm\sigma)$ is basically the same, which transforms as
\begin{equation}
	\bar\sigma^\mu_{a \dot b} \rightarrow
	{\Lambda^\mu}_\nu(\bm\theta,\bm\beta) \bar\sigma^\nu_{c \dot d}
	{\left[\Lambda_L(-\bm\theta, -\bm\beta)\right]^{c}}_{a} {\left[\Lambda_L(-\bm\theta, \bm\beta)\right]^{\dot d}}_{\dot b}.
\end{equation}
Finally, we note that the Levi-Civita symbol $\varepsilon^{\mu\nu\rho\sigma}$ can also be useful.
One invariant involving this is $\varepsilon^{\mu\nu\rho\sigma} F_{\mu\nu} F_{\rho\sigma}$, where $F_{\mu\nu}$ is antisymmetry under ($\mu \leftrightarrow \nu$).

If we choose $(j_L=j_R=1/2)$, the field is transformed as Lorentz vector.
We denote the field as $A^\mu(x)$.
Some possible quadratic forms for the vector field that forms singlets are
\begin{equation}
	A^\mu A_\mu,\ (\partial_\mu A^\mu)^2,\ A^\nu \partial^2 A_\nu,\ 
	\varepsilon_{\mu\nu\rho\lambda} \partial^\mu A^\nu \partial^\rho A^\lambda.
\end{equation}
For the field theory describe the electromagnetic field, we require the theory to further have gauge symmetry, i.e., invariant under
\begin{equation}
	A^\mu(x) \rightarrow A^\mu(x) + \partial^\mu \alpha(x).
\end{equation}
The gauge invariant forbids the first term, and forces the second and third term to combine as
\begin{equation*}
	(\partial_\mu A^\mu)^2 - A^\nu \partial^2 A_\nu
	\sim \frac{1}{2}(\partial^\mu A^\nu - \partial^\nu A^\mu)(\partial_\mu A^\nu-\partial_\nu A_\mu)
	\equiv \frac{1}{2} F^{\mu\nu}F_{\mu\nu}.
\end{equation*}
where we have define a field-strength tensor
\begin{equation}
	F^{\mu\nu}\equiv (\partial^\mu A^\nu - \partial^\nu A^\mu)
	= \left[\begin{array}{cccc}
		0 & -E_1 & -E_2 & -E_3 \\
		E_1 & 0 & -B_3 & B_2 \\
		E_2 & B_3 & 0 & -B_1 \\
		E_3 & -B_2 & B_1 & 0
	\end{array} \right]^{\mu\nu},
\end{equation}
where we notice that from Maxwell equations:
\begin{equation}
	E^i = \partial_t \vec A = -\vec\nabla A^0, \quad B^i = \nabla \times \vec A.
\end{equation}
Note that the fourth term is called the \textit{theta term}, which can be written as a boundary term
\begin{equation}
	\varepsilon_{\mu\nu\rho\lambda} \partial^\mu A^\nu \partial^\rho A^\lambda
	= \partial^\mu (\varepsilon_{\mu\nu\rho\lambda} A^\nu \partial^\rho A^\lambda).
\end{equation}
The Lagrangian describing the electromagnetic field is given by
\begin{equation}
	\mathcal{L}_{\mathrm{Maxwell}} = -\frac{1}{4}F_{\mu\nu}F^{\mu\nu}.
\end{equation}




\section{Path Integral Formalism}

\subsection{Generating Functionals}
In this section, we are discussing general interacting field theory.
Consider the action for free field with source
\begin{equation}
	S[\phi,J]
	= \int d^dx\left[\mathcal{L}(\phi) + J(x)\cdot\phi(x) \right].
\end{equation}
In the path integral formalism, we consider the partition function with source: $Z[J] = \int D[\phi]\ e^{iS[\phi,J]}$.
The central quantity a field theory produces is the correlation function of field values at different spacetime points:
\begin{equation}
	C(x_1,\cdots,x_n) \equiv \langle \phi(x_1)\cdots \phi(x_n)\rangle,
\end{equation}
where $\langle \cdots \rangle$ denotes the (time-ordered) average defined by the functional integral:
\begin{equation}
	\langle \cdots \rangle \equiv \frac{1}{Z[0]}\int D[\phi]\ e^{-iS[\phi]} (\cdots).
\end{equation}
We see that the partition function produces all possible correlation functions:
\begin{equation}
	C(x_1,\cdots,x_n) = \left. \prod_{i=1}^n \left[\frac{\delta}{i\delta J(x_i)}\right] Z[J] \right|_{J=0}.
\end{equation}
The partition function $Z[J]$ is thus a generating functional.
For interaction theory, the perturbation approach is that we first evaluate the generating functional for free field $Z_0[J]$, then the generating functional for interacting field can be formally expressed as:
\begin{equation}
	Z[J] = \exp\left(i\int d^dx \mathcal{L}_{\mathrm{int}}\left[\frac{\delta}{i\delta J(x)}\right]\right)Z_0[J],
\end{equation}
which can then be Taylor expanded order by order.
Since the unconnected diagram can be absorbed into $Z[0]$, we only need to calculate the connected diagrams.
The procedure of perturbative expansion with only connected diagrams can be formally represented by introducing the quantity 
\begin{equation}
	Z[J] = Z[0]\exp\left(i W[J]\right).
\end{equation}
The perturbative expansion of $W[J]$ contains only the connected diagrams.
Note that for the free theory,
\begin{equation*}
	\frac{Z_0[J]}{Z_0[0]} = \exp\left[-\frac{i}{2}\int d^d x_1 d^d x_2 J(x_1) \Delta(x_1-x_2)J(x_2)\right],
\end{equation*}
which means $W_0 = -\frac{1}{2}\int d^d x_1 d^d x_2 J(x_1) \Delta(x_1-x_2)J(x_2)$.


Consider the four-point connected correlation $iV_4 \equiv \langle \mathcal{T}\phi(x_1) \phi(x_2) \phi(x_3) \phi(x_4)\rangle_c$.
Following the same procedure,
\begin{equation}
\begin{aligned}
	iV_4 
	&= i\left.\frac{\delta^4 W[J]}{i\delta J(x_1)i\delta J(x_2)i\delta J(x_3)i\delta J(x_4)}\right|_{J=0} \\
	&= \frac{1}{Z[0]}\left.\frac{\delta^4 Z[J]}{i\delta J_1 i\delta J_2 i\delta J_3 i\delta J_4}\right|_{J=0} 
	-i\Delta_{1,2} i\Delta_{3,4} -i\Delta_{1,3} i\Delta_{2,4} -i\Delta_{1,4} i\Delta_{2,3}.
\end{aligned}
\end{equation}
The connected correlation function automatically omits those disconnected components.

\subsection{Free Scalar Field}

Now we evaluate the propagator in the path-integral formalism.
In momentum space, the free action (with source) is\footnote{The space-time Fourier transformation is: 
\begin{equation*}
	\tilde{f}(k) = \int d^{d}x\ e^{ik\cdot x} f(x),\quad
	f(x) = \int \frac{d^{d}k}{(2\pi)^{d}}\ e^{-ik\cdot x}\tilde{f}(k),
\end{equation*}
where the inner product of two $d$-vectors is defined as $a \cdot b = a^0 b^b - \bm a \cdot \bm b$.}
\begin{equation*}
	\frac{1}{V}\sum_k \left[\frac{1}{2}\tilde\phi^*(k)( k^2-m^2)\tilde\phi(k)+\tilde J^*(k)\cdot\tilde\phi(k)+\tilde\phi^*(k)\cdot\tilde J(k)\right].
\end{equation*}
For real field, $\tilde\phi^*(k) = \tilde\phi(-k)$.
For our convenience, we have expressed the momentum integral as summation.
Actually, consider the $d$-dimensional box of size $L^d$, the momentum along each axis is multiple of $2\pi/L$, so when $L\rightarrow \infty$, the summation approaches in integral, $\frac{1}{V}\sum_k \rightarrow \int \frac{d^d k}{(2\pi)^d}$.
Let us omit the $1/V$ factor, the summation can be formally expressed as $\frac{1}{4}\mathbf{v}^T \cdot \mathbf M\cdot \mathbf{v} + \frac{1}{2}\mathbf{j}^T \cdot \mathbf{v}$, where
\begin{equation*}
	\mathbf v = \bigoplus_{|\mathbf k|} \left[
	\begin{array}{c}
		\tilde{\phi}(k) \\ 
		\tilde{\phi}^*(k) 
	\end{array}\right], \quad
	\mathbf M = \bigoplus_{|\mathbf k|} \left[
	\begin{array}{cc} 
		0 & k^2-m^2 \\ 
		k^2-m^2 & 0 
	\end{array}\right], \quad
	\mathbf j = \bigoplus_{|\mathbf k|} \left[
	\begin{array}{c}
		\tilde{J}^*(k) \\ 
		\tilde{J}(k) 
	\end{array}\right].
\end{equation*}
Note that in the above expression, we have made an infinitesimal shift of mass ($m^2 \rightarrow m^2 - i\epsilon$) to ensure the convergence of the Gaussian integral.
The integrated variables $v_i$ is not real.
To use the real Gaussian integral formula, we make use of a unitary transformation: 
\begin{equation*}
	\mathbf U = \frac{1}{\sqrt 2} \left[\begin{array}{cc}
		1 & 1 \\
		-i & i
	\end{array}\right], \quad
	\mathbf U \cdot \left[
	\begin{array}{c}
		\tilde{\phi}(k) \\ 
		\tilde{\phi}^*(k) 
	\end{array}\right] 
	= \frac{1}{\sqrt 2}\left[
	\begin{array}{c}
		\tilde\phi(k)+\tilde\phi^*(k) \\ 
		-i\tilde\phi(k)+i\tilde\phi^*(k)
	\end{array}\right]
	\equiv \left[
	\begin{array}{c}
		\tilde\phi_1(k) \\ 
		\tilde\phi_2(k) 
	\end{array}\right]
\end{equation*}
The path integral then becomes a real field integral.
Recall the real Gaussian integral formula:
\begin{equation}
	\int d\mathbf x \exp\left(-\frac{1}{2}\mathbf{x}^T \cdot \mathbf A \cdot \mathbf{x} + \mathbf{B}^T \cdot \mathbf{x}\right) 
	= \sqrt{\frac{(2\pi)^N}{\det{\mathbf A}}}\exp\left(\frac{1}{2}\mathbf{B}^T \cdot \mathbf{A}^{-1} \cdot \mathbf{B}\right),
	\label{eq:real-gaussian-integral}
\end{equation}
For the field integral, we absorbed the $(2\pi)^{N/2}$ term into the measure, and express the path integral for the Gaussian field as:
\begin{equation}
	W_0[J] 
	= -\frac{i}{4}\int \frac{d^d k}{(2\pi)^d} \mathbf j^T_k \cdot \mathbf M^{-1}_k \cdot \mathbf j_k
	= -\frac{1}{2} \int \frac{d^d k}{(2\pi)^d}  \tilde{J}^*(k) \tilde{\Delta}_0(k) \tilde{J}(k).
\end{equation}
This gives the propagator in the momentum space:
\begin{equation}
	\tilde{\Delta}_0(k) = \frac{i}{k^2-m^2}
	\quad \Longrightarrow \quad 
	\Delta_0(x_1-x_2) = i\int\frac{d^{4} k}{(2\pi)^{4}} \frac{e^{-i k\cdot (x_1-x_2)}}{k^2-m^2}.
\end{equation}


\subsection{From Field to Force}
Consider two separate particle described by the delta function $J_a(x) = \delta^{(3)}(\bm x - \bm x_a)$, together the source is $J(x) = J_1(x) + J_2(x)$.
Adding the source,
\begin{equation*}
	W_0[J] = -\frac{1}{2}\int d^4x_1 d^4 x_2 J(x_1) \Delta_0(x_1-x_2) J(x_2)
\end{equation*}
Omit the self energy terms $J_1^2(x), J_2^2(x)$, $W_0[J]$ is
\begin{equation}
\begin{aligned}
	W_0[J] &= -\int d^4 y_1 d^4 y_2\ e^{-ik^0(y_1^0-y_2^0)}\int \frac{d^4 k}{(2\pi)^4} J_1(y_1)\frac{e^{i\bm k\cdot (\bm y_1-\bm y_2)}}{k^2-m^2} J_2(y_2) \\
	&= -\int  dt \int d (y_1^0 - y_2^0) \ e^{-ik^0(y_1^0-y_2^0)}\int \frac{d^4 k}{(2\pi)^4} \frac{e^{i\bm k\cdot (\bm y_1-\bm y_2)}}{k^2-m^2} \\
	&= \left(\int dt \right)\int \frac{d^3 k}{(2\pi)^3} \frac{e^{i\bm k\cdot (\bm y_1-\bm y_2)}}{\bm k^2 + m^2}
\end{aligned}
\end{equation}
Recall that the partition function is actually infinite:
\begin{equation}
	Z_0 \sim \langle 0| e^{-i H_0 T} |0\rangle \quad \Longrightarrow \quad
	W_0 = -i E T,
\end{equation}
where $E$ is the energy.
Writing $\bm r \equiv \bm y_1 - \bm y_2$, and $u \equiv \cos\theta$ with $\theta$ the angle between $\bm k$ and $\bm r$, the volume form is $dk \cdot kd\theta \cdot  2\pi k \sin \theta = 2\pi k^2 dk du$, and the integral is
\begin{equation}
	E = -\int \frac{d^3 k}{(2\pi)^3} \frac{e^{i k r u}}{k^2 + m^2} 
	= - \frac{1}{(2\pi)^2} \int_0^\infty k^2 dk \int_{-1}^1 du \frac{e^{ikru}}{k^2 +m^2} 
	= -\frac{1}{2\pi^2 r} \int_0^\infty k  \frac{\sin kr}{k^2 +m^2} dk.
\end{equation}
Since the integral is even, we can extend the integral to
\begin{equation}
	E = -\frac{1}{4\pi^2 r} \int_{-\infty}^\infty k  \frac{\sin kr}{k^2 +m^2} dk 
	= \frac{i}{4\pi^2 r} \int_{-\infty}^\infty \frac{k e^{ikr}}{k^2 +m^2} dk
\end{equation}
The residue theorem gives
\begin{equation}
	\int_{-\infty}^\infty \frac{k e^{ikr}}{k^2 +m^2} dk = \pi ie^{-mr}
\end{equation}
So we get the potential of two particles:
\begin{equation}\label{eq:field-to-force}
	V(r) = -\frac{e^{-mr}}{4\pi r},
\end{equation}
and the attractive force is
\begin{equation}
	F(r) = -\frac{dV}{dr} = -(1+mr)\frac{e^{-mr}}{4\pi r^2}.
\end{equation}
We see that in the massless case, the force gives the long-range Coulomb force $F \propto 1/r^2$, while in the massive field theory, the force is short-ranged, with the decay length proportional to the mass.






\end{document}


