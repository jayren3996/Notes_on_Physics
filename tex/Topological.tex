\chapter{Topological Field Theory}

\section{Chern-Simons Theory}
Assume the action of the microscopical theory has the form $S[\psi_i]$, where $\{\psi_i\}$ denotes all degrees of microscopical freedom.
If the system has the U(1) symmetry, we can always rewrite the field theory as a gauge theory:
\begin{equation}
	S[\psi_i; A] = S[\psi_i] + \int d^dx\ j^\mu(x) A_\mu(x),
\end{equation}
where the current $j^\mu$ is the Noether current.
The gauge field $A^\mu(x)$ is regarded as the back ground field which has no dynamics.
If we are interested in the low-energy physics, especially for gapped system, the ground state physics, we can formally integrate out other degrees of freedom, the resulting effective theory has only the gauge degree of freedom:
\begin{equation}
	Z_{\mathrm{eff}}[A] = \int D[\psi_i] e^{i S[\psi_i;A]}.
\end{equation}
In this section, we consider the effective gauge field on $(2+1)$-dimensional space-time.
The effective action should also be gauge-invariant.
The allowed terms include
\begin{equation*}
	A \wedge dA,\ dA \wedge dA,\ \text{higher order terms}.
\end{equation*}
From dimensional analysis, the first term is most relevant in the low-energy.
Such effective theory is the \textit{Chern-Simons theory}:
\begin{equation}
	S_{\mathrm{CS}} = \frac{k}{4\pi}\int d^3 x\ \varepsilon_{\mu\nu\rho} A^\mu \partial^\nu A^\rho.
\end{equation}
