\chapter{Lattice Systems}

\section{Lattice Spins}

\subsection{The Ising Model}
The Ising model on the Euclidean space is described by the action
\begin{equation}
	S[\{s_k\}] = -K \sum_{\langle ij\rangle} s_i s_j,\quad s_k = \pm 1.
\end{equation}
where $\langle ij \rangle$ is the nearest-neighbor sites and the coupling is
\begin{equation}
	K = \beta J = \frac{J}{T}.
\end{equation}
The phase of the Ising model can be revealed by considering the correlation function:
\begin{equation}
	G_{ij} \equiv \langle \sigma_i^z \sigma_j^z\rangle
	= \frac{1}{Z} \sum_{\{s_k\}} e^{-S[\{s_k\}]} s_i s_j,
\end{equation}
where the partition function is
\begin{equation}
	Z = \sum_{\{s_k\}} e^{-S[\{s_k\}]}
\end{equation}


\subsubsection{Series Expansion}
The behavior of correlation in the high- and low- temperature limit can be seen using the series expansion.
Consider first the high-temperature limit where $K \rightarrow 0$, the partition function can be expanded in different order of $K$:
\begin{equation}
\begin{aligned}
	Z &= \sum_{\{s_k\}} \prod_{\langle ij\rangle} \cosh K (1 + s_i s_j \tanh K) \\
	&\sim \sum_{\{s_k\}} \prod_{\langle ij\rangle} (1 + s_i s_j K).
\end{aligned}
\end{equation}
The only terms that survive the averaging form a non-crossing path from site $i$ to site $j$.
In the small $K$ limit, the main contribution comes from the shortest path, i.e.,
\begin{equation}
	G_{ij} \propto K^{-r_{ij}},
\end{equation}
where $r_{ij}$ is the distance (Manhattan metric) from $i$ to $j$.
We see in high temperature the correlation is exponentially decaying, indicating a disorder phase.
Note that for $d=1$ case, there is only one path from $i$ to $j$, and the exact result is
\begin{equation}
	G_{ij} = \frac{(2\cosh K \tanh K)^{|i-j|}}{(2 \cosh K)^{|i-j|}}
	= \left(\tanh K\right)^{|i-j|},
\end{equation}
independent of the temperature, so the 1d Ising model has only one phase.

For $d\ge 2$ case, in the lower temperature limit, the dominant contribution to the partition function comes from the ferromagnetic configuration.
The excitations are the spin domains, whose energy is proportional to their perimeters.
For higher dimensional system, the creation of the domain is suppressed, leading to an ordered phase.



\section{Lattice Fermions}
In this section, we consider the system whose Hamiltonian composed of quadratic fermionic operators, i.e.,
\begin{equation}
	\hat H_{\mathrm{free}} = \sum_{i,j=1}^N A_{ij} c_i^\dagger c_j + \frac{1}{2}\sum_{i,j=1}^N B_{ij} c_i c_j + \frac{1}{2}\sum_{i,j=1}^N B_{ij}^* c_j^\dagger c_i^\dagger, \label{eq:lattice-free-fermion-hamiltonian}
\end{equation}
where $A$ is a Hermitian matrix, and matrix $B$ is anti-symmetric.
Without loss of generality, in the following we always assume that the sum of chemical potential is zero, i.e., $\mathrm{Tr} A=0$.
In the Nambu basis 
\begin{equation}
	\Psi = (c_1,\dots,c_N,c_1^\dagger,\dots,c_N^\dagger)^T,
\end{equation}
the Hamiltonian has the BdG form:
\begin{equation}
	\hat H_{\mathrm{free}} = \frac{1}{2} \sum_{i,j=1}^{2N} \Psi^\dagger_i H_{ij}^{\Psi} \Psi_j + \frac{1}{2}\mathrm{Tr}A,
\end{equation}
where the single-body matrix $H^{\Psi}$ is a $2N\times 2N$ Hermitian matrix
\begin{equation}
	H^{\Psi} = \left[\begin{array}{cc} 
		A & B \\
		-B^* & -A^* 
	\end{array}\right].
\end{equation}
Note that in the Nambu basis, the single-body Hamiltonian matrix has the particle-hole symmetry
\begin{equation}
	P = \sigma_x \mathcal K, 
	\quad \Longrightarrow \quad
	P H^{\Psi} P = -H^{\Psi}.
\end{equation}
This means the spectrum of the BdG Hamiltonian is symmetric with respect to zero.


\subsection{Majorana Representation}

The Majorana operators are defined as:
\begin{equation}
	\left[\begin{array}{c} \omega_{i} \\ \omega_{i+N} \end{array}\right]
	= \left[\begin{array}{cc} 
		1 & 1 \\ 
		i & -i 
	\end{array}\right] \left[\begin{array}{c} 
		c_i \\ c_i^\dagger 
	\end{array}\right], \quad 
	\left[\begin{array}{c} c_i \\ c_i^\dagger \end{array}\right]
	= \frac{1}{2} \left[\begin{array}{cc} 
		1 & -i \\ 
		1 & i 
	\end{array}\right] \left[\begin{array}{c} 
		\omega_{i} \\ \omega_{i+N}
	\end{array}\right].
\end{equation}
The Majorana operator satisfies the Fermion-like commutation relation
\begin{equation}
	\{\omega_i, \omega_j\} = 2\delta_{ij}.
\end{equation}
The fermionic bilinear in the Majorana basis has the form
\begin{equation}
	\hat H = -\frac{i}{4} \sum_{i,j=1}^{2N} H_{ij} \omega_i \omega_j
\end{equation}
where the single-body matrix $H$ is a $2N \times 2N$ real anti-symmetric matrix:
\begin{equation}
	H = \left[\begin{array}{cc} 
		-A^I - B^I & A^R - B^R \\
    	-A^R - B^R &  -A^I + B^I 
	\end{array}\right].
\end{equation}
where we have define $A^{R/I} = \mathrm{Re} A / \mathrm{Im} A$ and $B^{R/I} = \mathrm{Re} B / \mathrm{Im} B$.
Conversely, if we have a Majorana bilinear 
\begin{equation}
	\frac{i}{2} \sum_{i,j=1}^{2N} M_{ij}\omega_i \omega_j, \quad
	M = \left[\begin{array}{cc}
		M^{11} & M^{12} \\ M^{21} & M^{22}
	\end{array} \right],
\end{equation}
it can be transformed back to ordinary fermionic bilinear (\ref{eq:lattice-free-fermion-hamiltonian}) where
\begin{equation}
\begin{aligned}
	A &= M^{21} - M^{12} + i M^{11} + i M^{22}, \\
	B &= M^{21} + M^{12} + i M^{11} - i M^{22}.
	\label{eq:lattice-majorana-bilinear-to-fermion}
\end{aligned}
\end{equation}
A real anti-symmetric matrix can be transformed to standard form by an orthogonal transformation $O$:
\begin{equation}
\begin{aligned}
	H &= O \cdot \Sigma(\bm \lambda) \cdot O^T, \\
	\Sigma(\bm \lambda) &= i\sigma_y \otimes \mathrm{diag}(\lambda_1,\cdots,\lambda_n).
\end{aligned}
\end{equation}
Make the basis transformation
\begin{equation}
	\gamma_n = \sum_{j=1}^{2N} O_{jn} \omega_j,
\end{equation}
the Hamiltonian becomes the standard form:
\begin{equation}
\begin{aligned}
	H &= -\frac{i}{4} \sum_{i=1}^N \lambda_i (\gamma_i \gamma_{i+N}-\gamma_{i+N} \gamma_i) \\
	&= -\frac{i}{2} \sum_{i=1}^N \lambda_i \gamma_i \gamma_{i+N}.
\end{aligned}
\end{equation}
Each $\gamma_i \gamma_{i+N}$ pair can then transforms to independent fermion mode:
\begin{equation}
\begin{aligned}
	-\frac{i}{2}\gamma_i \gamma_{i+N} 
	&= -\frac{i}{2}(d_i + d_i^\dagger)(id_i-id_i^\dagger) \\ 
	&= d_i^\dagger d_i-\frac{1}{2}.
\end{aligned}
\end{equation}



\subsection{Gaussian States}
The Fermionic Gaussian states are those states with Gaussian form density operator:
\begin{equation}
	\hat \rho \propto \exp \left(\frac{i}{2}\sum_{i,j=1}^{2N}M_{ij}\omega_i \omega_j \right),
\end{equation}
where the matrix $M$ is real and anti-symmetric.\footnote{In particular, any thermal state has this form, with $M = \beta H/2$. The ground state of the free fermion system, though being pure state, can be regarded as the Gaussian state in the limit $M = \lim_{\beta \rightarrow \infty} \beta H$.}
If we expand the Gaussian form, the density operator becomes a Majorana polynomial:\footnote{Note that the coefficient $\Gamma$ in each order is not the direct expansion of the matrix $M$, since the direct expansion contains identical Majorana operators. That is, the $n$-th order expansion of the Majorana Gaussian form may contribute to the ($n-2m$)-th order term in the Majorana polynomial.}
\begin{equation}
	\hat{\rho} = \frac{\mathbb{I}}{2^N} + \sum_{n=1}^{N}\frac{i^n}{2^N}\sum_{1\le i_{1}<\cdots<i_{2n} \le 2N}\Gamma_{i_{1}\cdots i_{2n}} \omega_{i_1}\cdots\omega_{i_{2n}},
\end{equation}
where the coefficient $\Gamma_{i_1 \cdots i_{2n}}$ is the $2n$-point correlation function:
\begin{equation}
	\Gamma_{i_1 \cdots i_{2n}} = i^n \langle \omega_{i_1} \cdots \omega_{i_{2n}}\rangle, \quad i_m \ne i_n.
\end{equation}
In particular, the 2-point function 
\begin{equation}
	\Gamma_{ij} = i\langle \omega_i \omega_j\rangle - i\delta_{ij} = \frac{i}{2}\langle [\omega_i, \omega_j]\rangle
\end{equation}
is also called the \textit{covariance matrix}. 
For Gaussian state all $2n$-point correlation is determined by the covariance matrix by the Wick theorem.
\begin{framedrmk}[Two-point Correlation Function]
We are usually more familiar with the ordinary fermionic two-point correlation function $\langle c^\dagger_i c_j\rangle$ or $\langle c_i c_j\rangle$, which is related to the Majorana covariance matrix by:
\begin{equation}
\begin{aligned}
	\langle c_i^\dagger c_j\rangle &= \frac{1}{4}(
		\Gamma^{21}_{ij} - \Gamma^{12}_{ij} + 
		i \Gamma^{11}_{ij} + i \Gamma^{22}_{ij})
		+\frac{1}{2}\mathbb \delta_{ij}, \\
	\langle c_i c_j\rangle &= \frac{1}{4}(
		\Gamma^{21}_{ij} + \Gamma^{12}_{ij} + 
		i \Gamma^{11}_{ij} - i \Gamma^{22}_{ij}), \\
	\langle c_i^\dagger c_j^\dagger\rangle &= \frac{1}{4}(
		-\Gamma^{21}_{ij} - \Gamma^{12}_{ij} + 
		i \Gamma^{11}_{ij} - i \Gamma^{22}_{ij}).
\end{aligned}
\end{equation}
\end{framedrmk}

The relation of the correlation in each order can be neatly captured by the Grassmannian Gaussian form:
\begin{equation}
\begin{aligned}
	\omega(\hat \rho, \theta) 
	&= \frac{1}{2^N} \exp \left(\frac{i}{2} \sum_{i,j=1}^{2N}\Gamma_{ij}\theta_i \theta_j \right) \\
	&=\frac{1}{2^N} + \sum_{n=1}^{N}\frac{i^n}{2^N}\sum_{1\le i_{1}<\cdots<i_{2n} \le 2N}\Gamma_{i_{1}\cdots i_{2n}} \theta_{i_1} \cdots \theta_{i_{2n}}.
\end{aligned}
\end{equation}
When the covariance matrix is obtained, we can use the same routine to canonicalize the skew-symmetric matrix $\Gamma$:
\begin{equation*}
	\Gamma = O \cdot \Sigma(\bm \lambda) \cdot O^T, \quad
	\tilde\theta_n = \sum_i O_{in} \theta_i,
\end{equation*}
and the density matrix in the Grassmann representation is
\begin{equation}
	\omega(\hat \rho, \theta) 
	= \prod_{n=1}^N \left(\frac{1}{2} e^{i \lambda_n \tilde\theta_n \tilde\theta_{n+N}} \right)
	= \prod_{n=1}^N \left(\frac{1+i\lambda_n \tilde\theta_n\tilde\theta_{n+N}}{2}  \right).
\end{equation}
This state correspond to a product state $\rho = \otimes_n \rho_n$ where
\begin{equation}
	\rho_n = \frac{1}{2} \left[\begin{array}{cc}
		1 + \lambda_n & 0 \\
		0 & 1 - \lambda_n
	\end{array} \right].
\end{equation}
The entanglement entropy is then
\begin{equation}
	S=\sum_n S_n = -\sum_n \left[
	\left(\frac{1+\lambda_n}{2}\right)\ln\left(\frac{1+\lambda_n}{2}\right)
	+ \left(\frac{1-\lambda_n}{2}\right)\ln\left(\frac{1-\lambda_n}{2}\right)\right].
\end{equation}


\subsection{Jordan-Wigner Transformation}
Some lattice spin model can be mapped to fermion one by the Jordan-Wigner (J-W) transformation, which defines the isomorphism between the fermion and spin Hilbert space.
One a single site, we map the 




\section{Lattice Gauges}



