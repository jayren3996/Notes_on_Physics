\chapter{Lattice Systems}

\section{Free Fermion Systems}
In this section, we consider the system whose Hamiltonian composed of quadratic fermionic operators, i.e.,
\begin{equation}
	\hat H_{\mathrm{free}} = \sum_{i,j=1}^N A_{ij} c_i^\dagger c_j + \frac{1}{2}\sum_{i,j=1}^N B_{ij} c_i c_j + \frac{1}{2}\sum_{i,j=1}^N B_{ij}^* c_j^\dagger c_i^\dagger, \label{eq:lattice-free-fermion-hamiltonian}
\end{equation}
where $t_{ij}$ is a Hermitian matrix, and $\Delta_{ij}$ is anti-symmetric.
In the Nambu basis 
\begin{equation}
	\Psi = (c_1,\dots,c_N,c_1^\dagger,\dots,c_N^\dagger)^T,
\end{equation}
the Hamiltonian has the form\footnote{Without loss of generality, in the following we always assume that the sum of chemical potential is zero, i.e., $\mathrm{Tr} A=0$.}
\begin{equation}
	\hat H_{\mathrm{free}} = \frac{1}{2} \sum_{i,j=1}^{2N} \Psi^\dagger_i H_{ij}^{\Psi} \Psi_j + \frac{1}{2}\mathrm{Tr}A,
\end{equation}
where the single-body matrix $H^{\Psi}$ is a $2N\times 2N$ Hermitian matrix
\begin{equation}
	H^{\Psi} = \left[\begin{array}{cc} 
		A & B \\
		-B^* & -A^* 
	\end{array}\right].
\end{equation}

\subsection{Majorana Representation}

The Majorana operators are defined as:
\begin{equation}
	\left[\begin{array}{c} \omega_{i} \\ \omega_{i+N} \end{array}\right]
	= \left[\begin{array}{cc} 
		1 & 1 \\ 
		i & -i 
	\end{array}\right] \left[\begin{array}{c} 
		c_i \\ c_i^\dagger 
	\end{array}\right], \quad 
	\left[\begin{array}{c} c_i \\ c_i^\dagger \end{array}\right]
	= \frac{1}{2} \left[\begin{array}{cc} 
		1 & -i \\ 
		1 & i 
	\end{array}\right] \left[\begin{array}{c} 
		\omega_{i} \\ \omega_{i+N}
	\end{array}\right].
\end{equation}
The fermionic bilinear in the Majorana basis has the form
\begin{equation}
	\hat H = -\frac{i}{4} \sum_{i,j=1}^{2N} H_{ij} \omega_i \omega_j
\end{equation}
where the single-body matrix $H$ is a $2N \times 2N$ real anti-symmetric matrix:
\begin{equation}
	H = \left[\begin{array}{cc} 
		-A^I - B^I & A^R - B^R \\
    	-A^R - B^R &  -A^I + B^I 
	\end{array}\right].
\end{equation}
where we have define $A^{R/I} = \mathrm{Re} A / \mathrm{Im} A$ and $B^{R/I} = \mathrm{Re} B / \mathrm{Im} B$.
Conversely, if we have a Majorana bilinear 
\begin{equation}
	\frac{i}{2} \sum_{i,j=1}^{2N} M_{ij}\omega_i \omega_j, \quad
	M = \left[\begin{array}{cc}
		M^{11} & M^{12} \\ M^{21} & M^{22}
	\end{array} \right],
\end{equation}
it can be transformed back to ordinary fermionic bilinear (\ref{eq:lattice-free-fermion-hamiltonian}) where
\begin{equation}
\begin{aligned}
	A &= M^{21} - M^{12} + i M^{11} + i M^{22}, \\
	B &= M^{21} + M^{12} + i M^{11} - i M^{22}.
	\label{eq:lattice-majorana-bilinear-to-fermion}
\end{aligned}
\end{equation}
A real anti-symmetric matrix can be transformed to standard form by an orthogonal transformation $O$:
\begin{equation}
\begin{aligned}
	H &= O \cdot \Sigma(\bm \lambda) \cdot O^T, \\
	\Sigma(\bm \lambda) &= i\sigma_y \otimes \mathrm{diag}(\lambda_1,\cdots,\lambda_n).
\end{aligned}
\end{equation}
Make the basis transformation
\begin{equation}
	\gamma_n = \sum_{j=1}^{2N} O_{jn} \omega_j,
\end{equation}
the Hamiltonian becomes the standard form:
\begin{equation}
\begin{aligned}
	H &= -\frac{i}{4} \sum_{i=1}^N \lambda_i (\gamma_i \gamma_{i+N}-\gamma_{i+N} \gamma_i) \\
	&= -\frac{i}{2} \sum_{i=1}^N \lambda_i \gamma_i \gamma_{i+N}.
\end{aligned}
\end{equation}
Each $\gamma_i \gamma_{i+N}$ pair can then transforms to independent fermion mode:
\begin{equation}
\begin{aligned}
	-\frac{i}{2}\gamma_i \gamma_{i+N} 
	&= -\frac{i}{2}(d_i + d_i^\dagger)(id_i-id_i^\dagger) \\ 
	&= d_i^\dagger d_i-\frac{1}{2}.
\end{aligned}
\end{equation}



\subsection{Gaussian States}
The Fermionic Gaussian states are those states with Gaussian form density operator:
\begin{equation}
	\hat \rho \propto \exp \left(\frac{i}{2}\sum_{i,j=1}^{2N}M_{ij}\omega_i \omega_j \right),
\end{equation}
where the matrix $M$ is real and anti-symmetric.\footnote{In particular, any thermal state has this form, with $M = \beta H/2$. The ground state of the free fermion system, though being pure state, can be regarded as the Gaussian state in the limit $M = \lim_{\beta \rightarrow \infty} \beta H$.}
If we expand the Gaussian form, the density operator becomes a Majorana polynomial:\footnote{Note that the coefficient $\Gamma$ in each order is not the direct expansion of the matrix $M$, since the direct expansion contains identical Majorana operators. That is, the $n$-th order expansion of the Majorana Gaussian form may contribute to the ($n-2m$)-th order term in the Majorana polynomial.}
\begin{equation}
	\hat{\rho} = \frac{\mathbb{I}}{2^N} + \sum_{n=1}^{N}\frac{i^n}{2^N}\sum_{1\le i_{1}<\cdots<i_{2n} \le 2N}\Gamma_{i_{1}\cdots i_{2n}} \omega_{i_1}\cdots\omega_{i_{2n}},
\end{equation}
where the coefficient $\Gamma_{i_1 \cdots i_{2n}}$ is the $2n$-point correlation function:
\begin{equation}
	\Gamma_{i_1 \cdots i_{2n}} = i^n \langle \omega_{i_1} \cdots \omega_{i_{2n}}\rangle, \quad i_m \ne i_n.
\end{equation}
In particular, the 2-point function 
\begin{equation}
	\Gamma_{ij} = i\langle \omega_i \omega_j\rangle - i\delta_{ij} = \frac{i}{2}\langle [\omega_i, \omega_j]\rangle
\end{equation}
is also called the \textit{covariance matrix}. 
For Gaussian state all $2n$-point correlation is determined by the covariance matrix by the Wick theorem.
\begin{framedrmk}[Two-point Correlation Function]
We are usually more familiar with the ordinary fermionic two-point correlation function $\langle c^\dagger_i c_j\rangle$ or $\langle c_i c_j\rangle$, which is related to the Majorana covariance matrix by:
\begin{equation}
\begin{aligned}
	\langle c_i^\dagger c_j\rangle &= \frac{1}{4}(
		\Gamma^{21}_{ij} - \Gamma^{12}_{ij} + 
		i \Gamma^{11}_{ij} + i \Gamma^{22}_{ij})
		+\frac{1}{2}\mathbb \delta_{ij}, \\
	\langle c_i c_j\rangle &= \frac{1}{4}(
		\Gamma^{21}_{ij} + \Gamma^{12}_{ij} + 
		i \Gamma^{11}_{ij} - i \Gamma^{22}_{ij}), \\
	\langle c_i^\dagger c_j^\dagger\rangle &= \frac{1}{4}(
		-\Gamma^{21}_{ij} - \Gamma^{12}_{ij} + 
		i \Gamma^{11}_{ij} - i \Gamma^{22}_{ij}).
\end{aligned}
\end{equation}
\end{framedrmk}

The relation of the correlation in each order can be neatly captured by the Grassmannian Gaussian form:
\begin{equation}
\begin{aligned}
	\omega(\hat \rho, \theta) 
	&= \frac{1}{2^N} \exp \left(\frac{i}{2} \sum_{i,j=1}^{2N}\Gamma_{ij}\theta_i \theta_j \right) \\
	&=\frac{1}{2^N} + \sum_{n=1}^{N}\frac{i^n}{2^N}\sum_{1\le i_{1}<\cdots<i_{2n} \le 2N}\Gamma_{i_{1}\cdots i_{2n}} \theta_{i_1} \cdots \theta_{i_{2n}}.
\end{aligned}
\end{equation}
When the covariance matrix is obtained, we can use the same routine to canonicalize the skew-symmetric matrix $\Gamma$:
\begin{equation*}
	\Gamma = O \cdot \Sigma(\bm \lambda) \cdot O^T, \quad
	\tilde\theta_n = \sum_i O_{in} \theta_i,
\end{equation*}
and the density matrix in the Grassmann representation is
\begin{equation}
	\omega(\hat \rho, \theta) 
	= \prod_{n=1}^N \left(\frac{1}{2} e^{i \lambda_n \tilde\theta_n \tilde\theta_{n+N}} \right)
	= \prod_{n=1}^N \left(\frac{1+i\lambda_n \tilde\theta_n\tilde\theta_{n+N}}{2}  \right).
\end{equation}
This state correspond to a product state $\rho = \otimes_n \rho_n$ where
\begin{equation}
	\rho_n = \frac{1}{2} \left[\begin{array}{cc}
		1 + \lambda_n & 0 \\
		0 & 1 - \lambda_n
	\end{array} \right].
\end{equation}
The entanglement entropy is then
\begin{equation}
	S=\sum_n S_n = -\sum_n \left[
	\left(\frac{1+\lambda_n}{2}\right)\ln\left(\frac{1+\lambda_n}{2}\right)
	+ \left(\frac{1-\lambda_n}{2}\right)\ln\left(\frac{1-\lambda_n}{2}\right)\right].
\end{equation}



\subsection{Lindblad Master Equation}
For Lindblad equation
\begin{equation}
	\frac{d}{dt} \hat\rho = -i[\hat H, \hat \rho] + \sum_{\mu=1}^{m} \hat L_\mu \hat\rho \hat L_\mu^\dagger -\frac{1}{2} \sum_{\mu=1}^{m} \{\hat L_\mu^\dagger \hat L_\mu, \hat \rho\}
\end{equation}
When the \textit{jump operator} $\hat L_\mu$ contains only the linear Majorana operator, the Lindblad equation preserve Gaussianity. 
For \textit{jump operator} contains up to quadratic Majorana terms, the evolution will break the Gaussian form, however, the $2n$-point correlation is still solvable for free fermion system.

\subsubsection*{Dynamics of Covariance Matrix}

We assume that the jump operator has up to quadratic Majorana terms. 
In particular, we denote the linear terms and the Hermitian quadratic terms as
\begin{equation}
	\hat L_r = \sum_{j=1}^{2N} L^r_{j} \omega_j, \quad
	\hat L_s = \sum_{j,k=1}^{2N} M^s_{jk} \omega_j \omega_k.
\end{equation}
Now consider the dynamics of the expectation value $\langle\hat O\rangle$:
\begin{equation}
\begin{aligned}
	\frac{d}{dt}\langle \hat O\rangle
	&= -i \mathrm{Tr} [\hat O (\hat H \hat\rho-\hat\rho \hat H)] 
		+ \sum_\mu \mathrm{Tr}[\hat O \hat L_\mu \hat\rho \hat L_\mu^\dagger]
		- \frac{1}{2}\sum_\mu \mathrm{Tr}[\hat O \hat L_\mu^\dagger \hat L_\mu \hat\rho
		+ \hat O \hat\rho \hat L_\mu^\dagger \hat L_\mu] \\
	&= \left\langle
		i[\hat H, \hat O] + \sum_\mu \hat L_\mu^\dagger \hat O\hat L_\mu - \frac{1}{2} \sum_\mu\{\hat L_\mu^\dagger \hat L_\mu, \hat O \}
		\right\rangle.
\end{aligned}
\end{equation}
We can express the dynamics of operator as in the Heisenberg picture:
\begin{equation}
	\frac{d\hat O}{dt} = i[\hat H, \hat O] + \mathcal D_r[\hat O] + \mathcal D_s[\hat O],
\end{equation}
where
\begin{equation}
\begin{aligned}
	\mathcal D_r[\hat O] 
	&= \sum_r \hat L_r^\dagger \hat O\hat L_r - \frac{1}{2} \sum_r\{\hat L_r^\dagger \hat L_r, \hat O \}
	= \frac{1}{2}\sum_r [\hat L_r^\dagger L_r, \hat O] - \sum_r \hat L_r^\dagger[\hat L_r,\hat O],  \\
	\mathcal D_s[\hat O] 
	&= \sum_s \hat L_s \hat O\hat L_s - \frac{1}{2} \sum_r\{\hat L_s^2, \hat O \}
	= -\frac{1}{2} \sum_s [\hat L_s,[\hat L_s,\hat O]].
\end{aligned}
\end{equation}
The equation of motion can be further simplified to:
\begin{equation}
	\frac{d\hat O}{dt} 
	= i[\hat H_{\mathrm{eff}}, \hat O] - \sum_r \hat L_r^\dagger[\hat L_r,\hat O] -\frac{1}{2} \sum_s [\hat L_s,[\hat L_s,\hat O]],
\end{equation}
where the effective Hamiltonian is
\begin{equation}
	\hat H_{\mathrm{eff}} = \sum_{ij} \left(-\frac{i}{4}H_{ij}-\frac{1}{2} B^I_{ij}\right)\omega_i\omega_j,
\end{equation}
where we have defined $B_{ij} = \sum_r L^r_i L^{r*}_j$.
Using the commutation relation $\{\omega_i, \omega_j\} = 2\delta_{ij}$, we have the following relation
\begin{equation}
\begin{aligned}[]
	[\omega_k,\omega_i \omega_j] &= 2(\delta_{ki}\omega_j-\delta_{kj}\omega_i), \\
	[\omega_k \omega_l, \omega_i \omega_j] 
	&= 2(\delta_{ki}\omega_j \omega_l-\delta_{kj} \omega_i \omega_l + \delta_{li}\omega_k \omega_j - \delta_{lj}\omega_k\omega_i),
\end{aligned}
\end{equation}
and let $\hat O_{ij} = \omega_i\omega_j - \delta_{ij}\mathbb I$.
The first term of EOM is:
\begin{equation*}
\begin{aligned}[]
	i\langle[\hat H_{\mathrm{eff}}, \hat O_{ij}]\rangle_t
	&= \sum_{kl}\left(\frac{1}{4}H-\frac{i}{2}B^I \right)_{kl} \langle[\omega_k \omega_l, \omega_i \omega_j]\rangle_t \\
	&= \sum_{kl} \left(\frac{1}{2}H-i B^I\right)_{kl} \langle 
		\delta_{ki}\omega_j \omega_l-\delta_{kj} \omega_i \omega_l + 
		\delta_{li}\omega_k \omega_j - \delta_{lj}\omega_k\omega_i
	\rangle_t \\
	&= \left[
		(H-2iB^I)^T \cdot \langle\hat O\rangle_t + 
		\langle\hat O\rangle_t \cdot (H-2iB^I)
	\right]_{ij}.
\end{aligned}
\end{equation*}
The second term is
\begin{equation*}
\begin{aligned}
	-\sum_r \langle L_r^\dagger[L_r, \hat O_{ij}] \rangle_t
	&= -\sum_{kl} B_{kl}^* \langle \omega_k [\omega_l, \omega_i \omega_j] \rangle_t \\
	&= -2\sum_{kl} B_{kl}^* \langle 
		\delta_{li} \omega_k \omega_j - 
		\delta_{lj} \omega_k \omega_i
	\rangle_t \\
	&= -\left[2B\cdot \langle\hat O\rangle_t + 2\langle\hat O\rangle_t\cdot B^* + 4i B^I \right]_{ij}.
\end{aligned}
\end{equation*}
And the third term is
\begin{equation*}
\begin{aligned}
	-\frac{1}{2}\sum_s \langle[\hat L_s,[\hat L_s, \hat O_{ij}]]\rangle_t
	&= -\frac{1}{2} \sum_s \sum_{kl} M^s_{kl}\langle[\hat L_s,[\omega_k \omega_l, \omega_i \omega_j]]\rangle_t \\
	&= 2\sum_s \sum_{k} \left\langle M^s_{ik}[\hat L_s,\omega_k \omega_j]-[\hat L_s,\omega_i \omega_k]M^s_{kj} \right\rangle_t \\
	&= 8\sum_{s,kl} \left\langle M^s_{ik}[-M^s_{kl}\omega_l\omega_j+\omega_k\omega_l M^s_{lj}]+[M^s_{il}\omega_l\omega_k-\omega_i\omega_l M^s_{lk}]M^s_{kj} \right\rangle_t \\
	&= 8\sum_s \left[2 M^s \cdot \langle\hat O\rangle_t\cdot M^s-(M^s)^2 \cdot \langle\hat O\rangle_t - \langle\hat O\rangle_t\cdot(M^s)^2 \right]_{ij}.
\end{aligned}
\end{equation*}
In together, we get the EOM of variance matrix $\Gamma_{ij}(t)=i\langle\hat O_{ij}\rangle_t$, the result is:
\begin{equation}
	\partial_t \Gamma = X^T\cdot\Gamma + \Gamma \cdot X + \sum_s (Z^s)^T \cdot \Gamma\cdot Z^s + Y,
\end{equation}
where
\begin{equation}
	X = H - 2B^R + 8 \sum_s (\mathrm{Im} M^s)^2, \quad
	Y = 4B^I, \quad 
	Z = 4 \mathrm{Im} M^s.
\end{equation}






