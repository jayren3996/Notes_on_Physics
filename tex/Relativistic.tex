\chapter{Relativistic Quantum Field Theory}

\section{Lorentz Invariance}

\subsection{The Lorentz Algebra}
The metric is chosen to be 
\begin{equation}
	g_{\mu\nu}=g^{\mu\nu}=\mathrm{diag}(+1,-1,-1,-1).
\end{equation}
The Lorentz transformation ${\Lambda^{\mu}}_{\nu}$ satisfies
\begin{equation}
{\Lambda^{\mu}}_{\alpha}{\Lambda^{\nu}}_{\beta} g^{\alpha\beta} = g^{\mu\nu}.
\end{equation}
From this we have
\begin{equation*}
	g^{\gamma\alpha}{\Lambda^{\mu}}_{\alpha}{\Lambda^{\nu}}_{\beta} g_{\mu\nu} 
	= g^{\gamma\alpha}g_{\alpha\beta} 
	\quad \Longrightarrow \quad
	{\Lambda_{\nu}}^{\gamma}{\Lambda^{\nu}}_{\beta} 
	= {\delta^{\gamma}}_{\beta},
\end{equation*}
The inverse Lorentz transformation satisfies:
\begin{equation}
	{(\Lambda^{-1})^{\mu}}_{\nu} = {\Lambda_{\nu}}^{\mu}.
\end{equation}
The infinitesimal transformation is denoted as
\begin{equation*}
\begin{aligned}
	{\Lambda^{\mu}}_{\nu} &= {\delta^{\mu}}_{\nu}+\delta{\omega^{\mu}}_{\nu} \\
	{(\Lambda^{-1})^\mu}_\nu &= {\delta^{\mu}}_{\nu}-\delta{\omega^\mu}_\nu
\end{aligned}
	\quad \Longrightarrow \quad
	g_{\alpha\nu}\delta{\omega^{\nu}}_{\beta}+\delta{\omega^{\mu}}_{\alpha}g_{\mu\beta}
	=\delta\omega_{\alpha\beta} + \delta\omega_{\beta\alpha} = 0.
\end{equation*}
A representation of Lorentz group $U(\Lambda)$ can be parametrized as:
\begin{equation}
	U(\Lambda) = \exp\left(\frac{i}{2}\omega_{\mu\nu}M^{\mu\nu}\right).
\end{equation}
Another useful parametrization is
\begin{equation}
	\theta_i \equiv \frac{1}{2}\varepsilon_{ijk}\omega_{jk}, \ 
	\beta_i \equiv \omega_{0i}.
\end{equation}
A new set of generators are:
\begin{equation}
	J_i \equiv \frac{1}{2}\varepsilon_{ijk}M^{jk},\ 
	K_i \equiv M^{i0},
\end{equation}
where $J_i$'s are the generators of the spatial rotations, and $K_i$'s are the generators of Lorentz boosts.

In the fundamental representation, the generators are represented by
\begin{equation*}
\begin{aligned}
	J_1 &= \left[\begin{array}{cccc} 0 & & & \\ & 0 & & \\ & & 0 & -i \\ & & i & 0 \end{array}\right], & 
	J_2 &= \left[\begin{array}{cccc} 0 & & & \\ & 0 & & i \\ & & 0 & \\ & -i & & 0 \end{array}\right], &
	J_3 &= \left[\begin{array}{cccc} 0 & & & \\ & 0 & -i & \\ & i & 0 & \\ & & & 0 \end{array}\right], \\
	K_1 &= \left[\begin{array}{cccc} 0 & -i & & \\ -i & 0 & & \\ & & 0 & \\ & & & 0 \end{array}\right], & 
	K_2 &= \left[\begin{array}{cccc} 0 & & -i & \\ & 0 & & \\ -i & & 0 & \\ & & & 0 \end{array}\right], &
	K_3 &= \left[\begin{array}{cccc} 0 & & & -i \\ & 0 & & \\ & & 0 & \\ -i & & & 0 \end{array}\right].
\end{aligned}
\end{equation*}
The Lie algebra of the Lorentz algebra can be explicitly done using the fundamental representation. 
The result is
\begin{equation}
\begin{aligned}
	\left[J_i, J_j\right] &= i \varepsilon_{ijk} J_k, \\
	\left[J_i, K_j\right] &= i \varepsilon_{ijk} K_k, \\
	\left[K_i, K_j\right] &= -i\varepsilon_{ijk} J_k.
\end{aligned}
\end{equation}
By defining a new set of generators:
\begin{equation}
	N_i^{L} \equiv \frac{J_i - i K_i}{2},\ 
	N_i^{R} \equiv \frac{J_i + i K_i}{2}.
\end{equation}
They satisfies two independent $\mathfrak{su}(2)$ algebra:
\begin{equation}
\begin{aligned}
	\left[N_i^L, N_j^L \right] &= i\varepsilon_{ijk}N_k^L, \\
	\left[N_i^R, N_j^R \right] &= i\varepsilon_{ijk}N_k^R, \\
	\left[N_i^L, N_j^R \right] &= 0.
\end{aligned}
\end{equation}
That is, the Lorentz algebra is isomorphic to two $\mathfrak{su}(2)$ algebra,
\begin{equation}
	\mathfrak{so}(3,1) \approx \mathfrak{su}_L(2)\oplus\mathfrak{su}_R(2).
	\label{eq:Lorentz-alg-decomp}
\end{equation}
From Eq.~(\ref{eq:Lorentz-alg-decomp}), we know that the representation of the Lorentz algebra can be labelled by $j_L$ and $j_R$.
Note that the fundamental representation correspond to
\begin{equation*}
	\left(j_L=\frac{1}{2},j_R=\frac{1}{2}\right).
\end{equation*}
The specific form of the group is
\begin{equation}
	\Lambda(\vec\theta,\vec\beta)
	=\exp\left[i(\vec\theta+i\vec\beta)\cdot \vec N^L + i(\vec\theta-i\vec\beta)\cdot \vec N^R\right].
\end{equation}
The spinor representations are those with $j_L=1/2$ or $j_R=1/2$. 
Specifically, we define the left-hand spinor $\psi_L$ and right-hand spinor $\psi_R$ that transform as:
\begin{equation}
\begin{aligned}
	\Lambda_L(\vec\theta,\vec\beta)\psi_L 
	&= \exp\left(\frac{i}{2}\vec\theta\cdot\vec\sigma-\frac{1}{2}\vec\beta\cdot\vec\sigma \right) \psi_L, \\
	\Lambda_R(\vec\theta,\vec\beta)\psi_R 
	&= \exp\left(\frac{i}{2}\vec\theta\cdot\vec\sigma+\frac{1}{2}\vec\beta\cdot\vec\sigma \right) \psi_R.
\end{aligned}
\end{equation}
Using the fact $\sigma^2 \cdot \vec\sigma^* \cdot\sigma^2 = -\vec\sigma$, the left-hand and the right-hand representations are related by:
\begin{equation}
\begin{aligned}
	\sigma^2 \Lambda_L^* \sigma^2 &= \Lambda_R, & \sigma^2 \Lambda_L^T \sigma^2 &= \Lambda_L^{-1}, \\
	\sigma^2 \Lambda_R^* \sigma^2 &= \Lambda_L, & \sigma^2 \Lambda_R^T \sigma^2 &= \Lambda_R^{-1}.
\end{aligned}
\end{equation}
For this reason, the left-hand and right-hand spinor can be interchanged by
\begin{equation}
\begin{aligned}
	\sigma^2 \psi_L^* &\sim \chi_R, & \psi_L^\dagger \sigma^2 &\sim \chi^\dagger_R \\
	\sigma^2 \psi_R^* &\sim \chi_L, & \psi^\dagger_R \sigma^2 &\sim \chi^\dagger_L.
	\label{eq:left-right-spinor-rel}
\end{aligned}
\end{equation}



\subsection{The Invariant Symbols}
The invariant symbols can be thought as the Clebsch-Gordan coefficients that help to form singlets.
The first singlet comes from the decomposition
\begin{equation*}
	\frac{1}{2}\otimes \frac{1}{2} \approx 0 \oplus 1.
\end{equation*}
Correspondingly, we can check that for each-hand-side spinor, the quadratic forms
\begin{equation}
	\psi_L^T\sigma^2\chi_L \quad \text{or} \quad 
	\psi_R^T\sigma^2\chi_R
	\label{eq:inner-product-inv-symbol}
\end{equation}
are singlets.
We can define the first invariant symbol as\footnote{We use the dotted symbol to denote the right-hand spinor indices.}
\begin{equation}
	\varepsilon^{ab} = \varepsilon^{\dot a \dot b} = i(\sigma^2)_{ab}, \quad
	\varepsilon_{ab} = \varepsilon_{\dot a \dot b} = -i(\sigma^2)_{ab}.
\end{equation}
The symbol $\varepsilon^{ab}$ or $\varepsilon_{ab}$ also serve as the index raising/lowering symbol, i.e.,
\begin{equation}
	\varepsilon^{ab}\psi_b = \psi^a,\ 
	\varepsilon_{ab}\psi^b = \psi_a.
\end{equation}
The singlet (\ref{eq:inner-product-inv-symbol}) is then defined as the inner product of two spinors:
\begin{equation}
	\psi\cdot\chi 
	\equiv \varepsilon_{ab}\psi^a\chi^b
	= \psi^a\chi_{a}
	= -\varepsilon_{ba}\psi^a\chi^b
	= -\psi_b\chi^b.
\end{equation}
In addition, because of (\ref{eq:left-right-spinor-rel}), the expressions
\begin{equation*}
	\psi_L^\dagger \chi_R \quad \text{and} \quad \psi_R^\dagger \chi_L
\end{equation*}
are also singlets.

Besides, we know there should be another invariant symbol from the decomposition
\begin{equation*}
	\left(\frac{1}{2}, 0\right) \otimes \left(0,\frac{1}{2}\right)
	\approx \left(0, 0\right) \oplus \cdots.
\end{equation*}
For this reason, we are searching for the symbol $M$ that the expression
\begin{equation*}
	M^\mu_{a\dot b} \psi^a_L \chi^{\dot b}_R
\end{equation*}
transforms as the Lorentz vector.
The matrix $M^\mu$ should transform as
\begin{equation*}
	M^\mu \longrightarrow \Lambda_L^T \cdot M^\mu \cdot \Lambda_R = {\Lambda^\mu}_\nu M^\nu.
\end{equation*}
Use the fact that $\sigma^2 \cdot \Lambda_L^T \cdot \sigma^2 = \Lambda_L^{-1}$, the above equation transforms to
\begin{equation*}
	\left(\sigma^2 M^\mu\right) \longrightarrow \Lambda_L^{-1} \cdot \left(\sigma^2 M^\mu\right)\cdot \Lambda_R.
\end{equation*}
We then show the matrices $\sigma^\mu = (\sigma^0,\vec\sigma)$ satisfies the requirement.
Firstly, for the spatial rotation,
\begin{equation*}
	\Lambda_L(\vec\theta,\vec 0) = \Lambda_R(\theta,\vec 0) = \exp\left(i\vec\theta\cdot \frac{\vec\sigma}{2}\right)
\end{equation*}
The Pauli matrix transform as
\begin{equation*}
	\left(1-i\delta\vec\theta\cdot\frac{\vec\sigma}{2}\right)\sigma^j\left(1+i\delta\vec\theta\cdot \frac{\vec\sigma}{2}\right)
	= \sigma^j + i\delta\theta_i \left(-i \varepsilon_{ijk}\sigma^k \right)
\end{equation*}
Secondly, for the boosts,
\begin{equation*}
	\Lambda_L(\vec 0, \vec\beta) = \exp\left(-\vec\beta\cdot \frac{\vec\sigma}{2}\right),\ 
	\Lambda_R(\vec 0, \vec\beta) = \exp\left(+\vec\beta\cdot \frac{\vec\sigma}{2}\right)
\end{equation*}
The Pauli matrix transform as
\begin{equation*}
	\left(1+\delta\vec\beta\cdot\frac{\vec\sigma}{2}\right)\sigma^\mu \left(1+\delta\vec\beta\cdot \frac{\vec\sigma}{2}\right) = \begin{cases}
		 \sigma^0 + i\delta\beta_i \cdot (-i\sigma^i), & \mu = 0 \\
		 \sigma^j + i\delta\beta_j (-i\sigma^0), & \mu = j
	\end{cases}.
\end{equation*}
We thus have shown indeed that
\begin{equation}
	\psi_L^T \sigma^2 \sigma^\mu \chi_R
\end{equation}
is a Lorentz vector.
Further more, from (\ref{eq:left-right-spinor-rel}), we know that
\begin{equation}
	\eta_R^\dagger \sigma^\mu \chi_R
\end{equation}
is also a Lorentz vector.
Similarly, consider the Lorentz vector 
\begin{equation*}
	N^\mu_{\dot a b} \psi^{\dot a}_R \chi^{b}_R,
\end{equation*}
which together with $\sigma^2$ should transforms as
\begin{equation*}
	\left(\sigma^2 N^\mu\right) \longrightarrow 
	\Lambda_R^{-1} \cdot \left(\sigma^2 N^\mu\right)\cdot \Lambda_L.
\end{equation*}
We can check that $\bar\sigma^\mu = (\sigma^0,-\vec\sigma)$ satisfies the requirement, and thus 
\begin{equation}
	\eta_L^\dagger \bar\sigma^\mu \chi_L
\end{equation}
is also a Lorentz vector.


\section{Klein-Gordon Field}

In relativistic quantum field theory, the Lagrangian should be a singlet under Lorentz transformation.
Different free fields correspond to different representation of the Lorentz algebra.
The symmetry under Lorentz transformation also restrict the possible terms that can appear in the Lagrangian.


The simplest case is when $j_L=j_R = 0$, corresponding to the scalar field, which we denote as $\phi(x)$.
Since the field it self is singlet, any polynomial of the field in principle can appear in the theory.
When considering the free theory, we restrict our attention to the quadratic terms.
We require the field theory to have a dynamical term, which contains derivative the the field.
The derivative operator $\partial^\mu$ transforms as the fundamental representation.
To be Lorentz invariant, the allowed free theory can only be
\begin{equation}
	\mathcal L_{\mathrm{K-G}} = \frac{1}{2}\partial^\mu \phi \partial_\mu \phi -\frac{m^2}{2}\phi^2 
	\simeq -\frac{1}{2}\phi (\partial^2+m^2) \phi.
\end{equation}

For general discussion, we consider the field theory on $d$-dimensional space-time.
Note that the space-time Fourier transformation is defined as
\begin{equation}
\begin{aligned}
	\tilde{\phi}(k) &= \int d^{d}x e^{ik\cdot x} \phi(x), \\ 
	\phi(x) &= \int \frac{d^{d}k}{(2\pi)^{d}} e^{-ik\cdot x}\tilde{\phi}(k),
\end{aligned}
\end{equation}
where the inner product of two 4-momentum and 4-coordinate is
\begin{equation}
	k\cdot x=\omega t-\vec k\cdot \vec x.
\end{equation}


\subsection{Canonical Quantization}
The classical equation of motion 
\begin{equation*}
	\partial_\mu \left[\frac{\partial}{\partial(\partial_\mu \phi)}\right] - \frac{\partial \mathcal L}{\partial \phi} = 0
\end{equation*}
for Klein-Gordon field is 
\begin{equation}
	(\partial_t^2-\nabla^2+m^2)\phi(\vec x,t) = 0. 
	\label{eq:rkg-eom}
\end{equation}
The solution to Eq.~(\ref{eq:rkg-eom}) is proportional to the plane wave:
\begin{equation*}
	\phi(\vec x, t) \propto e^{-i\omega_{\bm{k}}t+i\vec{p}\cdot\vec{x}} + e^{i\omega_{\vec{k}}t-i\bm{p}\cdot\vec{x}},
\end{equation*}
where the energy is $\omega_{\bm{k}}=\bm{k}^2+m^2$ and $\vec k$ is the momentum as the conserved quantity.
The general solution to the EOM is
\begin{equation}
	\phi(\vec x,t) \propto \int \frac{d^{3} k}{(2\pi)^{3}} \left(
		a_{k}e^{-i\omega_{\bm{k}}t+i\vec{k}\cdot\vec{x}} + 
		a^*_{k}e^{i\omega_{\bm{k}}t-i\vec{k}\cdot\vec{x}} 
	\right).
\end{equation}

\subsubsection{Single Particle State}
The canonical quantization promote the coefficient $a_{k}/a_{k}^*$ to the particle annihilation/creation operator $a_{k}/a_{k}^\dagger$, with the commutation relation
\begin{equation}
	[a_{k}, a_{p}^\dagger] = (2\pi)^{3} \delta^{3}(\vec{k}-\vec{p}).
\end{equation}

The single-particle state with momentum $\vec k$ is created by $a_{k}^{\dagger}$ operators acting on the vacuum:
\begin{equation}
	|\vec{k}\rangle \equiv \sqrt{2\omega_{\bm k}} a_{k}^{\dagger}|0\rangle,
	\label{eq:rel-single-particle}
\end{equation}
where $|\vec{k}\rangle$ is a state with a single particle of momentum $\vec{k}$.
The factor of $\sqrt{2 \omega_{\bm k}}$ in Eq.~(\ref{eq:rel-single-particle}) is just a convention, but it will make some calculations easier. 
To compute the normalization of one-particle states, we start with
\begin{equation}
	\langle 0|0\rangle=1,
\end{equation}
which leads to
\begin{equation}
	\langle\vec{p}|\vec{k}\rangle 
	= 2\sqrt{\omega_{\bm p} \omega_{\bm k}}\left\langle 0\left|a_{p} a_{k}^{\dagger}\right| 0\right\rangle
	= 2 \omega_{\bm p}(2\pi)^{3} \delta^{3}(\vec{p}-\vec{k}).
\end{equation}
The identity operator for one-particle states is
\begin{equation}
	1=\int \frac{d^{3} p}{(2\pi)^{3}} \frac{1}{2\omega_{\bm p}}|\vec{p}\rangle\langle\vec{p}|, \label{eq:rel-identity}
\end{equation}
which we can check with
\begin{equation*}
	|\vec{k}\rangle
	=\int \frac{d^{3} p}{(2\pi)^{3}} \frac{1}{2\omega_{\bm p}}|\vec{p}\rangle\langle\vec{p}|\vec{k}\rangle
	=\int \frac{d^{3} p}{(2\pi)^{3}} \frac{1}{2\omega_{\bm p}} 2\omega_{\bm p}(2\pi)^3 \delta^3(\vec{p}-\vec{k})|\vec{p}\rangle
	=|\vec{k}\rangle.
\end{equation*}
The identity operator Eq.~(\ref{eq:rel-identity}) is Lorentz invariant since it can be expressed as
\begin{equation}
	1 = \int \frac{d^{3} p d\omega}{(2\pi)^{4}} 2\pi\delta(\omega^2-{\bm{p}}^2-m^2) |\vec p\rangle\langle \vec p|.
\end{equation}

\subsubsection{Field Expansion}
We fix the normalization by requiring 
\begin{equation}
	\langle \vec k|\phi(\vec x,0)|0\rangle = e^{-i \vec k\cdot \vec x},
\end{equation}
and the quantized field operator is
\begin{equation}
	\phi(\vec{x}, t)
	=\int \frac{d^{3} k}{(2\pi)^{3}} \frac{1}{\sqrt{2\omega_{\bm k}}}\left(a_k 
	e^{-i k \cdot x}+a_k^{\dagger} e^{i k \cdot x}\right).
\end{equation}

Consider the two-point correlation (propagator):
\begin{equation*}
\begin{aligned}
	i\Delta(x_1-x_2) &= \langle 0|T \phi(x_1) \phi(x_2) |0\rangle \\
	&= \theta(t_1-t_2) \langle 0|\phi(x_1) \phi(x_2) |0\rangle 
	+ \theta(t_2-t_1) \langle 0|\phi(x_2) \phi(x_1) |0\rangle.
\end{aligned}
\end{equation*}
Note that
\begin{equation}
	\langle 0|\phi(x_1) \phi(x_2) |0\rangle
	= \int\frac{d^{3} k}{(2\pi)^{3}}\frac{1}{2\omega_k} e^{i\vec k\cdot (\vec x_1-\vec x_2)-i\omega_{\vec k}\tau},
\end{equation}
where $\tau =t_1-t_2$.
The propagator can be written as
\begin{equation}
\begin{aligned}
	i\Delta(x_1-x_2) 
	&= \int\frac{d^{3} k}{(2\pi)^{3}}\frac{1}{2\omega_k} e^{i\vec k\cdot (\vec x_1-\vec x_2)}\left[e^{-i\omega_{\vec k}\tau}\theta(\tau)+e^{i\omega_{\vec k}\tau}\theta(-\tau)\right] \\
	&= \int\frac{d^{3} k}{(2\pi)^{3}} e^{i\vec k\cdot (\vec x_1-\vec x_2)}\int \frac{d\omega}{2\pi i}\frac{-e^{i\omega\tau}}{\omega^2-\omega_k^2+i\epsilon} \\
	&= \int\frac{d^{4} k}{(2\pi)^{4}} e^{-i k\cdot (x_1-x_2)}\frac{i}{k^2-m^2+i\epsilon}.
\end{aligned}
\end{equation}


\subsection{Path-integral Formalism}
Consider the action for free field with source
\begin{equation}
	S_0[\phi,J]
	= \int d^dx\left[\mathcal{L}_0 + J(x)\cdot\phi(x) \right].
\end{equation}
In the path integral formalism, we consider the partition function 
\begin{equation}
	Z[J] = \int D[\phi] \exp(iS[\phi,J])
	\equiv Z[0] \exp(iW[J]).
\end{equation}
where we have introduced a new quantity
\begin{equation}
\begin{aligned}
	W[J] = -\frac{1}{2}\int d^dx_1 d^dx_2 J(x_1)\Delta(x_1-x_2)J(x_2).
\end{aligned}
\end{equation}
For free field, the free propagator $\Delta_0(x_1-x_2)$ is:
\begin{equation}
	i\Delta_0(x_1-x_2) = \int \frac{d^dk}{(2\pi)^d} \frac{e^{-ik\cdot x}}{k^2-m^2+i\epsilon},
	\label{eq:free-rkg-pgt}
\end{equation}
where the extra $i\epsilon$ term is use to bring the singularities infinitesimally below the real axis. 
This infinitesimal value can be absorbed into the mass term, by regarding the mass term $m^2$ as $m^2-i\epsilon$.

Note that $\Delta_0(x_1-x_2)$ is related to the correlation function:
\begin{equation}
	\Delta_0(x_1-x_2) = \frac{1}{i} \langle 0| T\phi(x_1)\phi(x_2)|0\rangle
	= \frac{\delta}{i\delta J(x_1)}\frac{\delta}{i\delta J(x_2)} W_0[J].
\end{equation}




\subsubsection{Gaussian Integral}
Now we evaluate the propagator in the path-integral formalism.
In momentum space, the free action (with source) is 
\begin{equation*}
	\frac{1}{V}\sum_k \left[\tilde\phi^*(k)( k^2-m^2)\tilde\phi(k)+\tilde J^*(k)\cdot\tilde\phi(k)+\tilde\phi^*(k)\cdot\tilde J(k)\right].
\end{equation*}
For real field, $\tilde\phi^*(k) = \tilde\phi(-k)$.
For our convenience, we have expressed the momentum integral as summation.
Actually, consider the $d$-dimensional box of size $L^d$, the momentum along each axis is multiple of $2\pi/L$, so when $L\rightarrow \infty$, the summation approaches in integral,
\begin{equation*}
	\frac{1}{V}\sum_k \rightarrow \int \frac{d^d k}{(2\pi)^d}.
\end{equation*}
Let us omit the $1/V$ factor, the summation can be formally expressed as
\begin{equation}
	-\frac{1}{2}\mathbf{v}^T \cdot \mathbf A\cdot \mathbf{v} + \mathbf{b}^T \cdot \mathbf{v}
\end{equation}
where
\begin{equation*}
	\mathbf v = \bigoplus_{|\mathbf k|} \left[
	\begin{array}{c}
		\tilde{\phi}(k) \\ 
		\tilde{\phi}^*(k) 
	\end{array}\right],\ 
	\mathbf A = \bigoplus_{|\mathbf k|} \left[
	\begin{array}{cc} 
		0 & k^2-m^2 \\ 
		k^2-m^2 & 0 
	\end{array}\right],\ 
	\mathbf b = \bigoplus_{|\mathbf k|} \left[
	\begin{array}{c}
		\tilde{J}^*(k) \\ 
		\tilde{J}(k) 
	\end{array}\right].
\end{equation*}
We can use a unitary transformation to tranform
\begin{equation*}
	\mathbf U = \frac{1}{\sqrt 2} \left[\begin{array}{cc}
		1 & 1 \\
		-i & i
	\end{array}\right], \quad
	\mathbf U \cdot \left[
	\begin{array}{c}
		\tilde{\phi}(k) \\ 
		\tilde{\phi}^*(k) 
	\end{array}\right] 
	= \frac{1}{\sqrt 2}\left[
	\begin{array}{c}
		\tilde\phi(k)+\tilde\phi^*(k) \\ 
		-i\tilde\phi(k)+i\tilde\phi^*(k)
	\end{array}\right]
	\equiv \left[
	\begin{array}{c}
		\tilde\phi_1(k) \\ 
		\tilde\phi_2(k) 
	\end{array}\right]
\end{equation*}
The path integral then becomes a real field integral.
Recall the real Gaussian integral formula:
\begin{equation}
	\int d\mathbf v \exp\left(-\frac{1}{2}\mathbf{v}^T \cdot \mathbf A\cdot \mathbf{v} + \mathbf{b}^T \cdot \bm{v}\right) 
	= \sqrt{\frac{(2\pi)^N}{\det{\mathbf A}}}\exp\left(\frac{1}{2}\mathbf{b}^T \cdot \mathbf{A}^{-1} \cdot \mathbf{b}\right),
	\label{eq:real-gaussian-integral}
\end{equation}
For the field integral, we absorbed the $(2\pi)^{N/2}$ term into the measure, and express the path integral for the Gaussian field as:
\begin{equation}
	W_0[J] 
	= -\frac{i}{2}\int \frac{d^d k}{(2\pi)^d} \mathbf b^T_k \cdot \mathbf A^{-1}_k \cdot \mathbf b_k.
\end{equation}
This gives the propagator in the momentum space:
\begin{equation*}
	\tilde{\Delta}_0(k) = \frac{1}{k^2-m^2}.
\end{equation*}


\subsubsection{From Field to Force}
Consider two separate particle described by the delta function $J_a(x) = \delta^{(3)}(\bm x - \bm x_a)$, together the source is
\begin{equation}
	J(x) = J_1(x) + J_2(x).
\end{equation}
Adding the source,
\begin{equation*}
	W[J] = -\frac{1}{2}\int d^4x_1 d^4 x_2 J(x_1) \Delta(x_1-x_2) J(x_2)
\end{equation*}
Omit the self energy terms $J_1^2(x), J_2^2(x)$, $W[J]$ is
\begin{equation}
\begin{aligned}
	W[J] &= -\int d^4 y_1 d^4 y_2\ e^{-ik^0(y_1^0-y_2^0)}\int \frac{d^4 k}{(2\pi)^4} J_1(y_1)\frac{e^{i\bm k\cdot (\bm y_1-\bm y_2)}}{k^2-m^2} J_2(y_2) \\
	&= -\int  dt \int d (y_1^0 - y_2^0) \ e^{-ik^0(y_1^0-y_2^0)}\int \frac{d^4 k}{(2\pi)^4} \frac{e^{i\bm k\cdot (\bm y_1-\bm y_2)}}{k^2-m^2} \\
	&= \left(\int dt \right)\int \frac{d^3 k}{(2\pi)^3} \frac{e^{i\bm k\cdot (\bm y_1-\bm y_2)}}{\bm k^2 + m^2}
\end{aligned}
\end{equation}
Recall that the partition function is actually infinite:
\begin{equation}
	Z \sim \langle 0| e^{-iHT} |0\rangle \quad \Longrightarrow \quad
	W = -iET,
\end{equation}
where $E$ is the energy.
Writing $\bm r \equiv \bm y_1 - \bm y_2$, and $u \equiv \cos\theta$ with $\theta$ the angle between $\bm k$ and $\bm r$, the volume form is $dk \cdot kd\theta \cdot  2\pi k \sin \theta = 2\pi k^2 dk du$, and the integral is
\begin{equation}
\begin{aligned}
	E &= -\int \frac{d^3 k}{(2\pi)^3} \frac{e^{i k r u}}{k^2 + m^2} \\
	&= - \frac{1}{(2\pi)^2} \int_0^\infty k^2 dk \int_{-1}^1 du \frac{e^{ikru}}{k^2 +m^2} \\
	&= -\frac{1}{2\pi^2 r} \int_0^\infty k  \frac{\sin kr}{k^2 +m^2} dk.
\end{aligned}
\end{equation}
Since the integral is even, we can extend the integral to
\begin{equation}
\begin{aligned}
	E &= -\frac{1}{4\pi^2 r} \int_{-\infty}^\infty k  \frac{\sin kr}{k^2 +m^2} dk \\
	&= \frac{i}{4\pi^2 r} \int_{-\infty}^\infty \frac{k e^{ikr}}{k^2 +m^2} dk
\end{aligned}
\end{equation}
The residue theorem gives
\begin{equation*}
	\int_{-\infty}^\infty \frac{k e^{ikr}}{k^2 +m^2} dk = \pi ie^{-mr}
\end{equation*}
So we get the potential of two particles:
\begin{equation}
	E(r) = -\frac{e^{-mr}}{4\pi r},
\end{equation}
and the attractive force is
\begin{equation}
	F(r) = -\frac{dE}{dr} = -(1+mr)\frac{e^{-mr}}{4\pi r^2}.
\end{equation}





\section{Vector Field}

If we can choose $j_L=j_R=1/2$, the field is transformed as Lorentz vector.
We denote the field as $A^\mu(x)$.
Some possible quadratic forms for the vector field that forms singlets are
\begin{equation*}
	A^\mu A_\mu,\ (\partial_\mu A^\mu)^2,\ A^\nu \partial^2 A_\nu,\ 
	\varepsilon_{\mu\nu\rho\lambda} \partial^\mu A^\nu \partial^\rho A^\lambda.
\end{equation*}
For the field theory describe the electromagnetic field, we require the theory to further have gauge symmetry, i.e., invariant under
\begin{equation}
	A^\mu(x) \rightarrow A^\mu(x) + \partial^\mu \alpha(x).
\end{equation}
The gauge invariant forbids the first term, and forces the second and third term to combine as
\begin{equation*}
	(\partial_\mu A^\mu)^2 - A^\nu \partial^2 A_\nu
	\sim \frac{1}{2}(\partial^\mu A^\nu - \partial^\nu A^\mu)(\partial_\mu A^\nu-\partial_\nu A_\mu)
	\equiv \frac{1}{2} F^{\mu\nu}F_{\mu\nu}.
\end{equation*}
where we have define a field-strength tensor
\begin{equation}
	F^{\mu\nu}\equiv (\partial^\mu A^\nu - \partial^\nu A^\mu)
	= \left[\begin{array}{cccc}
		0 & B_1 & B_2 & B_3 \\
		-B_1 & 0 & E_3 & -E_2 \\
		-B_2 & -E_3 & 0 & E_1 \\
		-B_3 & E_2 & -E_1 & 0
	\end{array} \right].
\end{equation}
Note that the fourth term is called the \textit{theta term}, which can be written as a boundary term
\begin{equation*}
	\varepsilon_{\mu\nu\rho\lambda} \partial^\mu A^\nu \partial^\rho A^\lambda
	= \partial^\mu (\varepsilon_{\mu\nu\rho\lambda} A^\nu \partial^\rho A^\lambda).
\end{equation*}
The Lagrangian describing the electromagnetic field is given by
\begin{equation}
	\mathcal{L}_{\mathrm{Maxwell}} = -\frac{1}{4}F_{\mu\nu}F^{\mu\nu}.
\end{equation}



\subsection{Path-integral Formalism}
We define the gauge fixing function
\begin{equation*}
	G(A) = \partial_\mu A^\mu(x) -\omega(x) = 0
\end{equation*}
The gauge transformation has the form:
\begin{equation*}
	A^\alpha_\mu(x) = A_\mu(x) + \partial_\mu \alpha(x).
\end{equation*}
We then have
\begin{equation*}
	1 \propto \int D[\alpha] \det\left(\frac{\delta G(A^\alpha)}{\delta \alpha}\right) \delta(G(A)).
\end{equation*}
Inset the identity operator into the path integral formula
\begin{equation*}
	Z[J] \propto \det\left(\partial^2 \right) \int D[\alpha]D[A] e^{iS[A,J]} \delta(\partial_\mu A^\mu -\omega(x)).
\end{equation*}
The above equation does not depend on $\omega(x)$.
We can then integrate over $\omega(x)$ with gaussian weight
\begin{equation*}
\begin{aligned}
	Z[J] &\propto \int D[\omega] e^{-i\int d^d x \frac{\omega^2}{2\xi}} \int D[\alpha]D[A] e^{iS[A,J]}
	\delta(\partial_\mu A^\mu-\omega) \\
	&= \int D[A] e^{iS[A,J]} \exp\left\{i \left[S[A,J]-\int d^d x \frac{1}{2\xi}(\partial_\mu A^\mu)^2 \right]\right\}.
\end{aligned}
\end{equation*}
In momentum space, the modified Langriangian is 
\begin{equation*}
	\tilde{\mathcal{L}}_\xi(k) = \tilde{A}^\mu(k)\left[
		-k^2 g_{\mu\nu}+\left(1-\frac{1}{\xi}\right)k_\mu k_\nu
		\right] \tilde{A}^\nu(-k) +
		\tilde{J}_\mu(k) \tilde{A}^\mu(-k) +
		\tilde{A}^\mu(k) \tilde{J}_\mu(-k).
\end{equation*}
We can check that
\begin{equation}
	\left[-k^2 g_{\mu\nu}+\left(1-\frac{1}{\xi}\right)k_\mu k_\nu\right]^{-1}
	= \frac{-g^{\mu\nu}+(1-\xi)k^\mu k^\nu}{k^2}.
\end{equation}
Thus, the partition function is
\begin{equation}
	\frac{Z_{\mathrm{maxwell}}[J]}{Z_{\mathrm{maxwell}}[0]}
	= \exp\left[-\frac{i}{2}\int d^dx_1 d^dx_2 J_\mu(x_1) \Pi^{\mu\nu}(x_1-x_2) J_\nu(x_2) \right],
\end{equation}
where
\begin{equation}
	\Pi^{\mu\nu}(x_1-x_2) = \int \frac{d^d k}{(2\pi)^d} e^{-ik\cdot(x_1-x_2)}\frac{-g^{\mu\nu}+(1-\xi)k^\mu k^\nu}{k^2}.
\end{equation}
The propagator is
\begin{equation}
\begin{aligned}
	\langle 0|T A^\mu(x_1) A^\nu(x_2) |0\rangle
	&= \left.\frac{1}{Z_{\mathrm{Maxwell}}[0]}\frac{\delta}{iJ_\mu(x_1)}\frac{\delta}{iJ_\nu(x_2)} Z_{\mathrm{Maxwell}}[J]\right|_{J=0} \\
	&= i\Pi^{\mu\nu}(x_1-x_2).
\end{aligned}
\end{equation}


\subsection{Canonical Quantization}
In momentum space, the Lagrangian transforms to
\begin{equation}
	\tilde{A}^\mu(k)\left(-k^2 g_{\mu\nu}+k_\mu k_\nu\right) \tilde{A}^\nu(-k).
\end{equation}
The EOM in momentum space is
\begin{equation*}
	(-k^2 g_{\mu\nu}+k_\mu k_\nu) \tilde{A}^\nu(k) = 0.
\end{equation*}
Since the linear operator $(-k^2 g_{\mu\nu}+k_\mu k_\nu)$ is singular, i.e.,
\begin{equation*}
	(-k^2 g_{\mu\nu}+k_\mu k_\nu)k^\nu = 0.
\end{equation*}
The gauge freedom can be used to further restrict
\begin{equation*}
	A^0 = 0.
\end{equation*}
In this way, there are only two independent polarization for EOM solution
\begin{equation}
	A^\mu = e^{-ik\cdot x} \epsilon^\mu_j,\ j=1,2,
\end{equation}
where
\begin{equation*}
	\epsilon_1 = (0,1,0,0),\
	\epsilon_2 = (0,0,1,0).
\end{equation*}
The field expansion is then
\begin{equation}
	A^\mu = \int \frac{d^{3} k}{(2\pi)^{3}}\frac{1}{\sqrt{2\omega_k}}
	\sum_{j=1}^2 \left(\epsilon^\mu_j a_{k,j} e^{-ik\cdot x} + 
	\epsilon^{\mu*}_j a^\dagger_{k,j} e^{ik\cdot x}\right).
\end{equation}
A single-particle state with polarization vector $\epsilon_j$ is defined as
\begin{equation}
	|k,\epsilon_j\rangle = \sqrt{2\omega_k}	\vec\epsilon_j a^\dagger_{k,j}|0\rangle.
\end{equation}
Note that then the field is off shell (internal photon line), the photon can be space-like or time-like, and then there are an additional polarization.
In general, 
\begin{equation*}
	\sum_{j=1}^3 \epsilon^{\mu*}_j \epsilon^\nu_j = -(1 - P_{k}) = -(g^{\mu\nu}-k^\mu k^\nu),
\end{equation*}
where $P_k$ is the projection to 4-momentum $k$.
The propagator is then
\begin{equation}
	i\Pi(x_1-x_2)= \int \frac{d^4 k}{(2\pi)^4} e^{-ik\cdot(x_1-x_2)}\frac{-i(g^{\mu\nu}-k^\mu k^\nu)}{k^2+i\epsilon}.
\end{equation}



\section{Dirac Field}

Based on previous discussion, the Lagrangian for spinor field can have
\begin{equation*}
	\psi_L^\dagger \bar\sigma^\mu \partial_\mu \psi_L,\ 
	\psi_R^\dagger \sigma^\mu \partial_\mu \psi_R,\ 
	\psi_L^\dagger \psi_R,\ \psi_R^\dagger \psi_L,\ 
	\psi_L \cdot \psi_L,\ \psi_R \cdot \psi_R.
\end{equation*}
The Dirac field describe the theory with both left-hand and right-hand spinors.
The Lagrangian is
\begin{equation}
	\mathcal{L}_{\mathrm{Dirac}}
	= \bar\psi \left(i\gamma^\mu \partial_\mu - m\right)\psi,
\end{equation}
where
\begin{eqnarray}
	\psi = \left(\begin{array}{c}
		\psi_L \\ \psi_R
	\end{array}\right),\ 
	\bar\psi = \left(\begin{array}{cc}
		\psi_R^\dagger & \psi_L^\dagger
	\end{array}\right),\ 
	\gamma^\mu = \left(\begin{array}{cc}
		0 & \sigma^\mu \\
		\bar\sigma^\mu & 0
	\end{array}\right).
\end{eqnarray}
In addition, we could consider using the last two terms as the mass, the result theory is the \textit{Majorana field theory}:
\begin{equation}
\begin{aligned}
	\mathcal{L}^{L}_{\mathrm{Majorana}}
	&= \psi_L^\dagger \left(i\bar\sigma^\mu \partial_\mu -m \sigma^2 \right) \psi_L, \\
	\mathcal{L}^{R}_{\mathrm{Majorana}}
	&= \psi_R^\dagger \left(i\sigma^\mu \partial_\mu -m \sigma^2 \right) \psi_R. 
\end{aligned}
\end{equation} 



\subsection{Path-integral Formalism}

Consider the partition function with source
\begin{equation}
	Z_{\mathrm{Dirac}}[J]
	= \int D[\bar\psi,\psi] \exp\left[i\int d^dx \left(\mathcal{L}_{\mathrm{Dirac}}+\bar{\eta}\psi + \bar\psi\eta \right) \right].
\end{equation}
In momentum space:
\begin{equation}
	S = \int\frac{d^d k}{(2\pi)^d} \left[
		\tilde{\bar\psi}(k)(\cancel{k}-m)\tilde{\psi}(k) +
		\tilde{\bar\eta}(k) \tilde{\psi}(k) +
		\tilde{\bar\psi}(k) \tilde{\eta}(k)
	\right].
\end{equation}
Using the Gaussian integral formula (for Grassman variables), the partition function is:
\begin{equation}
\begin{aligned}
	\frac{Z_{\mathrm{Dirac}}[J]}{Z_{\mathrm{Dirac}}[0]}
	&= \exp\left[-i\int \frac{d^d k}{(2\pi)^d} \tilde{\bar\eta}(k)\frac{1}{\cancel{k}-m}\tilde\eta(k)\right] \\
	&= \exp\left[-i\int d^dx_1 d^d x_2 \bar{\eta}(x_1)\cdot D_F(x_1-x_2)\cdot \eta(x_2) \right]
\end{aligned}
\end{equation}

where
\begin{equation}
	D_F(x_1-x_2) = \int \frac{d^d k}{(2\pi)^d} \frac{e^{-i k \cdot (x_1-x_2)}}{\cancel{k}-m}
	= \int \frac{d^d k}{(2\pi)^d} \frac{\cancel{k}+m}{k^2-m^2} e^{-i k \cdot (x_1-x_2)}.
\end{equation}
Note that the propagator is
\begin{equation}
\begin{aligned}
	\langle 0| T \psi^\alpha(x_1) \bar\psi^\beta(x_2) |0\rangle
	&= \left.\frac{1}{Z_{\mathrm{Dirac}}[0]}\frac{\delta}{i\delta \bar{\eta}_\alpha(x_1))}\frac{i\delta}{\delta\eta_\beta(x_2)} Z_{\mathrm{Dirac}}[\bar\eta,\eta]\right|_{\eta=\bar\eta=0} \\
	&= i D^{\alpha\beta}_F(x_1-x_2),
\end{aligned}
\end{equation}
where the sign in the variational derivative comes from the anti-commutation relation of the fermionic fields.


\subsection{Canonical Quantization}
In momentum space, the Lagrangian ie=s:
\begin{equation*}
	\tilde{\bar \psi}(p)(\cancel{p} - m)\tilde\psi(p),
\end{equation*}
The EOM is
\begin{equation}
	(\cancel p -m)\tilde\psi(p) = 0
\end{equation}


The general solution of the Dirac equation can be written as a linear combination of plane waves. 
The positive frequency waves are of the form
\begin{equation*}
	\psi(x)=u(p) e^{-i p \cdot x}, \quad p^{2}=m^{2}, \quad p^{0}>0
\end{equation*}
There are two linearly independent solutions for $u(p)$,
\begin{equation*}
	u^{s}(p)=\left(\begin{array}{c}
	\sqrt{p \cdot \sigma} \xi^{s} \\
	\sqrt{p \cdot \bar{\sigma}} \xi^{s}
	\end{array}\right), \quad s=1,2
\end{equation*}
which we normalize according to
\begin{equation*}
	\bar{u}^{r}(p) u^{s}(p)=2 m \delta^{r s} \quad \text { or } \quad u^{r \dagger}(p) u^{s}(p)=2 \omega_{\bm p} \delta^{r s}
\end{equation*}
In exactly the same way, we can find the negative-frequency solutions:
\begin{equation*}
	\psi(x)=v(p) e^{+i p \cdot x}, \quad p^{2}=m^{2}, \quad p^{0}>0 \text {. (3.61) }
\end{equation*}
Note that we have chosen to put the $+$ sign into the exponential, rather than having $p^{0}<0$.
There are two linearly independent solutions for $v(p)$,
\begin{equation*}
	v^{s}(p)=\left(\begin{array}{c}
		\sqrt{p \cdot \sigma} \eta^{s} \\
		-\sqrt{p \cdot \bar{\sigma}} \eta^{s}
	\end{array}\right), \quad s=1,2
\end{equation*}
where $\eta^{s}$ is another basis of two-component spinors. These solutions are normalized according to
\begin{equation*}
	\bar{v}^{r}(p) v^{s}(p)=-2 m \delta^{r s} \quad \text { or } \quad v^{r \dagger}(p) v^{s}(p)=+2 \omega_{\bm{p}} \delta^{r s}
\end{equation*}
The $u$'s and $v$'s are also orthogonal to each other:
\begin{equation*}
\begin{aligned}
	\bar{u}^{r}(p) v^{s}(p) &=0, & 
	u^{r\dagger}(\bm p,\omega_{\bm p}) v^{s}(-\bm p,\omega_{\bm p}) &=0, \\
	\bar{v}^{r}(p) u^{s}(p) &=0, & 
	v^{r\dagger}(\bm p,\omega_{\bm p}) u^{s}(-\bm p,\omega_{\bm p}) &=0.
\end{aligned}
\end{equation*}
A useful identity is
\begin{equation*}
\begin{aligned}
	\sum_{s} u^{s}(p) \bar{u}^{s}(p) &= \cancel p+m, \\
	\sum_{s} v^{s}(p) \bar{v}^{s}(p) &= \cancel p-m.
\end{aligned}
\end{equation*}


The Dirac field expansion is
\begin{equation}
\begin{aligned}
	\psi(x) &=\int \frac{d^{3} p}{(2 \pi)^{3}} \frac{1}{\sqrt{2 \omega_{\mathbf{p}}}} 
		\sum_{s}\left(a_{\mathbf{p}}^{s} u^{s}(p) e^{-i p \cdot x}
		+b_{\mathbf{p}}^{s \dagger} v^{s}(p) e^{i p \cdot x}\right), \\
	\bar{\psi}(x) &=\int \frac{d^{3} p}{(2 \pi)^{3}} \frac{1}{\sqrt{2 \omega_{\mathbf{p}}}} 
		\sum_{s}\left(b_{\mathbf{p}}^{s} \bar{v}^{s}(p) e^{-i p \cdot x}
		+a_{\mathbf{p}}^{s \dagger} \bar{u}^{s}(p) e^{i p \cdot x}\right).
\end{aligned}
\end{equation}
Now let us investigate the propagator
\begin{equation}
\begin{aligned}
	iD_{F,\alpha\beta}(x_1-x_2) &= \langle0|T\psi_\alpha(x_1)\bar\psi_\beta(x_2)|0\rangle \\
	&= \theta(\tau) \langle0|\psi_\alpha(x_1)\bar\psi_\beta(x_2)|0\rangle - \theta(-\tau) \langle0|\bar\psi_\beta(x_2)\psi_\alpha(x_1)|0\rangle.
\end{aligned}
\end{equation}
On the RHS, the first term is
\begin{equation*}
\begin{aligned}
	\langle0|\psi_\alpha(x_1)\bar\psi_\beta(x_2)|0\rangle 
	&= \int \frac{d^{3} p}{(2 \pi)^{3}} \frac{1}{\sqrt{2 \omega_{\mathbf{p}}}} \left[\sum_s u_\alpha^s(p)\bar u_\beta^s(p)\right]e^{-i p\cdot (x_1-x_2)} \\
	&= (i\cancel \partial+m)_{\alpha\beta}\int \frac{d^{3} p}{(2 \pi)^{3}} \frac{1}{\sqrt{2 \omega_{\mathbf{p}}}} e^{-i p\cdot (x_1-x_2)}.
\end{aligned}
\end{equation*}
For the second term:
\begin{equation*}
\begin{aligned}
	\langle0|\bar\psi_\beta(x_2)\psi_\alpha(x_1)|0\rangle
	&= \int \frac{d^{3} p}{(2 \pi)^{3}} \frac{1}{\sqrt{2 \omega_{\mathbf{p}}}} \left[\sum_s \bar v_\beta^s(p)v_\alpha^s(p)\right]e^{i p\cdot (x_1-x_2)} \\
	&= -(i\cancel \partial + m)_{\alpha\beta}\int \frac{d^{3} p}{(2 \pi)^{3}} \frac{1}{\sqrt{2 \omega_{\mathbf{p}}}} e^{i p\cdot (x_1-x_2)}.
\end{aligned}
\end{equation*}
Together, the Dirac propagator is:
\begin{equation*}
\begin{aligned}
	iD_F(x_1-x_2) &= (i\cancel \partial+m)i\Delta(x_1-x_2) \\
	&= \int\frac{d^{4} p}{(2\pi)^{4}} e^{-i p\cdot (x_1-x_2)}\frac{i(\cancel p+m)}{p^2-m^2+i\epsilon}.
\end{aligned}
\end{equation*}



\section{Symmetries}

\subsection{Global Symmetries and Conserved Quantities}
If a field theory has a global symmetry, it means that under the infinitesimal transformation:
\begin{equation*}
\begin{aligned}
	x^{\mu} & \rightarrow x^{\mu}+\delta x^{\mu},\\
	\phi_{r}\left(x\right) & \rightarrow \phi_{r}\left(x\right)+\delta\phi_{r}\left(x\right),
\end{aligned}
\end{equation*}
the action is invariant, i.e.,
\begin{equation}
	\int_{\Omega'}d^{4}x'\mathcal{L}'\left(x'\right)=\int_{\Omega}d^{4}x\mathcal{L}\left(x\right).
\end{equation}
Here, instead of writing the infinite integral, we require that for any space-time region $\Omega$ (which is transformed to $\Omega'$ under symmetry transformation), the Lagrangian integral is invariant under the symmetry action.




