\chapter{Relativistic Free Field Theories}

\section{The Lorentz Invariance}

The metric for ($3+1$)-dimensional flat space-time is chosen to be
\begin{equation}
	g_{\mu\nu}=g^{\mu\nu}=\mathrm{diag}(+1,-1,-1,-1).
\end{equation}
The Lorentz transformation ${\Lambda^{\mu}}_{\nu}$ satisfies
\begin{equation}
{\Lambda^{\mu}}_{\alpha}{\Lambda^{\nu}}_{\beta} g^{\alpha\beta} = g^{\mu\nu}.
\end{equation}
From this we have
\begin{equation}
	g^{\gamma\alpha}{\Lambda^{\mu}}_{\alpha}{\Lambda^{\nu}}_{\beta} g_{\mu\nu} 
	= g^{\gamma\alpha}g_{\alpha\beta} 
	\quad \Longrightarrow \quad
	{\Lambda_{\nu}}^{\gamma}{\Lambda^{\nu}}_{\beta} 
	= {\delta^{\gamma}}_{\beta},
\end{equation}
The inverse Lorentz transformation satisfies:
\begin{equation}
	{(\Lambda^{-1})^{\mu}}_{\nu} = {\Lambda_{\nu}}^{\mu}.
\end{equation}
The infinitesimal transformation is denoted as
\begin{equation}
\begin{aligned}
	{\Lambda^{\mu}}_{\nu} &= \delta^{\mu}_{\nu}+\delta{\omega^{\mu}}_{\nu}, \\
	{(\Lambda^{-1})^\mu}_\nu &= \delta^{\mu}_{\nu}-\delta{\omega^\mu}_\nu.
\end{aligned}
\end{equation}
which means $\delta {\omega^\mu}_\nu = -\delta {\omega_\nu}^\mu$.
We can further use the metric tensor $g_{\mu\nu}$ to lower the indices and get $\delta\omega_{\alpha\beta} = -\delta\omega_{\beta\alpha}$, i.e., the infinitesimal parameter $\delta \omega_{\mu\nu}$ is anti-symmetric for ($\mu \leftrightarrow \nu$).

In general, a representation of Lorentz group $U_R(\Lambda)$ can be parametrized as:
\begin{equation}
	U_R(\Lambda) = \exp\left(\frac{i}{2}\omega_{\mu\nu}M_R^{\mu\nu}\right).
\end{equation}
Another useful parametrization is
\begin{equation}
	\theta_i \equiv \frac{1}{2}\varepsilon_{ijk}\omega_{jk}, \quad 
	\beta_i \equiv \omega_{i0}.
\end{equation}
A new set of generators are:
\begin{equation}
	J_i \equiv \frac{1}{2}\varepsilon_{ijk}M^{jk}, \quad 
	K_i \equiv M^{i0},
\end{equation}
where $J_i$'s are the generators of the spatial rotations, and $K_i$'s are the generators of Lorentz boosts.

In the fundamental representation, the generators are represented by
\begin{equation}
\begin{aligned}
	J_1 &= \left[\begin{array}{cccc} 0 & & & \\ & 0 & & \\ & & 0 & -i \\ & & i & 0 \end{array}\right], & 
	J_2 &= \left[\begin{array}{cccc} 0 & & & \\ & 0 & & i \\ & & 0 & \\ & -i & & 0 \end{array}\right], &
	J_3 &= \left[\begin{array}{cccc} 0 & & & \\ & 0 & -i & \\ & i & 0 & \\ & & & 0 \end{array}\right], \\
	K_1 &= \left[\begin{array}{cccc} 0 & -i & & \\ -i & 0 & & \\ & & 0 & \\ & & & 0 \end{array}\right], & 
	K_2 &= \left[\begin{array}{cccc} 0 & & -i & \\ & 0 & & \\ -i & & 0 & \\ & & & 0 \end{array}\right], &
	K_3 &= \left[\begin{array}{cccc} 0 & & & -i \\ & 0 & & \\ & & 0 & \\ -i & & & 0 \end{array}\right].
\end{aligned}
\end{equation}
The Lie algebra of the Lorentz algebra can be explicitly done using the fundamental representation. 
The result is
\begin{equation}
\begin{aligned}
	\left[J_i, J_j\right] &= i \varepsilon_{ijk} J_k, \\
	\left[J_i, K_j\right] &= i \varepsilon_{ijk} K_k, \\
	\left[K_i, K_j\right] &= -i\varepsilon_{ijk} J_k.
\end{aligned}
\end{equation}



\subsection{Left and Right Spinors}
We introduce a new set of generators:
\begin{equation}
	N_i^{L} \equiv \frac{J_i - i K_i}{2}, \quad
	N_i^{R} \equiv \frac{J_i + i K_i}{2}.
\end{equation}
They satisfies two independent $\mathfrak{su}(2)$ algebra:
\begin{equation}
\begin{aligned}
	\left[N_i^L, N_j^L \right] &= i\varepsilon_{ijk}N_k^L, \\
	\left[N_i^R, N_j^R \right] &= i\varepsilon_{ijk}N_k^R, \\
	\left[N_i^L, N_j^R \right] &= 0.
\end{aligned}
\end{equation}
That is, the Lorentz algebra is isomorphic to two $\mathfrak{su}(2)$ algebra,
\begin{equation}
	\mathfrak{so}(3,1) \approx \mathfrak{su}_L(2)\oplus\mathfrak{su}_R(2).
	\label{eq:Lorentz-alg-decomp}
\end{equation}
From Eq.~(\ref{eq:Lorentz-alg-decomp}), we know that the representation of the Lorentz algebra can be labelled by $j_L$ and $j_R$.
Note that the fundamental representation correspond to
\begin{equation*}
	\left(j_L=\frac{1}{2},j_R=\frac{1}{2}\right).
\end{equation*}
The specific form of the group is
\begin{equation}
	\Lambda(\vec\theta,\vec\beta)
	=\exp\left[i(\vec\theta+i\vec\beta)\cdot \vec N^L + i(\vec\theta-i\vec\beta)\cdot \vec N^R\right].
\end{equation}
The spinor representations are those with $j_L=1/2$ or $j_R=1/2$. 
Specifically, we define the left-hand spinor $\psi_L$ and right-hand spinor $\psi_R$ that transform as:
\begin{equation}\label{eq:qft-left-right-spinor-rep}
\begin{aligned}
	\Lambda_L(\vec\theta,\vec\beta)\psi_L 
	&= \exp\left(\frac{i}{2}\vec\theta\cdot\vec\sigma-\frac{1}{2}\vec\beta\cdot\vec\sigma \right) \psi_L, \\
	\Lambda_R(\vec\theta,\vec\beta)\psi_R 
	&= \exp\left(\frac{i}{2}\vec\theta\cdot\vec\sigma+\frac{1}{2}\vec\beta\cdot\vec\sigma \right) \psi_R.
\end{aligned}
\end{equation}
Using the fact $\sigma^2 \cdot \vec\sigma^* \cdot\sigma^2 = -\vec\sigma$, the left-hand and the right-hand representations are related by:
\begin{equation}
\begin{aligned}
	\sigma^2 \Lambda_L^* \sigma^2 &= \Lambda_R, & \sigma^2 \Lambda_L^T \sigma^2 &= \Lambda_L^{-1}, \\
	\sigma^2 \Lambda_R^* \sigma^2 &= \Lambda_L, & \sigma^2 \Lambda_R^T \sigma^2 &= \Lambda_R^{-1}.
\end{aligned}
\end{equation}
For this reason, the left-hand and right-hand spinor can be interchanged by
\begin{equation}
\begin{aligned}
	\sigma^2 \psi_L^* &\sim \chi_R, & \psi_L^\dagger \sigma^2 &\sim \chi^\dagger_R, \\
	\sigma^2 \psi_R^* &\sim \chi_L, & \psi^\dagger_R \sigma^2 &\sim \chi^\dagger_L.
	\label{eq:left-right-spinor-rel}
\end{aligned}
\end{equation}



\subsection{The Invariant Symbols}
The invariant symbols can be thought as the Clebsch-Gordan coefficients that help to form singlets.
The first singlet comes from the decomposition
\begin{equation*}
	\frac{1}{2}\otimes \frac{1}{2} \approx 0 \oplus 1.
\end{equation*}
Correspondingly, we can check that for each-hand-side spinor, the quadratic forms
\begin{equation}
	\psi_L^T\sigma^2\chi_L \quad \text{or} \quad 
	\psi_R^T\sigma^2\chi_R
	\label{eq:inner-product-inv-symbol}
\end{equation}
are singlets.
We can define the first invariant symbol as\footnote{We use the dotted symbol to denote the right-hand spinor indices.}
\begin{equation}
	\varepsilon^{ab} = \varepsilon^{\dot a \dot b} = i(\sigma^2)_{ab}, \quad
	\varepsilon_{ab} = \varepsilon_{\dot a \dot b} = -i(\sigma^2)_{ab}.
\end{equation}
The symbol $\varepsilon^{ab}$ or $\varepsilon_{ab}$ also serve as the index raising/lowering symbol, i.e.,
\begin{equation}
	\varepsilon^{ab}\psi_b = \psi^a,\ 
	\varepsilon_{ab}\psi^b = \psi_a.
\end{equation}
The singlet (\ref{eq:inner-product-inv-symbol}) is then defined as the inner product of two spinors:
\begin{equation}
	\psi\cdot\chi 
	\equiv \varepsilon_{ab}\psi^a\chi^b
	= \psi^a\chi_{a}
	= -\varepsilon_{ba}\psi^a\chi^b
	= -\psi_b\chi^b.
\end{equation}
In addition, because of (\ref{eq:left-right-spinor-rel}), the expressions
\begin{equation*}
	\psi_L^\dagger \chi_R \quad \text{and} \quad \psi_R^\dagger \chi_L
\end{equation*}
are also singlets.

Besides, we know there should be another invariant symbol from the decomposition
\begin{equation*}
	\left(\frac{1}{2}, 0\right) \otimes \left(0,\frac{1}{2}\right)
	\approx \left(0, 0\right) \oplus \cdots.
\end{equation*}
For this reason, we are searching for the symbol $M$ that the expression
\begin{equation*}
	M^\mu_{a\dot b} \psi^a_L \chi^{\dot b}_R
\end{equation*}
transforms as the Lorentz vector.
The matrix $M^\mu$ should transform as
\begin{equation*}
	M^\mu \longrightarrow \Lambda_L^T \cdot M^\mu \cdot \Lambda_R = {\Lambda^\mu}_\nu M^\nu.
\end{equation*}
Use the fact that $\sigma^2 \cdot \Lambda_L^T \cdot \sigma^2 = \Lambda_L^{-1}$, the above equation transforms to
\begin{equation*}
	\left(\sigma^2 M^\mu\right) \longrightarrow \Lambda_L^{-1} \cdot \left(\sigma^2 M^\mu\right)\cdot \Lambda_R.
\end{equation*}
We then show the matrices $\sigma^\mu = (\sigma^0,\vec\sigma)$ satisfies the requirement.
Firstly, for the spatial rotation,
\begin{equation}
	\Lambda_L(\vec\theta,\vec 0) = \Lambda_R(\theta,\vec 0) = \exp\left(i\vec\theta\cdot \frac{\vec\sigma}{2}\right)
\end{equation}
The Pauli matrix transform as
\begin{equation*}
	\left(1-i\delta\vec\theta\cdot\frac{\vec\sigma}{2}\right)\sigma^j\left(1+i\delta\vec\theta\cdot \frac{\vec\sigma}{2}\right)
	= \sigma^j + i\delta\theta_i \left(-i \varepsilon_{ijk}\sigma^k \right)
\end{equation*}
Secondly, for the boosts,
\begin{equation}
	\Lambda_{L/R}(\vec 0, \vec\beta) = 
	\exp\left(\mp\vec\beta\cdot \frac{\vec\sigma}{2}\right).
\end{equation}
The Pauli matrix transform as
\begin{equation*}
	\left(1+\delta\vec\beta\cdot\frac{\vec\sigma}{2}\right)\sigma^\mu \left(1+\delta\vec\beta\cdot \frac{\vec\sigma}{2}\right) = \begin{cases}
		 \sigma^0 + i\delta\beta_i \cdot (-i\sigma^i), & \mu = 0 \\
		 \sigma^j + i\delta\beta_j (-i\sigma^0), & \mu = j
	\end{cases}.
\end{equation*}
We thus have shown indeed that
\begin{equation}
	\psi_L^T \sigma^2 \sigma^\mu \chi_R
\end{equation}
is a Lorentz vector.
Further more, from (\ref{eq:left-right-spinor-rel}), we know that
\begin{equation}
	\eta_R^\dagger \sigma^\mu \chi_R
\end{equation}
is also a Lorentz vector.
Similarly, consider the Lorentz vector 
\begin{equation*}
	N^\mu_{\dot a b} \psi^{\dot a}_R \chi^{b}_R,
\end{equation*}
which together with $\sigma^2$ should transforms as
\begin{equation*}
	\left(\sigma^2 N^\mu\right) \longrightarrow 
	\Lambda_R^{-1} \cdot \left(\sigma^2 N^\mu\right)\cdot \Lambda_L.
\end{equation*}
We can check that $\bar\sigma^\mu = (\sigma^0,-\vec\sigma)$ satisfies the requirement, and thus 
\begin{equation}
	\eta_L^\dagger \bar\sigma^\mu \chi_L
\end{equation}
is also a Lorentz vector.



\subsection{Lorentz-invariant Lagrangian}
In relativistic quantum field theory, the Lagrangian should be a singlet under Lorentz transformation.
Different free fields correspond to different representation of the Lorentz algebra.
The symmetry under Lorentz transformation also restrict the possible terms that can appear in the Lagrangian.

\subsubsection{Scalar Field}
The simplest case is when $j_L=j_R = 0$, corresponding to the scalar field, which we denote as $\phi(x)$.
Since the field it self is singlet, any polynomial of the field in principle can appear in the theory.
When considering the free theory, we restrict our attention to the quadratic terms.
We require the field theory to have a dynamical term, which contains derivative the the field.
The derivative operator $\partial^\mu$ transforms as the fundamental representation.
To be Lorentz invariant, the allowed free theory can only be
\begin{equation}
	\mathcal L_{\mathrm{KG}} = \frac{1}{2}\partial^\mu \phi \partial_\mu \phi -\frac{m^2}{2}\phi^2 
	\simeq -\frac{1}{2}\phi (\partial^2+m^2) \phi.
\end{equation}

For general discussion, we consider the field theory on $d$-dimensional space-time.
Note that the space-time Fourier transformation is defined as
\begin{equation}
\begin{aligned}
	\tilde{\phi}(k) &= \int d^{d}x e^{ik\cdot x} \phi(x), \\ 
	\phi(x) &= \int \frac{d^{d}k}{(2\pi)^{d}} e^{-ik\cdot x}\tilde{\phi}(k),
\end{aligned}
\end{equation}
where the inner product of two $d$-momentum and $d$-coordinate is
\begin{equation}
	k\cdot x \equiv \omega t-\vec k\cdot \vec x.
\end{equation}
The action can be expressed as
\begin{equation}
	S_{\mathrm{KG}} = \int \frac{d^d k}{(2\pi)^d} \frac{1}{2} \tilde{\phi}^*(k)(k^2-m^2)\tilde{\phi}(k).
\end{equation}

\subsubsection{Vector Field}
If we can choose $j_L=j_R=1/2$, the field is transformed as Lorentz vector.
We denote the field as $A^\mu(x)$.
Some possible quadratic forms for the vector field that forms singlets are
\begin{equation}
	A^\mu A_\mu,\ (\partial_\mu A^\mu)^2,\ A^\nu \partial^2 A_\nu,\ 
	\varepsilon_{\mu\nu\rho\lambda} \partial^\mu A^\nu \partial^\rho A^\lambda.
\end{equation}
For the field theory describe the electromagnetic field, we require the theory to further have gauge symmetry, i.e., invariant under
\begin{equation}
	A^\mu(x) \rightarrow A^\mu(x) + \partial^\mu \alpha(x).
\end{equation}
The gauge invariant forbids the first term, and forces the second and third term to combine as
\begin{equation*}
	(\partial_\mu A^\mu)^2 - A^\nu \partial^2 A_\nu
	\sim \frac{1}{2}(\partial^\mu A^\nu - \partial^\nu A^\mu)(\partial_\mu A^\nu-\partial_\nu A_\mu)
	\equiv \frac{1}{2} F^{\mu\nu}F_{\mu\nu}.
\end{equation*}
where we have define a field-strength tensor
\begin{equation}
	F^{\mu\nu}\equiv (\partial^\mu A^\nu - \partial^\nu A^\mu)
	= \left[\begin{array}{cccc}
		0 & -E_1 & -E_2 & -E_3 \\
		E_1 & 0 & -B_3 & B_2 \\
		E_2 & B_3 & 0 & -B_1 \\
		E_3 & -B_2 & B_1 & 0
	\end{array} \right],
\end{equation}
where we notice that from Maxwell equations:
\begin{equation}
	E^i = \partial_t \vec A = -\vec\nabla A^0, \quad B^i = \nabla \times \vec A.
\end{equation}
Note that the fourth term is called the \textit{theta term}, which can be written as a boundary term
\begin{equation}
	\varepsilon_{\mu\nu\rho\lambda} \partial^\mu A^\nu \partial^\rho A^\lambda
	= \partial^\mu (\varepsilon_{\mu\nu\rho\lambda} A^\nu \partial^\rho A^\lambda).
\end{equation}
The Lagrangian describing the electromagnetic field is given by
\begin{equation}
	\mathcal{L}_{\mathrm{Maxwell}} = -\frac{1}{4}F_{\mu\nu}F^{\mu\nu}.
\end{equation}


\subsubsection{Spinor Field}

Based on previous discussion, the Lagrangian for spinor field can have
\begin{equation}
	\psi_L^\dagger \bar\sigma^\mu \partial_\mu \psi_L,\ 
	\psi_R^\dagger \sigma^\mu \partial_\mu \psi_R,\ 
	\psi_L^\dagger \psi_R,\ \psi_R^\dagger \psi_L,\ 
	\psi_L \cdot \psi_L,\ \psi_R \cdot \psi_R.
\end{equation}
The Dirac field describe the theory with both left-hand and right-hand spinors.
The Lagrangian is
\begin{equation}
	\mathcal{L}_{\mathrm{Dirac}}
	= \bar\psi \left(i\gamma^\mu \partial_\mu - m\right)\psi,
\end{equation}
where
\begin{eqnarray}
	\psi = \left(\begin{array}{c}
		\psi_L \\ \psi_R
	\end{array}\right),\ 
	\bar\psi = \left(\begin{array}{cc}
		\psi_R^\dagger & \psi_L^\dagger
	\end{array}\right),\ 
	\gamma^\mu = \left(\begin{array}{cc}
		0 & \sigma^\mu \\
		\bar\sigma^\mu & 0
	\end{array}\right).
\end{eqnarray}
In addition, we could consider using the last two terms as the mass, the result theory is the \textit{Majorana field theory}:
\begin{equation}
\begin{aligned}
	\mathcal{L}^{L}_{\mathrm{Majorana}}
	&= \psi_L^\dagger \left(i\bar\sigma^\mu \partial_\mu -m \sigma^2 \right) \psi_L, \\
	\mathcal{L}^{R}_{\mathrm{Majorana}}
	&= \psi_R^\dagger \left(i\sigma^\mu \partial_\mu -m \sigma^2 \right) \psi_R. 
\end{aligned}
\end{equation} 
For the spinor basis, the Dirac Algebra is generated by
\begin{equation}\label{eq:qft-diract-generator}
	S^{\mu\nu} = \frac{i}{4}[\gamma^\mu, \gamma^\nu].
\end{equation}
The Lorentz group is represented by
\begin{equation}\label{eq:qft-dirac-rep}
	\Lambda_{\frac{1}{2}} = \exp\left(\frac{i}{2}\omega_{\mu\nu} S^{\mu\nu}\right).
\end{equation}
Using the familiar parametrization,
\begin{equation}
	S^{i0} = \frac{i}{2}\left[\begin{array}{cc}
		\sigma^i & 0 \\ 0 & -\sigma^i
	\end{array}\right], \quad 
	S^{ij} = \frac{1}{2}\epsilon^{ijk} \left[\begin{array}{cc}
		\sigma^k & 0 \\ 0 & -\sigma^k
	\end{array}\right],
\end{equation}
which agree with the transformation property (\ref{eq:qft-left-right-spinor-rep}).



\section{Canonical Quantization}

\subsection{Scalar Field}

For the Klein-Gordon Lagrangian
\begin{equation}
	\mathcal L = -\frac{1}{2}\phi(x)(\partial^2+m^2)\phi(x),
\end{equation}
the equation of motion is:
\begin{equation}\label{eq:rkg-eom}
\begin{aligned}
	&\ \partial_\mu \left[\frac{\partial \mathcal L}{\partial(\partial_\mu \phi)}\right] - \frac{\partial \mathcal L}{\partial \phi} = 0 \\
	\Rightarrow &\
	(\partial_t^2-\nabla^2+m^2)\phi(\bm x,t) = 0.
\end{aligned}
\end{equation}
The (classical) solution to Eq.~(\ref{eq:rkg-eom}) is proportional to the plane wave:
\begin{equation}
	\phi_{\bm k}(\bm x, t) \propto e^{-i\omega_{\bm{k}}t+i\bm{k}\cdot\bm{x}} + e^{i\omega_{\bm{k}}t-i\bm{k}\cdot\bm{x}},
\end{equation}
where the energy is $\omega_{\bm{k}}=\bm{k}^2+m^2$ and $\bm k$ is the momentum as the conserved quantity.
The general solution to Eq.~(\ref{eq:rkg-eom}) is
\begin{equation}
	\phi(\bm x,t) \propto \int \frac{d^{3} k}{(2\pi)^{3}} \left(
		a_{k}e^{-i\omega_{\bm{k}}t+i\bm{k}\cdot\bm{x}} + 
		a^*_{k}e^{i\omega_{\bm{k}}t-i\bm{k}\cdot\bm{x}} 
	\right),
\end{equation}
where $a_k$'s are arbitrary c-numbers.

The canonical quantization promote the coefficient $a_{k}/a_{k}^*$ to the particle annihilation/creation operator $a_{k}/a_{k}^\dagger$, with the commutation relation
\begin{equation}
	[a_{k}, a_{p}^\dagger] = (2\pi)^{3} \delta^{3}(\bm{k}-\bm{p}).
\end{equation}

\subsubsection{Single-particle States}
The single-particle state with momentum $\bm k$ is created by $a_{k}^{\dagger}$ operators acting on the vacuum:
\begin{equation}
	|\bm{k}\rangle \equiv \sqrt{2\omega_{\bm k}} a_{k}^{\dagger}|0\rangle,
	\label{eq:rel-single-particle}
\end{equation}
where $|\bm{k}\rangle$ is a state with a single particle of momentum $\bm{k}$.
The factor of $\sqrt{2 \omega_{\bm k}}$ in (\ref{eq:rel-single-particle}) is a convention to ensure Lorenz invariant.
To compute the normalization of one-particle states, we start by requiring the vacuum state to be of unit norm:
\begin{equation}
	\langle 0|0\rangle=1,
\end{equation}
which, together with the canonical commutation relation of particle annihilation and creation operators leads to
\begin{equation}
	\langle\bm{p}|\bm{k}\rangle 
	= 2\sqrt{\omega_{\bm p} \omega_{\bm k}}\left\langle 0\left|a_{p} a_{k}^{\dagger}\right| 0\right\rangle
	= 2 \omega_{\bm p}(2\pi)^{3} \delta^{3}(\bm{p}-\bm{k}).
\end{equation}
The identity operator for one-particle states under such norm is
\begin{equation}
	1=\int \frac{d^{3} p}{(2\pi)^{3}} \frac{1}{2\omega_{\bm p}}|\bm{p}\rangle\langle\bm{p}|, \label{eq:rel-identity}
\end{equation}
which we can check with
\begin{equation*}
	|\bm{k}\rangle
	=\int \frac{d^{3} p}{(2\pi)^{3}} \frac{1}{2\omega_{\bm p}}|\bm{p}\rangle\langle\bm{p}|\bm{k}\rangle
	=\int \frac{d^{3} p}{(2\pi)^{3}} \frac{1}{2\omega_{\bm p}} 2\omega_{\bm p}(2\pi)^3 \delta^3(\bm{p}-\bm{k})|\bm{p}\rangle
	=|\bm{k}\rangle.
\end{equation*}
We see that the identity operator (\ref{eq:rel-identity}) under such convention is Lorentz invariant, since it can be expressed as
\begin{equation}
	1 = 2\pi \int \frac{d^{3} p d\omega}{(2\pi)^{4}} \delta(\omega^2-{\bm{p}}^2-m^2) |\bm p\rangle\langle \bm p|.
\end{equation}

The single-particle defined above can be used to fix the normalization:
\begin{equation}
	\langle \bm k|\phi(\bm x,0)|0\rangle = e^{-i \bm k\cdot \bm x},
\end{equation}
leading to the field expansion
\begin{equation}
	\phi(x)
	=\int \frac{d^{3} k}{(2\pi)^{3}} \frac{1}{\sqrt{2\omega_{\bm k}}}\left(a_k 
		e^{-i k \cdot x}+a_k^{\dagger} e^{i k \cdot x}\right).
\end{equation}

\subsubsection{Hamiltonian}

We can obtain the Hamiltonian for the Klein-Gordon field using the Legendre transformation:
\begin{equation}
\begin{aligned}
	H &= \int d^4 x\ \left[\pi(x) \dot{\phi}(x) - \mathcal L(x) \right] \\
	&= \int d^4 x\ \frac{1}{2} \left[\pi^2 + (\nabla \phi)^2 + m^2 \phi^2 \right]
\end{aligned}
\end{equation}
where the canonical momentum is defined as
\begin{equation}
\begin{aligned}
	\pi(x) &= \frac{\partial \mathcal L}{\partial \dot{\phi}} = \dot{\phi}(x) \\
	&= -i\int \frac{d^{3} k}{(2\pi)^{3}} \sqrt{\frac{\omega_{k}}{2}}\left(a_k 
		e^{-i k \cdot x} - a_k^{\dagger} e^{i k \cdot x}\right)
\end{aligned}
\end{equation}
The $\pi^2$ term expands as
\begin{equation}
	\pi^2(x) = \int \frac{d^{3} k_1}{(2\pi)^{3}} \frac{d^{3} k_2}{(2\pi)^{3}}
		\frac{\sqrt{\omega_{k_1} \omega_{k_2}}}{2} \left(a^\dagger_{k_1}a_{k_2}e^{i(k_1-k_2)x} - a^\dagger_{k_1} a^\dagger_{k_2} e^{i(k_1+k_2)x} + h.c.\right).
\end{equation}
We note that after integrate over $x$, the phase factor $e^{i(k_1-k_2)x}$ produce a delta function for $k_1$ and $k_2$.
The $a^\dagger_{k_1} a^\dagger_{k_2}$ terms will finally be cancelled by other terms.
We temporally ignore such term.
The contribution from the first term is then
\begin{equation}
	\int d^4 x\ \pi^2(x) = \int \frac{d^3 k}{(2\pi)^3} \frac{\omega_k}{2} a_k^\dagger a_k + h.c.
\end{equation}
The second term is
\begin{equation}
	(\nabla \phi)^2 = \int \frac{d^{3} k_1}{(2\pi)^{3}} \frac{d^{3} k_2}{(2\pi)^{3}}
		\frac{\bm k_1 \bm k_2}{2\sqrt{\omega_{k_1}\omega_{k_2}}} \left(a^\dagger_{k_1}a_{k_2}e^{i(k_1-k_2)x} - a^\dagger_{k_1} a^\dagger_{k_2} e^{i(k_1+k_2)x} + h.c.\right).
\end{equation}
The third term is
\begin{equation}
	m^2 \phi^2 = \int \frac{d^{3} k_1}{(2\pi)^{3}} \frac{d^{3} k_2}{(2\pi)^{3}}
		\frac{m^2}{2\sqrt{\omega_{k_1}\omega_{k_2}}} \left(a^\dagger_{k_1}a_{k_2}e^{i(k_1-k_2)x} + a^\dagger_{k_1} a^\dagger_{k_2} e^{i(k_1+k_2)x} + h.c.\right).
\end{equation}
All three contributions sum up as
\begin{equation}
\begin{aligned}
	H &= \int \frac{d^3 k}{(2\pi)^3} \frac{1}{2}\left(\frac{\omega_k}{2} + \frac{\bm k^2+m^2}{2 \omega_k}\right) \left(a_k^\dagger a_k + h.c. \right) \\
	&= \int \frac{d^3 k}{(2\pi)^3} \omega_k \left(a_k^\dagger a_k +\frac{1}{2} \right).
\end{aligned}
\end{equation}

We can now check that the $a^\dagger a^\dagger$ terms indeed have no contributions, as the total contribution for each momentum $k$ is
\begin{equation}
	-\frac{\omega_k}{2} + \frac{\bm k^2}{2\omega_k} + \frac{m^2}{2\omega_k} = 0.
\end{equation}

The Hamiltonian in the operator form also make it manifest that
\begin{equation}
	H |\bm k\rangle = \omega_k |\bm k\rangle.
\end{equation}




\subsubsection{Correlation Function}
Consider the two-point correlation (propagator):
\begin{equation}
\begin{aligned}
	i\Delta(x_1-x_2) &\equiv \langle 0|T \phi(x_1) \phi(x_2) |0\rangle \\
	&= \theta(t_1-t_2) \langle 0|\phi(x_1) \phi(x_2) |0\rangle 
	+ \theta(t_2-t_1) \langle 0|\phi(x_2) \phi(x_1) |0\rangle.
\end{aligned}
\end{equation}
Note that
\begin{equation}
	\langle 0|\phi(x_1) \phi(x_2) |0\rangle
	= \int\frac{d^{3} k}{(2\pi)^{3}}\frac{1}{2\omega_k} e^{i\bm k\cdot (\bm x_1-\bm x_2)-i\omega_{\bm k}\tau},
\end{equation}
where $\tau =t_1-t_2$.
The propagator can be written as
\begin{equation}
\begin{aligned}
	i\Delta(x_1-x_2) 
	&= \int\frac{d^{3} k}{(2\pi)^{3}}\frac{1}{2\omega_k} e^{i\bm k\cdot (\bm x_1-\bm x_2)}\left[e^{-i\omega_{\bm k}\tau}\theta(\tau)+e^{i\omega_{\bm k}\tau}\theta(-\tau)\right] \\
	&= \int\frac{d^{3} k}{(2\pi)^{3}} e^{i\bm k\cdot (\bm x_1-\bm x_2)}\int \frac{d\omega}{2\pi i}\frac{-e^{i\omega\tau}}{\omega^2-\omega_k^2+i\epsilon} \\
	&= \int\frac{d^{4} k}{(2\pi)^{4}} e^{-i k\cdot (x_1-x_2)}\frac{i}{k^2-m^2+i\epsilon}.
\end{aligned}
\end{equation}
We have used the identity
\begin{equation*}
	\frac{1}{2\omega_k} \left[e^{-i\omega_{\bm k}\tau}\theta(\tau)+e^{i\omega_{\bm k}\tau}\theta(-\tau)\right] 
	= \int \frac{d\omega}{2\pi i} \frac{-e^{i\omega\tau}}{\omega^2-\omega_k^2+i\epsilon},
\end{equation*}
where an infinitesimal number $\epsilon$ is included to move the singularities away from the real axis.
Any final result shall take the ($\epsilon \rightarrow 0^+$) limit.
Sometimes the infinitesimal $\epsilon$ will be absorbed into the mass, i.e., $m^2 \rightarrow m^2-i\epsilon$.


\subsection{Vector Field}

Although forbid by gauge invariance, we consider a vector field with nonzero mass term.
Actually the vector field can obtain mass from the spontaneous symmetry breaking.
For example the W and Z boson in the week interaction have nonzero mass.
The action with mass term is:
\begin{equation}
	S = \int \frac{d^4 k}{(2\pi)^4} \tilde{A}^{\mu *}(k) \left(-k^2 g_{\mu\nu}+k_\mu k_\nu + m^2 \right) \tilde{A}^\nu(k).
\end{equation}
The equation of motion for the action is
\begin{equation}
\begin{aligned}
	\frac{\delta S}{\delta \tilde{A}^{\mu *}(k)} = 0 
	\quad \Longrightarrow \quad 
	\left(-k^2 g_{\mu\nu}+k_\mu k_\nu + m^2 \right) \tilde{A}^\nu(k) = 0.
\end{aligned}
\end{equation}
Such equation is sometimes called the \textit{Proca equation}.
Note that the Proca equation implies
\begin{equation}
	\partial_\mu A^\mu = k_\nu \tilde{A}^\nu = 0.
\end{equation}
So the Proca equation becomes
\begin{equation}
	\left(-k^2 g_{\mu\nu} + m^2 \right) \tilde{A}^\nu(k) = 0,
\end{equation}
which is similar to the Klein-Gordon field with multiple components.

\subsubsection{Polarization Vectors}
Since we are now dealing with a field with space-time indices, it is helpful to introduce a set of basis vectors.
The general solution to the Proca equation is also a plane wave labelled by momentum $\bm k$.
For each momentum, we introduce a \textit{longitudinal polarization vector}
\begin{equation}
	\epsilon(\bm k,3) \equiv \left(\frac{|\bm k|}{m}, \frac{\bm k}{|\bm k|}\frac{k_0}{m} \right)
\end{equation}
and two \textit{transverse polarization vectors}
\begin{equation}
	\epsilon(\bm k, 1) \equiv (0, \bm \epsilon(\bm k, 1)), \quad
	\epsilon(\bm k, 2) \equiv (0, \bm \epsilon(\bm k, 2)),
\end{equation}
satisfying the orthogonal relation
\begin{equation}
	\bm \epsilon(\bm k, 1)\cdot \bm k = \bm \epsilon(\bm k, 2)\cdot \bm k = k^\mu \epsilon_\mu(\bm k, 3) = 0.
\end{equation}
So these three polarization vector together with 4-momentum $k$ form a basis for the space-time.
For the notational convenience, we define
\begin{equation}
	\epsilon(\bm k,0) \equiv \frac{k}{m}.
\end{equation}
The vector field can be 
\begin{equation}
	A_\mu(\bm k, \lambda; x) \propto e^{-i\omega_k t + \bm k \cdot \bm x} \epsilon_\mu(\bm k, \lambda)
\end{equation}

The condition $k \cdot \tilde A = 0$ is satisfied if we require no mode in the $\epsilon(\bm k, 0)$ polarization.
Then the vector field is basically three independent scalar field, leading to the  field expansion:
\begin{equation}
	A^\mu(x) = \int \frac{d^{3} k}{(2\pi)^{3}}\frac{1}{\sqrt{2\omega_k}}
	\sum_{j=1}^3 \left[\epsilon^\mu(\bm k, j) a_{k,j} e^{-ik\cdot x} + 
	\epsilon^{\mu}(\bm k, j) a^\dagger_{k,j} e^{ik\cdot x}\right].
\end{equation}
A single-particle state with polarization vector $\epsilon(\bm k, j)$ is defined as
\begin{equation}
	|k,\epsilon_j\rangle = \epsilon(\bm k, j) \sqrt{2\omega_k} a^\dagger_{j}|0\rangle.
\end{equation}


\subsubsection{Massless Polarization Vectors}

For the massless vector field, the polarization $\epsilon(\bm k, 3)$ is not well-defined.
We modify the definition to
\begin{equation}
\begin{aligned}
	\epsilon(\bm k, 0) &\equiv (1,0,0,0), \\
	\epsilon(\bm k, 3) &\equiv (0,0,0,1).
\end{aligned}
\end{equation}
Note that we have choose the spatial direction so that the momentum point to the z-direction.

We can add two types of virtual particles generated by $a^\dagger_{k,0}$ and $a^\dagger_{k,3}$ respectively, which are usually called the \textit{scalar photons} and \textit{longitudinal photons}.
However, the gauge fixing condition requires
\begin{equation}
	\partial_\mu A^\mu(x)|\psi\rangle = 0 \quad \Longrightarrow \quad
	(a_{k,0}-a_{k,3})|\psi\rangle = 0
\end{equation}
for all state $|\psi\rangle$ in the gauge-fixed Hilbert space.

We can show that the scalar and longitudinal modes are just the result of gauge transformation.
The physical polarization are the transverse polarization modes, and the field expansion is
\begin{equation}
	A^\mu(x) = \int \frac{d^{3} k}{(2\pi)^{3}}\frac{1}{\sqrt{2\omega_k}}
	\sum_{j=1}^2 \left[\epsilon^\mu(\bm k, j) a_{k,j} e^{-ik\cdot x} + 
	\epsilon^{\mu}(\bm k, j) a^\dagger_{k,j} e^{ik\cdot x}\right].
\end{equation}


\subsubsection{Correlation Function}
To obtain the correlation for the massless vector field, we consider a modified Lagrangian:
\begin{equation}
	\mathcal L = -\frac{1}{4}F_{\mu\nu}F^{\mu\nu}-\frac{\xi}{2}(\partial_\mu A^\mu)^2.
\end{equation}
In the momentum space:
\begin{equation}
	\tilde{\mathcal L}_k = \tilde{A}^\mu(-k)\left(-k^2 g_{\mu\nu}+(1-\xi)k_\mu k_\nu\right) \tilde{A}^\nu(k)
\end{equation}
To construct the inverse matrix we make a general symmetric ansatz
\begin{equation}
	(G_\gamma^{-1})^{\mu\nu}(k)=A\left(k^{2}\right) g^{\mu\nu}+B\left(k^{2}\right) k^{\mu} k^{\nu}.
\end{equation}
Requiring that
\begin{equation}
	(G_\gamma)_{\mu\sigma}(k)(G_\gamma^{-1})^{\sigma\nu}(k) = \delta_\mu^\nu,
\end{equation}
and comparing the coefficients, we get the conditions
\begin{equation}\label{eq:qft-propagator-equation}
\begin{aligned}
	-k^{2} A\left(k^{2}\right) &=1, \\
	\xi k^{2} B\left(k^{2}\right) &=(\xi-1) A\left(k^{2}\right).
\end{aligned}
\end{equation}
In the case $\xi=0$ these equations are not compatible. Without the gauge-fixing term the matrix $(G_\gamma)_{\mu \nu}$ cannot be inverted (since the determinant vanishes) and the Feynman propagator cannot be constructed. If $\xi \neq 0$, however, no problems arise and the system of equations (\ref{eq:qft-propagator-equation}) is solved by
\begin{equation}
	A\left(k^{2}\right)=-\frac{1}{k^{2}}, \quad B\left(k^{2}\right)=\frac{\xi-1}{\xi} \frac{1}{\left(k^{2}\right)^{2}},
\end{equation}
which leads to
\begin{equation}
	G_\gamma(k) = \frac{-g^{\mu\nu}+(1-\xi)k^\mu k^\nu}{k^2}.
\end{equation}
Different choice of $\xi$ correspond to different gauge fixing.
The Landau gauge choose $\xi=1$, and the propagator has the simplest form
\begin{equation}
	G_\gamma(k) = \frac{-g_{\mu\nu}}{k^2}.
\end{equation}



\subsection{Spinor Field}
The equation of motion for Dirac field is
\begin{equation}
\begin{aligned}
	\partial_\mu\frac{\partial\mathcal{L}}{\partial(\partial_\mu\psi)} - \frac{\partial \mathcal L}{\partial \psi} = 0 
	\quad &\Longrightarrow \quad
	\bar\psi(i\overleftarrow{\cancel \partial}-m) = 0, \\
	\partial_\mu\frac{\partial\mathcal{L}}{\partial(\partial_\mu\bar\psi)} - \frac{\partial \mathcal L}{\partial \bar\psi} = 0 
	\quad &\Longrightarrow \quad
	(i\overrightarrow{\cancel \partial}-m)\psi = 0.
\end{aligned}
\end{equation}
This EOM is a matrix equation.
The general solution of the Dirac equation can be written as a linear combination of plane waves (with positive and negative energy):\footnote{Note that we have chosen to put the $+$ sign into the exponential, rather than having $p^{0}<0$.}
\begin{equation}
	\psi_p(x) = \begin{cases}
		u(p) e^{-i p \cdot x} & p^{0}>0 \\
		v(p) e^{+i p \cdot x} & p^{0}<0
	\end{cases}, \quad p^{2}=m^{2}.
\end{equation}
In momentum space, $u(p)$ and $v(p)$ satisfies:
\begin{equation}
	\left[\begin{array}{cc}
		-m & p \cdot \sigma \\ p\cdot \bar\sigma & -m
	\end{array} \right] u_s(p) = 
	\left[\begin{array}{cc}
		-m & -p \cdot \sigma \\ -p\cdot \bar\sigma & -m
	\end{array} \right] v_s(p) = 0
\end{equation}
For massive Dirac field, we can choose the rest frame where $p = (m,0,0,0)$, the matrix equation is\footnote{We first consider the case where there is only one spatial dimension. It correspond to the choice of coordinate such that the momentum point to the $z$ direction.}
\begin{equation}
\begin{aligned}
	\left[\begin{array}{cc}
		-m & m \\
		m & -m
	\end{array}\right] u_s = 0 
	\quad &\Longrightarrow \quad
	u_s = \sqrt{m}\left[\begin{array}{c}
		\xi_s \\ \xi_s
	\end{array}\right], \\
	\left[\begin{array}{cc}
		m & m \\
		m & m
	\end{array}\right] v_s = 0 
	\quad &\Longrightarrow \quad
	v_s = \sqrt{m}\left[\begin{array}{c}
		\eta_s \\ -\eta_s
	\end{array}\right],
\end{aligned}
\end{equation}
where $\xi$ and $\eta$ has two independent solutions.
For example, four linearly independent solutions are
\begin{equation}
	u_{\uparrow} = \left[\begin{array}{c} 1 \\ 0 \\ 1 \\ 0 \end{array}\right], \quad
	u_{\downarrow} = \left[\begin{array}{c} 0 \\ 1 \\ 0 \\ 1 \end{array}\right], \quad
	v_{\uparrow} = \left[\begin{array}{c} -1 \\ 0 \\ 1 \\ 0 \end{array}\right], \quad
	v_{\downarrow} = \left[\begin{array}{c} 0 \\ 1 \\ 0 \\ -1 \end{array}\right].
\end{equation}
The Dirac spinor is a complex four-component object, with eight real degrees of freedom. 
The equations of motion reduce it to four degrees of freedom, which, as we will see, can be interpreted as spin up and spin down for particle and antiparticle.


\subsubsection{Solution in General Frame}
To derive a more general expression, we can solve the equations again in the boosted frame and match the normalization. 
If $p=(E,0,0,p_z)$ then
\begin{equation}
	p \cdot \sigma=\left[\begin{array}{cc}
		E-p_{z} & 0 \\
		0 & E+p_{z}
	\end{array}\right], \quad 
	p \cdot \bar{\sigma}=\left[\begin{array}{cc}
		E+p_{z} & 0 \\
		0 & E-p_{z}
	\end{array}\right].
\end{equation}
Let $a=\sqrt{E-p_{z}}$ and $b=\sqrt{E+p_{z}}$, then $m^{2}=\left(E-p_{z}\right)\left(E+p_{z}\right)=a^{2} b^{2}$ and Dirac equation becomes
\begin{equation}
	\left[\begin{array}{cccc}
		-a b & 0 & a^{2} & 0 \\
		0 & -a b & 0 & b^{2} \\
		b^{2} & 0 & -a b & 0 \\
		0 & a^{2} & 0 & -a b
	\end{array}\right] u_{s}(p) = 
	\left[\begin{array}{cccc}
		a b & 0 & a^{2} & 0 \\
		0 & a b & 0 & b^{2} \\
		b^{2} & 0 & a b & 0 \\
		0 & a^{2} & 0 & a b
	\end{array}\right] v_{s}(p) = 0.
\end{equation}

The solutions are
\begin{equation}
	u_{s}=\left(\begin{array}{ll}
	\left[\begin{array}{ll}
		a & 0 \\
		0 & b
	\end{array}\right] \xi_{s} \\
	\left[\begin{array}{ll}
		b & 0 \\
		0 & a
	\end{array}\right] \xi_{s}
	\end{array}\right), \quad 
	v_{s}=\left(\begin{array}{ll}
	\left[\begin{array}{ll}
		a & 0 \\
		0 & b
	\end{array}\right] \eta_{s} \\
	-\left[\begin{array}{ll}
		b & 0 \\
		0 & a
	\end{array}\right] \eta_{s}
	\end{array}\right).
\end{equation}
Using
\begin{equation}
	\sqrt{p \cdot \sigma}=\left[\begin{array}{cc}
		\sqrt{E-p_{z}} & 0 \\
		0 & \sqrt{E+p_{z}}
	\end{array}\right], \quad 
	\sqrt{p \cdot \bar{\sigma}}=\left[\begin{array}{cc}
		\sqrt{E+p_{z}} & 0 \\
		0 & \sqrt{E-p_{z}}
	\end{array}\right],
\end{equation}
we can write more generally
\begin{equation}
	u_{s}(p) = \left(\begin{array}{c}
		\sqrt{p \cdot \sigma} \xi_{s} \\
		\sqrt{p \cdot \bar{\sigma}} \xi_{s}
	\end{array}\right), \quad 
	v_{s}(p) = \left(\begin{array}{c}
		\sqrt{p \cdot \sigma} \eta_{s} \\
		-\sqrt{p \cdot \bar{\sigma}} \eta_{s}
	\end{array}\right),
\end{equation}
where the square root of a matrix can be defined by changing to the diagonal basis, taking the square root of the eigenvalues, then changing back to the original basis. 
In practice, we will usually pick $p$ along the $z$ axis, so we do not need to know how to make sense of $\sqrt{p \cdot \sigma}$. Then the four solutions are
\begin{equation}\label{eq:qft-dirac-solutions}
\begin{aligned}
	u^{1}(p) &= \left(\begin{array}{c}
		\sqrt{E-p_{z}} \\ 0 \\
		\sqrt{E+p_{z}} \\ 0
	\end{array}\right), & 
	u^{2}(p) &= \left(\begin{array}{c}
		0 \\ \sqrt{E-p_{z}} \\
		0 \\ \sqrt{E+p_{z}}
	\end{array}\right), \\
	v^{1}(p) &= \left(\begin{array}{c}
		\sqrt{E-p_{z}} \\ 0 \\
		-\sqrt{E+p_{z}} \\ 0
	\end{array}\right), & 
	v^{2}(p) &= \left(\begin{array}{c}
		0 \\ \sqrt{E-p_{z}} \\
		0 \\ -\sqrt{E+p_{z}}
	\end{array}\right).
\end{aligned}
\end{equation}
In any frame $u^{s}$ are the positive frequency electrons, and the $v^{s}$ are negative frequency electrons, or equivalently, positive frequency positrons.

For massless spinors, $p_{z}=\pm E$ and the explicit solutions in Eq. (\ref{eq:qft-dirac-solutions}) are 4-vectors with one non-zero component describing spinors with fixed helicity. 
The spinor solutions for massless electrons are sometimes called polarizations, and are useful for computing polarized electron scattering amplitudes.

For Weyl spinors, there are only four real degrees of freedom off-shell and two real degrees of freedom on-shell. 
Recalling that the Dirac equation splits up into separate equations for $\psi_{L}$ and $\psi_{R}$, the Dirac spinors with zeros in the bottom two rows will be $\psi_{L}$ and those with zeros in the top two rows will be $\psi_{R}$. 
Since $\psi_{L}$ and $\psi_{R}$ have two degrees of freedom each, these must be particle and antiparticle for the same helicity. 
The embedding of Weyl spinors into fields this way induces irreducible unitary representations of the Poincare group for $m=0$.

\subsubsection{Normalization and Spin Sum}
The normalization chosen this way gives the orthogonal relation:
\begin{equation}\label{eq:qft-dirac-otho-1}
\begin{aligned}
	\bar{u}^{r}(p) u^{s}(p) &= +2 m \delta^{r s}, \\
	\bar{v}^{r}(p) v^{s}(p) &= -2 m \delta^{r s}.
\end{aligned}
\end{equation} 
This is the (conventional) normalization for the spinor inner product for massive Dirac spinors. 
It is also easy to check that
\begin{equation}
	\bar u_s(p) v_{s'}(p) = \bar v_s(p) u_{s'}(p) = 0.
\end{equation}
We can further check that an additional orthogonal relation hold
\begin{equation}
\begin{aligned}
	u^{r \dagger}(p) u^{s}(p) &= -2 \omega_{\bm p} \delta^{r s}, \\
	v^{r \dagger}(p) v^{s}(p) &= +2 \omega_{\bm p} \delta^{r s}.
\end{aligned}
\end{equation}
And if we define $\bar p \equiv (E,-\vec p)$, there is another set of orthogonal relation:
\begin{equation}
	u^{r\dagger}(p) v^{s}(\bar p) = 
	v^{r\dagger}(p) u^{s}(\bar p) =0.
\end{equation}
A useful identity is the spin sum identity:
\begin{equation}
\begin{aligned}
	\sum_{s} u^{s}(p) \bar{u}^{s}(p) &= \cancel p+m, \\
	\sum_{s} v^{s}(p) \bar{v}^{s}(p) &= \cancel p-m.
\end{aligned}
\end{equation}


\subsubsection{Field Expansion and Correlation}
The Dirac field expansion is
\begin{equation}
\begin{aligned}
	\psi(x) &=\int \frac{d^{3} p}{(2 \pi)^{3}} \frac{1}{\sqrt{2 \omega_{\mathbf{p}}}} 
		\sum_{s}\left(a_{\mathbf{p}}^{s} u^{s}(p) e^{-i p \cdot x}
		+b_{\mathbf{p}}^{s \dagger} v^{s}(p) e^{i p \cdot x}\right), \\
	\bar{\psi}(x) &=\int \frac{d^{3} p}{(2 \pi)^{3}} \frac{1}{\sqrt{2 \omega_{\mathbf{p}}}} 
		\sum_{s}\left(b_{\mathbf{p}}^{s} \bar{v}^{s}(p) e^{-i p \cdot x}
		+a_{\mathbf{p}}^{s \dagger} \bar{u}^{s}(p) e^{i p \cdot x}\right).
\end{aligned}
\end{equation}
Now let us investigate the propagator
\begin{equation}
\begin{aligned}
	iD_{F,\alpha\beta}(x_1-x_2) &= \langle0|T\psi_\alpha(x_1)\bar\psi_\beta(x_2)|0\rangle \\
	&= \theta(\tau) \langle0|\psi_\alpha(x_1)\bar\psi_\beta(x_2)|0\rangle - \theta(-\tau) \langle0|\bar\psi_\beta(x_2)\psi_\alpha(x_1)|0\rangle.
\end{aligned}
\end{equation}
On the RHS, the first term is
\begin{equation*}
\begin{aligned}
	\langle0|\psi_\alpha(x_1)\bar\psi_\beta(x_2)|0\rangle 
	&= \int \frac{d^{3} p}{(2 \pi)^{3}} \frac{1}{\sqrt{2 \omega_{\mathbf{p}}}} \left[\sum_s u_\alpha^s(p)\bar u_\beta^s(p)\right]e^{-i p\cdot (x_1-x_2)} \\
	&= (i\cancel \partial+m)_{\alpha\beta}\int \frac{d^{3} p}{(2 \pi)^{3}} \frac{1}{\sqrt{2 \omega_{\mathbf{p}}}} e^{-i p\cdot (x_1-x_2)}.
\end{aligned}
\end{equation*}
For the second term:
\begin{equation*}
\begin{aligned}
	\langle0|\bar\psi_\beta(x_2)\psi_\alpha(x_1)|0\rangle
	&= \int \frac{d^{3} p}{(2 \pi)^{3}} \frac{1}{\sqrt{2 \omega_{\mathbf{p}}}} \left[\sum_s \bar v_\beta^s(p)v_\alpha^s(p)\right]e^{i p\cdot (x_1-x_2)} \\
	&= -(i\cancel \partial + m)_{\alpha\beta}\int \frac{d^{3} p}{(2 \pi)^{3}} \frac{1}{\sqrt{2 \omega_{\mathbf{p}}}} e^{i p\cdot (x_1-x_2)}.
\end{aligned}
\end{equation*}
Together, the Dirac propagator is:
\begin{equation}
\begin{aligned}
	iD_F(x_1-x_2) &= (i\cancel \partial+m)i\Delta(x_1-x_2) \\
	&= \int\frac{d^{4} p}{(2\pi)^{4}} e^{-i p\cdot (x_1-x_2)}\frac{i(\cancel p+m)}{p^2-m^2+i\epsilon}.
\end{aligned}
\end{equation}




\section{Path-integral Quantization}

\subsection{Scalar Field}
Consider the action for free field with source
\begin{equation}
	S_0[\phi,J]
	= \int d^dx\left[\mathcal{L}_0 + J(x)\cdot\phi(x) \right].
\end{equation}
In the path integral formalism, we consider the partition function 
\begin{equation}
	Z_0[J] = \int D[\phi] \exp(iS_0[\phi,J])
	\equiv Z[0] \exp(iW_0[J]).
\end{equation}
where we have introduced a new quantity
\begin{equation}
\begin{aligned}
	W_0[J] = -\frac{1}{2}\int d^dx_1 d^dx_2 J(x_1)\Delta_0(x_1-x_2)J(x_2).
\end{aligned}
\end{equation}
For free field, the free propagator $\Delta_0(x_1-x_2)$ is:
\begin{equation}
	i\Delta_0(x_1-x_2) = \langle 0| T\phi(x_1)\phi(x_2)|0\rangle
	= \frac{\delta}{i\delta J(x_1)}\frac{\delta}{i\delta J(x_2)} iW_0[J].
\end{equation}
Now we evaluate the propagator in the path-integral formalism.
In momentum space, the free action (with source) is 
\begin{equation*}
	\frac{1}{V}\sum_k \left[\frac{1}{2}\tilde\phi^*(k)( k^2-m^2)\tilde\phi(k)+\tilde J^*(k)\cdot\tilde\phi(k)+\tilde\phi^*(k)\cdot\tilde J(k)\right].
\end{equation*}
For real field, $\tilde\phi^*(k) = \tilde\phi(-k)$.
For our convenience, we have expressed the momentum integral as summation.
Actually, consider the $d$-dimensional box of size $L^d$, the momentum along each axis is multiple of $2\pi/L$, so when $L\rightarrow \infty$, the summation approaches in integral,
\begin{equation*}
	\frac{1}{V}\sum_k \rightarrow \int \frac{d^d k}{(2\pi)^d}.
\end{equation*}
Let us omit the $1/V$ factor, the summation can be formally expressed as
\begin{equation}
	\frac{1}{4}\mathbf{v}^T \cdot \mathbf M\cdot \mathbf{v} + \frac{1}{2}\mathbf{j}^T \cdot \mathbf{v}
\end{equation}
where
\begin{equation*}
	\mathbf v = \bigoplus_{|\mathbf k|} \left[
	\begin{array}{c}
		\tilde{\phi}(k) \\ 
		\tilde{\phi}^*(k) 
	\end{array}\right],\ 
	\mathbf M = \bigoplus_{|\mathbf k|} \left[
	\begin{array}{cc} 
		0 & k^2-m^2 \\ 
		k^2-m^2 & 0 
	\end{array}\right],\ 
	\mathbf j = \bigoplus_{|\mathbf k|} \left[
	\begin{array}{c}
		\tilde{J}^*(k) \\ 
		\tilde{J}(k) 
	\end{array}\right].
\end{equation*}
Note that in the above expression, we have made an infinitesimal shift of mass ($m^2 \rightarrow m^2 - i\epsilon$) to ensure the convergence of the Gaussian integral.
The integrated variables $v_i$ is not real.
To use the real Gaussian integral formula, we make use of a unitary transformation: 
\begin{equation*}
	\mathbf U = \frac{1}{\sqrt 2} \left[\begin{array}{cc}
		1 & 1 \\
		-i & i
	\end{array}\right], \quad
	\mathbf U \cdot \left[
	\begin{array}{c}
		\tilde{\phi}(k) \\ 
		\tilde{\phi}^*(k) 
	\end{array}\right] 
	= \frac{1}{\sqrt 2}\left[
	\begin{array}{c}
		\tilde\phi(k)+\tilde\phi^*(k) \\ 
		-i\tilde\phi(k)+i\tilde\phi^*(k)
	\end{array}\right]
	\equiv \left[
	\begin{array}{c}
		\tilde\phi_1(k) \\ 
		\tilde\phi_2(k) 
	\end{array}\right]
\end{equation*}
The path integral then becomes a real field integral.
Recall the real Gaussian integral formula:
\begin{equation}
	\int d\mathbf x \exp\left(-\frac{1}{2}\mathbf{x}^T \cdot \mathbf A \cdot \mathbf{x} + \mathbf{B}^T \cdot \mathbf{x}\right) 
	= \sqrt{\frac{(2\pi)^N}{\det{\mathbf A}}}\exp\left(\frac{1}{2}\mathbf{B}^T \cdot \mathbf{A}^{-1} \cdot \mathbf{B}\right),
	\label{eq:real-gaussian-integral}
\end{equation}
For the field integral, we absorbed the $(2\pi)^{N/2}$ term into the measure, and express the path integral for the Gaussian field as:
\begin{equation}
	W_0[J] 
	= -\frac{i}{4}\int \frac{d^d k}{(2\pi)^d} \mathbf j^T_k \cdot \mathbf M^{-1}_k \cdot \mathbf j_k
	= -\frac{1}{2} \int \frac{d^d k}{(2\pi)^d}  \tilde{J}^*(k) \tilde{\Delta}_0(k) \tilde{J}(k).
\end{equation}
This gives the propagator in the momentum space:
\begin{equation}
	\tilde{\Delta}_0(k) = \frac{i}{k^2-m^2}
	\quad \Longrightarrow \quad 
	\Delta_0(x_1-x_2) = i\int\frac{d^{4} k}{(2\pi)^{4}} \frac{e^{-i k\cdot (x_1-x_2)}}{k^2-m^2}.
\end{equation}


\subsubsection{From Field to Force}
Consider two separate particle described by the delta function $J_a(x) = \delta^{(3)}(\bm x - \bm x_a)$, together the source is
\begin{equation}
	J(x) = J_1(x) + J_2(x).
\end{equation}
Adding the source,
\begin{equation*}
	W_0[J] = -\frac{1}{2}\int d^4x_1 d^4 x_2 J(x_1) \Delta_0(x_1-x_2) J(x_2)
\end{equation*}
Omit the self energy terms $J_1^2(x), J_2^2(x)$, $W_0[J]$ is
\begin{equation}
\begin{aligned}
	W_0[J] &= -\int d^4 y_1 d^4 y_2\ e^{-ik^0(y_1^0-y_2^0)}\int \frac{d^4 k}{(2\pi)^4} J_1(y_1)\frac{e^{i\bm k\cdot (\bm y_1-\bm y_2)}}{k^2-m^2} J_2(y_2) \\
	&= -\int  dt \int d (y_1^0 - y_2^0) \ e^{-ik^0(y_1^0-y_2^0)}\int \frac{d^4 k}{(2\pi)^4} \frac{e^{i\bm k\cdot (\bm y_1-\bm y_2)}}{k^2-m^2} \\
	&= \left(\int dt \right)\int \frac{d^3 k}{(2\pi)^3} \frac{e^{i\bm k\cdot (\bm y_1-\bm y_2)}}{\bm k^2 + m^2}
\end{aligned}
\end{equation}
Recall that the partition function is actually infinite:
\begin{equation}
	Z_0 \sim \langle 0| e^{-i H_0 T} |0\rangle \quad \Longrightarrow \quad
	W_0 = -i E T,
\end{equation}
where $E$ is the energy.
Writing $\bm r \equiv \bm y_1 - \bm y_2$, and $u \equiv \cos\theta$ with $\theta$ the angle between $\bm k$ and $\bm r$, the volume form is $dk \cdot kd\theta \cdot  2\pi k \sin \theta = 2\pi k^2 dk du$, and the integral is
\begin{equation}
\begin{aligned}
	E &= -\int \frac{d^3 k}{(2\pi)^3} \frac{e^{i k r u}}{k^2 + m^2} \\
	&= - \frac{1}{(2\pi)^2} \int_0^\infty k^2 dk \int_{-1}^1 du \frac{e^{ikru}}{k^2 +m^2} \\
	&= -\frac{1}{2\pi^2 r} \int_0^\infty k  \frac{\sin kr}{k^2 +m^2} dk.
\end{aligned}
\end{equation}
Since the integral is even, we can extend the integral to
\begin{equation}
\begin{aligned}
	E &= -\frac{1}{4\pi^2 r} \int_{-\infty}^\infty k  \frac{\sin kr}{k^2 +m^2} dk \\
	&= \frac{i}{4\pi^2 r} \int_{-\infty}^\infty \frac{k e^{ikr}}{k^2 +m^2} dk
\end{aligned}
\end{equation}
The residue theorem gives
\begin{equation}
	\int_{-\infty}^\infty \frac{k e^{ikr}}{k^2 +m^2} dk = \pi ie^{-mr}
\end{equation}
So we get the potential of two particles:
\begin{equation}\label{eq:field-to-force}
	V(r) = -\frac{e^{-mr}}{4\pi r},
\end{equation}
and the attractive force is
\begin{equation}
	F(r) = -\frac{dV}{dr} = -(1+mr)\frac{e^{-mr}}{4\pi r^2}.
\end{equation}
We see that in the massless case, the force gives the long-range Coulomb force $F \propto 1/r^2$, while in the massful field theory, the force is short-ranged, with the decay length proportional to the mass.


\subsection{Vector Field}
We define the gauge fixing function
\begin{equation*}
	G(A) = \partial_\mu A^\mu(x) -\omega(x) = 0
\end{equation*}
The gauge transformation has the form:
\begin{equation*}
	A^\alpha_\mu(x) = A_\mu(x) + \partial_\mu \alpha(x).
\end{equation*}
We then have
\begin{equation*}
	1 \propto \int D[\alpha] \det\left(\frac{\delta G(A^\alpha)}{\delta \alpha}\right) \delta(G(A)).
\end{equation*}
Inset the identity operator into the path integral formula
\begin{equation*}
	Z[J] \propto \det\left(\partial^2 \right) \int D[\alpha]D[A] e^{iS[A,J]} \delta(\partial_\mu A^\mu -\omega(x)).
\end{equation*}
The above equation does not depend on $\omega(x)$.
We can then integrate over $\omega(x)$ with gaussian weight
\begin{equation*}
\begin{aligned}
	Z[J] &\propto \int D[\omega] e^{-i\int d^d x \frac{\omega^2}{2\xi}} \int D[\alpha]D[A] e^{iS[A,J]}
	\delta(\partial_\mu A^\mu-\omega) \\
	&= \int D[A] e^{iS[A,J]} \exp\left\{i \left[S[A,J]-\int d^d x \frac{1}{2\xi}(\partial_\mu A^\mu)^2 \right]\right\}.
\end{aligned}
\end{equation*}
In momentum space, the modified Langriangian is 
\begin{equation*}
	\tilde{\mathcal{L}}_\xi(k) = \tilde{A}^\mu(k)\left[
		-k^2 g_{\mu\nu}+\left(1-\frac{1}{\xi}\right)k_\mu k_\nu
		\right] \tilde{A}^\nu(-k) +
		\tilde{J}_\mu(k) \tilde{A}^\mu(-k) +
		\tilde{A}^\mu(k) \tilde{J}_\mu(-k).
\end{equation*}
In the momentum space, the photon propagator is
\begin{equation}\label{eq:qft-photon-momentum-propagator}
\begin{aligned}
	\tilde G^{\mu\nu}(k) 
	&= \left[-k^2 g_{\mu\nu}+\left(1-\frac{1}{\xi}\right)k_\mu k_\nu\right]^{-1} \\
	&= \frac{-g^{\mu\nu}+(1-\xi)k^\mu k^\nu}{k^2}.
\end{aligned}
\end{equation}
Thus, the partition function is
\begin{equation}
	\frac{Z_{\mathrm{maxwell}}[J]}{Z_{\mathrm{maxwell}}[0]}
	= \exp\left[-\frac{i}{2}\int d^dx_1 d^dx_2 J_\mu(x_1) G^{\mu\nu}(x_1-x_2) J_\nu(x_2) \right],
\end{equation}
where the real-space propagator is
\begin{equation}
	G^{\mu\nu}(x_1-x_2) = \int \frac{d^d k}{(2\pi)^d} e^{-ik\cdot(x_1-x_2)}\frac{-g^{\mu\nu}+(1-\xi)k^\mu k^\nu}{k^2}.
\end{equation}
Note that the propagator is related to the two-point correaltion:
\begin{equation}
\begin{aligned}
	\langle 0|T A^\mu(x_1) A^\nu(x_2) |0\rangle
	&= \left.\frac{1}{Z_{\mathrm{Maxwell}}[0]}\frac{\delta}{iJ_\mu(x_1)}\frac{\delta}{iJ_\nu(x_2)} Z_{\mathrm{Maxwell}}[J]\right|_{J=0} \\
	&= iG^{\mu\nu}(x_1-x_2).
\end{aligned}
\end{equation}


\subsection{Spinor Field}
Consider the partition function with source
\begin{equation}
	Z_{\mathrm{Dirac}}[J]
	= \int D[\bar\psi,\psi] \exp\left[i\int d^dx \left(\mathcal{L}_{\mathrm{Dirac}}+\bar{\eta}\psi + \bar\psi\eta \right) \right].
\end{equation}
In momentum space:
\begin{equation}
	S = \int\frac{d^d k}{(2\pi)^d} \left[
		\tilde{\bar\psi}(k)(\cancel{k}-m)\tilde{\psi}(k) +
		\tilde{\bar\eta}(k) \tilde{\psi}(k) +
		\tilde{\bar\psi}(k) \tilde{\eta}(k)
	\right].
\end{equation}
Using the Gaussian integral formula (for Grassman variables), the partition function is:
\begin{equation}
\begin{aligned}
	\frac{Z_{\mathrm{Dirac}}[J]}{Z_{\mathrm{Dirac}}[0]}
	&= \exp\left[-i\int \frac{d^d k}{(2\pi)^d} \tilde{\bar\eta}(k)\frac{1}{\cancel{k}-m}\tilde\eta(k)\right] \\
	&= \exp\left[-i\int d^dx_1 d^d x_2 \bar{\eta}(x_1)\cdot D_F(x_1-x_2)\cdot \eta(x_2) \right]
\end{aligned}
\end{equation}

where
\begin{equation}
	D_F(x_1-x_2) = \int \frac{d^d k}{(2\pi)^d} \frac{e^{-i k \cdot (x_1-x_2)}}{\cancel{k}-m}
	= \int \frac{d^d k}{(2\pi)^d} \frac{\cancel{k}+m}{k^2-m^2} e^{-i k \cdot (x_1-x_2)}.
\end{equation}
Note that the propagator is
\begin{equation}
\begin{aligned}
	\langle 0| T \psi^\alpha(x_1) \bar\psi^\beta(x_2) |0\rangle
	&= \left.\frac{1}{Z_{\mathrm{Dirac}}[0]}\frac{\delta}{i\delta \bar{\eta}_\alpha(x_1))}\frac{i\delta}{\delta\eta_\beta(x_2)} Z_{\mathrm{Dirac}}[\bar\eta,\eta]\right|_{\eta=\bar\eta=0} \\
	&= i D^{\alpha\beta}_F(x_1-x_2),
\end{aligned}
\end{equation}
where the sign in the variational derivative comes from the anti-commutation relation of the fermionic fields.




