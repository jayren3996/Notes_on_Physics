\chapter{Scattering Theory}

\section{Canonical Quantization of Free Fields}
\subsection{Klein-Gordon Field}
For the real Klein-Gordon field
\begin{equation*}
	\mathcal L_{\mathrm{K-G}} = -\frac{1}{2}\phi (\partial^2+m^2) \phi,
\end{equation*}
the classical equation of motion is 
\begin{equation}
	(-\partial_t^2+\nabla^2-m^2)\phi(\vec x,t) = 0. 
	\label{eq:rkg-eom}
\end{equation}
The solution to Eq.~(\ref{eq:rkg-eom}) is proportional to the plane wave:
\begin{equation*}
	\phi(\vec x, t) \propto e^{-i\omega_{\bm{k}}t+i\vec{p}\cdot\vec{x}} + e^{i\omega_{\vec{k}}t-i\bm{p}\cdot\vec{x}},
\end{equation*}
where the energy is $\omega_{\bm{k}}=\bm{k}^2+m^2$ and $\vec k$ is the momentum as the conserved quantity.
The general solution to the EOM is
\begin{equation}
	\phi(\vec x,t) \propto \int \frac{d^{3} k}{(2\pi)^{3}} \left(
		a_{k}e^{-i\omega_{\bm{k}}t+i\vec{k}\cdot\vec{x}} + 
		a^*_{k}e^{i\omega_{\bm{k}}t-i\vec{k}\cdot\vec{x}} 
	\right).
\end{equation}
The canonical quantization promote the coefficient $a_{k}/a_{k}^*$ to the particle annihilation/creation operator $a_{k}/a_{k}^\dagger$, with the commutation relation
\begin{equation}
	[a_{k}, a_{p}^\dagger] = (2\pi)^{3} \delta^{3}(\vec{k}-\vec{p}).
\end{equation}

The single-particle state with momentum $\vec k$ is created by $a_{k}^{\dagger}$ operators acting on the vacuum:
\begin{equation}
	|\vec{k}\rangle \equiv \sqrt{2\omega_{\bm k}} a_{k}^{\dagger}|0\rangle,
	\label{eq:rel-single-particle}
\end{equation}
where $|\vec{k}\rangle$ is a state with a single particle of momentum $\vec{k}$.

\begin{framedrmk}[Lorentz Invariance of Single-particle State]
The factor of $\sqrt{2 \omega_{\bm k}}$ in Eq.~(\ref{eq:rel-single-particle}) is just a convention, but it will make some calculations easier. 
To compute the normalization of one-particle states, we start with
\begin{equation}
	\langle 0|0\rangle=1,
\end{equation}
which leads to
\begin{equation}
	\langle\vec{p}|\vec{k}\rangle 
	= 2\sqrt{\omega_{\bm p} \omega_{\bm k}}\left\langle 0\left|a_{p} a_{k}^{\dagger}\right| 0\right\rangle
	= 2 \omega_{\bm p}(2\pi)^{3} \delta^{3}(\vec{p}-\vec{k}).
\end{equation}
The identity operator for one-particle states is
\begin{equation}
	1=\int \frac{d^{3} p}{(2\pi)^{3}} \frac{1}{2\omega_{\bm p}}|\vec{p}\rangle\langle\vec{p}|, \label{eq:rel-identity}
\end{equation}
which we can check with
\begin{equation*}
	|\vec{k}\rangle
	=\int \frac{d^{3} p}{(2\pi)^{3}} \frac{1}{2\omega_{\bm p}}|\vec{p}\rangle\langle\vec{p}|\vec{k}\rangle
	=\int \frac{d^{3} p}{(2\pi)^{3}} \frac{1}{2\omega_{\bm p}} 2\omega_{\bm p}(2\pi)^3 \delta^3(\vec{p}-\vec{k})|\vec{p}\rangle
	=|\vec{k}\rangle.
\end{equation*}
The identity operator Eq.~(\ref{eq:rel-identity}) is Lorentz invariant since it can be expressed as
\begin{equation}
	1 = \int \frac{d^{3} p d\omega}{(2\pi)^{4}} 2\pi\delta(\omega^2-{\bm{p}}^2-m^2) |\vec p\rangle\langle \vec p|.
\end{equation}
\end{framedrmk}

We fix the normalization by requiring 
\begin{equation*}
	\langle \vec k|\phi(\vec x,0)|0\rangle = e^{-i \vec k\cdot \vec x},
\end{equation*}
and the quantized field operator is
\begin{equation}
	\phi(\vec{x}, t)
	=\int \frac{d^{3} k}{(2\pi)^{3}} \frac{1}{\sqrt{2\omega_{\bm k}}}\left(a_k 
	e^{-i k \cdot x}+a_k^{\dagger} e^{i k \cdot x}\right).
\end{equation}

Consider the two-point correlation:
\begin{equation*}
\begin{aligned}
	i\Delta(x_1-x_2) &= \langle 0|T \phi(x_1) \phi(x_2) |0\rangle \\
	&= \theta(t_1-t_2) \langle 0|\phi(x_1) \phi(x_2) |0\rangle 
	+ \theta(t_2-t_1) \langle 0|\phi(x_2) \phi(x_1) |0\rangle.
\end{aligned}
\end{equation*}
Note that
\begin{equation*}
	\langle 0|\phi(x_1) \phi(x_2) |0\rangle
	= \int\frac{d^{3} k}{(2\pi)^{3}}\frac{1}{2\omega_k} e^{i\vec k\cdot (\vec x_1-\vec x_2)-i\omega_{\vec k}\tau},
\end{equation*}
where $\tau =t_1-t_2$.
The propagator can be written as
\begin{equation*}
\begin{aligned}
	i\Delta(x_1-x_2) 
	&= \int\frac{d^{3} k}{(2\pi)^{3}}\frac{1}{2\omega_k} e^{i\vec k\cdot (\vec x_1-\vec x_2)}\left[e^{-i\omega_{\vec k}\tau}\theta(\tau)+e^{i\omega_{\vec k}\tau}\theta(-\tau)\right] \\
	&= \int\frac{d^{3} k}{(2\pi)^{3}} e^{i\vec k\cdot (\vec x_1-\vec x_2)}\int \frac{d\omega}{2\pi i}\frac{-e^{i\omega\tau}}{\omega^2-\omega_k^2+i\epsilon} \\
	&= \int\frac{d^{4} k}{(2\pi)^{4}} e^{-i k\cdot (x_1-x_2)}\frac{i}{k^2-m^2+i\epsilon}.
\end{aligned}
\end{equation*}



\subsection{Maxwell Field}
For Maxwell field
\begin{equation}
	\tilde{\mathcal L}_{\xi}(k) = \tilde{A}^\mu(k)\left(-k^2 g_{\mu\nu}+k_\mu k_\nu\right) \tilde{A}^\nu(-k).
\end{equation}
The EOM in momentum space is
\begin{equation*}
	(-k^2 g_{\mu\nu}+k_\mu k_\nu) \tilde{A}^\nu(k) = 0.
\end{equation*}
Since the linear operator $(-k^2 g_{\mu\nu}+k_\mu k_\nu)$ is singular, i.e.,
\begin{equation*}
	(-k^2 g_{\mu\nu}+k_\mu k_\nu)k^\nu = 0.
\end{equation*}
The gauge freedom can be used to further restrict
\begin{equation*}
	A^0 = 0.
\end{equation*}
In this way, there are only two independent polarization for EOM solution
\begin{equation}
	A^\mu = e^{-ik\cdot x} \epsilon^\mu_j,\ j=1,2,
\end{equation}
where
\begin{equation*}
	\epsilon_1 = (0,1,0,0),\
	\epsilon_2 = (0,0,1,0).
\end{equation*}
The field expansion is then
\begin{equation}
	A^\mu = \int \frac{d^{3} k}{(2\pi)^{3}}\frac{1}{\sqrt{2\omega_k}}
	\sum_{j=1}^2 \left(\epsilon^\mu_j a_{k,j} e^{-ik\cdot x} + 
	\epsilon^{\mu*}_j a^\dagger_{k,j} e^{ik\cdot x}\right).
\end{equation}
A single-particle state with polarization vector $\epsilon_j$ is defined as
\begin{equation}
	|k,\epsilon_j\rangle = \sqrt{2\omega_k}	\vec\epsilon_j a^\dagger_{k,j}|0\rangle.
\end{equation}
Note that then the field is off shell (internal photon line), the photon can be space-like or time-like, and then there are an additional polarization.
In general, 
\begin{equation*}
	\sum_{j=1}^3 \epsilon^{\mu*}_j \epsilon^\nu_j = -(1 - P_{k}) = -(g^{\mu\nu}-k^\mu k^\nu),
\end{equation*}
where $P_k$ is the projection to 4-momentum $k$.
The propagator is then
\begin{equation}
	i\Pi(x_1-x_2)= \int \frac{d^4 k}{(2\pi)^4} e^{-ik\cdot(x_1-x_2)}\frac{-i(g^{\mu\nu}-k^\mu k^\nu)}{k^2+i\epsilon}.
\end{equation}


\subsection{Dirac Field}

For Dirac field
\begin{equation*}
	\tilde{\mathcal L}(p) = \tilde{\bar \psi}(p)(\cancel{p} - m)\tilde\psi(p),
\end{equation*}
The EOM is
\begin{equation}
	(\cancel p -m)\tilde\psi(p) = 0
\end{equation}

\begin{framedrmk}[Solution to Dirac Function]
The general solution of the Dirac equation can be written as a linear combination of plane waves. 
The positive frequency waves are of the form
\begin{equation*}
	\psi(x)=u(p) e^{-i p \cdot x}, \quad p^{2}=m^{2}, \quad p^{0}>0
\end{equation*}
There are two linearly independent solutions for $u(p)$,
\begin{equation*}
	u^{s}(p)=\left(\begin{array}{c}
	\sqrt{p \cdot \sigma} \xi^{s} \\
	\sqrt{p \cdot \bar{\sigma}} \xi^{s}
	\end{array}\right), \quad s=1,2
\end{equation*}
which we normalize according to
\begin{equation*}
	\bar{u}^{r}(p) u^{s}(p)=2 m \delta^{r s} \quad \text { or } \quad u^{r \dagger}(p) u^{s}(p)=2 \omega_{\bm p} \delta^{r s}
\end{equation*}
In exactly the same way, we can find the negative-frequency solutions:
\begin{equation*}
	\psi(x)=v(p) e^{+i p \cdot x}, \quad p^{2}=m^{2}, \quad p^{0}>0 \text {. (3.61) }
\end{equation*}
Note that we have chosen to put the $+$ sign into the exponential, rather than having $p^{0}<0$.
There are two linearly independent solutions for $v(p)$,
where $\eta^{s}$ is another basis of two-component spinors. These solutions are normalized according to
\begin{equation*}
	\bar{v}^{r}(p) v^{s}(p)=-2 m \delta^{r s} \quad \text { or } \quad v^{r \dagger}(p) v^{s}(p)=+2 \omega_{\bm{p}} \delta^{r s}
\end{equation*}
The $u$'s and $v$'s are also orthogonal to each other:
\begin{equation*}
	\bar{u}^{r}(p) v^{s}(p)=\bar{v}^{r}(p) u^{s}(p)=0.
\end{equation*}
A useful identity is
\begin{equation*}
\begin{aligned}
	\sum_{s} u^{s}(p) \bar{u}^{s}(p) &= \cancel p+m, \\
	\sum_{s} v^{s}(p) \bar{v}^{s}(p) &= \cancel p-m.
\end{aligned}
\end{equation*}
\end{framedrmk}

The Dirac field expansion is
\begin{equation}
\begin{aligned}
	\psi(x) &=\int \frac{d^{3} p}{(2 \pi)^{3}} \frac{1}{\sqrt{2 \omega_{\mathbf{p}}}} 
		\sum_{s}\left(a_{\mathbf{p}}^{s} u^{s}(p) e^{-i p \cdot x}
		+b_{\mathbf{p}}^{s \dagger} v^{s}(p) e^{i p \cdot x}\right), \\
	\bar{\psi}(x) &=\int \frac{d^{3} p}{(2 \pi)^{3}} \frac{1}{\sqrt{2 \omega_{\mathbf{p}}}} 
		\sum_{s}\left(b_{\mathbf{p}}^{s} \bar{v}^{s}(p) e^{-i p \cdot x}
		+a_{\mathbf{p}}^{s \dagger} \bar{u}^{s}(p) e^{i p \cdot x}\right).
\end{aligned}
\end{equation}
Now let us investigate the propagator
\begin{equation}
\begin{aligned}
	iD_{F,\alpha\beta}(x_1-x_2) &= \langle0|T\psi_\alpha(x_1)\bar\psi_\beta(x_2)|0\rangle \\
	&= \theta(\tau) \langle0|\psi_\alpha(x_1)\bar\psi_\beta(x_2)|0\rangle - \theta(-\tau) \langle0|\bar\psi_\beta(x_2)\psi_\alpha(x_1)|0\rangle.
\end{aligned}
\end{equation}
On the RHS, the first term is
\begin{equation*}
\begin{aligned}
	\langle0|\psi_\alpha(x_1)\bar\psi_\beta(x_2)|0\rangle 
	&= \int \frac{d^{3} p}{(2 \pi)^{3}} \frac{1}{\sqrt{2 \omega_{\mathbf{p}}}} \left[\sum_s u_\alpha^s(p)\bar u_\beta^s(p)\right]e^{-i p\cdot (x_1-x_2)} \\
	&= (i\cancel \partial+m)_{\alpha\beta}\int \frac{d^{3} p}{(2 \pi)^{3}} \frac{1}{\sqrt{2 \omega_{\mathbf{p}}}} e^{-i p\cdot (x_1-x_2)}.
\end{aligned}
\end{equation*}
For the second term:
\begin{equation*}
\begin{aligned}
	\langle0|\bar\psi_\beta(x_2)\psi_\alpha(x_1)|0\rangle
	&= \int \frac{d^{3} p}{(2 \pi)^{3}} \frac{1}{\sqrt{2 \omega_{\mathbf{p}}}} \left[\sum_s \bar v_\beta^s(p)v_\alpha^s(p)\right]e^{i p\cdot (x_1-x_2)} \\
	&= -(i\cancel \partial + m)_{\alpha\beta}\int \frac{d^{3} p}{(2 \pi)^{3}} \frac{1}{\sqrt{2 \omega_{\mathbf{p}}}} e^{i p\cdot (x_1-x_2)}.
\end{aligned}
\end{equation*}
Together, the Dirac propagator is:
\begin{equation*}
\begin{aligned}
	iD_F(x_1-x_2) &= (i\cancel \partial+m)i\Delta(x_1-x_2) \\
	&= \int\frac{d^{4} p}{(2\pi)^{4}} e^{-i p\cdot (x_1-x_2)}\frac{i(\cancel p+m)}{p^2-m^2+i\epsilon}.
\end{aligned}
\end{equation*}



\section{Interaction and Scattering}

\subsection{Lehmann Representation}
The interacting Hamiltonian do not conserve particle number, and the ground state $|\Omega\rangle$ is no longer the vacuum $|0\rangle$.

Consider the Green's function
\begin{equation*}
	iG(x_1-x_2) = \begin{cases}
		\langle\Omega|T\phi(x_1)\phi(x_2)|\Omega\rangle & \mathrm{Klein-Gordon} \\
		\langle\Omega|T A(x_1)A(x_2)|\Omega\rangle & \mathrm{Maxwell} \\
		\langle\Omega|T\psi(x_1)\bar\psi(x_2)|\Omega\rangle & \mathrm{Dirac}
	\end{cases}.
\end{equation*}
We can insert a complete basis 
\begin{equation*}
	1 = |\Omega\rangle\langle\Omega| + \sum_\lambda\int\frac{d^3 k}{(2\pi)^3}\frac{1}{2\omega_k}|\lambda_{\vec k}\rangle \langle\lambda_{\vec k}|
\end{equation*}
into the correlation function,\footnote{Here we assume $\langle\Omega|\phi(x)|\Omega\rangle=0$ unless there is spontaneously symmetry breaking happening.} the Green's function takes the form (take K-G field as the example):
\begin{equation*}
	iG(x_1-x_2) = \sum_\lambda \int\frac{d^3 k}{(2\pi)^3}
	\left[\theta(t_1-t_2)\langle\Omega|\phi(x_1)|\lambda_{\vec k}\rangle\langle\lambda_{\vec k}|\phi(x_2)|\Omega\rangle + (t_1\leftrightarrow t_2, x_1 \leftrightarrow x_2)\right].
\end{equation*}
Note that
\begin{equation*}
	\langle\lambda_{\vec k}|\phi(x)|\Omega\rangle 
	= \langle\lambda_{\vec k}|e^{iP\cdot x}\phi(0) e^{-iP\cdot x}|\Omega\rangle
	= e^{ik\cdot x} \left.\langle\lambda_{0}|\phi(0)|\Omega\rangle\right|_{k^0=\omega_{\vec k}}.
\end{equation*}
Following the same procedure as we do for the free field theory, 
\begin{equation}
	G(x_1-x_2) = \int_0^\infty \frac{dM^2}{2\pi} \rho(M^2) G_0(x_1-x_2;M^2),
\end{equation}
where the spectral function $\rho(M^2)$ is
\begin{equation*}
	\rho(M^2) = \sum_\lambda(2\pi)\delta(M^2-m_\lambda^2)|\langle\Omega|\phi(0)|\lambda_0\rangle|^2.
\end{equation*}
In particle, near the one-particle state the Green's function looks like:
\begin{equation*}
	i\tilde G(k) = \frac{iZ_{\phi}}{k^2-m^2+i\epsilon} + \mathrm{regular\ terms}.
\end{equation*}
If we renormalize the field strength 
\begin{equation*}
	\phi_R(x) = \frac{1}{\sqrt{Z_\phi}}\phi_0(x),
\end{equation*}
the Green's function then has exactly the same form as free theory.
This normalization factor $Z_\phi$ is exactly what we obtained in the loop correction to the propagator.



\subsection{Scattering Amplitude}

Consider the scattering process in the interaction picture,
\begin{equation}
\begin{aligned}
	\langle f| e^{-iHt} |i\rangle &= \langle f| T \exp \left(-i\int dt V_{\mathrm{int}}(t) \right)|i\rangle \\
	&= \langle f| T \exp \left(i\int d^d x \mathcal{L}_{\mathrm{int}}(t) \right)|i\rangle.
\end{aligned}
\end{equation}
The S-matrix is defined as
\begin{equation}
	\langle f|S|i\rangle 
	= \langle f|\mathcal{T} \exp \left(i\int d^d x \mathcal{L}_{\mathrm{int}}(t) \right)|i\rangle 
	= 1 + i \langle f|\mathcal{T}|i\rangle .
\end{equation}
Because of the additional momentum conservation,
\begin{equation}
	\langle f|\mathcal{T}|i\rangle = (2\pi)^d \delta^d\left(\sum p\right) \mathcal M_{fi}.
\end{equation}


\subsection{LSZ for Klein-Gordon Field}
For free theory, the particle annihilation operator is
\begin{equation}
\begin{aligned}
	\sqrt{2\omega_k} a_k &= i \int d^{d-1} x\ e^{ik\cdot x}(-i\omega_k+\partial_t)\phi_0(x), \\
	\sqrt{2\omega_k} a^\dagger_k &= -i \int d^{d-1} x\ e^{-ik\cdot x}(i\omega_k+\partial_t)\phi_0(x).
\end{aligned}
\end{equation}
When interaction is turned on, the field operator $\phi(x)$ is renormalized as
\begin{equation*}
	\phi(x) \sim \sqrt{Z_{\phi}} \phi_{\mathrm{in}}(x) \sim \sqrt{Z_{\phi}} \phi_{\mathrm{out}}(x).
\end{equation*}
In this way, we have
\begin{equation*}
\begin{aligned}
	\sqrt{2\omega_k}(a_{\mathrm{in}}^\dagger - a_{\mathrm{out}}^\dagger)
	&= i Z_\phi^{-1/2} \int dt \partial_t \left(\int d^{d-1}x\ e^{-ikx}(i\omega_k+\partial_t)\phi_0(x)\right) \\
	&= i Z_\phi^{-1/2} \int d^d x e^{-ik\cdot x}(\omega_k^2+\partial_t^2)\phi_0(x) \\
	&= i Z_\phi^{-1/2} \int d^d x e^{-ik\cdot x}\partial_t^2\phi_0(x) + \phi_0(x)(-\nabla^2+m^2)e^{-i k\cdot x} \\
	&= i Z_\phi^{-1/2} \int d^d x e^{-ik\cdot x}(\partial^2+m^2)\phi_0(x)
\end{aligned}
\end{equation*}
The initial and final states are:
\begin{equation}
\begin{aligned}
	|k_1, \cdots, k_m; \mathrm{in}\rangle &= \left[\prod_{j=1}^m \sqrt{2\omega_{k_j}} a^\dagger_{\mathrm{in}}(k_j)\right] |0\rangle, \\
	|p_1, \cdots, p_n, \mathrm{out}\rangle &= \left[\prod_{j=1}^n \sqrt{2\omega_{p_j}}a^\dagger_{\mathrm{out}}(p_j)\right] |0\rangle.
\end{aligned}
\end{equation}
The S-matrix is
\begin{equation*}
\begin{aligned}
	S_{fi} &= \langle p_1, \cdots, p_n;\mathrm{out}| S |k_1, \cdots, k_m; \mathrm{in}\rangle \\
	&= \frac{\langle 0|T 
		\left(\prod \sqrt{2\omega_{p_j}} a_{p_j;\mathrm{out}} \right)
		\int d^dx \exp(i\mathcal{L}_{\mathrm{int}})
		\left(\prod \sqrt{2\omega_{k_j}} a^\dagger_{k_j;\mathrm{in}} \right)|0\rangle}
		{\langle 0|T\int d^dx \exp(i\mathcal{L}_{\mathrm{int}})|0\rangle}
\end{aligned}
\end{equation*}
Since the scattering process correspond to the connected diagram, meaning that the initial and final state has distinct momentum particles.
We are free to make the substitution
\begin{equation*}
	a^\dagger_{\mathrm{in}} \rightarrow (a_{\mathrm{in}}^\dagger - a_{\mathrm{out}}^\dagger),\ 
	a_{\mathrm{out}} \rightarrow -(a_{\mathrm{in}}^\dagger - a_{\mathrm{out}}^\dagger)^\dagger.
\end{equation*}
In this way, the S-matrix is
\begin{equation*}
\begin{aligned}
	& \langle p_1, \cdots, p_n| S |k_1, \cdots, k_m\rangle  \\
	=& \prod_{i=1}^{m}\left[ \int d^dx_i \ e^{ip_i\cdot x_i}\frac{-\partial^2-m_i^2}{i\sqrt{Z_\phi}}\right]
	\prod_{j=m+1}^{m+n}\left[\int d^dx_j \ e^{-ik_j\cdot x_j}\frac{-\partial^2-m_j^2}{i\sqrt{Z_\phi}}\right] \\
	& \times \frac{\langle 0| T \phi_0(x_1)\cdots \phi_0(x_{m+n}) \int d^dx \exp(i\mathcal{L}_{\mathrm{int}})|0\rangle}
		{\langle 0|T\int d^dx \exp(i\mathcal{L}_{\mathrm{int}})|0\rangle} \\
	=& \prod_{i=1}^{m}\left[\frac{p_i^2-m_i^2}{i\sqrt{Z_\phi}}\right]
		\prod_{j=m+1}^{m+n}\left[\frac{k_j^2-m_j^2}{i\sqrt{Z_\phi}}\right]
		\tilde{G}(-p_1,\cdots,-p_n,k_1,\cdots,k_m).
	\label{eq:K-G-LSZ}
\end{aligned}
\end{equation*}
Note that in the second equality, we move the operator $\partial^2$ out of the time-ordering operator, which will actually create contact terms.
The contact terms can be shown to have no contribution to the S-matrix.
Also, the Green function is defined as
\begin{equation*}
\begin{aligned}
	G(x_1,\cdots,x_{m+n}) 
	&\equiv \langle \Omega|T \phi(x_1)\cdots \phi(x_{m+n})|\Omega\rangle \\
	&= \frac{\langle 0| T \phi_0(x_1)\cdots \phi_0(x_{m+n}) \int d^dx \exp(i\mathcal{L}_{\mathrm{int}})|0\rangle}
		{\langle 0|T\int d^dx \exp(i\mathcal{L}_{\mathrm{int}})|0\rangle}.
\end{aligned}
\end{equation*}

The extra factor before the momentum-space Green's function effectively cancel out the external propagator. 
Thus the LSZ formula (\ref{eq:K-G-LSZ}) means that the S-matrix is the amputated Green's function.


\begin{framedrmk}[Contact Terms]
We first consider the time-ordered two-point function:
\begin{equation*}
	\langle 0|T\phi(x_1)\phi(x_2)|0\rangle
	= \theta(t_1-t_2)\langle 0|\phi(x_1)\phi(x_2)|0\rangle -
	\theta(t_2-t_1)\langle 0|\phi(x_2)\phi(x_1)|0\rangle.
\end{equation*}	
Take time derivative on both side:
\begin{equation*}
\begin{aligned}
	\partial_{t_1} \langle 0|T\phi(x_1)\phi(x_2)|0\rangle
	&= \langle 0|T\partial_{t_1}\phi(x_1)\phi(x_2)|0\rangle +
	\delta(t_1-t_2)\langle 0|[\phi(x_1),\phi(x_2)]|0\rangle \\
	&= \langle 0|T\partial_{t_1}\phi(x_1)\phi(x_2)|0\rangle.
\end{aligned}
\end{equation*}
The second equality follows from the fact that $x_1,x_2$ is equal-time.
Take the the time derivative once more:
\begin{equation*}
	\partial^2_{t_1} \langle 0|T\phi(x_1)\phi(x_2)|0\rangle
	= \langle 0|T\partial^2_{t_1}\phi(x_1)\phi(x_2)|0\rangle +
	\delta(t_1-t_2)\langle 0|[\partial_{t_1}\phi(x_1),\phi(x_2)]|0\rangle.
\end{equation*}
The second term on the right hand side is the contact term.
For free theory, $\partial_{t_1}\phi(x_1)$ is the canonical momentum, meaning that
\begin{equation*}
	[\phi(\vec x_1, t),\partial_{t}\phi(\vec x_1,t)] = i \hbar \delta^{3}(\vec x_1-\vec x_2).
\end{equation*}
In general, for n-point correlation,
\begin{equation*}
\begin{aligned}
	 \partial_{t_1}^2 \langle 0|T\phi_{x_1}\cdots\phi_{x_n}|0\rangle 
	=& \langle 0|T\partial_{t_1}^2\phi_{x_1}\cdots\phi_{x_n}|0\rangle \\
	 &-i\hbar \sum_j \delta^4(x_1-x_j)\langle 0|T\phi_{x_2}\cdots\cancel{\phi_{x_j}}\cdots\phi_{x_n}|0\rangle.
\end{aligned}
\end{equation*}
In the LSZ formula, the contact term do not have any singularity.
When the external legs approach to momentum shell, these regular terms vanishes, so the contact will not contribute to the S-matrix.
\end{framedrmk}


