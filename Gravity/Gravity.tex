\documentclass[aps,prb,superscriptaddress,nofootinbib]{revtex4}
\usepackage{amsfonts}
\usepackage{amsmath}
\usepackage{amssymb}
\usepackage{graphbox}
\usepackage{graphicx}
\usepackage{caption}
\usepackage{bm}
\usepackage{bbm}
\usepackage{cancel}
\usepackage{color}
\usepackage{mathrsfs}
\usepackage[colorlinks,bookmarks=true,citecolor=blue,linkcolor=red,urlcolor=blue]{hyperref}
\usepackage{simpler-wick}
\usepackage{appendix}
\usepackage{float}
\usepackage{array}
\usepackage{booktabs}
\usepackage[export]{adjustbox}
\setlength{\parindent}{10 pt}
\setlength{\parskip}{2 pt}
\setcounter{MaxMatrixCols}{30}
\bibliographystyle{apsrev}
\newcommand{\RNum}[1]{\uppercase\expandafter{\romannumeral #1\relax}}
\newcommand{\normord}[1]{{:\mathrel{#1}:}}
\def\tbs{\textbackslash}
\def \tr{\operatorname{tr}}
\def \Tr{\operatorname{Tr}}


\begin{document}
\title{Gravity}
\author{Jie Ren}


\maketitle


\tableofcontents

\section{Riemann Geometry}

\subsection{Connection}
For a general coordinate system, we can choose a coordinate basis $\{e_\mu \equiv \frac{\partial}{\partial x^\mu}\}$ and define the connection: 
\begin{equation}\label{eq:connection}
	\nabla_\mu e_\nu = e_\lambda{\Gamma^\lambda}_{\mu\nu}.
\end{equation}
We immediately know the covariant derivative for vector field:
\begin{equation}
	\nabla_\mu (W^\nu e_\nu) = \frac{\partial W^\nu}{\partial x^\mu} e_\nu + W^\nu e_\lambda {\Gamma^\lambda}_{\mu\nu} = \left(\frac{\partial W^\lambda}{\partial x^\mu} +{\Gamma^\lambda}_{\mu\nu} W^\nu \right) e_\lambda,  
\end{equation}
For the dual vector, consider the equation:
\begin{equation}
	\nabla_\mu (W^\nu V_\nu) = (\partial_\mu W^\nu) V_\nu + W^\nu (\partial_\mu V_\nu)
	= \left(\frac{\partial W^\lambda}{\partial x^\mu} +{\Gamma^\lambda}_{\mu\nu} W^\nu \right)V_\lambda + W^\nu (\nabla_\mu V)_\nu,
\end{equation}
which leads to $\nabla_\mu V_\nu = \partial_\mu V_\nu -V_\lambda {\Gamma^\lambda}_{\mu\nu}$.
The covariant derivative is used to defined the \textit{parallel transport}: let $X$ be a vector field defined along curve $c(t)$.
$X$ is said to be parallel transported if
\begin{equation}
	\nabla_V X = 0 \quad \Longrightarrow \quad 
	\frac{d x^\mu}{dt}\left(\frac{\partial X^\lambda}{\partial x^\mu} +{\Gamma^\lambda}_{\mu\nu} X^\nu \right) = \frac{d X^\lambda}{dt} +{\Gamma^\lambda}_{\mu\nu} \frac{dx^\mu}{dt} X^\nu = 0.
\end{equation}
where $V^\mu = \frac{d}{dt} x^\mu|_{c(t)}$.
Further, a curve $c(t)$ is a geodesic if
\begin{equation}
	\nabla_V V = 0 \quad \Longrightarrow \quad
	\frac{d^2 x^\mu}{dt^2} + {\Gamma^\mu}_{\nu\lambda} \frac{dx^\nu}{dt}\frac{dx^\lambda}{dt} = 0.
\end{equation}
Note that the connection is not a tensor: for a coordinate transformation $x \rightarrow \tilde x$, the basis transforms as $\tilde e_\nu = (\partial x^\mu/\partial \tilde x^\nu) e_\mu$.
According to (\ref{eq:connection}), the new connection is
\begin{equation}
	\nabla_{\tilde e_\mu} \tilde e_\nu 
	= \tilde{e}_\lambda {\tilde \Gamma^\lambda}_{\mu\nu} 
	= \frac{\partial x^\sigma}{\partial \tilde{x}^\mu} \nabla_\sigma \left(\frac{\partial x^\tau}{\partial\tilde x^\nu} e_\tau\right)
	= \frac{\partial x^\sigma}{\partial \tilde{x}^\mu} \left(\frac{\partial}{\partial x^\sigma}\frac{\partial x^\tau}{\partial\tilde x^\nu} e_\tau + {\Gamma^\rho}_{\sigma\tau}\frac{\partial x^\tau}{\partial\tilde x^\nu}e_\rho\right)
\end{equation}
Therefore, the connection transforms as:
\begin{equation}
	{\tilde \Gamma^\lambda}_{\mu\nu}
	= {\Gamma^\rho}_{\sigma\tau} \frac{\partial \tilde x^\lambda}{\partial x^\rho} \frac{\partial x^\sigma}{\partial \tilde{x}^\mu} \frac{\partial x^\tau}{\partial\tilde x^\nu} + \frac{\partial \tilde x^\lambda}{\partial x^\tau} \frac{\partial^2 x^\tau}{\partial \tilde x^\mu \partial \tilde x^\nu}.
\end{equation}



\subsection{Curvature}


\subsection{Vielbeins}



\section{The Einstein Equations}


\subsection{The Einstein-Hilbert Action}

Given a Ricci scalar $R$, the action for gravitational field is
\begin{equation}
	S = \frac{M^2_\mathrm{pl}}{2} \int d^4 x \sqrt{-g} R.
\end{equation}

\subsection{Schwarzschild Spacetime}

\subsection{de Sitter Space}





\end{document}


