\documentclass[aps,prb,superscriptaddress,nofootinbib]{revtex4}
\usepackage{amsfonts}
\usepackage{amsmath}
\usepackage{amssymb}
\usepackage{graphbox}
\usepackage{graphicx}
\usepackage{caption}
\usepackage{bm}
\usepackage{bbm}
\usepackage{cancel}
\usepackage{color}
\usepackage{mathrsfs}
\usepackage[colorlinks,bookmarks=true,citecolor=blue,linkcolor=red,urlcolor=blue]{hyperref}
\usepackage{simpler-wick}
\usepackage{appendix}
\usepackage{float}
\usepackage{array}
\usepackage{booktabs}
\usepackage[export]{adjustbox}
\setlength{\parindent}{10 pt}
\setlength{\parskip}{2 pt}
\setcounter{MaxMatrixCols}{30}
\bibliographystyle{apsrev}
\newcommand{\RNum}[1]{\uppercase\expandafter{\romannumeral #1\relax}}
\newcommand{\normord}[1]{{:\mathrel{#1}:}}
\def\tbs{\textbackslash}
\def \tr{\operatorname{tr}}
\def \Tr{\operatorname{Tr}}


\begin{document}
\title{Gravity}
\author{Jie Ren}


\maketitle


\tableofcontents

\section{Riemann Geometry}

\subsection{Connection}
For a general coordinate system, we can choose a coordinate basis $\{e_\mu \equiv \partial_\mu\}$ and define the connection as $\nabla_\mu e_\nu = \Gamma^\lambda_{\mu\nu} e_\lambda$, where $\nabla_\mu$ is the covariant derivative along the $x^\mu$ direction.
We immediately know the covariant derivative for the vector field:
\begin{equation*}
	\nabla_\mu (W^\nu e_\nu) = \frac{\partial W^\nu}{\partial x^\mu} e_\nu + W^\nu e_\lambda \Gamma^\lambda_{\mu\nu} = \left(\frac{\partial W^\lambda}{\partial x^\mu} +\Gamma^\lambda_{\mu\nu} W^\nu \right) e_\lambda,  
\end{equation*}
For the dual vector, consider the expression $\nabla_\mu (W^\nu V_\nu)$.
Since $W^\nu V_\nu$ is a scalar, the covariant derivative is the same as the ordinary derivative, $\nabla_\mu(W^\nu V_\nu) = \partial_\mu(W^\nu V_\nu)$.
On the other hand,
\begin{equation*}
	\nabla_\mu (W^\nu V_\nu) = (\partial_\mu W^\nu) V_\nu + W^\nu (\partial_\mu V_\nu)
	= \left(\frac{\partial W^\lambda}{\partial x^\mu} +\Gamma^\lambda_{\mu\nu} W^\nu \right)V_\lambda + W^\nu (\nabla_\mu V)_\nu,
\end{equation*}
which leads to $\nabla_\mu V_\nu = \partial_\mu V_\nu -V_\lambda {\Gamma^\lambda}_{\mu\nu}$.
In general, the covariant derivative on a tensor $T$ is
\begin{equation}
	\nabla_\rho T_{\nu_1 \cdots \nu_q}^{\mu_1 \cdots \mu_p}
	= \partial_\rho T_{\nu_1 \cdots \nu_q}^{\mu_1\cdots \mu_p}+
	(\Gamma^{\mu_1}_{\rho \sigma} T^{\sigma \mu_2 \cdots \mu_p}_{\nu_1 \ldots \nu_q}+\cdots+
	\Gamma^{\mu_p}_{\rho \sigma} T^{\mu_1 \cdots \mu_{p-1} \sigma}_{\nu_1 \cdots \nu_q}) - 
	(\Gamma^\sigma_{\rho \nu_1} T^{\mu_1 \cdots \mu_p}_{\sigma \nu_2 \cdots \nu_q}+\cdots+
	\Gamma^\sigma_{\rho \nu_q} T_{\nu_1 \cdots \nu_{q-1} \sigma}^{\mu_1 \cdots \mu_p}).
\end{equation}



In the space-time manifold with a metric $g_{\mu\nu}$, there exists a unique, torsion-free connection such that $\nabla_\rho g_{\mu\nu} = 0$.
To see this, let us first write the $\nabla_\rho g_{\mu\nu} = 0$ condition in three equivalent ways:
\begin{eqnarray}
	\partial_\rho g_{\mu\nu} -\Gamma^\sigma_{\rho\mu}g_{\sigma\nu}-\Gamma^\sigma_{\rho\nu}g_{\mu\sigma} &=& 0, \label{eq:christoffel-1}\\
	\partial_\mu g_{\nu\rho} -\Gamma^\sigma_{\mu\nu}g_{\sigma\rho}-\Gamma^\sigma_{\mu\rho}g_{\nu\sigma} &=& 0, \label{eq:christoffel-2}\\
	\partial_\nu g_{\rho\mu} -\Gamma^\sigma_{\nu\rho}g_{\sigma\mu}-\Gamma^\sigma_{\nu\mu}g_{\rho\sigma} &=& 0. \label{eq:christoffel-3}
\end{eqnarray}
The torsion is defined as $T^\sigma_{\mu\nu} = \Gamma^\sigma_{\mu\nu}-\Gamma^\sigma_{\nu\mu}$.
The torsion-free condition helps reduced the above equations.
We simply add Eq.~(\ref{eq:christoffel-2}) and Eq.~(\ref{eq:christoffel-3}) and subtract Eq.~(\ref{eq:christoffel-1}), then we have
\begin{equation}
	2g_{\rho\sigma} \Gamma^\sigma_{\mu\nu} = \partial_\mu g_{\nu\rho}+\partial_\nu g_{\mu\rho}-\partial_\rho g_{\mu\nu}
	\quad\Longrightarrow\quad
	\Gamma^\sigma_{\mu\nu} = \frac{1}{2}g^{\sigma\rho}(\partial_\mu g_{\nu\rho}+\partial_\nu g_{\mu\rho}-\partial_\rho g_{\mu\nu}).
\end{equation}
The torsion-free connection is called the \textit{Christoffel symbol}.
Note that the Christoffel symbol satisfies
\begin{equation}
	\Gamma^\mu_{\mu\nu} = \frac{1}{\sqrt{|g|}}\partial_\nu \sqrt{|g|}.
\end{equation}
The proof is straightforward, since
\begin{equation*}
	\Gamma^\mu_{\mu\nu} = \frac{1}{2} g^{\rho\mu} \partial_\nu g_{\mu\rho}
	= \frac{1}{2}\tr[g^{-1}\partial_\nu g]
	= \frac{1}{2}\partial_\nu \tr[\log g]
	= \partial_\nu \log \sqrt{|g|}
	= \frac{1}{\sqrt{|g|}}\partial_\nu \sqrt{|g|},
\end{equation*}
where we have used the fact $\tr \log A = \log \det A$, and we can replace $\det g$ with $|\det g| = |g|$ since the additional phase, upon the action of logarithm and derivative, vanished.

\paragraph*{The vielbeins}
There is a neat way to represent the connection.
First, we introduce a set of local frame called \textit{vielbeins} or \textit{tetrads}:
\begin{equation*}
	\hat e_a = e^\mu_a \partial_\mu, \quad
	g_{\mu\nu} e_a^\mu e_b^\nu = \eta_{ab}.
\end{equation*}
The vielbeins convert a general metric to the Minkowski metric (locally).
We can also raise and lower the indices by $e^a_\mu = \eta^{ab}e^\mu_b g_{\mu\nu}$.
Now consider the one form $\theta^a \equiv e^a_\mu dx^\mu$, satisfying $\eta_{ab}\theta^a\theta^b = g_{\mu\nu}dx^\mu dx^\nu$.
We define the matrix-valued connection one-form as
\begin{equation}
	\omega^a{}_b = \Gamma^a_{bc}\theta^c,
\end{equation}
where $\Gamma^c_{ab}$ is defined by $\nabla_{\hat e_a} \hat e_b = \Gamma^c_{ab} \hat e_c$.
There is a rather simple way to compute the connection one-forms, at least for a torsion-free connection. 
This follows from the first of two Cartan structure relations.

\noindent\textbf{Claim:} for torsion-free connection, 
\begin{equation}
	d{\theta}^a+\omega^a{}_b \wedge {\theta}^b=0.
\end{equation}
\textbf{Proof:} We first look at the second term
$\omega_b^a \wedge \hat{\theta}^b = \Gamma^a_{bc}\left(e_\mu^c d x^\mu\right) \wedge\left(e^b_\nu d x^\nu\right)$.
According to its definition, the components of $\Gamma_{c b}^a$ are related to the coordinate basis components by
$$
	\Gamma^c_{a b}  = e_\rho^c e_a^\mu \nabla_\mu e_b^\rho = e_\rho^c e_a^\mu(\partial_\mu e_b^\rho+ \Gamma^\rho_{\mu \nu}e_b^\nu).
$$
So $\omega^a{}_b \wedge {\theta}^b 
	=e_\rho^a e_c^\lambda e_\mu^c e^b_\nu\left(\partial_\lambda e_b^\rho+e_b^\sigma \Gamma^\rho_{\lambda \sigma}\right) d x^\mu \wedge d x^\nu$.
We can further simplify the expression using the fact $e_c^\lambda e_\mu^c=\delta_\mu^\lambda$ and the fact that the connection is torsion-free.
Therefore, the connection term vanished:
$$
	\omega^a{}_b \wedge {\theta}^b 
	=e_\rho^a e_\nu^b \partial_\mu e_b^\rho d x^\mu \wedge d x^\nu
$$
Now we use the fact that $e_\nu^b e_b^\rho=\delta_\nu^\rho$, so $e_\nu^b \partial_\mu e_b{ }^\rho=-e_b^\rho \partial_\mu e^b{ }_\nu$. We have
$$
	\omega^a{}_b \wedge {\theta}^b = -e^a_\rho e_b^\rho \partial_\mu e^b_\nu d x^\mu \wedge d x^\nu 
	=-\partial_\mu e^a_\nu d x^\mu \wedge d x^\nu=-d {\theta}^a,
$$
which completes the proof.

\noindent\textbf{Claim:} For the Levi-Civita connection, the connection one-form is anti-symmetric:
\begin{equation}
	\omega_{a b}=-\omega_{b a}.
\end{equation}
\textbf{Proof:} This follows from the expression for the components $\Gamma_{b c}^a$. 
Lowering an index, we have
$$
\Gamma_{a b c}=\eta_{a d} e^d_\rho e_b^\mu \nabla_\mu e_c^\rho=-\eta_{a d} e_c^\rho e_b^\mu \nabla_\mu e^d_\rho=-\eta_{c f} e^f_\sigma e_b^\mu \nabla_\mu\left(\eta_{a d} g^{\rho \sigma} e^d_\rho\right)
$$
where, in the final equality, we've used the fact that the connection is compatible with the metric to raise the indices of $e_\rho^d$ inside the covariant derivative. Finishing off the derivation, we then have
$$
\Gamma_{a b c}=-\eta_{c f} e^f_\rho e_b^\mu \nabla_\mu e_a^\rho=-\Gamma_{c b a}.
$$
The result then follows from the definition $\omega_{a b}=\Gamma_{a c b} \hat{\theta}^c$.

As a concrete example, consider the metric of the general form
\begin{equation}\label{eq:general-sphere-metric}
	ds^2 = - f(r)^2 dt^2 + f(r)^{-2} dr^2 + r^2 \left(d\theta^2+\sin^2\theta d\phi^2 \right).
\end{equation}
The basis of coordinate one-forms is\footnote{Note that we have put a hat on the one-form to avoid confusion with the $\theta$ angle.}
$$
	\hat{\theta} = \left(f(r) d t,\ \frac{1}{f(r)} d r,\ r d \theta,\ r \sin \theta d \phi \right).
$$
The exterior derivatives are
$$
	d \hat{\theta} = \left(\frac{d}{dr}f(r)\ d r \wedge d t, \ 0, \ d r \wedge d \theta, \ \sin \theta\ d r \wedge d \phi+r \cos \theta\ d \theta \wedge d \phi \right).
$$
Then we can simply read out the non-vanishing component of the connection one form:
\begin{equation*}
	\omega^0{}_1 =  \omega^1{}_0 = f^{\prime}(r) \hat{\theta}^0, \quad 
	\omega^2{}_1 = -\omega^1{}_2 = \frac{f}{r} \hat{\theta}^2, \quad
	\omega^3{}_1 = -\omega^1{}_3 = \frac{f}{r} \hat{\theta}^3, \quad 
	\omega^3{}_2 = -\omega^2{}_3 = \frac{\cot \theta}{r} \hat{\theta}^3.
\end{equation*}





\subsection{Curvature}
The curvature $R$ can be viewed as a map from $\mathfrak{X}(M) \times \mathfrak{X}(M)$ to a differential operator acting on $\mathfrak{X}(M)$,
\begin{equation}
	R(X, Y) = \nabla_X \nabla_Y-\nabla_Y \nabla_X-\nabla_{[X, Y]}
\end{equation}
We can evaluate these tensors in a coordinate basis $\left\{e_\mu\right\}=\left\{\partial_\mu\right\}$, with the dual basis $\left\{f^\mu\right\}=\left\{d x^\mu\right\}$. The components of $R$ are
\begin{eqnarray}
	R_{\rho \mu \nu}^\sigma 
	&=&f^\sigma\left(\nabla_\mu \nabla_\nu e_\rho-\nabla_\nu \nabla_\mu e_\rho-\nabla_{\left[e_\mu, e_\nu\right]} e_\rho\right)
 	=f^\sigma\left(\nabla_\mu \nabla_\nu e_\rho-\nabla_\nu \nabla_\mu e_\rho\right) \nonumber \\
	&=&f^\sigma\left[\nabla_\mu\left(\Gamma_{\nu \rho}^\lambda e_\lambda\right)-\nabla_\nu\left(\Gamma_{\mu \rho}^\lambda e_\lambda\right)\right] 
	=f^\sigma\left[\left(\partial_\mu \Gamma_{\nu \rho}^\lambda\right) e_\lambda+\Gamma_{\nu \rho}^\lambda \Gamma_{\mu \lambda}^\tau e_\tau-\left(\partial_\nu \Gamma_{\mu \rho}^\lambda\right) e_\lambda-\Gamma_{\mu \rho}^\lambda \Gamma_{\nu \lambda}^\tau e_\tau\right] \nonumber \\
	&=&\partial_\mu \Gamma_{\nu \rho}^\sigma-\partial_\nu \Gamma_{\mu \rho}^\sigma+\Gamma_{\nu \rho}^\lambda \Gamma_{\mu \lambda}^\sigma-\Gamma_{\mu \rho}^\lambda \Gamma_{\nu \lambda}^\sigma, \label{eq:Reimann-tensor}
\end{eqnarray}
where we've used the fact that, in a coordinate basis, $\left[e_\mu, e_\nu\right]=\left[\partial_\mu, \partial_\nu\right]=0$. 

There is a closely related calculation in which both the torsion and Riemann tensors appears. We look at the commutator of covariant derivatives acting on vector fields. Written in an orgy of anti-symmetrised notation, this calculation gives\footnote{We use the notation $A_{[\mu\nu]} = \frac{1}{2}(A_{\mu\nu}-A_{\nu\mu})$.}
$$
\begin{aligned}
\nabla_{[\mu} \nabla_{\nu]} Z^\sigma= & \partial_{[\mu}\left(\nabla_{\nu]} Z^\sigma\right)+\Gamma_{[\mu|\lambda|}^\sigma \nabla_{\nu]} Z^\lambda-\Gamma_{[\mu \nu]}^\rho \nabla_\rho Z^\sigma \\
= & \partial_{[\mu} \partial_{\nu]} Z^\sigma+\left(\partial_{[\mu} \Gamma_{\nu] \rho}^\sigma\right) Z^\rho+\left(\partial_{[\mu} Z^\rho\right) \Gamma_{\nu] \rho}^\sigma+\Gamma_{[\mu|\lambda|}^\sigma \partial_{\nu]} Z^\lambda +\Gamma_{[\mu|\lambda|}^\sigma \Gamma_{\nu] \rho}^\lambda Z^\rho-\Gamma_{[\mu \nu]}^\rho \nabla_\rho Z^\sigma.
\end{aligned}
$$
The first term vanishes, while the third and fourth terms cancel against each other. We're left with
\begin{equation}\label{eq:Ricci-id}
	(\nabla_\mu\nabla_\nu - \nabla_\nu \nabla_\mu) Z^\sigma=R_{\rho \mu \nu}^\sigma Z^\rho-T_{\mu \nu}^\rho \nabla_\rho Z^\sigma,
\end{equation}
where the torsion tensor is $T_{\mu \nu}^\rho= \Gamma_{\mu\nu}^\rho - \Gamma_{\nu\mu}^\rho$ and the Riemann tensor coincides with Eq.~(\ref{eq:Reimann-tensor}). 
The expression Eq.~(\ref{eq:Ricci-id}) is known as the Ricci identity.

We can compute the components of the Riemann tensor in our non-coordinate basis,
$$
	R_{b c d}^a=R\left(\hat{\theta}^a ; \hat{e}_c, \hat{e}_d, \hat{e}_b\right).
$$
The anti-symmetry of the last two indices, $R_{b c d}^a=-R_{b d c}^a$, makes this ripe for turning into a matrix of two-forms,
\begin{equation}
	\mathcal{R}^a{}_b=\frac{1}{2} R_{b c d}^a \hat{\theta}^c \wedge \hat{\theta}^d.
\end{equation}
The second of the two Cartan structure relations states that this can be written in terms of the curvature one-form as
\begin{equation}
	\mathcal{R}^a{}_b = d \omega^a{}_b + \omega^a{}_c \wedge \omega^c{}_b.
\end{equation}
Consider the metric in Eq.~(\ref{eq:general-sphere-metric}).
Now we can use this to compute the curvature two-form. We will focus on $\mathcal{R}^0{}_1=d \omega^0{}_1+\omega^0{}_c \wedge \omega^c{}_1$.
We have
$$
	d \omega^0{}_1 = f^{\prime} d \hat{\theta}^0+f^{\prime \prime} d r \wedge \hat{\theta}^0=\left[\left(f^{\prime}\right)^2+f^{\prime \prime} f\right] d r \wedge d t.
$$
The second term in the curvature 2-form is $\omega^0{}_c \wedge \omega^c{}_1=\omega^0{}_1 \wedge \omega^1{}_1=0$. So we're left with
$$
\mathcal{R}^0{}_1=\left[\left(f^{\prime}\right)^2+f^{\prime \prime} f\right] d r \wedge d t=\left[\left(f^{\prime}\right)^2+f^{\prime \prime} f\right] \hat{\theta}^1 \wedge \hat{\theta}^0.
$$


\subsection{Dynamics}

The covariant derivative defines the \textit{parallel transport}: let $X$ be a vector field defined along curve $c(t)$.
$X$ is said to be parallel transported if $\nabla_V X = 0$, which leads to the parallel transportation equation:
\begin{equation*}
	\frac{d x^\mu}{dt}\left(\frac{\partial X^\lambda}{\partial x^\mu} +{\Gamma^\lambda}_{\mu\nu} X^\nu \right) = \frac{d}{dt}X^\lambda +{\Gamma^\lambda}_{\mu\nu} V^\mu X^\nu = 0,\quad \text{where}\quad V^\mu = \frac{d}{dt} x^\mu|_{c(t)}.
\end{equation*}
Further, a curve $c(t)$ is a geodesic if $\nabla_V V = 0$, which leads to the geodesic equation:
\begin{equation*}
	\frac{d^2 x^\mu}{dt^2} + {\Gamma^\mu}_{\nu\lambda} \frac{dx^\nu}{dt}\frac{dx^\lambda}{dt} = 0.
\end{equation*}




\section{The Einstein Equations}


\subsection{The Einstein-Hilbert Action}

Given a Ricci scalar $R$, the action for the gravitational field is
\begin{equation}
	S = \frac{M^2_\mathrm{pl}}{2} \int d^4 x \sqrt{|g|} R.
\end{equation}
Note that $S$ is non-renormalizable.
In the following, we will choose the unit so that $M^2_\text{pl}/2=1$.

We would like to determine the Euler-Lagrange equations arising from the action. 
We do this in the usual way, by starting with some fixed metric $g_{\mu \nu}(x)$ and seeing how the action changes when we shift $g_{\mu \nu}(x) \rightarrow g_{\mu \nu}(x)+\delta g_{\mu \nu}(x)$.
Writing the Ricci scalar as $R=g^{\mu \nu} R_{\mu \nu}$, the Einstein-Hilbert action clearly changes as
$$
	\delta S=\int d^4 x\left[(\delta \sqrt{|g|}) g^{\mu \nu} R_{\mu \nu}+\sqrt{|g|}\left(\delta g^{\mu \nu}\right) R_{\mu \nu}+\sqrt{|g|} g^{\mu \nu} \delta R_{\mu \nu}\right]
$$
It turns out that it's slightly easier to think of the variation in terms of the inverse metric $\delta g^{\mu \nu}$. 
This is equivalent to the variation of the metric $\delta g_{\mu \nu}$; the two are related by
$$
g_{\rho \mu} g^{\mu \nu}=\delta_\rho^\nu \Rightarrow\left(\delta g_{\rho \mu}\right) g^{\mu \nu}+g_{\rho \mu} \delta g^{\mu \nu}=0 \quad \Rightarrow \quad \delta g^{\mu \nu}=-g^{\mu \rho} g^{\nu \sigma} \delta g_{\rho \sigma}.
$$
To proceed, we will need to calculate $\delta \sqrt{|g|}$.
Using the identity $\log \operatorname{det} A=\operatorname{tr} \log A$, we have
$$
\frac{1}{\operatorname{det} A} \delta(\operatorname{det} A)=\operatorname{tr}\left(A^{-1} \delta A\right).
$$
Applying this to the metric, we have
$$
\delta \sqrt{|g|}=\frac{1}{2} \frac{1}{\sqrt{|g|}}|g| g^{\mu \nu} \delta g_{\mu \nu}=\frac{1}{2} \sqrt{|g|} g^{\mu \nu} \delta g_{\mu \nu}
=-\frac{1}{2} \sqrt{|g|} g_{\mu \nu} \delta g^{\mu \nu}.
$$
Now we turn to $\delta R_{\mu\nu}$.
We claim that $\delta R_{\mu \nu}=\nabla_\rho \delta \Gamma_{\mu \nu}^\rho-\nabla_\nu \delta \Gamma_{\mu \rho}^\rho$, where
$$
\delta \Gamma_{\mu \nu}^\rho=\frac{1}{2} g^{\rho \sigma}\left(\nabla_\mu \delta g_{\sigma \nu}+\nabla_\nu \delta g_{\sigma \mu}-\nabla_\sigma \delta g_{\mu \nu}\right).
$$
is a tensor.
The last expression now becomes a total derivative 
$$
g^{\mu \nu} \delta R_{\mu \nu}=\nabla_\mu X^\mu \quad \text { with } \quad X^\mu=g^{\rho \nu} \delta \Gamma_{\rho \nu}^\mu-g^{\mu \nu} \delta \Gamma_{\nu \rho}^\rho.
$$
The variation of the action can then be written as
$$
\delta S=\int d^4 x \sqrt{-g}\left[\left(R_{\mu \nu}-\frac{1}{2} R g_{\mu \nu}\right) \delta g^{\mu \nu}+\nabla_\mu X^\mu\right].
$$
This final term is a total derivative and, by the divergence, we ignore it. 
Requiring $\delta S=0$, we have the equations of motion
$$
G_{\mu \nu}:=R_{\mu \nu}-\frac{1}{2} R g_{\mu \nu}=0.
$$




\subsection{Schwarzschild Spacetime}

\subsection{de Sitter Space}





\end{document}


