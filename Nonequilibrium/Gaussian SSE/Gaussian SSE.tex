\documentclass{SciPost}

% Prevent all line breaks in inline equations.
%\binoppenalty=10000
%\relpenalty=10000

\hypersetup{
    colorlinks,
    linkcolor={red!50!black},
    citecolor={blue!50!black},
    urlcolor={blue!80!black}
}

\usepackage[bitstream-charter]{mathdesign}
\urlstyle{same}

% Fix \cal and \mathcal characters look (so it's not the same as \mathscr)
\DeclareSymbolFont{usualmathcal}{OMS}{cmsy}{m}{n}
\DeclareSymbolFontAlphabet{\mathcal}{usualmathcal}

\fancypagestyle{SPstyle}{
\fancyhf{}
\lhead{\colorbox{scipostblue}{\bf \color{white} ~Notes on Physics}}
\rhead{{\bf \color{scipostdeepblue} ~Nonequilibrium}}
\renewcommand{\headrulewidth}{1pt}
\fancyfoot[C]{\textbf{\thepage}}
}

\begin{document}

\pagestyle{SPstyle}

\begin{center}{\Large \textbf{\color{scipostdeepblue}{
Gaussian Lindbladian\\
}}}\end{center}

\begin{center}
\textbf{Jie Ren}
\end{center}

\tableofcontents

\section{Gaussian System}

\subsection{Dirac Free Fermion}

For Dirac Free Fermion systems, a Gaussian state can be represented as a matrix $B$:\footnote{Note that the matrix $B$ representing the Gaussian state has the unitary degree of freedom $$|B\rangle = |B'\rangle, \quad B'_{ij} = \sum_k B_{ik}U_{kj},$$ where $U_{kj}$ is an $N\times N$ unitary matrix. It means that the Gaussian state is determined by the linear subspace that columns of $B$ span. The columns of $B$ need not be orthogonal, while the canonical form can be obtained by the QR decomposition: $B_{L\times N} = Q_{L\times N} \cdot R_{N\times N},$ where the $Q$ matrix is orthonormal and we can set $B' = Q$.} 
\begin{equation}
	|B\rangle \equiv \prod_{j=1}^N \sum_i B_{ij} c_{i}^\dagger |0\rangle 
	\equiv \bigotimes_{j=1}^N |B_j\rangle.
\end{equation}
The Hamiltonian evolution $|\psi(t)\rangle = e^{-iH \Delta t} |\psi\rangle$ becomes evolution of the matrix $B$:\footnote{Using the Baker-Campbell-Hausdorff formula, $e^X Y e^{-X} = \exp(\operatorname{ad} X) Y$, therefore $e^{-iH \Delta t} c_j^\dagger e^{iH \Delta t} = c_k^\dagger[e^{-iH \Delta t}]_{kj}$. Here the Hamiltonian is not necessarily Hermitian.}
\begin{equation}
	e^{-iH \Delta t}|B\rangle = \prod_{j=1}^N \sum_i \left[e^{-iH \Delta t}\right]_{ki} B_{ij} c_{k}^\dagger |0\rangle = \left|e^{-iH \Delta t}\cdot B \right\rangle.
\end{equation}
The density matrix for $m$-site subregion has the Gaussian form $\rho = \exp(-M_{ij}c_i^\dagger c_j)$. If we diagonalize matrix $M$
\begin{equation}
	M_{ij} = \sum_k \lambda_k U_{ik}  U^*_{jk},
\end{equation}
and consider the new basis $a_k^\dagger = \sum_i U_{ik}c_i^\dagger$, then $\rho$ will be in diagonal form
\begin{equation}
	\rho = \frac{1}{Z}\exp\left(-\sum_k \lambda_k a_k^\dagger a_k\right) 
	= \bigotimes_{k=1}^n \begin{pmatrix} 
		\mu_k & 0 \\ 0 & 1-\mu_k 
	\end{pmatrix},
\end{equation}
where $\mu_k = (1+e^{-\lambda_k})^{-1}$. 
Under basis $\{a_k\}$, the correlation function is diagonal: 
\begin{equation}
	\langle a_k^\dagger a_l\rangle = \delta_{kl}\mu_k = U_{ik}\langle c_i^\dagger c_j\rangle U_{jk}^* = U^T\cdot G\cdot U^*.
\end{equation}
It means the spectrum of $\rho$ can be obtained from correlation function $G$.
The entropy is
\begin{equation}
	S(\rho) = -\Tr[\rho \log \rho] = \sum_k H(\mu_k).
\end{equation}
where $H(x) = -x\log x - (1-x)\log(1-x)$.

\subsection{BdG Hamiltonian}
The BdG Hamiltonian has the form
\begin{equation}
	\hat H = \sum_{i,j=1}^n A_{ij} c_i^\dagger c_j+\frac{1}{2}\sum_{i,j=1}^n (B_{ij}c_i^\dagger c_j^\dagger-B_{ij}^*c_ic_j),
\end{equation}
where $A$ is Hermitian and $B$ is antisymmetric. 
Define the Nambu basis $\underline\Psi$ where $\Psi_i = c_i$ when $1 \le i \le n$, and $\Psi_i = c_i^\dagger$ when $n+1 \le i \le 2n$.
The Hamiltonian under $\underline\Psi$ becomes
\begin{equation}
	\hat H = \frac{1}{2}\sum_{i,j=1}^{2n} \Psi_i^\dagger H_{ij} \Psi_j, \quad
	H= \begin{bmatrix} A & B \\ -B^* & -A^* \end{bmatrix},
\end{equation}

The action of particle-hole symmetry $\mathcal C$ on the Nambu spinor is $\mathcal C \cdot \Psi = \sigma^x \cdot\Psi^*$, on Hamiltonian is $C H C^{-1} = \sigma^x H^* \sigma^x = -H$. 
Consider a unitary transformation\footnote{Since $T$ is unitary, the submatrix $U$ and $V$ satisfy: $U^\dagger \cdot U + V^\dagger \cdot V = \mathbb I$, and $U^\dagger \cdot V^* + V^\dagger \cdot U^* = 0$. In numerical diagonalization, one shall only keep half of the eigenvectors to obtain the matrix $U, V$ and discard the rest as a way of gauge fixing.} $T$ conserving particle-hole symmetry that diagonalizes the Hamiltonian matrix:
\begin{equation}
	T^\dagger H T = \mathrm{diag}(\varepsilon_1,\cdots,\varepsilon_N,-\varepsilon_1,\cdots,-\varepsilon_N),\quad
	T = \begin{bmatrix}
		U & V^* \\ V & U^*
	\end{bmatrix}.
\end{equation}
Define a set of new fermionic modes
\begin{equation}
	d_k^\dagger = \sum_{j=1}^{2n} \Psi_j^\dagger T_{jk}
	= \sum_{j=1}^n (U_{jk} c_j^\dagger + V_{jk} c_j),
\end{equation}
where $k = 1,2,\dots,n$.
The Hamiltonian is
\begin{equation}
	\hat H = \frac{1}{2} \sum_n \varepsilon_n (d_n^\dagger d_n- d_n d_n^\dagger) 
	= \sum_n \varepsilon_n\left(d_n^\dagger d_n-\frac{1}{2}\right).
\end{equation}

\paragraph{Gaussian State}
A fermionic Gaussian state $|\psi\rangle$ can be regarded as the ground state of a BdG Hamiltonian $H$, which is annihilated by all $d$'s:
\begin{equation}
	\sum_{j=1}^n U_{jk}^* c_j|\psi\rangle = - \sum_{j=1}^n V_{jk}^* c_j^\dagger|\psi\rangle, \quad\forall k =1,2,\dots,n.
\end{equation}
For this linear form, we immediately see that $|\psi\rangle$ has the Gaussian form:
\begin{equation}
	|\psi\rangle = \frac{1}{\mathcal N} \exp\left(\sum_{ij}M_{ij}c_i^\dagger c_j^\dagger\right)|0\rangle.
\end{equation}
The equation leads to
\begin{equation*}
	c_j |\psi\rangle = e^{\sum_{kl}M_{kl}c_k^\dagger c_l^\dagger} \left(e^{-\sum_{kl}M_{kl}c_k^\dagger c_l^\dagger} c_j e^{\sum_{kl}M_{kl}c_k^\dagger c_l^\dagger}\right) |0\rangle 
	= -2 M_{kj} c_k^\dagger|\psi\rangle.
\end{equation*}
Therefore $M = \frac{1}{2} V^* \cdot (U^*)^{-1}$, or equivalently, $M = -\frac{1}{2}(U^\dagger)^{-1} \cdot V^\dagger$.
The matrix $M$ is antisymmetric since
\begin{equation*}
\begin{aligned}
	M + M^T 
	&= \frac{1}{2} \left[V^* \cdot (U^*)^{-1} + (U^\dagger)^{-1} \cdot V^\dagger\right] 
	= \frac{1}{2} \left[V^*  + (U^\dagger)^{-1} \cdot V^\dagger \cdot U^*\right] \cdot  (U^*)^{-1} \\
	&= \frac{1}{2} \left[V^*  - (U^\dagger)^{-1} \cdot  U^\dagger \cdot V^*\right] \cdot  (U^*)^{-1} =0.
\end{aligned}
\end{equation*}
For BdG system, a Gaussian state $|\psi\rangle$ is specified by the matrix $U$ and $V$, or equivalently by the $\{d^\dagger_k\}$ which annihilate $|\psi\rangle$.
The evolution of the system is equivalently described by the evolution of the operator
\begin{equation}
	d_k^\dagger(t) = e^{-iHt} d_k^\dagger e^{iHt}.
\end{equation}
Formally, the expression resembles the case in the Dirac free fermion system, but the BdG form has the redundancy that shall be eliminated by imposing particle-hole symmetry. Effectively,  the apparent ``energy" is one-half of the true energy of $d_k$. From the commutation relation point of view, the factor 2 comes from the commutator $[\Psi_i, \Psi_{i+n}]$. As a result, the evolution follows
\begin{equation*}
	\begin{bmatrix} U(t) \\ V(t) \end{bmatrix}
	= \exp(-iH_\text{BdG}t)\begin{bmatrix} U \\ V \end{bmatrix}.
\end{equation*}


\subsection{Majorana Basis}
Under the Majorana basis $\underline\Omega=(\omega_1,\dots,\omega_{2n})^T$, where
\begin{equation}
	\omega_i = \begin{cases}
		c_i + c_i^\dagger & 1 \le i \le n \\
		i(c_i-c_i^\dagger) & n+1 \le i \le 2n
	\end{cases}.
\end{equation}
The Majorana operators satisfy the commutation relation $\{\omega_i,\omega_j\} = 2\delta_{ij}$.
The Hamiltonian under $\underline\Omega$ is
\begin{equation}
	\hat H = \frac{1}{8} \underline\Omega 
	\begin{bmatrix}
		\mathbb{I} & \mathbb{I} \\ i\mathbb{I} & -i\mathbb{I}
	\end{bmatrix}\cdot
	\begin{bmatrix}
		A & B \\ -B^* & -A^*
	\end{bmatrix}\cdot
	\begin{bmatrix}
		\mathbb{I} & -i\mathbb{I} \\ \mathbb{I} & i\mathbb{I}
	\end{bmatrix}
	\underline\Omega 
	= -\frac{i}{4} \sum_{i,j=1}^{2n} \Omega_i H_{ij} \Omega_j.
\end{equation}
Here, the Hamiltonian matrix under the Majorana basis is
\begin{equation}
	H = \begin{bmatrix}
		-\im{A} - \im{B} & \re{A} - \re{B} \\
    	-\re{A} - \re{B} &  -\im{A} + \im{B}
	\end{bmatrix},
\end{equation}
where $\re{A} = \mathrm{Re}[A]$, $\im{A} = \mathrm{Im}[A]$, $\re{B} = \mathrm{Re}[B]$, $\im{B} = \mathrm{Im}[B]$. 
The canonical form of antisymmetric matrix $H$ is:
\begin{equation}
	H = R \begin{bmatrix}
		0 & \mathrm{diag}(\varepsilon_1,\cdots,\varepsilon_n) \\ 
		-\mathrm{diag}(\varepsilon_1,\cdots,\varepsilon_n) & 0
	\end{bmatrix} R^T,
\end{equation}
where $R \in \mathrm{SO}(2n)$.
By defining the new Majorana operator $\gamma_k = \sum_{j=1}^{2n} \omega_j R_{jk}$, the Hamiltonian becomes a diagonal form
\begin{equation}
	H = -\frac{i}{2} \sum_{k=1}^n \varepsilon_k \gamma_k\gamma_{k+n}
	= \sum_{k=1}^n \varepsilon_k \left(d_k^\dagger d_k -\frac{1}{2}\right).
\end{equation}

\paragraph{Grassmann representation}
A general operator in Fermionic Fock space can be expanded on the Majorana basis:
\begin{equation}
	\hat{X}=\alpha\hat{I}+\sum_{p=1}^{2n}\sum_{1\le a_{1}<\cdots<a_{p}\le2n}\alpha_{a_{1}\cdots a_{p}}\hat{\omega}_{a_{1}}\cdots\hat{\omega}_{a_{p}}.
\end{equation}
Define a linear map from Fermionic operator space to Grassmann algebra:
\begin{equation}
	\hat X \mapsto X(\theta)=\alpha + \sum_{1\le a_{1}<\cdots<a_{p}\le2n}\alpha_{a_{1}\cdots a_{p}}\theta_{a_{1}}\cdots \theta_{a_{p}}.
\end{equation}
This mapping is called the Grassmann representation\cite{bravyi2004lagrangian} of $\hat X$.\footnote{One can formally define calculus on Grassmann algebra: $\partial_{\theta_i} \theta_j = \int d{\theta_i}\theta_j = \delta_{ij}$ and $\int d{\theta}=0$.}
The Gaussian integral of Grassmann algebra is
\begin{equation}
	\int D[\theta]\ \exp\left(\eta^T\cdot\theta+\frac{i}{2}\theta^T \cdot M\cdot\theta \right)
	=i^n \operatorname{Pf}(M) \exp\left(-\frac{i}{2}\eta^T\cdot M^{-1}\cdot\eta \right).
\end{equation}
A Gaussian state $\hat \rho$ has Gaussian Grassmann representation:
\begin{equation}
	\rho(\theta)=\frac{1}{2^{n}}\exp\left(\frac{i}{2}\theta^{T}\cdot M\cdot\theta\right),
\end{equation}
where the antisymmetric matrix $M_{ab}=\frac{i}{2}\Tr(\hat \rho[\hat \omega_a,\hat\omega_b])$ is the \textbf{covariance matrix}.
All higher correlations of a Gaussian state are determined by the Wick theorem, namely
$$
\Tr(i^p\hat\rho\hat\omega_{a_1}\cdots\hat\omega_{a_p})=\operatorname{Pf}(M|_{a_1,\dots,a_p}).
$$
The canonical form of antisymmetric matrix $M$ is:
\begin{equation}
	M = R \begin{bmatrix}
	0 & \mathrm{diag}(\lambda_1,\cdots,\lambda_n) \\ 
	-\mathrm{diag}(\lambda_1,\cdots,\lambda_n) & 0
\end{bmatrix} R^T,
\end{equation}
where $R \in \mathrm{SO}(2n)$.
Under the new Grassmann variance $\mu = R\theta$, $\rho$ has the form
\begin{equation}
	\rho(\mu)
	=\frac{1}{2^{n}}\prod_{j}\exp\left(i\lambda_j\mu_{j} \mu_{j+n}\right)
	=\frac{1}{2^{n}}\prod_{j}\left(1+i\lambda_{j} \mu_{j} \mu_{j+n}\right).
\end{equation}
We can then obtain the operator form $\hat\rho = 2^{-n} \prod_{j=1}^n(1+i\lambda_j \hat\gamma_j\hat\gamma_{j+n})$, where $\hat \gamma$'s are a new set of Majorana operators. In the fermion basis
\begin{equation}
	\hat d_j = \frac{\hat\gamma_j - i \hat\gamma_{j+n}}{2},\quad
	\hat d_j^\dagger = \frac{\hat\gamma_j + i \hat\gamma_{j+n}}{2},
\end{equation}
the density matrix has the form
\begin{equation}\label{eq:gaussian-std-form}
	\hat\rho
	= \prod_{j}\left(\frac{1+\lambda_j}{2}-\lambda_{j}d_{j}^{\dagger}d_{j}\right) 
	=\bigotimes_j \begin{bmatrix}
		\frac{1+\lambda_j}{2} & 0 \\
		0 & \frac{1-\lambda_j}{2}
	\end{bmatrix}_j.
\end{equation}
Without loss of generality, we assume $\lambda_i \ge 0$. For pure state, $\lambda_i =1,\ \forall i$.
For mixed state, the entropy of $\rho$ is just 
\begin{equation}
	S(\hat\rho)=\sum_j H\left(\frac{1+\lambda_j}{2}\right),
\end{equation}


\section{Lindblad Master Equation}
\subsection{General Markovian Form}
For general open quantum evolution, suppose the system and environment are separable initially: $\rho_T=\rho\otimes\rho_B$, where we assume $\rho_B=\sum_\alpha \lambda_\alpha |\phi_\alpha\rangle\langle\phi_\alpha |$. Then the evolution of system-bath is unitary: $\rho_T(t) = U(t)\rho_TU^\dagger(t)$. Trace out the environment's degrees of freedom, we have the quantum channel expression: $\rho(t) = \sum_{\alpha\beta} W_{\alpha\beta} \rho W^\dagger_{\alpha\beta}$, where the \textbf{Kraus operator} $W$ can be expressed formally as $W_{\alpha\beta} = \sqrt{\lambda_\beta} \langle\phi_\alpha|U(t)|\phi_\beta\rangle$.

The Lindblad equation assumes a semi-group relation: 
\begin{equation}
	\rho_t = \mathcal{L}_t \rho_0 = \lim_{N \rightarrow \infty} \mathcal{L}_{t/N}\cdot\mathcal{L}_{t/N}\cdots \mathcal{L}_{t/N}\rho_0.
\end{equation}
Such time decimation implies that the evolution is Markovian. We will show that Markovian approximation leads directly to the Lindblad equation. First, we choose a complete operator basis $\{F_i\}$ in $N$-dimensional Hilbert space, satisfying $\Tr[F_i^\dagger F_j] = \delta_{ij}$, where we choose $F_0=N^{-1/2} \cdot\mathbb I$. For a quantum channel, the channel operator $K_\mu$ can be expanded as 
\begin{equation}
	K_\mu = \sum_i \Tr[F_i^\dagger K_\mu]F_i.
\end{equation}
In general, we have: $\mathcal{L}_t[\rho] = \sum_{ij}c_{ij}(t)F_i\rho F_j^\dagger$, where the Hermitian coefficient $c_{ij}(t)$ is 
\begin{equation}
	c_{ij}(t) = \sum_{\mu} \Tr[F_i^\dagger K_\mu]\cdot \Tr[F_j^\dagger K_\mu]^*.
\end{equation}
Our target is to compute the limit $\partial_t \rho \equiv \lim_{t\rightarrow 0} \frac{1}{t}(\mathcal{L}_t[\rho]-\rho)$.
For this purpose, we define the (Hermitian) coefficient $a_{ij}$ as:
\begin{equation*}
	a_{00} = \lim_{t\rightarrow 0} \frac{c_{00}(t)-N}{t}, \quad
	a_{ij} = \lim_{t\rightarrow 0} \frac{c_{ij}(t)}{t}.
\end{equation*}
The limit is then
$$
	\frac{d}{dt}\rho 
	= \frac{a_{00}}{N}\rho + \frac{1}{\sqrt N} \sum_{i>0} \left(a_{i0} F_i \rho + a_{i0}^*\rho F_i^\dagger\right) + \sum_{i,j>0} a_{ij} F_i \rho F_j^\dagger. 
$$
To further simplify the expression, we define
\begin{equation*}
	F = \frac{1}{\sqrt N} \sum_{i=1}^{N^2-1} a_{i0} F_i, \quad
	G = \frac{a_{00}}{2N}\mathbb I +\frac{1}{2}(F^\dagger+F), \quad
	H = \frac{1}{2i}(F^\dagger-F).
\end{equation*}
The limit can be expressed by $G,H$ in a compact form:
\begin{equation*}
	\frac{d\rho}{dt} = -i[H,\rho]+\{G, \rho\}+\sum_{i,j=1}^{N^2-1}a_{ij}F_i\rho F_j^\dagger.
\end{equation*}
Note the $[H,\rho]$ part is the traceless part, and the $\{G,\rho\}$ is the trace part. Since the quantum channel preserves the trace (for any $\rho$):
$$
\Tr\left[\frac{d\rho}{dt}\right]= \Tr\left[ \left(2G+\sum_{i,j=1}^{N^2-1}a_{ij}F_j^\dagger F_i \right)\rho \right]=0.
$$
Therefore $G = -\frac{1}{2}\sum_{i,j=1}^{N^2-1}a_{ij}F_j^\dagger F_i$. We thus obtain the Lindblad form:
\begin{equation}
	\frac{d\rho}{dt} = -i[H,\rho]+\sum_{i,j=1}^{N^2-1}a_{ij} \left(F_i\rho F_j^\dagger-\frac{1}{2}\{F_j^\dagger F_i, \rho\} \right).
\end{equation}
We can simplify the form by diagonalizing the matrix $a_{ij}$. It is a convention to take the norm of $a_{ij}$ out to indicate the strength of the system-bath coupling, and the diagonalized Lindblad equation is
\begin{equation}\label{eq:lindbladian}
	\frac{d\rho}{dt} = -i[H,\rho]+ \gamma\sum_{m} \left(L_m\rho L_m^\dagger-\frac{1}{2}\{L_m^\dagger L_m, \rho\} \right).
\end{equation}







\subsection{Quasi-Free Lindbladian}
Consider the Lindblad in the Heisenberg picture:
\begin{equation}
	\frac{d}{dt} \hat O
	= i[\hat H, \hat O] + \sum_\mu \hat L_\mu^\dagger \hat O\hat L_\mu - \frac{1}{2} \sum_\mu\{\hat L_\mu^\dagger \hat L_\mu, \hat O \},
\end{equation}
where we choose $\hat O_{ij} = \omega_i\omega_j$ satisfying the relation $\hat O^T = 2\mathbb I - \hat O$. The covariance matrix is then $\Gamma_{ij} = i\langle \hat O\rangle - i\delta_{ij}$.

We assume that the jump operator has up to quadratic Majorana terms. In particular, we denote the linear terms and the Hermitian quadratic terms as
\begin{equation}
	\hat L_r = \sum_{j=1}^{2N} L^r_{j} \omega_j, \quad
	\hat L_s = -\frac{i}{4} \sum_{j,k=1}^{2N} M^s_{jk} \omega_j \omega_k.
\end{equation}
When the \textbf{jump operator} $\hat L_\mu$ contains only the linear Majorana operator, the Lindblad equation preserves Gaussianity. The evolution will break the Gaussian form for jump operators containing up to quadratic Majorana terms. 
However, the $2n$-point correlation is still solvable for free fermion systems.


Assume only linear terms in jump operators,
\begin{equation}
	\partial_t \hat O = \left[i\hat H,\hat O\right] + \mathcal D_r[\hat O] 
	= \left[i\hat H - \frac{1}{2}\sum_r \hat L_r^\dagger L_r,\hat O\right] + \sum_r \left[\hat L_r^\dagger,\hat O\right]\hat L_r.
\end{equation}
For the future convenience, we define 
\begin{equation}
	\sum_r L_i^r L_j^{r*} = B_{ij} = \re B_{ij} + i \im B_{ij}. 
\end{equation}
The first term of EOM is:
\begin{equation*}
	\left[i\hat H - \frac{1}{2}\sum_r \hat L_r^\dagger L_r,\hat O_{ij}\right]
	= \sum_{kl}\left(\frac{1}{4}H-\frac{1}{2}B \right)_{kl} [\omega_k \omega_l, \omega_i \omega_j].
\end{equation*}
Use the commutation relation $\{\omega_i, \omega_j\} = 2\delta_{ij}$, we have the relation 
\begin{equation}
\begin{aligned}[]
	[\omega_k,\omega_i \omega_j] &= 2(\delta_{ki}\omega_j-\delta_{kj}\omega_i),\\
	[\omega_k \omega_l, \omega_i \omega_j] &= 2(\delta_{ki}\omega_j \omega_l-\delta_{kj} \omega_i \omega_l + \delta_{li}\omega_k \omega_j - \delta_{lj}\omega_k\omega_i).
\end{aligned}
\end{equation}
Therefore
\begin{equation*}
\begin{aligned}[]
	\left[i\hat H - \frac{1}{2}\sum_r \hat L_r^\dagger L_r,\hat O_{ij}\right]
	&= \left[
		\left(\frac{H}{2}- B\right) \cdot \hat O^T + \left(\frac{H}{2}- B\right)^T \cdot \hat O
		- \hat O \cdot \left(\frac{H}{2}- B\right)^T- \hat O^T \cdot \left(\frac{H}{2}- B\right)
	\right]_{ij} \\
	&= \left[
		(-H+2\im B) \cdot \hat O + \hat O \cdot (H-2\im B)
	\right]_{ij}.
\end{aligned}
\end{equation*}
The second term is
\begin{equation*}
\begin{aligned}
	\sum_r \left[L_r^\dagger, \hat O_{ij}\right] \hat L_r
	&= \sum_{kl} B_{kl} [\omega_k, \omega_i \omega_j]\omega_l
	= 2\sum_{kl} B_{kl}\left(
		\delta_{ki} \omega_j \omega_l - 
		\delta_{kj} \omega_i \omega_l\right) \\
	&= \left[2B\cdot \hat O^T - 2\hat O\cdot B^T\right]_{ij}
	= \left[-2B\cdot \hat O - 2\hat O\cdot B^* + 4B\right]_{ij}.
\end{aligned}
\end{equation*}
Therefore
\begin{equation}
	\partial_t \hat O_{ij} = \left[
		(-H-2\re B) \cdot \hat O + \hat O \cdot (H-2\re{B}) + 4B
	\right]_{ij}.
\end{equation}
The EOM of the covariance matrix is then
\begin{equation}
	\partial_t \Gamma = X^T\cdot\Gamma + \Gamma \cdot X + Y,\quad
	X = H - 2\re{B},\quad Y = 4\im{B}.
\end{equation}
Note that the constant part is replaced by its anti-symmetric part.

The steady state of the system is solved by the Lyapunov equation
\begin{equation}
	X^T\cdot\Gamma + \Gamma \cdot X = - Y.
\end{equation}


Now include the Hermitian quadratic quantum jumps:
\begin{equation}
\begin{aligned}
	\partial_t \hat O &= i[\hat H, \hat O] + \mathcal D_r[\hat O] + \mathcal D_s[\hat O], \\
	\mathcal D_s[\hat O] 
	&= \sum_s \hat L_s \hat O\hat L_s - \frac{1}{2} \sum_r\{\hat L_s^2, \hat O \}
	= -\frac{1}{2} \sum_s [\hat L_s,[\hat L_s,\hat O]].
\end{aligned}
\end{equation}



\subsection{Closed Hierachy}
A direct calculation gives
\begin{equation}
\begin{aligned}
	D_s[\hat O]
	&= \frac{i}{8} \sum_s \sum_{kl} M^s_{kl} [\hat L_s,[\omega_k \omega_l, \omega_i \omega_j]] \\
	&= -\frac{i}{2}\sum_s \sum_{k} \left\{ M^s_{ik}[\hat L_s,\omega_k \omega_j]-[\hat L_s,\omega_i \omega_k]M^s_{kj} \right\} \\
	&= -\sum_{s,kl} \left[ M^s_{ik}(-M^s_{kl}\omega_l\omega_j+\omega_k\omega_l M^s_{lj})+(M^s_{il}\omega_l\omega_k-\omega_i\omega_l M^s_{lk})M^s_{kj} \right] \\
	&= \frac{1}{2}\sum_s \left[(M^s)^2 \cdot \hat O + \hat O\cdot(M^s)^2 -2 M^s \cdot\hat O\cdot M^s\right]_{ij}.
\end{aligned}
\end{equation}
Since $M^s$ is imaginary anti-symmetric matrix, $(M^s)^2 = - (\im M^s)^2$, and $M^s \cdot O \cdot M^s = - \im M^s \cdot O \cdot \im M^s$.
Together, we get the EOM of the variance matrix $\Gamma_{ij}$:
\begin{equation}
	\partial_t \Gamma = X^T\cdot\Gamma + \Gamma \cdot X - \sum_s M^s \cdot \Gamma\cdot M^s + Y,
\end{equation}
where
\begin{equation}
\begin{aligned}
	X = H - 2\re{B} + \frac{1}{2} \sum_s (M^s)^2, \quad
	Y = 4\im{B}.
\end{aligned}
\end{equation}


\paragraph{Dirac Fermion Case}

This section considers the free fermion system preserving the U(1) charge. The jump operators are assumed to be quadratic: $\hat L_s = \sum_{jk} M^s_{jk} c_j^\dagger c_k$ where $\{M^s\}$ are Hermitian matrices.

For the fermion case, we choose $\hat O_{ij} = c_i^\dagger c_j$, and consider the Lindbladian
\begin{equation}
	\partial_t \hat O = i[\hat H, \hat O] + \mathcal D_s[\hat O]
	= i[\hat H, \hat O] - \frac{1}{2} \sum_s [\hat L_s,[\hat L_s,\hat O]],
\end{equation}
where each $\hat L_s = M^s_{ij} c_i^\dagger c_j$ is a Hermitian fermion bilinear.


The Hamiltonian part is:\footnote{Using the fact
$[c_k^\dagger c_l, c_i^\dagger c_j] = c_k^\dagger[c_l,c_i^\dagger c_j] + [c_k^\dagger,c_i^\dagger c_j]c_l =\delta_{il}c_k^\dagger c_j -\delta_{jk}c_i^\dagger c_l$, we know that for a quadratic form $\hat A = \sum_{ij} A_{ij} c_i^\dagger c_j$, $[\hat A, \hat O_{ij}] = [A^T, \hat O]_{ij}$.}
\begin{equation}
	i \sum_{kl}H_{kl}[c_k^\dagger c_l, c_i^\dagger c_j]
	= i \sum_{kl} H_{kl} (\delta_{il}c_k^\dagger c_j -\delta_{jk}c_i^\dagger c_l) 
	= i [H^T\cdot \hat O - \hat O\cdot H^T]_{ij}.
\end{equation}
Similarly, the double commutation in the second term is:
\begin{equation}
	\mathcal D_s[\hat O]
	= -\frac{1}{2}\sum_s [(M^{s*})^2\cdot\hat O + \hat O\cdot (M^{s*})^2 - 2 M^{s*}\cdot \hat O \cdot M^{s*}].
\end{equation}
Together, the EOM of correlation $G_{ij} = \langle c_i^\dagger c_j\rangle$ is
\begin{equation}
	\partial_t G = X^\dagger \cdot G + G \cdot X + \sum_s M^{s*}\cdot G \cdot M^{s*},
\end{equation}
where 
\begin{equation}
	X = -i H^* - \frac{1}{2}\sum_s (M^{s*})^2.
\end{equation}





\section{Stochastic Schr\"{o}dinger Equation}

The Lindblad form Eq.~(\ref{eq:lindbladian}) is equivalent to the stochastic Schr\"{o}dinger equation (SSE):
\begin{equation}
	d|\psi\rangle = -iH|\psi\rangle + A[\psi]dt + \sum_m B[\psi]dW_m,
\end{equation}
where $dW_m$ is a stochastic infinitesimal element. The expectation value is then the average over all possible evolution path (trajectory): 
$$\langle O(t) \rangle = \overline{\langle\psi(t)|O|\psi(t)\rangle}.$$
For simplicity, in this section, we consider the jump operator $L_x$ labeled by coordinate $x$. 
The SSE can be Trotterized as
\begin{equation}
	\rho' = \left(\prod_x \mathcal{M}_{x} \right) \left[e^{-iH\Delta t} \rho e^{iH\Delta t}\right].
\end{equation}


\subsection{Quantum Jump}
Consider a small time interval $\Delta t$; the Lindblad equation is equivalent to the quantum channel 
\begin{equation}
	\mathcal{L}_{\Delta t}[\rho] = M_0 \rho M_0^\dagger + M_x \rho M_x^\dagger,
\end{equation}
where the jump operators are:
\begin{equation}
\begin{aligned}
	M_x = \sqrt{\gamma\Delta t} L_x, \quad
	M_0 = \sqrt{1 - \gamma \Delta t L_x^\dagger L_x}.
\end{aligned}
\end{equation}
A quantum channel can be simulated by a stochastic evolution of pure states:
\begin{equation}
	|\psi(t+\Delta t)\rangle \propto \left(\prod_{x} \mathcal{M}_x\right) e^{-i H\Delta t}|\psi(t)\rangle
\end{equation}
where each weak measurement is
\begin{equation}
	\mathcal{M}_x|\psi\rangle \propto \begin{cases}
		M_x |\psi\rangle & p = \langle L_x^\dagger L_x\rangle \gamma\Delta t \\
		M_0 |\psi\rangle & p = 1-\langle L_x^\dagger L_x\rangle \gamma\Delta t
	\end{cases},
\end{equation}
We can introduce a Poisson variable $dW_x$ satisfying 
\begin{equation}
	dW_x dW_y = \delta_{xy} dW_x,\quad \overline{dW_x} = \langle L_x^\dagger L_x\rangle\gamma dt,
\end{equation}
and the evolution can be cast into the stochastic differential equation
\begin{equation}
\begin{aligned}
	d|\psi\rangle = & -iHdt |\psi\rangle + \sum_x \frac{L_x-\sqrt{\langle L_x^\dagger L_x\rangle}}{\sqrt{\langle L_x^\dagger L_x\rangle}}dW_x|\psi\rangle - \frac{\gamma}{2} \sum_x \left(L_x^\dagger L_x-\langle L_x^\dagger L_x\rangle\right)dt 
	  |\psi\rangle.
\end{aligned}
\end{equation}
The $-\langle L_x^\dagger L_x\rangle dt |\psi\rangle$ comes from the renormalization. 
For numerical simulation, we can ignore it.

Note that in the numerical simulation, after each quantum jump, the state should be renormalized so that the jump probability for other $L_x$ can be computed correctly; this requires several renormalization procedures in a single time step. 

When applied to a quasi-particle creation/annihilation operator, a Diract free fermion state maintains its structure. Consider a general quasi-particle $b^\dagger = \sum_i b_i c_i^\dagger$, creating a quasiparticle is simply adding a column to $B$:\footnote{The new column $b$ is not orthogonal to linear space $B$. Therefore, an orthogonalization procedure is needed to obtain canonical form.}
\begin{equation}
	b^\dagger|B\rangle = \sum_k b_k c^\dagger_k \prod_{j=1}^N \sum_i c_i^\dagger B_{ij} |0\rangle
	= \prod_{j=1}^{N+1} \sum_i c_i^\dagger \left[b|B\right]_{ij} |0\rangle.
\end{equation}
For the quasiparticle annihilation operator $b$, 
\begin{equation*}
	b|B\rangle = \sum_k b_k^* c_k \prod_{j} \sum_i c_i^\dagger B_{ij} |0\rangle
	=\sum_j \langle b|B_j\rangle \bigotimes_{l\ne j}|B_l\rangle.
\end{equation*}
We can use the gauge freedom to restrict $\langle b| B'_{j}\rangle = 0$ for $j>1$. Such matrix $B'$ always exists since we can always find a column $j$ that $\langle b| B_{j}\rangle \ne 0$ (otherwise $p_m=0$ and the jump is impossible). We then move the column to the first and define the column as
\begin{equation}
	|B'_{j}\rangle = |B_{j}\rangle - \frac{\langle a|B_{j}\rangle}{\langle a|B_{1}\rangle} |B_{1}\rangle, \quad j>1.
\end{equation}
In the weak measurement setting, the jump operator is $d^\dagger d$, which first finds the null space of $b$ in $B$, then adds $b$ to $B$. The following function \lstinline{replace_vector!} do the routine:
\begin{lstlisting}
function replace_vector!(B, b)
	c = vec(b'*B)
	p = argmax(abs.(c))
	Bp = B[:, p] / c[p]
	B[:, p] = b
	for i in axes(B,2)
		isequal(i, p) && continue
		for j in axes(B,1)
			B[j, i] -= c[i] * Bp[j]
		end
	end
end	
\end{lstlisting}
While for jump operator $dd^\dagger$, the function \lstinline{avoid_vector!} do the routine:
\begin{lstlisting}
function avoid_vector!(B, b)
	c = vec(b'*B)
	for i in axes(B, 2), j in axes(B, 1)
		B[j, i] -= c[i] * b[j]
	end
end
\end{lstlisting}
Such column transformations do not alter the linear space $B$ spans, while the orthogonality and the normalization might be affected.

The story for BdG system is similar. That is, for jump operator $b^\dagger b$ and $b b^\dagger$, we can treat them as particles in Nabu basis in the equal footing.  




\subsection{Quantum State Diffusion}
Gaussian SSE is another way to unravel the Lindblad evolution.
It is often numerically more efficient since it only requires one renormalization in a single time step.
We first introduce the Wiener processes $dW_x$ satisfying 
\begin{equation}
	\overline{dW_x} = 0,\quad \overline{dW_x dW_y} = \delta_{xy} \gamma dt.
\end{equation}
Here we assume $L_x$ is Hermitian. The Gaussian SSE is then
\begin{equation}
	d |\psi\rangle = -i H dt |\psi\rangle + 
	\sum_x \left(L_x-\langle L_x\rangle \right) d W_x \left|\psi\right\rangle - \frac{\gamma}{2} \sum_x  (L_x - \langle L_x \rangle)^2 dt  \left|\psi\right\rangle.
\end{equation}
To retain the Lindblad, note that 
\begin{equation*}
	d\rho = \overline{|d\psi\rangle\langle\psi|} + \overline{|\psi\rangle \langle d\psi|}+ \overline{|d\psi\rangle\langle d\psi|}.
\end{equation*}
The first two terms give
\begin{equation*}
	-i[H,\rho]dt - \frac{\gamma}{2} \sum_x \left\{L_x^2- 2\langle L_x\rangle L_x + \langle L_x\rangle^2,\rho \right\}dt.
\end{equation*}
The second term gives
\begin{equation*}
	\gamma \sum_x \left[L_x \rho L_x  - \left\{\langle L_x\rangle L_x - \frac{\langle L_x\rangle^2}{2},\rho \right\}\right]dt + O(dt^2).
\end{equation*}
We, therefore, recover the Lindblad equation after averaging the SSE.

In numerical simulation, we exponentiate the expression, 
\begin{equation}
	|\psi'\rangle \sim \exp\left(A dt + \sum_x B_x dW_x \right) e^{-iH\Delta t}|\psi\rangle.
\end{equation}
The Taylor expansion of the exponent to the lowest order gives
\begin{equation*}
	e^{A dt + B \sum_x dW_x} = 1 + \left(A + \frac{1}{2} \sum_x B_x^2\right) dt + \sum_x B_x dW_x
\end{equation*}
Compared with the SSE, we get
\begin{equation*}
	A = -\gamma \sum_x (L_x-\langle L_x\rangle)^2,\quad
	B_x = L_x-\langle L_x\rangle.
\end{equation*}
The simplest example is when $L_x = n_x$ is a (quasi-)particle number operator, $n_x^2=n_x$, then we can simulate the SSE by the following form
\begin{equation}
	|\psi'\rangle \propto \exp\left\{\sum_x [dW_x + \gamma (2\langle n_x\rangle-1)]n_x dt \right\} e^{-iH\Delta t}|\psi\rangle,
\end{equation} 
where we have ignored the normalization term. After each time step, there should be a normalization procedure.

Consider local Lindbladian $\partial_t \rho = -i[H,\rho] + \sum_x \mathcal{M}_x[\rho]$.
The Krause operator of $\mathcal M_x$ is
\begin{equation}
	M_x^{(1)} = L_x \sqrt{\gamma \Delta t},\quad
	M_x^{(0)} = \sqrt{1-L_x^\dagger L_x \gamma \Delta t}.
\end{equation}
We consider the case where the jump operator $L_x$ the form:
\begin{equation}
	L_x = U_x d_x^\dagger d_x,\quad
	L_x^\dagger L_x = d_x^\dagger d_x,
\end{equation}
where $U_x$ is a Gaussian unitary operator. In this case, the Kraus operator is 
\begin{equation*}
\begin{aligned}
	M_x^{(0)} &= \sqrt{(1-\gamma\Delta t)d_x^\dagger d_x + d_x d_x^\dagger} 
	= \sqrt{1-\gamma\Delta t} d_x^\dagger d_x + d_x d_x^\dagger \\
	&= 1 - \left(1-\sqrt{1-\gamma\Delta t}\right) d_x^\dagger d_x
	= \exp\left[\log\left(1-\sqrt{1-\gamma\Delta t}\right) d_x^\dagger d_x\right],
\end{aligned}
\end{equation*}
which preserves Gaussianity. 









\end{document}
