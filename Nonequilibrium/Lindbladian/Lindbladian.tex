\documentclass[aps,prx,superscriptaddress,nofootinbib]{revtex4}
\usepackage{amsfonts}
\usepackage{amsmath}
\usepackage{amsthm}
\usepackage{amssymb}
\usepackage{graphbox}
\usepackage{graphicx}
\usepackage{caption}
\usepackage{bm}
\usepackage{bbm}
\usepackage{cancel}
\usepackage{color}
\usepackage{mathrsfs}
\usepackage[colorlinks,bookmarks=true,citecolor=blue,linkcolor=red,urlcolor=blue]{hyperref}
\usepackage{simpler-wick}
\usepackage{appendix}
\usepackage{float}
\usepackage{array}
\usepackage{booktabs}
\usepackage{filecontents}
\usepackage[export]{adjustbox}
\setlength{\parindent}{10 pt}
\setlength{\parskip}{2 pt}
\setcounter{MaxMatrixCols}{30}
\bibliographystyle{apsrev}
\newcommand{\RNum}[1]{\uppercase\expandafter{\romannumeral #1\relax}}
\newcommand{\normord}[1]{{:\mathrel{#1}:}}
\newcolumntype{M}[1]{>{\centering\arraybackslash}m{#1}}
\def\tbs{\textbackslash}
\def \tr{\operatorname{tr}}
\def \Tr{\operatorname{Tr}}
\def \Pf{\operatorname{Pf}}

\newtheorem{theorem}{Theorem}
\newtheorem{definition}{Definition}

\begin{filecontents}{references.bib}
@misc{bravyi2004lagrangian,
      title={Lagrangian representation for fermionic linear optics}, 
      author={Sergey Bravyi},
      year={2004},
      eprint={quant-ph/0404180},
      archivePrefix={arXiv},
      primaryClass={quant-ph}
}
\end{filecontents}


\begin{document}
\title{Lindblad Equation}
\author{Jie Ren}




\maketitle


\tableofcontents


\section{Lindblad Master Equation}
\subsection{General Markovian Form}
For general open quantum evolution, suppose the system and environment are separable initially: $\rho_T=\rho\otimes\rho_B$, where we assume $\rho_B=\sum_\alpha \lambda_\alpha |\phi_\alpha\rangle\langle\phi_\alpha |$. Then the evolution of system-bath is unitary: $\rho_T(t) = U(t)\rho_TU^\dagger(t)$. Trace out the environment's degrees of freedom, we have the quantum channel expression: 
\begin{equation}
	\rho(t) = \sum_{\alpha\beta} W_{\alpha\beta} \rho W^\dagger_{\alpha\beta},\quad 
	W_{\alpha\beta} = \sqrt{\lambda_\beta} \langle\phi_\alpha|U(t)|\phi_\beta\rangle.
\end{equation}
In general, the evolution of an open quantum system has the form $\rho(t) = \mathcal{L}_t[\rho]$. The Lindblad equation assumes a semi-group relation: $\mathcal{L}_t = \lim_{N \rightarrow \infty} \mathcal{L}_{t/N}\cdot\mathcal{L}_{t/N}\cdots \mathcal{L}_{t/N}$. Such time decimation implies that the evolution is Markovian. We will show that Markovian approximation leads directly to the Lindblad equation. First, we choose a complete operator basis $\{F_i\}$ in $N$-dimensional Hilbert space, satisfying $\Tr[F_i^\dagger F_j] = \delta_{ij}$, where we choose $F_0=N^{-1/2} \cdot\mathbb I$. For a quantum channel, the channel operator $K_\mu$ can be expanded as $K_\mu = \sum_i \Tr[F_i^\dagger K_\mu]F_i$. In general, we have:
$$
\mathcal{L}_t[\rho] = \sum_{ij}c_{ij}(t)F_i\rho F_j^\dagger,
$$
where the Hermitian coefficient $c_{ij}(t)$ is $c_{ij}(t) = \sum_{\mu} \Tr[F_i^\dagger K_\mu]\cdot \Tr[F_j^\dagger K_\mu]^*$. Our target is to compute the limit
$$
\frac{d}{dt} \rho \equiv \lim_{t\rightarrow 0} \frac{1}{t}(\mathcal{L}_t[\rho]-\rho).
$$
For this purpose, we define the (Hermitian) coefficient $a_{ij}$ as:
$$
a_{00} = \lim_{t\rightarrow 0} \frac{c_{00}(t)-N}{t}, \quad
a_{ij} = \lim_{t\rightarrow 0} \frac{c_{ij}(t)}{t}.
$$
The limit is then
$$
	\frac{d}{dt}\rho 
	= \frac{a_{00}}{N}\rho + \frac{1}{\sqrt N} \sum_{i>0} \left(a_{i0} F_i \rho + a_{i0}^*\rho F_i^\dagger\right) + \sum_{i,j>0} a_{ij} F_i \rho F_j^\dagger. 
$$
To further simplify the expression, we define
$$
	F = \frac{1}{\sqrt N} \sum_{i=1}^{N^2-1} a_{i0} F_i, \quad
	G = \frac{1}{2N}a_{00}\mathbb I +\frac{1}{2}(F^\dagger+F), \quad
	H = \frac{1}{2i}(F^\dagger-F).
$$
The limit can be expressed by $G,H$ in a compact form:
\begin{equation}
	\frac{d\rho}{dt} = -i[H,\rho]+\{G, \rho\}+\sum_{i,j=1}^{N^2-1}a_{ij}F_i\rho F_j^\dagger.
\end{equation}
Note the $[H,\rho]$ part is the traceless part and the $\{G,\rho\}$ is the trace part. Since the quantum channel preserves the trace (for any $\rho$):
$$
\Tr\left[\frac{d\rho}{dt}\right]= \Tr\left[ \left(2G+\sum_{i,j=1}^{N^2-1}a_{ij}F_j^\dagger F_i \right)\rho \right]=0.
$$
Therefore $G = -\frac{1}{2}\sum_{i,j=1}^{N^2-1}a_{ij}F_j^\dagger F_i$. We thus obtain the Lindblad form:
$$
\frac{d\rho}{dt} = -i[H,\rho]+\sum_{i,j=1}^{N^2-1}a_{ij} \left(F_i\rho F_j^\dagger-\frac{1}{2}\{F_j^\dagger F_i, \rho\} \right).
$$
We can further simplify the form by diagonalizing the matrix $a_{ij}$. It is a convention to take the norm of $a_{ij}$ out to indicate the strength of the system-bath coupling, and the diagonalized Lindblad equation is
\begin{equation}\label{eq:lindbladian}
	\frac{d\rho}{dt} = -i[H,\rho]+ \gamma\sum_{m} \left(L_m\rho L_m^\dagger-\frac{1}{2}\{L_m^\dagger L_m, \rho\} \right).
\end{equation}




\subsection{First Principal Deduction}
In this section, we consider a general system-bath coupling:\footnote{Without loss of generality, we can also assume $\Vert A_k \Vert =1$, $\Tr[\rho_B B_k]=0$.}
\begin{equation}
	H_T = H + H_B + V, \quad V = \sum_k A_k \otimes B_k.
\end{equation}
We will show under certain condition, the dynamics of the system is well approximated by the Lindblad equation. We first assume that initially, the total system is a product state 
$$\rho_T(0) = \rho(0) \otimes \rho_B.$$ 
In the following, we will adopt the interacting picture, where the density operator evolves as 
$$\partial_t \rho_T(t) = -i[V(t), \rho_T(t)] \equiv -i\mathcal V(t) |\rho_T(t) \rangle.$$
Note that in the last equality, $\rho_T$ is expressed as a ket in the Hilbert space of linear operator, and the commutator with $V$ is expressed as a superoperator $\mathcal V$. This notation can simplify the expression. For example, the inner product in the operator space is the trace, so the partial trace operation can be denoted as $|\rho\rangle = \langle \mathbb I_B|\rho_T\rangle$. The evolution of the system is then 
\begin{equation*}
\begin{aligned}
	\frac{d}{dt} |\rho(t)\rangle &= -i \langle \mathbb I_B|\mathcal V(t) |\rho_T(t)\rangle 
	= -i \langle \mathbb I_B|\mathcal{V}(t) |\rho_T(0)\rangle - \int_0^t \langle \mathbb I_B| \mathcal{V}(t) \mathcal{V}(\tau) |\rho_T(\tau)\rangle d\tau \\
	&= - \int_0^t \langle \mathbb I_B| \mathcal{V}(t) \mathcal{V}(\tau) |\rho_T(\tau)\rangle d\tau.
\end{aligned}
\end{equation*}
Now we are taking the \textbf{Born approximation}, which states when the coupling is weak enough compared with the energy scale of the system and the bath, the total density matrix is approximated by the product state $|\rho_T(t)\rangle \approx |\rho(t)\rangle \otimes |\rho_B\rangle$. The evolution is now
\begin{equation*}
\begin{aligned}
	\frac{d}{dt} \rho(t) &\approx \int_0^t \mathrm{Tr}_B\left[ V(t) \rho_T(\tau) V(\tau)- \rho_T(\tau) V(\tau) V(t) \right]d\tau +h.c. \\
	&= \sum_{kl}\int_0^t d\tau\ C_{lk}(\tau - t) \left[A_k(t)\rho(\tau)A_l(\tau)-\rho(\tau)A_l(\tau)A_k(t)\right]+h.c.,
\end{aligned}
\end{equation*}
where $C_{kl}(t) \equiv \mathrm{Tr}_B[\rho_B B_k(t) B_l ]$ is the correlation function of $B_k$'s.  We then take the \textbf{Markovian approximation} which assumes that the correlations of the bath decay fast in time. We can thus make the substitution $\rho(\tau) \rightarrow \rho(t)$, the result equation of motion is Markovian:
\begin{equation*}
\begin{aligned}
	\frac{d}{dt} \rho(t) &\approx \sum_{kl}\int_{0}^{t}dt' C_{lk}(-t') \left[A_k(t)\rho(t)A_l(t-t')-\rho(t)A_l(t-t')A_k(t)\right]+h.c. \\
	&= \sum_{k} \int_0^t dt \left[A_k \rho B_{k}-\rho B_{k} A_k+h.c.\right],
\end{aligned}
\end{equation*}
where we have defined $B_{k}(t) = \sum_l \int_0^{\infty} dt' A_l(t-t')C_{lk}(-t')$. Now we switch to the frequency domain,
$$
	A_k(t) = \sum_\omega A_{k}(\omega) e^{-i\omega t}, \quad
	B_k(t) = \sum_{l,\omega} e^{-i\omega t} A_l(\omega)\Gamma_{lk}(\omega), \quad
	\Gamma_{kl}(\omega) = \int_0^\infty dt\ e^{i\omega t}C_{kl}(t).
$$
We then take the \textbf{rotating wave approximation}, where we only keep the contributions from canceling frequency of operator $A$ and $B$,
\begin{equation}
\begin{aligned}
	\frac{d}{dt}\rho(t) &= \sum_{\omega} \left[\Gamma_{lk}(\omega) A_k(\omega) \rho A_l(\omega) - \Gamma_{lk}(\omega)\rho A_l(\omega) A_k(\omega) + h.c. \right] \\
	&= \sum_{\omega} \gamma_{kl}(\omega)(A_{l,\omega}\rho A_{k,\omega}^\dagger-\frac{1}{2}\{\rho,A_{k,\omega}^\dagger A_{l,\omega}\}) -i\left[\sum_{\omega}S_{kl}(\omega)A_{k,\omega}^\dagger A_{l,\omega},\rho\right],
\end{aligned}
\end{equation}
where we defined
$$
\gamma_{kl}(\omega) = \Gamma_{kl}(\omega) +\Gamma^*_{lk}(\omega), \quad
S_{kl}(\omega) = \frac{\Gamma_{kl}(\omega) - \Gamma^*_{lk}(\omega)}{2i}.
$$
The matrices $\gamma(\omega)$ are positive, we can then take the square root of them. The jump operator is then 
$$L_{i,\omega} = \sum_j \sqrt{\gamma_{ij}(\omega)}A_{j,\omega}.$$ 
The evolution is then in the Lindblad form.



\subsection{Stochastic Schr\"{o}dinger Equation}

The Lindblad form Eq.~(\ref{eq:lindbladian}) is equivalent to the stochastic Schr\"{o}dinger equation (SSE):
\begin{equation}
	d|\psi\rangle = -iH|\psi\rangle + A[\psi]dt + B[\psi]dW,
\end{equation}
where $dW$ is a stochastic infinitesimal element. The expectation value is then the average over all possible evolution path (trajectory): $\langle O(t) \rangle = \overline{\langle\psi(t)|O|\psi(t)\rangle}$.
\subsubsection{Poisson SSE}
Consider a small time interval $\Delta t$, the Lindblad equation is equivalent to the quantum channel $\rho(t+\Delta t) = M_0 \rho(t) M_0^\dagger + \sum_m M_m \rho M_m^\dagger$, where
$$
	M_0 = 1 - i\left(H - i\frac{\gamma}{2} \sum_m L_m^\dagger L_m\right)\Delta t, \quad
	M_m = \sqrt{\gamma\Delta t} L_m.
$$
A quantum channel can be simulated by a stochastic evolution of pure states:
\begin{equation}
	|\psi(t+\Delta t)\rangle \propto \begin{cases}
		L_m |\psi(t)\rangle & p = p_m(t) \gamma\Delta t \\
		\exp(-iH_{\mathrm{eff}}\Delta t)|\psi(t)\rangle & p = 1-\sum_m p_m(t)
	\end{cases},\quad\text{where }p_m(t)= \langle\psi(t)|L_m^\dagger L_m|\psi(t)\rangle.
\end{equation}
Here the effective (non-Hermitian) Hamiltonian is
\begin{equation}
	H_{\mathrm{eff}} = H -i\frac{\gamma}{2}\sum_m L_m^\dagger L_m.
\end{equation}
We can introduce a Poisson variable $dW_m$ satisfying 
$$dW_m dW_n = \delta_{mn} dW_m,\quad \overline{dW_m} = \langle L_m^\dagger L_m\rangle\gamma dt,$$
and the evolution can be cast into the stochastic differential equation
\begin{equation}
	d|\psi\rangle = -iHdt |\psi\rangle + \sum_m \left[\left(\frac{L_m}{\langle L_m^\dagger L_m\rangle^{\frac{1}{2}}}-1\right)dW_m -\frac{\gamma}{2} \left(L_m^\dagger L_m-\langle L_m^\dagger L_m\rangle\right)dt \right]
	  |\psi\rangle.
\end{equation}
Note that the $-\langle L_m^\dagger L_m\rangle dt |\psi\rangle$ comes from the renormalization. For numerical simulation, we can ignore it.



\subsubsection{Gaussian SSE}

We can also use the Wiener processes $dW_m$ satisfying 
$$\overline{dW_m} = 0,\quad \overline{dW_m dW_n} = \delta_{mn} \gamma dt.$$
The Gaussian SSE is
\begin{equation}
	d |\psi\rangle = -i H dt |\psi\rangle + 
	\sum_m \left[\left(L_m-\langle L_m\rangle\right) \mathrm{d} W_{m}-\frac{\gamma}{2}\left(L_m^\dagger-\langle L_m\rangle\right)\left(L_m-\langle L_m\rangle\right) dt\right]\left|\psi\right\rangle.
\end{equation}
To retain the Lindblad, note that $d\rho = \overline{|d\psi\rangle\langle\psi|} + \overline{|\psi\rangle \langle d\psi|}+ \overline{|d\psi\rangle\langle d\psi|}$.
Without going into the detail, we note that $L_m dW_m$ term in $\overline{|d\psi\rangle\langle d\psi|}$ will contribute a term $\gamma L_m \rho L_m^\dagger dt$; $-\frac{\gamma}{2}L_m^\dagger L_m dt$ term in $\overline{|d\psi\rangle\langle\psi|} + \overline{|\psi\rangle \langle d\psi|}$ contribute a tern $-\frac{\gamma}{2}\{L_m^\dagger L_m, \rho\}$ term. All terms involving expectation value can be regarded as coming from the renormalization.



\section{Quadratic Lindbladian}

Consider the Lindblad in the Heisenberg picture:
\begin{equation}
	\frac{d}{dt} \hat O
	= i[\hat H, \hat O] + \sum_\mu \hat L_\mu^\dagger \hat O\hat L_\mu - \frac{1}{2} \sum_\mu\{\hat L_\mu^\dagger \hat L_\mu, \hat O \},
\end{equation}
where we choose $\hat O_{ij} = \omega_i\omega_j$ satisfying the relation $\hat O^T = 2\mathbb I - \hat O$. The covariance matrix is then $\Gamma_{ij} = i\langle \hat O\rangle - i\delta_{ij}$.

We assume that the jump operator has up to quadratic Majorana terms. In particular, we denote the linear terms and the Hermitian quadratic terms as
$$
	\hat L_r = \sum_{j=1}^{2N} L^r_{j} \omega_j, \quad
	\hat L_s = \sum_{j,k=1}^{2N} M^s_{jk} \omega_j \omega_k.
$$
When the \textbf{jump operator} $\hat L_\mu$ contains only the linear Majorana operator, the Lindblad equation preserves Gaussianity. For jump operators containing up to quadratic Majorana terms, the evolution will break the Gaussian form, however, the $2n$-point correlation is still solvable for free fermion systems.

\subsection{Third Quantization}

Assume only linear terms in jump operators,
$$
	\partial_t \hat O = \left[i\hat H,\hat O\right] + \mathcal D_r[\hat O] 
	= \left[i\hat H - \frac{1}{2}\sum_r \hat L_r^\dagger L_r,\hat O\right] + \sum_r \left[\hat L_r^\dagger,\hat O\right]\hat L_r.
$$
Define $B\equiv \sum_r L_i^r L_j^{r*}$, the first term of EOM is:\footnote{Use the commutation relation $\{\omega_i, \omega_j\} = 2\delta_{ij}$, we have the relation $[\omega_k,\omega_i \omega_j] = 2(\delta_{ki}\omega_j-\delta_{kj}\omega_i)$ and $[\omega_k \omega_l, \omega_i \omega_j] = 2(\delta_{ki}\omega_j \omega_l-\delta_{kj} \omega_i \omega_l + \delta_{li}\omega_k \omega_j - \delta_{lj}\omega_k\omega_i)$.}
\begin{equation*}
\begin{aligned}[]
	\left[i\hat H - \frac{1}{2}\sum_r \hat L_r^\dagger L_r,\hat O_{ij}\right]
	&= \sum_{kl}\left(\frac{1}{4}H-\frac{1}{2}B \right)_{kl} [\omega_k \omega_l, \omega_i \omega_j] \\
	&= \sum_{kl} \left(\frac{1}{2}H- B\right)_{kl} (
		\delta_{ki}\omega_j \omega_l-\delta_{kj} \omega_i \omega_l + 
		\delta_{li}\omega_k \omega_j - \delta_{lj}\omega_k\omega_i
	) \\
	&= \left[
		\left(\frac{1}{2}H- B\right) \cdot \hat O^T + \left(\frac{1}{2}H- B\right)^T \cdot \hat O
		- \hat O \cdot \left(\frac{1}{2}H- B\right)^T- \hat O^T \cdot \left(\frac{1}{2}H- B\right)
	\right]_{ij} \\
	&= \left[
		(-H+2B^I) \cdot \hat O + \hat O \cdot (H-2B^I)
	\right]_{ij}
\end{aligned}
\end{equation*}
The second term is
\begin{equation*}
\begin{aligned}
	\sum_r \left[L_r^\dagger, \hat O_{ij}\right] \hat L_r
	&= \sum_{kl} B_{kl} [\omega_k, \omega_i \omega_j]\omega_l
	= 2\sum_{kl} B_{kl}\left(
		\delta_{ki} \omega_j \omega_l - 
		\delta_{kj} \omega_i \omega_l\right) \\
	&= \left[2B\cdot \hat O^T - 2\hat O\cdot B^T\right]_{ij}
	= \left[-2B\cdot \hat O - 2\hat O\cdot B^* + 4B\right]_{ij}
\end{aligned}
\end{equation*}
Therefore
$$
	\partial_t \hat O_{ij} = \left[
		(-H-2B^R) \cdot \hat O + 
		\hat O \cdot (H-2B^R) + 4B
	\right]_{ij}
$$
The EOM of the covariance matrix is then
\begin{equation}
	\partial_t \Gamma = X^T\cdot\Gamma + \Gamma \cdot X + Y,
\end{equation}
where $X = H - 2B^R$, $Y = 4B^I$. Note that the constant part is replaced by its anti-symmetric part.

The steady state of the system is solved by the Lyapunov equation
\begin{equation}
	X^T\cdot\Gamma + \Gamma \cdot X = - Y.
\end{equation}


\subsection{Quadratic Jump Operators}

Now include the Hermitian quadratic quantum jumps:
\begin{equation}
\begin{aligned}
	\partial_t \hat O &= i[\hat H, \hat O] + \mathcal D_r[\hat O] + \mathcal D_s[\hat O], \\
	\mathcal D_s[\hat O] 
	&= \sum_s \hat L_s \hat O\hat L_s - \frac{1}{2} \sum_r\{\hat L_s^2, \hat O \}
	= -\frac{1}{2} \sum_s [\hat L_s,[\hat L_s,\hat O]].
\end{aligned}
\end{equation}

\subsubsection{Majorana Case}
A direct calculation gives
\begin{equation*}
\begin{aligned}
	D_s[\hat O]
	&= -\frac{1}{2} \sum_s \sum_{kl} M^s_{kl}\langle[\hat L_s,[\omega_k \omega_l, \omega_i \omega_j]] \\
	&= 2\sum_s \sum_{k} \left\{ M^s_{ik}[\hat L_s,\omega_k \omega_j]-[\hat L_s,\omega_i \omega_k]M^s_{kj} \right\} \\
	&= 8\sum_{s,kl} \left[ M^s_{ik}(-M^s_{kl}\omega_l\omega_j+\omega_k\omega_l M^s_{lj})+(M^s_{il}\omega_l\omega_k-\omega_i\omega_l M^s_{lk})M^s_{kj} \right] \\
	&= 8\sum_s \left[2 M^s \cdot\hat O\cdot M^s-(M^s)^2 \cdot \hat O - \hat O\cdot(M^s)^2 \right]_{ij}.
\end{aligned}
\end{equation*}
Together, we get the EOM of the variance matrix $\Gamma_{ij}$:
\begin{equation}
	\partial_t \Gamma = X^T\cdot\Gamma + \Gamma \cdot X + \sum_s (Z^s)^T \cdot \Gamma\cdot Z^s + Y,
\end{equation}
where
\begin{equation}
	X = H - 2B^R + 8 \sum_s (\mathrm{Im} M^s)^2, \quad
	Y = 4B^I, \quad 
	Z = 4 M^s.
\end{equation}


\subsubsection{Dirac Fermion Case}

In this section, we consider the free fermion system preserving the U(1) charge. The jump operators are assumed to be quadratic: $\hat L_s = \sum_{jk} M^s_{jk} c_j^\dagger c_k$ where $\{M^s\}$ are Hermitian matrices.

For the fermion case, we choose $\hat O_{ij} = c_i^\dagger c_j$, and consider the Lindbladian
$$
	\partial_t \hat O = i[\hat H, \hat O] + \mathcal D_s[\hat O]
	= i[\hat H, \hat O] - \frac{1}{2} \sum_s [\hat L_s,[\hat L_s,\hat O]],
$$
where each $\hat L_s = M^s_{ij} c_i^\dagger c_j$ is a Hermitian fermion bilinear.


The Hamiltonian part is:\footnote{Using the fact
$[c_k^\dagger c_l, c_i^\dagger c_j] = c_k^\dagger[c_l,c_i^\dagger c_j] + [c_k^\dagger,c_i^\dagger c_j]c_l =\delta_{il}c_k^\dagger c_j -\delta_{jk}c_i^\dagger c_l$, we know that for a quadratic form $\hat A = \sum_{ij} A_{ij} c_i^\dagger c_j$, $[\hat A, \hat O_{ij}] = [A^T, \hat O]_{ij}$.}
$$
	i \sum_{kl}H_{kl}[c_k^\dagger c_l, c_i^\dagger c_j]
	= i \sum_{kl} H_{kl} (\delta_{il}c_k^\dagger c_j -\delta_{jk}c_i^\dagger c_l)
	= i [H^T\cdot \hat O - \hat O\cdot H^T]_{ij}.
$$
Similarly, the double commutation in the second term is:
$$
	\mathcal D_s[\hat O]
	= -\frac{1}{2}\sum_s [(M^{s*})^2\cdot\hat O + \hat O\cdot (M^{s*})^2 - 2 M^{s*}\cdot \hat O \cdot M^{s*}].
$$
Together, the EOM of correlation $G_{ij} = \langle c_i^\dagger c_j\rangle$ is
\begin{equation}
	\partial_t G = X^\dagger \cdot G + G \cdot X + \sum_s M^{s*}\cdot G \cdot M^{s*},
\end{equation}
where $X = -i H^* - \frac{1}{2}\sum_s (M^{s*})^2$.



\section{Fermionic Gaussian States}

In this section, we discuss the general fermionic Gaussian state, in the framework of the Grassmann representation.
We will closely follow Ref.~\cite{bravyi2004lagrangian}.

\subsection{Grassmann Representation}
The Majorana operators are defined as $\hat{\omega}^a_j =\hat{c}_{i}+\hat{c}_{i}^{\dagger}$, $\hat{\omega}^b_j = i(\hat{c}_{i}-\hat{c}_{i}^{\dagger})$. A general operator in Fermionic Fock space can be expanded on the Majorana basis:
\begin{equation}
	\hat{X}=\alpha\hat{I}+\sum_{p=1}^{2n}\sum_{1\le a_{1}<\cdots<a_{p}\le2n}\alpha_{a_{1}\cdots a_{p}}\hat{\omega}_{a_{1}}\cdots\hat{\omega}_{a_{p}}.
\end{equation}
Define a linear map from Fermionic operator space to Grassmann algebra:
\begin{equation}
	\hat X \mapsto X(\theta)=\alpha + \sum_{1\le a_{1}<\cdots<a_{p}\le2n}\alpha_{a_{1}\cdots a_{p}}\theta_{a_{1}}\cdots \theta_{a_{p}}.
\end{equation}
This mapping is called the Grassmann representation of $\hat X$. 

One can formally define calculus on Grassmann algebra:
\begin{equation}
	\frac{\partial}{\partial\theta_{i}}\theta_{j} = \int d\theta_{i}\theta_{j}=\delta_{ij},\quad
	\frac{\partial}{\partial\theta_{i}}1 = \int d\theta_{i}1=0.
\end{equation}
The Gaussian integral of Grassmann algebra is
\begin{equation}
	\int D\theta \exp\left(\eta^T\theta+\frac{i}{2}\theta^T M\theta\right)
	=i^n \operatorname{Pf}(M) \exp\left(-\frac{i}{2}\eta^T M^{-1}\eta\right).
\end{equation}
One useful result concerning the expectation value is
\begin{theorem}
For two operator $\hat X$ and $\hat Y$, we have the following identity 
$$
\Tr\left(\hat{X}\hat{Y}\right)=\left(-2\right)^{n}\int D[\theta,\mu] e^{\theta^{T}\cdot \mu} X(\theta) Y(\mu).
$$
where $\int D\theta=\int d\theta_{2n}\cdots\int d\theta_{1}$, $\int D\mu =\int d\mu_{2n}\cdots\int d\mu_{1}$.
\end{theorem}
\begin{proof}
We prove the statement by considering only $m$-th order monomial. On the one hand 
$$
\text{LHS}=\Tr[\hat\omega_1 \cdots \hat\omega_m \hat\omega_1 \cdots \hat\omega_m] = 2^{n} (-1)^{m(m-1)/2}.
$$
On the other hand,
\begin{equation*}
\begin{aligned}
	\text{RHS} &= \left(-2\right)^{n} \int D[\theta,\mu] \ \theta_1\cdots\theta_m (\theta_{m+1}\mu_{m+1}\cdots \theta_{2n}\mu_{2n})\mu_1 \cdots \mu_m \\
	&= \left(-2\right)^{n} (-1)^{(4n-m)m+(m+1+2n)(2n-m)/2} \\
	&= 2^n(-1)^{-m(m+3)/2} = 2^n(-1)^{m(m-1)/2}.
\end{aligned}
\end{equation*}
We therefore proved the statement.
\end{proof}


\subsubsection{Gaussian States}

\begin{definition}
A quantum state $\hat \rho$ is Gaussian if it has Gaussian Grassmann representation:
$$\rho(\theta)=\frac{1}{2^{n}}\exp\left(\frac{i}{2}\theta^{T}M\theta\right),$$
where the antisymmetric matrix $M_{ab}=\frac{i}{2}\Tr(\hat \rho[\hat \omega_a,\hat\omega_b])$ is the \textbf{covariance matrix}.
\end{definition}
All higher correlations of a Gaussian state are determined by the Wick theorem, namely
$$
\Tr(i^p\hat\rho\hat\omega_{a_1}\cdots\hat\omega_{a_p})=\operatorname{Pf}(M|_{a_1,\dots,a_p}).
$$
The canonical form of antisymmetric matrix $M$ is:
$$
M = R \begin{bmatrix}
	0 & \mathrm{diag}(\lambda_1,\cdots,\lambda_n) \\ 
	-\mathrm{diag}(\lambda_1,\cdots,\lambda_n) & 0
\end{bmatrix} R^T, \quad
R \in \mathrm{SO}(2n).
$$
Under the new Grassmann variance $\mu = R\theta$, $\rho$ has the form
\begin{equation}
	\rho(\mu)
	=\frac{1}{2^{n}}\prod_{j}\exp\left(i\lambda_j\mu_{j} \mu_{j+n}\right)
	=\frac{1}{2^{n}}\prod_{j}\left(1+i\lambda_{j} \mu_{j} \mu_{j+n}\right).
\end{equation}
We can then obtain the operator form: 
\begin{equation}
	\hat\rho = 2^{-n} \prod_{j=1}^n(1+i\lambda_j \hat\gamma_j\hat\gamma_{j+n})
\end{equation}
where $\hat \gamma$'s are a new set of Majorana operators. In the fermion basis
\begin{equation}
	\hat d_j = \frac{\hat\gamma_j - i \hat\gamma_{j+n}}{2},\quad
	\hat d_j^\dagger = \frac{\hat\gamma_j + i \hat\gamma_{j+n}}{2},
\end{equation}
the density matrix has the form
\begin{equation}\label{eq:gaussian-std-form}
	\hat\rho
	= \prod_{j}\left(\frac{1+\lambda_j}{2}-\lambda_{j}d_{j}^{\dagger}d_{j}\right) 
	=\bigotimes_j \begin{bmatrix}
		\frac{1+\lambda_j}{2} & 0 \\
		0 & \frac{1-\lambda_j}{2}
	\end{bmatrix}_j.
\end{equation}
Without loss of generality, we assume $\lambda_i \ge 0$. For pure state, $\lambda_i =1,\ \forall i$.
For mixed state, the entropy of $\rho$ is just 
\begin{equation}
	S(\hat\rho)=\sum_j H\left(\frac{1+\lambda_j}{2}\right)
	= -\sum_j \left[\left(\frac{1+\lambda_j}{2}\right)\log\left(\frac{1+\lambda_j}{2}\right)+\left(\frac{1-\lambda_j}{2}\right)\log\left(\frac{1-\lambda_j}{2}\right)\right].
\end{equation}



\subsubsection{Gaussian Operators}
\begin{definition}
An operator $\hat X$ (with nonzero trace) is Gaussian if 
$$
X(\theta)=C\exp\left(\frac{i}{2}\theta^TM\theta\right)
$$
for some complex number $C$ and some \textbf{complex antisymmetric} matrix $M$. $M$ is called a correlation matrix of $\hat X$. 
If $\hat X$ is traceless, it should be thought of as a limit $\hat X = \lim_{m\rightarrow\infty} \hat X_m$ for some converging sequence of Gaussian operators with nonzero trace. 
\end{definition}
Note that for traceless $\hat X$, the explicit form of $X(\theta)$ is
\begin{equation}
	X(\theta)=C \left(\prod_{a=1}^{2k}\mu_a\right)\exp\left(\frac{i}{2}\sum_{a,b=2k+1}^{2n} M_{ab}\mu_a \mu_b\right),
\end{equation}
where $\mu_a = \sum_b T_{ab}\theta_b$ for some invertible complex matrix $T$. The factor is a limiting point of the sequence:
$$
\prod_{a=1}^{2k} \mu_a = \lim_{t\rightarrow\infty} \prod_{a=1}^k \left(\mu_{2a-1}\mu_{2a}+\frac{1}{t}\right) = \lim_{t\rightarrow\infty} \frac{1}{t^k} \exp\left(t\sum_{a=1}^k \mu_{2a-1}\mu_{2a}\right).
$$
Introducing the operator $\hat\Lambda \equiv \sum_{a=1}^{2n} \hat \omega_a \otimes \hat \omega_a$, we have the following theorem:
\begin{theorem}
An operator $\hat X$ is Gaussian iff $\hat X$ is even and satisfies $$[\hat\Lambda, \hat X\otimes \hat X]=0.$$
\end{theorem}
\begin{proof}
The adjoint action of $\hat \Lambda$ in the Grassmann representation has the form:
\begin{equation}
	\Lambda_\text{ad} = 2\sum_a\left(\theta_a\otimes \frac{\partial}{\partial\theta_a}+\frac{\partial}{\partial\theta_a}\otimes \theta_a\right) \equiv \sum_a \Delta_a.
\end{equation}
That is, $[\hat\omega_a\otimes \hat \omega_a, Y \otimes Z](\theta)= \Delta_a \cdot Y(\theta)\otimes Z(\theta)$ for any operators $Y,Z$ having the same parity. Without loss of generality, both $Y$ and $Z$ are monomials in $\hat\omega$'s. In this case each of them either commutes or anticommutes with $\hat{\omega}_a$. Consider two cases:
\begin{enumerate}
	\item Both $Y$ and $Z$ contain $\hat{\omega}_a$, or both $Y$ and $Z$ do not contain $\hat{\omega}_a$. Then the commutator $[\hat{\omega}_a\otimes \hat{\omega}_a,Y\otimes Z]$ is zero since both factors yield the same sign. The right-hand side is also zero, since either $\theta_a$ or $\partial/\partial{\theta_a}$ annihilates both $Y$ and $Z$.
	\item $Y$ contains $\hat{\omega}_a$ while $Z$ does not contain $\hat{\omega}_a$ (or vice verse). In this case $\hat{\omega}_a\otimes\hat{\omega}_a$ anticommutes with $Y\otimes Z$. Let us write $Y=\hat{\omega}_a \tilde{Y}$, where $\tilde{Y}$ is a monomial which does not contain $\hat{\omega}_a$. We have: $$[\hat{\omega}_a\otimes \hat{\omega}_a,Y\otimes Z]=2(\hat{\omega}_a\otimes \hat{\omega}_a)(Y\otimes Z) = 2\tilde{Y}\otimes
(\hat{\omega}_a Z).$$
On the other hand,
$$\theta_a\otimes \frac{\partial}{\partial\theta_a} \cdot Y \otimes Z =0,\quad\frac{\partial}{\partial\theta_a}\otimes \theta_a \cdot Y\otimes Z = \tilde{Y}\otimes \theta_a Z.$$
We again get equality.
\end{enumerate}

\noindent\textbf{Necessity:}
Note that $\Lambda_\text{ad}$ is invariant under change of variables since
$$
\mu_a = \sum_b T_{ab}\theta_b,\quad \frac{\partial}{\partial\mu_a} = \sum_b(T^{-1})_{ab}\frac{\partial}{\partial\theta_b} \ \Longrightarrow\ \sum_a \theta_a\otimes\frac{\partial}{\partial\theta_a} = \sum_a \mu_a\otimes\frac{\partial}{\partial\mu_a}.
$$
Direct application of the operator to the general Gaussian form will prove the necessity.

\noindent\textbf{Sufficiency:} Denote $C=2^{-n}\tr{(X)}\equiv X(0)$ and represent $X(\theta)$ as
$$X(\theta)= C\cdot 1 + \frac{iC}2\sum_{a,b=1}^{2n} M_{ab}\,\theta_a \theta_b + \mbox{higher order terms}.$$
Applying a differential operator $1\otimes \frac{\partial}{\partial\theta_b}$ to both sides:
$$
\sum_{a=1}^{2n} \left(\theta_a X \otimes \frac{\partial^2}{\partial \theta_b \partial \theta_a} X - \frac{\partial}{\partial\theta_a} X \otimes \theta_a\frac{\partial}{\partial\theta_b} X \right) + \frac{\partial}{\partial\theta_b} X \otimes X = 0.
$$
Now let us put $\theta\equiv 0$ in the second factor:
$$
\frac{\partial}{\partial\theta_b} X = i\sum_{a=1}^{2n} M_{ba} \theta_a X.
$$
This differential equation can be easily solved by $X(\theta)=C \exp{\left( \frac{i}2\, \theta^T M \theta \right)}$. 

For general cases, we denote $\mathcal K \subseteq \mathcal M_1$ a subspace spanned by linear functions which annihilate $\hat X$, i.e.
$$
\mathcal K=\left\{ f \in \mathcal M_1 \; : \; f(\theta)X(\theta)=0\right\}.
$$
Let us perform a linear change of variables $\mu_a=\sum_b T_{ab} \theta_b$, with $T$ being an invertible complex matrix chosen such  that the first $k$ variables $\mu$ span the subspace $\mathcal K$, i.e. $\mathcal K=\operatorname{span}\, [ \mu_1,\ldots,\mu_{2k}]$. From equalities $\mu_j X=0$, $j\in [1,2k]$ it follows that 
$$
X(\theta(\mu))=\left(\prod_a \mu_a \right) \tilde{X}(\mu),
$$
where $\tilde{X}(\mu)$ depends only upon $\mu_{2k+1},\ldots,\mu_{2n}$. The function $\tilde{X}(\mu)$ satisfies the equation
$$
\sum_{a=2k+1}^{2n}\left(\mu_a\otimes \frac{\partial}{\partial\mu_a} + \frac{\partial}{\partial\mu_a}\otimes \mu_a\right)\, \tilde{X}\otimes \tilde{X}=0.
$$
Therefore we get the general Gaussian form.
\end{proof}

\subsubsection{Gaussian Linear Maps}
We define linear maps that preserve Gaussian states as the following:
\begin{definition}
A linear map $\mathcal E$ is Gaussian iff it admits an integral  representation
\begin{equation}
	\mathcal E(X)(\theta) = C \int D[\eta,\mu] \exp{\left[ S(\theta,\eta) + i\eta^T \mu \right]} X(\mu),
\end{equation}
where
\begin{equation}
	S(\theta,\eta)= \frac{i}{2} (\theta^T,\eta^T)
	\begin{pmatrix}
		A & B \\ -B^T & D
	\end{pmatrix}
	\begin{pmatrix}
		\theta \\ \eta
	\end{pmatrix}
\end{equation}
for some complex  $2n\times 2n$ matrices $A$, $B$, $D$, and some complex number $C$.
\end{definition}
Consider a Gaussian operator $\hat X$ which can be described by a correlation matrix $M$ and a Gaussian map $\mathcal E$. Applying the Gaussian integration, one can show that $\mathcal E(X)$ has a correlation matrix
$$
\mathcal E(M) = A + B \left(M^{-1} + D \right)^{-1} B^T = A + B \left(I + MD \right)^{-1} M B^T,
$$
while a pre-exponential factor of the operator $\mathcal E(X)$ can be found from an  identity
$$
\tr{ \left(\mathcal E(X)\right)} = C (-1)^n \Pf(M) \Pf(M^{-1}+D) \tr{(X)}.
$$
The value of $\tr{(\mathcal E(X))}$ can be found up to a factor $\pm 1$ using a regularized version:
$$
\tr{ \left( \mathcal E(X) \right) }^2 = C^2 \det{\left( I + M D \right) } \tr{(X)}^2.
$$



\subsection{Operator Form}


\subsubsection{Dirac Fermion Case}

For particle number conserving systems, the Gaussian state can be represented as a matrix: 
\begin{equation}
	|B\rangle \equiv \prod_{j=1}^N \sum_i B_{ij} c_{i}^\dagger |0\rangle 
	\equiv \bigotimes_{j=1}^N |B_j\rangle.
\end{equation}
Note that the matrix $B$ representing the Gaussian state has the unitary degree of freedom
$$|B\rangle = |B'\rangle, \quad B'_{ij} = \sum_k B_{ik}U_{kj},$$
where $U_{kj}$ is an $N\times N$ unitary matrix. It means that the Gaussian state is determined by the linear subspace that columns of $B$ span. The columns of $B$ need not to orthogonal, while the canonical form can be obtained by the QR decomposition: $B_{L\times N} = Q_{L\times N} \cdot R_{N\times N},$ where the $Q$ matrix is orthonormal and we can set $B' = Q$.

A free fermion state maintains its structure when applied to a quasi-particle creation/annihilation operator. Consider a general quasi-particle $b^\dagger = \sum_i b_i c_i^\dagger$, creating a quasiparticle is simply adding a column to $B$, since
\begin{equation}
	b^\dagger|B\rangle = \sum_k b_k c^\dagger_k \prod_{j=1}^N \sum_i c_i^\dagger B_{ij} |0\rangle
	= \prod_{j=1}^{N+1} \sum_i c_i^\dagger \left[b|B\right]_{ij} |0\rangle
\end{equation}
In general, the new column $b$ is not orthogonal to linear space $B$, therefore orthogonalization procedure is needed to obtain canonical form.

Using the Baker-Campbell-Hausdorff formula $e^X Y e^{-X} = \exp(\operatorname{ad} X) Y$, 
\begin{equation}
	e^{-iHt} c_j^\dagger e^{iHt} = c_k^\dagger[e^{-iHt}]_{kj}\ \Longrightarrow\ 
	e^{-iHt}|B\rangle = \prod_{j=1}^N \sum_i [e^{-iHt}]_{ki} B_{ij} c_{k}^\dagger |0\rangle
	= |e^{-iHt}\cdot B\rangle.
\end{equation}

For the quasiparticle annihilation operator $b$, 
\begin{equation}
	b|B\rangle = \sum_k b_k^* c_k \prod_{j} \sum_i c_i^\dagger B_{ij} |0\rangle
	=\sum_j \langle b|B_j\rangle \bigotimes_{l\ne j}|B_l\rangle.
\end{equation}
We can use the gauge freedom to restrict $\langle b| B'_{j}\rangle = 0$ for $j>1$. Such matrix $B'$ always exists since we can always find a column $j$ that $\langle b| B_{j}\rangle \ne 0$ (otherwise $p_m=0$ and the jump is impossible). We then move the column to the first and define the column as
\begin{equation}
	|B'_{j}\rangle = |B_{j}\rangle - \frac{\langle a|B_{j}\rangle}{\langle a|B_{1}\rangle} |B_{1}\rangle, \quad j>1.
\end{equation}
Such column transformations do not alter the linear space $B$ spans, while the orthogonality and the normalization might be affected. 


\subsubsection{Majorana Case}

For the Majorana case, the canonical form (\ref{eq:gaussian-std-form}) for a Gaussian pure state $|\psi$ can be reformulated as 
\begin{equation}
	|\psi\rangle\langle\psi| = \prod_{j=1}^n \hat d_j^\dagger \hat d_j, \quad
	\hat d_j^\dagger = \frac{\hat\gamma_j + i \hat\gamma_{j+n}}{2} = \sum_{i=1}^n \left(\frac{R_{i,j} + i R_{i,j+n}}{2}\right) \hat c_i + \left(\frac{R_{i+n,j} + iR_{i+n,j+n}}{2}\right) \hat c_i^\dagger.
\end{equation}
Note that the state is annihilated by $\{\hat d_j^\dagger\}$. We can store the information of $|\psi\rangle$ into a $2n\times n$ complex matrix
\begin{equation}
	|\psi\rangle \Longleftrightarrow B = \frac{1}{2} \begin{bmatrix}
		R_{11} + i R_{12} \\ R_{21} + i R_{22}
	\end{bmatrix}.
\end{equation}
The rest of the procedures are parallel to those of the Dirac fermion case.




\bibliography{references.bib}




\end{document}


