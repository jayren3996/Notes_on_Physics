\documentclass{SciPost}

% Prevent all line breaks in inline equations.
%\binoppenalty=10000
%\relpenalty=10000

\hypersetup{
    colorlinks,
    linkcolor={red!50!black},
    citecolor={blue!50!black},
    urlcolor={blue!80!black}
}

\usepackage[bitstream-charter]{mathdesign}
\urlstyle{same}

% Fix \cal and \mathcal characters look (so it's not the same as \mathscr)
\DeclareSymbolFont{usualmathcal}{OMS}{cmsy}{m}{n}
\DeclareSymbolFontAlphabet{\mathcal}{usualmathcal}

\fancypagestyle{SPstyle}{
\fancyhf{}
\lhead{\colorbox{scipostblue}{\bf \color{white} ~Notes on Physics}}
\rhead{{\bf \color{scipostdeepblue} ~Nonequilibrium}}
\renewcommand{\headrulewidth}{1pt}
\fancyfoot[C]{\textbf{\thepage}}
}

\begin{document}

\pagestyle{SPstyle}

\begin{center}{\Large \textbf{\color{scipostdeepblue}{
Lindblad Equation\\
}}}\end{center}

\begin{center}
\textbf{Jie Ren}
\end{center}

\tableofcontents

\section{Lindblad Master Equation}
\subsection{General Markovian Form}
For general open quantum evolution, suppose the system and environment are separable initially: $\rho_T=\rho\otimes\rho_B$, where we assume $\rho_B=\sum_\alpha \lambda_\alpha |\phi_\alpha\rangle\langle\phi_\alpha |$. Then the evolution of system-bath is unitary: $\rho_T(t) = U(t)\rho_TU^\dagger(t)$. Trace out the environment's degrees of freedom, we have the quantum channel expression: $\rho(t) = \sum_{\alpha\beta} W_{\alpha\beta} \rho W^\dagger_{\alpha\beta}$, where the \textbf{Kraus operator} $W$ can be expressed formally as $W_{\alpha\beta} = \sqrt{\lambda_\beta} \langle\phi_\alpha|U(t)|\phi_\beta\rangle$.

The Lindblad equation assumes a semi-group relation: 
\begin{equation}
	\rho_t = \mathcal{L}_t \rho_0 = \lim_{N \rightarrow \infty} \mathcal{L}_{t/N}\cdot\mathcal{L}_{t/N}\cdots \mathcal{L}_{t/N}\rho_0.
\end{equation}
Such time decimation implies that the evolution is Markovian. We will show that Markovian approximation leads directly to the Lindblad equation. First, we choose a complete operator basis $\{F_i\}$ in $N$-dimensional Hilbert space, satisfying $\Tr[F_i^\dagger F_j] = \delta_{ij}$, where we choose $F_0=N^{-1/2} \cdot\mathbb I$. For a quantum channel, the channel operator $K_\mu$ can be expanded as 
\begin{equation}
	K_\mu = \sum_i \Tr[F_i^\dagger K_\mu]F_i.
\end{equation}
In general, we have: $\mathcal{L}_t[\rho] = \sum_{ij}c_{ij}(t)F_i\rho F_j^\dagger$, where the Hermitian coefficient $c_{ij}(t)$ is 
\begin{equation}
	c_{ij}(t) = \sum_{\mu} \Tr[F_i^\dagger K_\mu]\cdot \Tr[F_j^\dagger K_\mu]^*.
\end{equation}
Our target is to compute the limit $\partial_t \rho \equiv \lim_{t\rightarrow 0} \frac{1}{t}(\mathcal{L}_t[\rho]-\rho)$.
For this purpose, we define the (Hermitian) coefficient $a_{ij}$ as:
\begin{equation*}
	a_{00} = \lim_{t\rightarrow 0} \frac{c_{00}(t)-N}{t}, \quad
	a_{ij} = \lim_{t\rightarrow 0} \frac{c_{ij}(t)}{t}.
\end{equation*}
The limit is then
$$
	\frac{d}{dt}\rho 
	= \frac{a_{00}}{N}\rho + \frac{1}{\sqrt N} \sum_{i>0} \left(a_{i0} F_i \rho + a_{i0}^*\rho F_i^\dagger\right) + \sum_{i,j>0} a_{ij} F_i \rho F_j^\dagger. 
$$
To further simplify the expression, we define
\begin{equation*}
	F = \frac{1}{\sqrt N} \sum_{i=1}^{N^2-1} a_{i0} F_i, \quad
	G = \frac{a_{00}}{2N}\mathbb I +\frac{1}{2}(F^\dagger+F), \quad
	H = \frac{1}{2i}(F^\dagger-F).
\end{equation*}
The limit can be expressed by $G,H$ in a compact form:
\begin{equation*}
	\frac{d\rho}{dt} = -i[H,\rho]+\{G, \rho\}+\sum_{i,j=1}^{N^2-1}a_{ij}F_i\rho F_j^\dagger.
\end{equation*}
Note the $[H,\rho]$ part is the traceless part, and the $\{G,\rho\}$ is the trace part. Since the quantum channel preserves the trace (for any $\rho$):
$$
\Tr\left[\frac{d\rho}{dt}\right]= \Tr\left[ \left(2G+\sum_{i,j=1}^{N^2-1}a_{ij}F_j^\dagger F_i \right)\rho \right]=0.
$$
Therefore $G = -\frac{1}{2}\sum_{i,j=1}^{N^2-1}a_{ij}F_j^\dagger F_i$. We thus obtain the Lindblad form:
\begin{equation}
	\frac{d\rho}{dt} = -i[H,\rho]+\sum_{i,j=1}^{N^2-1}a_{ij} \left(F_i\rho F_j^\dagger-\frac{1}{2}\{F_j^\dagger F_i, \rho\} \right).
\end{equation}
We can simplify the form by diagonalizing the matrix $a_{ij}$. It is a convention to take the norm of $a_{ij}$ out to indicate the strength of the system-bath coupling, and the diagonalized Lindblad equation is
\begin{equation}\label{eq:lindbladian}
	\frac{d\rho}{dt} = -i[H,\rho]+ \gamma\sum_{m} \left(L_m\rho L_m^\dagger-\frac{1}{2}\{L_m^\dagger L_m, \rho\} \right).
\end{equation}




\subsection{First Principal Deduction}
In this section, we consider a general system-bath coupling:\footnote{Without loss of generality, we can also assume $\Vert A_k \Vert =1$, $\Tr[\rho_B B_k]=0$.}
\begin{equation}
	H_T = H + H_B + V, \quad V = \sum_k A_k \otimes B_k.
\end{equation}
Under certain conditions, we will show that the dynamics of the system are well approximated by the Lindblad equation. We first assume that initially, the total system is a product state 
$$\rho_T(0) = \rho(0) \otimes \rho_B.$$ 
In the following, we will adopt the interacting picture, where the density operator evolves as 
$$\partial_t \rho_T(t) = -i[V(t), \rho_T(t)] \equiv -i\mathcal V(t) |\rho_T(t) \rangle.$$
In the last equality, $\rho_T$ is expressed as a ket in the Hilbert space of linear operator, and the commutator with $V$ is expressed as a superoperator $\mathcal V$. This notation can simplify the expression. For example, the inner product in the operator space is the trace so that the partial trace operation can be denoted as $|\rho\rangle = \langle \mathbb I_B|\rho_T\rangle$. The evolution of the system is then 
\begin{equation}
\begin{aligned}
	\partial_t |\rho(t)\rangle &= -i \langle \mathbb I_B|\mathcal V(t) |\rho_T(t)\rangle 
	= -i \langle \mathbb I_B|\mathcal{V}(t) |\rho_T(0)\rangle - \int_0^t \langle \mathbb I_B| \mathcal{V}(t) \mathcal{V}(\tau) |\rho_T(\tau)\rangle d\tau \\
	&= - \int_0^t \langle \mathbb I_B| \mathcal{V}(t) \mathcal{V}(\tau) |\rho_T(\tau)\rangle d\tau.
\end{aligned}
\end{equation}
Now we are taking the \textbf{Born approximation}, which states when the coupling is weak enough compared with the energy scale of the system and the bath, the total density matrix is approximated by the product state $|\rho_T(t)\rangle \approx |\rho(t)\rangle \otimes |\rho_B\rangle$. The evolution is now
\begin{equation}
\begin{aligned}
	\frac{d}{dt} \rho(t) 
	&\approx  \int_0^t \mathrm{Tr}_B\left[ V(t) \rho_T(\tau) V(\tau)- \rho_T(\tau) V(\tau) V(t) \right]d\tau +h.c. \\
	&= \sum_{kl}\int_0^t d\tau\ \left[C_{lk}(\tau - t) A_k(t)\rho(\tau)A_l(\tau) -  C_{lk}(\tau - t)\rho(\tau)A_l(\tau)A_k(t)+h.c.\right],
\end{aligned}
\end{equation}
where $C_{kl}(t) \equiv \mathrm{Tr}_B[\rho_B B_k(t) B_l ]$ is the correlation function of $B_k$'s.  We then take the \textbf{Markovian approximation}, which assumes that the correlations of the bath decay fast in time. We can thus make the substitution $\rho(\tau) \rightarrow \rho(t)$, the result equation of motion is Markovian:
\begin{equation}
\begin{aligned}
	\frac{d}{dt} \rho(t) &\approx \sum_{kl}\int_{0}^{t}dt' \left[C_{lk}(-t') A_k(t)\rho(t)A_l(t-t') - C_{lk}(-t')\rho(t)A_l(t-t')A_k(t)+h.c.\right] \\
	&= \sum_{k} \int_0^t dt \left[A_k \rho B_{k}-\rho B_{k} A_k+h.c.\right],
\end{aligned}
\end{equation}
where we have defined $B_{k}(t) = \sum_l \int_0^{\infty} dt' A_l(t-t')C_{lk}(-t')$. Now, we switch to the frequency domain,
\begin{equation*}
\begin{aligned}
	A_k(t) = \sum_\omega A_{k}(\omega) e^{-i\omega t}, \quad
	B_k(t) = \sum_{l,\omega} e^{-i\omega t} A_l(\omega)\Gamma_{lk}(\omega), \quad
	\Gamma_{kl}(\omega) = \int_0^\infty dt\ e^{i\omega t}C_{kl}(t).
\end{aligned}
\end{equation*}
We then take the \textbf{rotating wave approximation}, where we only keep the contributions from canceling frequency of operator $A$ and $B$,
\begin{equation}
\begin{aligned}
	\frac{d}{dt}\rho(t) &= \sum_{\omega} \left[\Gamma_{lk}(\omega) A_k(\omega) \rho A_l(\omega) - \Gamma_{lk}(\omega)\rho A_l(\omega) A_k(\omega) + h.c. \right] \\
	&= \sum_{\omega} \gamma_{kl}(\omega)(A_{l,\omega}\rho A_{k,\omega}^\dagger-\frac{1}{2}\{\rho,A_{k,\omega}^\dagger A_{l,\omega}\}) -i\left[\sum_{\omega}S_{kl}(\omega)A_{k,\omega}^\dagger A_{l,\omega},\rho\right],
\end{aligned}
\end{equation}
where we defined 
\begin{equation}
	\gamma_{kl}(\omega) = \Gamma_{kl}(\omega) +\Gamma^*_{lk}(\omega),\quad 
	S_{kl}(\omega) = \frac{1}{2i}[\Gamma_{kl}(\omega) - \Gamma^*_{lk}(\omega)].
\end{equation}
The matrices $\gamma(\omega)$ are positive; we can then take the square root of them. The jump operator is then 
$$L_{i,\omega} = \sum_j \sqrt{\gamma_{ij}(\omega)}A_{j,\omega}.$$ 
The evolution is then in the Lindblad form.



\section{Stochastic Schr\"{o}dinger Equation}

The Lindblad form Eq.~(\ref{eq:lindbladian}) is equivalent to the stochastic Schr\"{o}dinger equation (SSE):
\begin{equation}
	d|\psi\rangle = -iH|\psi\rangle + A[\psi]dt + \sum_m B[\psi]dW_m,
\end{equation}
where $dW_m$ is a stochastic infinitesimal element. The expectation value is then the average over all possible evolution path (trajectory): 
$$\langle O(t) \rangle = \overline{\langle\psi(t)|O|\psi(t)\rangle}.$$
For simplicity, in this section, we consider the jump operator $L_x$ labeled by coordinate $x$. 
The SSE can be Trotterized as
\begin{equation}
	\rho' = \left(\prod_x \mathcal{M}_{x} \right) \left[e^{-iH\Delta t} \rho e^{iH\Delta t}\right].
\end{equation}


\subsection{Poisson SSE}
Consider a small time interval $\Delta t$; the Lindblad equation is equivalent to the quantum channel 
\begin{equation}
	\mathcal{L}_{\Delta t}[\rho] = M_0 \rho M_0^\dagger + M_x \rho M_x^\dagger,
\end{equation}
where the jump operators are:
\begin{equation}
\begin{aligned}
	M_x = \sqrt{\gamma\Delta t} L_x, \quad
	M_0 = \sqrt{1 - \gamma \Delta t L_x^\dagger L_x}.
\end{aligned}
\end{equation}
A quantum channel can be simulated by a stochastic evolution of pure states:
\begin{equation}
	|\psi(t+\Delta t)\rangle \propto \left(\prod_{x} \mathcal{M}_x\right) e^{-i H\Delta t}|\psi(t)\rangle
\end{equation}
where each weak measurement is
\begin{equation}
	\mathcal{M}_x|\psi\rangle \propto \begin{cases}
		M_x |\psi\rangle & p = \langle L_x^\dagger L_x\rangle \gamma\Delta t \\
		M_0 |\psi\rangle & p = 1-\langle L_x^\dagger L_x\rangle \gamma\Delta t
	\end{cases},
\end{equation}
We can introduce a Poisson variable $dW_x$ satisfying 
\begin{equation}
	dW_x dW_y = \delta_{xy} dW_x,\quad \overline{dW_x} = \langle L_x^\dagger L_x\rangle\gamma dt,
\end{equation}
and the evolution can be cast into the stochastic differential equation
\begin{equation}
\begin{aligned}
	d|\psi\rangle = & -iHdt |\psi\rangle + \sum_x \frac{L_x-\sqrt{\langle L_x^\dagger L_x\rangle}}{\sqrt{\langle L_x^\dagger L_x\rangle}}dW_x|\psi\rangle - \frac{\gamma}{2} \sum_x \left(L_x^\dagger L_x-\langle L_x^\dagger L_x\rangle\right)dt 
	  |\psi\rangle.
\end{aligned}
\end{equation}
The $-\langle L_x^\dagger L_x\rangle dt |\psi\rangle$ comes from the renormalization. 
For numerical simulation, we can ignore it.

Note that in the numerical simulation, after each quantum jump, the state should be renormalized so that the jump probability for other $L_x$ can be computed correctly; this requires several renormalization procedures in a single time step. 




\subsection{Gaussian SSE}
Gaussian SSE is another way to unravel the Lindblad evolution.
It is often numerically more efficient since it only requires one renormalization in a single time step.
We first introduce the Wiener processes $dW_x$ satisfying 
\begin{equation}
	\overline{dW_x} = 0,\quad \overline{dW_x dW_y} = \delta_{xy} \gamma dt.
\end{equation}
Here we assume $L_x$ is Hermitian. The Gaussian SSE is then
\begin{equation}
	d |\psi\rangle = -i H dt |\psi\rangle + 
	\sum_x \left(L_x-\langle L_x\rangle \right) d W_x \left|\psi\right\rangle - \frac{\gamma}{2} \sum_x  (L_x - \langle L_x \rangle)^2 dt  \left|\psi\right\rangle.
\end{equation}
To retain the Lindblad, note that 
\begin{equation*}
	d\rho = \overline{|d\psi\rangle\langle\psi|} + \overline{|\psi\rangle \langle d\psi|}+ \overline{|d\psi\rangle\langle d\psi|}.
\end{equation*}
The first two terms give
\begin{equation*}
	-i[H,\rho]dt - \frac{\gamma}{2} \sum_x \left\{L_x^2- 2\langle L_x\rangle L_x + \langle L_x\rangle^2,\rho \right\}dt.
\end{equation*}
The second term gives
\begin{equation*}
	\gamma \sum_x \left[L_x \rho L_x  - \left\{\langle L_x\rangle L_x - \frac{\langle L_x\rangle^2}{2},\rho \right\}\right]dt + O(dt^2).
\end{equation*}
We, therefore, recover the Lindblad equation after averaging the SSE.

In numerical simulation, we exponentiate the expression, 
\begin{equation}
	|\psi'\rangle \sim \exp\left(A dt + \sum_x B_x dW_x \right) e^{-iH\Delta t}|\psi\rangle.
\end{equation}
The Taylor expansion of the exponent to the lowest order gives
\begin{equation*}
	e^{A dt + B \sum_x dW_x} = 1 + \left(A + \frac{1}{2} \sum_x B_x^2\right) dt + \sum_x B_x dW_x
\end{equation*}
Compared with the SSE, we get
\begin{equation*}
	A = -\gamma \sum_x (L_x-\langle L_x\rangle)^2,\quad
	B_x = L_x-\langle L_x\rangle.
\end{equation*}
The simplest example is when $L_x - n_x$ is a (quasi-)particle number operator, $n_x^2=n_x$, then we can simulate the SSE by the following form
\begin{equation}
	|\psi'\rangle \propto \exp\left\{\sum_x [dW_x + \gamma (2\langle n_x\rangle-1)]n_x dt \right\} e^{-iH\Delta t}|\psi\rangle,
\end{equation} 
where we have ignored the normalization term. After each time step, there should be a normalization procedure.


\section{Quadratic Lindbladian}

Consider the Lindblad in the Heisenberg picture:
\begin{equation}
	\frac{d}{dt} \hat O
	= i[\hat H, \hat O] + \sum_\mu \hat L_\mu^\dagger \hat O\hat L_\mu - \frac{1}{2} \sum_\mu\{\hat L_\mu^\dagger \hat L_\mu, \hat O \},
\end{equation}
where we choose $\hat O_{ij} = \omega_i\omega_j$ satisfying the relation $\hat O^T = 2\mathbb I - \hat O$. The covariance matrix is then $\Gamma_{ij} = i\langle \hat O\rangle - i\delta_{ij}$.

We assume that the jump operator has up to quadratic Majorana terms. In particular, we denote the linear terms and the Hermitian quadratic terms as
\begin{equation}
	\hat L_r = \sum_{j=1}^{2N} L^r_{j} \omega_j, \quad
	\hat L_s = -\frac{i}{4} \sum_{j,k=1}^{2N} M^s_{jk} \omega_j \omega_k.
\end{equation}
When the \textbf{jump operator} $\hat L_\mu$ contains only the linear Majorana operator, the Lindblad equation preserves Gaussianity. The evolution will break the Gaussian form for jump operators containing up to quadratic Majorana terms. 
However, the $2n$-point correlation is still solvable for free fermion systems.



\subsection{Third Quantization}

Assume only linear terms in jump operators,
\begin{equation}
	\partial_t \hat O = \left[i\hat H,\hat O\right] + \mathcal D_r[\hat O] 
	= \left[i\hat H - \frac{1}{2}\sum_r \hat L_r^\dagger L_r,\hat O\right] + \sum_r \left[\hat L_r^\dagger,\hat O\right]\hat L_r.
\end{equation}
For the future convenience, we define 
\begin{equation}
	\sum_r L_i^r L_j^{r*} = B_{ij} = \re B_{ij} + i \im B_{ij}. 
\end{equation}
The first term of EOM is:
\begin{equation*}
	\left[i\hat H - \frac{1}{2}\sum_r \hat L_r^\dagger L_r,\hat O_{ij}\right]
	= \sum_{kl}\left(\frac{1}{4}H-\frac{1}{2}B \right)_{kl} [\omega_k \omega_l, \omega_i \omega_j].
\end{equation*}
Use the commutation relation $\{\omega_i, \omega_j\} = 2\delta_{ij}$, we have the relation 
\begin{equation}
\begin{aligned}[]
	[\omega_k,\omega_i \omega_j] &= 2(\delta_{ki}\omega_j-\delta_{kj}\omega_i),\\
	[\omega_k \omega_l, \omega_i \omega_j] &= 2(\delta_{ki}\omega_j \omega_l-\delta_{kj} \omega_i \omega_l + \delta_{li}\omega_k \omega_j - \delta_{lj}\omega_k\omega_i).
\end{aligned}
\end{equation}
Therefore
\begin{equation*}
\begin{aligned}[]
	\left[i\hat H - \frac{1}{2}\sum_r \hat L_r^\dagger L_r,\hat O_{ij}\right]
	&= \left[
		\left(\frac{H}{2}- B\right) \cdot \hat O^T + \left(\frac{H}{2}- B\right)^T \cdot \hat O
		- \hat O \cdot \left(\frac{H}{2}- B\right)^T- \hat O^T \cdot \left(\frac{H}{2}- B\right)
	\right]_{ij} \\
	&= \left[
		(-H+2\im B) \cdot \hat O + \hat O \cdot (H-2\im B)
	\right]_{ij}.
\end{aligned}
\end{equation*}
The second term is
\begin{equation*}
\begin{aligned}
	\sum_r \left[L_r^\dagger, \hat O_{ij}\right] \hat L_r
	&= \sum_{kl} B_{kl} [\omega_k, \omega_i \omega_j]\omega_l
	= 2\sum_{kl} B_{kl}\left(
		\delta_{ki} \omega_j \omega_l - 
		\delta_{kj} \omega_i \omega_l\right) \\
	&= \left[2B\cdot \hat O^T - 2\hat O\cdot B^T\right]_{ij}
	= \left[-2B\cdot \hat O - 2\hat O\cdot B^* + 4B\right]_{ij}.
\end{aligned}
\end{equation*}
Therefore
\begin{equation}
	\partial_t \hat O_{ij} = \left[
		(-H-2\re B) \cdot \hat O + \hat O \cdot (H-2\re{B}) + 4B
	\right]_{ij}.
\end{equation}
The EOM of the covariance matrix is then
\begin{equation}
	\partial_t \Gamma = X^T\cdot\Gamma + \Gamma \cdot X + Y,\quad
	X = H - 2\re{B},\quad Y = 4\im{B}.
\end{equation}
Note that the constant part is replaced by its anti-symmetric part.

The steady state of the system is solved by the Lyapunov equation
\begin{equation}
	X^T\cdot\Gamma + \Gamma \cdot X = - Y.
\end{equation}


Now include the Hermitian quadratic quantum jumps:
\begin{equation}
\begin{aligned}
	\partial_t \hat O &= i[\hat H, \hat O] + \mathcal D_r[\hat O] + \mathcal D_s[\hat O], \\
	\mathcal D_s[\hat O] 
	&= \sum_s \hat L_s \hat O\hat L_s - \frac{1}{2} \sum_r\{\hat L_s^2, \hat O \}
	= -\frac{1}{2} \sum_s [\hat L_s,[\hat L_s,\hat O]].
\end{aligned}
\end{equation}

\subsection{Quadratic Lindbladian}
A direct calculation gives
\begin{equation}
\begin{aligned}
	D_s[\hat O]
	&= \frac{i}{8} \sum_s \sum_{kl} M^s_{kl} [\hat L_s,[\omega_k \omega_l, \omega_i \omega_j]] \\
	&= -\frac{i}{2}\sum_s \sum_{k} \left\{ M^s_{ik}[\hat L_s,\omega_k \omega_j]-[\hat L_s,\omega_i \omega_k]M^s_{kj} \right\} \\
	&= -\sum_{s,kl} \left[ M^s_{ik}(-M^s_{kl}\omega_l\omega_j+\omega_k\omega_l M^s_{lj})+(M^s_{il}\omega_l\omega_k-\omega_i\omega_l M^s_{lk})M^s_{kj} \right] \\
	&= \frac{1}{2}\sum_s \left[(M^s)^2 \cdot \hat O + \hat O\cdot(M^s)^2 -2 M^s \cdot\hat O\cdot M^s\right]_{ij}.
\end{aligned}
\end{equation}
Since $M^s$ is imaginary anti-symmetric matrix, $(M^s)^2 = - (\im M^s)^2$, and $M^s \cdot O \cdot M^s = - \im M^s \cdot O \cdot \im M^s$.
Together, we get the EOM of the variance matrix $\Gamma_{ij}$:
\begin{equation}
	\partial_t \Gamma = X^T\cdot\Gamma + \Gamma \cdot X - \sum_s M^s \cdot \Gamma\cdot M^s + Y,
\end{equation}
where
\begin{equation}
\begin{aligned}
	X = H - 2\re{B} + \frac{1}{2} \sum_s (M^s)^2, \quad
	Y = 4\im{B}.
\end{aligned}
\end{equation}


\subsection{Dirac Fermion Case}

This section considers the free fermion system preserving the U(1) charge. The jump operators are assumed to be quadratic: $\hat L_s = \sum_{jk} M^s_{jk} c_j^\dagger c_k$ where $\{M^s\}$ are Hermitian matrices.

For the fermion case, we choose $\hat O_{ij} = c_i^\dagger c_j$, and consider the Lindbladian
\begin{equation}
	\partial_t \hat O = i[\hat H, \hat O] + \mathcal D_s[\hat O]
	= i[\hat H, \hat O] - \frac{1}{2} \sum_s [\hat L_s,[\hat L_s,\hat O]],
\end{equation}
where each $\hat L_s = M^s_{ij} c_i^\dagger c_j$ is a Hermitian fermion bilinear.


The Hamiltonian part is:\footnote{Using the fact
$[c_k^\dagger c_l, c_i^\dagger c_j] = c_k^\dagger[c_l,c_i^\dagger c_j] + [c_k^\dagger,c_i^\dagger c_j]c_l =\delta_{il}c_k^\dagger c_j -\delta_{jk}c_i^\dagger c_l$, we know that for a quadratic form $\hat A = \sum_{ij} A_{ij} c_i^\dagger c_j$, $[\hat A, \hat O_{ij}] = [A^T, \hat O]_{ij}$.}
\begin{equation}
	i \sum_{kl}H_{kl}[c_k^\dagger c_l, c_i^\dagger c_j]
	= i \sum_{kl} H_{kl} (\delta_{il}c_k^\dagger c_j -\delta_{jk}c_i^\dagger c_l) 
	= i [H^T\cdot \hat O - \hat O\cdot H^T]_{ij}.
\end{equation}
Similarly, the double commutation in the second term is:
\begin{equation}
	\mathcal D_s[\hat O]
	= -\frac{1}{2}\sum_s [(M^{s*})^2\cdot\hat O + \hat O\cdot (M^{s*})^2 - 2 M^{s*}\cdot \hat O \cdot M^{s*}].
\end{equation}
Together, the EOM of correlation $G_{ij} = \langle c_i^\dagger c_j\rangle$ is
\begin{equation}
	\partial_t G = X^\dagger \cdot G + G \cdot X + \sum_s M^{s*}\cdot G \cdot M^{s*},
\end{equation}
where 
\begin{equation}
	X = -i H^* - \frac{1}{2}\sum_s (M^{s*})^2.
\end{equation}






%\bibliography{SciPost_Example_BiBTeX_File.bib}

%%%%%%%%%% END TODO: BIBLIOGRAPHY


\end{document}
