\documentclass[aps,prb,superscriptaddress,nofootinbib]{revtex4}
\usepackage{amsfonts}
\usepackage{amsmath}
\usepackage{amssymb}
\usepackage{graphbox}
\usepackage{graphicx}
\usepackage{caption}
\usepackage{bm}
\usepackage{bbm}
\usepackage{cancel}
\usepackage{color}
\usepackage{mathrsfs}
\usepackage[colorlinks,bookmarks=true,citecolor=blue,linkcolor=red,urlcolor=blue]{hyperref}
\usepackage{simpler-wick}
\usepackage{appendix}
\usepackage{float}
\usepackage{array}
\usepackage{booktabs}
\usepackage[export]{adjustbox}
\setlength{\parindent}{10 pt}
\setlength{\parskip}{2 pt}
\setcounter{MaxMatrixCols}{30}
\bibliographystyle{apsrev}
\newcommand{\RNum}[1]{\uppercase\expandafter{\romannumeral #1\relax}}
\newcommand{\normord}[1]{{:\mathrel{#1}:}}
\def\tbs{\textbackslash}
\def \tr{\operatorname{tr}}
\def \Tr{\operatorname{Tr}}


\begin{document}
\title{Non-equilibrium Field Theory}
\author{Jie Ren}



\maketitle


\tableofcontents



\section{Markovian Master Equation}

\subsection{General Markovian Form}

For general open quantum evolution, suppose the system and environment are separable initially:
\begin{equation}
	\rho_T=\rho\otimes\rho_B,\quad \rho_B=\sum_\alpha \lambda_\alpha |\phi_\alpha\rangle\langle\phi_\alpha |.
\end{equation}
Then the evolution of system-bath is unitary: $\rho_T(t) = U(t)\rho_TU^\dagger(t)$.
Trace out the environment's degrees of freedom, we have the quantum channel expression:
\begin{equation}
	\rho(t) = \sum_{\alpha\beta} W_{\alpha\beta} \rho W^\dagger_{\alpha\beta},
	\quad W_{\alpha\beta} = \sqrt{\lambda_\beta} \langle\phi_\alpha|U(t)|\phi_\beta\rangle.
\end{equation}
In general, the evolution of an open quantum system has the form $\rho(t) = \mathcal{L}_t[\rho]$.
The Lindblad equation assumes a semi-group relation:
\begin{equation}
	\mathcal{L}_t = \lim_{N \rightarrow \infty} \mathcal{L}_{t/N}\cdot\mathcal{L}_{t/N}\cdots \mathcal{L}_{t/N}.
\end{equation}
Such time decimation implies that the evolution is Markovian.
We will show that Markovian approximation leads directly to Lindblad equation.
First, we choose a complete operator basis $\{F_i\}$ in $N$-dimensional Hilbert space, satisfying
\begin{equation}
	\Tr[F_i^\dagger F_j] = \delta_{ij},
\end{equation}
where we choose $F_0=N^{-1/2} \cdot\mathbb I$. 
For a quantum channel, the channel operator $K_\mu$ can be expanded as
\begin{equation}
	K_\mu = \sum_i \Tr[F_i^\dagger K_\mu]F_i.
\end{equation}
In general, we have:
\begin{equation}
	\mathcal{L}_t[\rho] = \sum_{ij}c_{ij}(t)F_i\rho F_j^\dagger,
\end{equation}
where the Hermitian coefficient $c_{ij}(t)$ is
\begin{equation}
	c_{ij}(t) = \sum_{\mu} \Tr[F_i^\dagger K_\mu]\cdot \Tr[F_j^\dagger K_\mu]^*.
\end{equation}
Our target is to compute the limit
\begin{equation}
	\frac{d}{dt} \rho \equiv \lim_{t\rightarrow 0} \frac{1}{t}(\mathcal{L}_t[\rho]-\rho).
\end{equation}
For this purpose, we define the (Hermitian) coefficient $a_{ij}$ as:
\begin{equation}
	a_{00} = \lim_{t\rightarrow 0} \frac{c_{00}(t)-N}{t}, \quad
	a_{ij} = \lim_{t\rightarrow 0} \frac{c_{ij}(t)}{t}.
\end{equation}
The limit is then
\begin{equation}
\begin{aligned}
	\frac{d}{dt}\rho 
	&= \frac{a_{00}}{N}\rho + \frac{1}{\sqrt N} \sum_{i>0} \left(a_{i0} F_i \rho + a_{i0}^*\rho F_i^\dagger\right) + \sum_{i,j>0} a_{ij} F_i \rho F_j^\dagger. 
\end{aligned}
\end{equation}
Two further simplify the expression, we define
\begin{equation}
\begin{aligned}
	F &= \frac{1}{\sqrt N} \sum_{i=1}^{N^2-1} a_{i0} F_i, \\
	G &= \frac{1}{2N}a_{00}\mathbb I +\frac{1}{2}(F^\dagger+F), \\
	H &= \frac{1}{2i}(F^\dagger-F).
\end{aligned}
\end{equation}
The limit can be expressed by $G,H$ in a compact form:
\begin{equation}
	\frac{d\rho}{dt} = -i[H,\rho]+\{G, \rho\}+\sum_{i,j=1}^{N^2-1}a_{ij}F_i\rho F_j^\dagger.
\end{equation}
Note the $[H,\rho]$ part is the traceless part, and the $\{G,\rho\}$ is the trace part.
Since the quantum channel preserve the trace (for any $\rho$):
\begin{equation}
	\Tr\left[\frac{d\rho}{dt}\right]= \Tr\left[ \left(2G+\sum_{i,j=1}^{N^2-1}a_{ij}F_j^\dagger F_i \right)\rho \right]=0.
\end{equation}
Therefore,
\begin{equation}
	G = -\frac{1}{2}\sum_{i,j=1}^{N^2-1}a_{ij}F_j^\dagger F_i.
\end{equation}
We thus obtain the Lindblad form:
\begin{equation}
	\frac{d\rho}{dt} = -i[H,\rho]+\sum_{i,j=1}^{N^2-1}a_{ij} \left(F_i\rho F_j^\dagger-\frac{1}{2}\{F_j^\dagger F_i, \rho\} \right).
\end{equation}
We can further simplify the form by diagonalizing the matrix $a_{ij}$.
It is a convention to take the norm of $a_{ij}$ out to indicate the strength of the system-bath coupling, and the diagonalized Lindblad equation is
\begin{equation}
	\frac{d\rho}{dt} = -i[H,\rho]+ \gamma\sum_{m} \left(L_m\rho L_m^\dagger-\frac{1}{2}\{L_m^\dagger L_m, \rho\} \right).
\end{equation}


\subsection{First Principal Deduction}
In this section, we consider a general system-bath coupling
\begin{equation}
	H_T = H + H_B + V, \quad V = \sum_k A_k \otimes B_k.
\end{equation}
We will show under certain condition, the dynamics of the system is well approximated by the Lindblad equation.

We first assume that initially, the total system is product state:
\begin{equation}
	\rho_T(0) = \rho(0) \otimes \rho_B.
\end{equation}
Without loss of generality, we can also assume $\Vert A_k \Vert =1$, $\Tr[\rho_B B_k]=0$.
In the following, we will adopt the interacting picture, where the density operator evolves as
\begin{equation}
	\frac{d}{dt} \rho_T(t) = -i[V(t), \rho_T(t)] \equiv -i\mathcal V(t) |\rho_T(t) \rangle.
\end{equation}
Note that in the last equality, $\rho_T$ is expressed as a ket in the Hilbert space of linear operator, and the commutator with $V$ is expressed as a superoperator $\mathcal V$.
This notation can simplify the expression.
For example, the inner product in the operator space is the trace, so the partial trace operation can be denoted as
\begin{equation}
	|\rho\rangle = \langle \mathbb I_B|\rho_T\rangle.
\end{equation}
The evolution of the system is then 
\begin{equation}
\begin{aligned}
	\frac{d}{dt} |\rho(t)\rangle &= -i \langle \mathbb I_B|\mathcal V(t) |\rho_T(t)\rangle \\
	&= -i \langle \mathbb I_B|\mathcal{V}(t) |\rho_T(0)\rangle - \int_0^t \langle \mathbb I_B| \mathcal{V}(t) \mathcal{V}(\tau) |\rho_T(\tau)\rangle d\tau \\
	&= - \int_0^t \langle \mathbb I_B| \mathcal{V}(t) \mathcal{V}(\tau) |\rho_T(\tau)\rangle d\tau.
\end{aligned}
\end{equation}


Now we are taking the \textit{Born} approximation, which state when the coupling is weak enough compare with the energy scale of the system and the bath, the total density matrix is approximated by the product state:
\begin{equation}
	|\rho_T(t)\rangle \approx |\rho(t)\rangle \otimes |\rho_B\rangle.
\end{equation}
The evolution is now
\begin{equation}
\begin{aligned}
	\frac{d}{dt} \rho(t) &\approx \int_0^t \mathrm{Tr}_B\left[ V(t) \rho_T(\tau) V(\tau)- \rho_T(\tau) V(\tau) V(t) \right]d\tau +h.c. \\
	&= \sum_{kl}\int_0^t d\tau\ C_{lk}(\tau - t) \left[A_k(t)\rho(\tau)A_l(\tau)-\rho(\tau)A_l(\tau)A_k(t)\right]+h.c.,
\end{aligned}
\end{equation}
where $C_{kl}(t) \equiv \mathrm{Tr}_B[\rho_B B_k(t) B_l ]$ is the correlation function of $B_k$'s. 
We then take the \textit{Markovian} approximation which assumes that the correlations of the bath decay fast in time.
We can thus make the substitution $\rho(\tau) \rightarrow \rho(t)$, the result equation of motion is Markovian:
\begin{equation}
\begin{aligned}
	\frac{d}{dt} \rho(t) &\approx \sum_{kl}\int_{0}^{t}dt' C_{lk}(-t') \left[A_k(t)\rho(t)A_l(t-t')-\rho(t)A_l(t-t')A_k(t)\right]+h.c. \\
	&= \sum_{k} \int_0^t dt \left[A_k \rho B_{k}-\rho B_{k} A_k+h.c.\right],
\end{aligned}
\end{equation}
where we have defined
\begin{equation}
	B_{k}(t) = \sum_l \int_0^{\infty} dt' A_l(t-t')C_{lk}(-t'),
\end{equation}
Now we switch to the frequency domain,
\begin{equation}
\begin{aligned}
	A_k(t) &= \sum_\omega A_{k}(\omega) e^{-i\omega t}, \\
	B_k(t) &= \sum_{l,\omega} e^{-i\omega t} A_l(\omega)\Gamma_{lk}(\omega), \\
	\Gamma_{kl}(\omega) &= \int_0^\infty dt\ e^{i\omega t}C_{kl}(t).
\end{aligned}
\end{equation}
We then take the rotating wave approximation, where we only keep the contributions from canceling frequency of operator $A$ and $B$,
\begin{equation}
\begin{aligned}
	\frac{d}{dt}\rho(t) &= \sum_{\omega} \left[\Gamma_{lk}(\omega) A_k(\omega) \rho A_l(\omega) - \Gamma_{lk}(\omega)\rho A_l(\omega) A_k(\omega) + h.c. \right] \\
	&= \sum_{\omega} \gamma_{kl}(\omega)(A_{l,\omega}\rho A_{k,\omega}^\dagger-\frac{1}{2}\{\rho,A_{k,\omega}^\dagger A_{l,\omega}\}) -i\left[\sum_{\omega}S_{kl}(\omega)A_{k,\omega}^\dagger A_{l,\omega},\rho\right],
\end{aligned}
\end{equation}
where we defined
\begin{equation}
	\gamma_{kl}(\omega) = \Gamma_{kl}(\omega) +\Gamma^*_{lk}(\omega), \quad
	S_{kl}(\omega) = \frac{\Gamma_{kl}(\omega) - \Gamma^*_{lk}(\omega)}{2i}.
\end{equation}
The matrices $\gamma(\omega)$ are positive, we can then take the square root of them.
The jump operator is then
\begin{equation}
	L_{i,\omega} = \sum_j \sqrt{\gamma}_{ij}(\omega)A_{j,\omega}.
\end{equation}
The evolution is then in the Lindblad form.


\subsection{Stochastic Schr\"{o}dinger Equation}

Consider the Lindblad form
\begin{equation}
	\frac{d}{dt}\rho = -i[H,\rho] -\frac{\gamma}{2}\sum_m \{L_m^\dagger L_m, \rho\} + \gamma\sum_m L_m\rho L_m^\dagger.
\end{equation}
The evolution of the density matrix is equivalent to the stochastic Schr\"{o}dinger equation (SSE).
The general form of SSE is
\begin{equation}
	d|\psi\rangle = -iH|\psi\rangle + A[\psi]dt + B[\psi]dW,
\end{equation}
where $dW$ is a stochastic infinitesimal element.
The expectation value is then the average over all possible evolution path (trajectory):
\begin{equation}
	\langle O(t) \rangle = \overline{\langle\psi(t)|O|\psi(t)\rangle}.
\end{equation}

\paragraph*{Poisson SSE}
Consider a small time interval $\Delta t$, the Lindblad equation is equivalent to the quantum channel
\begin{equation}
	\rho(t+\Delta t) = M_0 \rho(t) M_0^\dagger + \sum_m M_m \rho M_m^\dagger, 
\end{equation}
where
\begin{equation}
\begin{aligned}
	M_0 &= 1 - i\left(H - i\frac{\gamma}{2} \sum_m L_m^\dagger L_m\right)\Delta t, \\
	M_m &= \sqrt{\gamma\Delta t} L_m.
\end{aligned}
\end{equation}

A such quantum channel can be simulated by a stochastic evolution of pure states:
\begin{equation}
	|\psi(t+\Delta t)\rangle \propto \begin{cases}
		L_m |\psi(t)\rangle & p = p_m(t) \gamma\Delta t \\
		\exp(-iH_{\mathrm{eff}}\Delta t)|\psi(t)\rangle & p = 1-\sum_m p_m(t)
	\end{cases},
\end{equation}
where $p_m(t)\equiv \langle\psi(t)|L_m^\dagger L_m|\psi(t)\rangle$, and the effective (non-Hermitian) Hamiltonian is
\begin{equation}
	H_{\mathrm{eff}} = H -i\frac{\gamma}{2}\sum_m L_m^\dagger L_m.
\end{equation}
We can introduce a Poisson variable $dW_m$ satisfying
\begin{equation}
	dW_m dW_n = \delta_{mn} dW_m,\quad
	\overline{dW_m} = \langle L_m^\dagger L_m\rangle\gamma dt,
\end{equation}
and the evolution can be cast into the stochastic differential equation
\begin{equation}\label{eq:OS-SSE-Poisson}
	d|\psi\rangle = -iHdt |\psi\rangle + \sum_m \left[\left(\frac{L_m}{\langle L_m^\dagger L_m\rangle^{\frac{1}{2}}}-1\right)dW_m -\frac{\gamma}{2} \left(L_m^\dagger L_m-\langle L_m^\dagger L_m\rangle\right)dt \right]
	  |\psi\rangle.
\end{equation}
Note that the $-\langle L_m^\dagger L_m\rangle dt |\psi\rangle$ comes from the renormalization.
For numerical simulation, we can ignore it.



\paragraph*{Gaussian SSE}

We can also use the Wiener processes $dW_m$ satisfying 
\begin{equation}
	\overline{dW_m} = 0,\quad
	\overline{dW_m dW_n} = \delta_{mn} \gamma dt.
\end{equation}
The Gaussian SSE is
\begin{equation}
	d |\psi\rangle = -i H dt |\psi\rangle + 
	\sum_m \left[\left(L_m-\langle L_m\rangle\right) \mathrm{d} W_{m}-\frac{\gamma}{2}\left(L_m^\dagger-\langle L_m\rangle\right)\left(L_m-\langle L_m\rangle\right) dt\right]\left|\psi\right\rangle.
\end{equation}
To retain the Lindblad, note that
\begin{equation}
	d\rho = \overline{|d\psi\rangle\langle\psi|} + \overline{|\psi\rangle \langle d\psi|}+ \overline{|d\psi\rangle\langle d\psi|}.
\end{equation}
Without going into the detail, we note that $L_m dW_m$ term in $\overline{|d\psi\rangle\langle d\psi|}$ will contribute a term $\gamma L_m \rho L_m^\dagger dt$; $-\frac{\gamma}{2}L_m^\dagger L_m dt$ term in $\overline{|d\psi\rangle\langle\psi|} + \overline{|\psi\rangle \langle d\psi|}$ contribute a tern $-\frac{\gamma}{2}\{L_m^\dagger L_m, \rho\}$ term.
All terms involving expectation value can be regarded as coming from the renormalization.


\subsection{Free Fermion Simulation}

In this section, we consider the system whose Hamiltonian is composed of quadratic fermionic operators, i.e.,
\begin{equation}
	\hat H_{\mathrm{free}} = \sum_{i,j=1}^N A_{ij} c_i^\dagger c_j + \frac{1}{2}\sum_{i,j=1}^N B_{ij} c_i c_j + \frac{1}{2}\sum_{i,j=1}^N B_{ij}^* c_j^\dagger c_i^\dagger, \label{eq:lattice-free-fermion-hamiltonian}
\end{equation}
where $t_{ij}$ is a Hermitian matrix, and $\Delta_{ij}$ is anti-symmetric.
In the Nambu basis 
\begin{equation}
	\Psi = (c_1,\dots,c_N,c_1^\dagger,\dots,c_N^\dagger)^T,
\end{equation}
the Hamiltonian has the form\footnote{Without loss of generality, in the following we always assume that the sum of chemical potential is zero, i.e., $\mathrm{Tr} A=0$.}
\begin{equation}
	\hat H_{\mathrm{free}} = \frac{1}{2} \sum_{i,j=1}^{2N} \Psi^\dagger_i H_{ij}^{\Psi} \Psi_j + \frac{1}{2}\mathrm{Tr}A,
\end{equation}
where the single-body matrix $H^{\Psi}$ is a $2N\times 2N$ Hermitian matrix
\begin{equation}
	H^{\Psi} = \left[\begin{array}{cc} 
		A & B \\
		-B^* & -A^* 
	\end{array}\right].
\end{equation}

\subsubsection{Majorana Representation}

The Majorana operators are defined as:
\begin{equation}
	\left[\begin{array}{c} \omega_{i} \\ \omega_{i+N} \end{array}\right]
	= \left[\begin{array}{cc} 
		1 & 1 \\ 
		i & -i 
	\end{array}\right] \left[\begin{array}{c} 
		c_i \\ c_i^\dagger 
	\end{array}\right], \quad 
	\left[\begin{array}{c} c_i \\ c_i^\dagger \end{array}\right]
	= \frac{1}{2} \left[\begin{array}{cc} 
		1 & -i \\ 
		1 & i 
	\end{array}\right] \left[\begin{array}{c} 
		\omega_{i} \\ \omega_{i+N}
	\end{array}\right].
\end{equation}
The fermionic bilinear in the Majorana basis has the form
\begin{equation}
	\hat H = -\frac{i}{4} \sum_{i,j=1}^{2N} H_{ij} \omega_i \omega_j
\end{equation}
where the single-body matrix $H$ is a $2N \times 2N$ real anti-symmetric matrix:
\begin{equation}
	H = \left[\begin{array}{cc} 
		-A^I - B^I & A^R - B^R \\
    	-A^R - B^R &  -A^I + B^I 
	\end{array}\right].
\end{equation}
where we have define $A^{R/I} = \mathrm{Re} A / \mathrm{Im} A$ and $B^{R/I} = \mathrm{Re} B / \mathrm{Im} B$.
Conversely, if we have a Majorana bilinear 
\begin{equation}
	\frac{i}{2} \sum_{i,j=1}^{2N} M_{ij}\omega_i \omega_j, \quad
	M = \left[\begin{array}{cc}
		M^{11} & M^{12} \\ M^{21} & M^{22}
	\end{array} \right],
\end{equation}
it can be transformed back to ordinary fermionic bilinear (\ref{eq:lattice-free-fermion-hamiltonian}) where
\begin{equation}
\begin{aligned}
	A &= M^{21} - M^{12} + i M^{11} + i M^{22}, \\
	B &= M^{21} + M^{12} + i M^{11} - i M^{22}.
	\label{eq:lattice-majorana-bilinear-to-fermion}
\end{aligned}
\end{equation}
A real anti-symmetric matrix can be transformed to standard form by an orthogonal transformation $O$:
\begin{equation}
\begin{aligned}
	H &= O \cdot \Sigma(\bm \lambda) \cdot O^T, \\
	\Sigma(\bm \lambda) &= i\sigma_y \otimes \mathrm{diag}(\lambda_1,\cdots,\lambda_n).
\end{aligned}
\end{equation}
Make the basis transformation
\begin{equation}
	\gamma_n = \sum_{j=1}^{2N} O_{jn} \omega_j,
\end{equation}
the Hamiltonian becomes the standard form:
\begin{equation}
\begin{aligned}
	H = -\frac{i}{4} \sum_{i=1}^N \lambda_i (\gamma_i \gamma_{i+N}-\gamma_{i+N} \gamma_i)
	= -\frac{i}{2} \sum_{i=1}^N \lambda_i \gamma_i \gamma_{i+N}.
\end{aligned}
\end{equation}
Each $\gamma_i \gamma_{i+N}$ pair can then transforms to independent fermion mode:
\begin{equation}
\begin{aligned}
	-\frac{i}{2}\gamma_i \gamma_{i+N} 
	= -\frac{i}{2}(d_i + d_i^\dagger)(id_i-id_i^\dagger)
	= d_i^\dagger d_i-\frac{1}{2}.
\end{aligned}
\end{equation}



\subsubsection{Gaussian States}
The Fermionic Gaussian states are those states with Gaussian form density operator:
\begin{equation}
	\hat \rho \propto \exp \left(\frac{i}{2}\sum_{i,j=1}^{2N}M_{ij}\omega_i \omega_j \right),
\end{equation}
where the matrix $M$ is real and anti-symmetric.\footnote{In particular, any thermal state has this form, with $M = \beta H/2$. The ground state of the free fermion system, though being pure state, can be regarded as the Gaussian state in the limit $M = \lim_{\beta \rightarrow \infty} \beta H$.}
If we expand the Gaussian form, the density operator becomes a Majorana polynomial:\footnote{Note that the coefficient $\Gamma$ in each order is not the direct expansion of the matrix $M$, since the direct expansion contains identical Majorana operators. That is, the $n$-th order expansion of the Majorana Gaussian form may contribute to the ($n-2m$)-th order term in the Majorana polynomial.}
\begin{equation}
	\hat{\rho} = \frac{\mathbb{I}}{2^N} + \sum_{n=1}^{N}\frac{i^n}{2^N}\sum_{1\le i_{1}<\cdots<i_{2n} \le 2N}\Gamma_{i_{1}\cdots i_{2n}} \omega_{i_1}\cdots\omega_{i_{2n}},
\end{equation}
where the coefficient $\Gamma_{i_1 \cdots i_{2n}}$ is the $2n$-point correlation function:
\begin{equation}
	\Gamma_{i_1 \cdots i_{2n}} = i^n \langle \omega_{i_1} \cdots \omega_{i_{2n}}\rangle, \quad i_m \ne i_n.
\end{equation}
In particular, the 2-point function 
\begin{equation}
	\Gamma_{ij} = i\langle \omega_i \omega_j\rangle - i\delta_{ij} = \frac{i}{2}\langle [\omega_i, \omega_j]\rangle
\end{equation}
is also called the \textit{covariance matrix}. 
For the Gaussian state all $2n$-point correlation is determined by the covariance matrix by the Wick theorem.

We are usually more familiar with the ordinary fermionic two-point correlation function $\langle c^\dagger_i c_j\rangle$ or $\langle c_i c_j\rangle$, which is related to the Majorana covariance matrix by:
\begin{equation}
\begin{aligned}
	\langle c_i^\dagger c_j\rangle &= \frac{1}{4}(
		\Gamma^{21}_{ij} - \Gamma^{12}_{ij} + 
		i \Gamma^{11}_{ij} + i \Gamma^{22}_{ij})
		+\frac{1}{2}\mathbb \delta_{ij}, \\
	\langle c_i c_j\rangle &= \frac{1}{4}(
		\Gamma^{21}_{ij} + \Gamma^{12}_{ij} + 
		i \Gamma^{11}_{ij} - i \Gamma^{22}_{ij}), \\
	\langle c_i^\dagger c_j^\dagger\rangle &= \frac{1}{4}(
		-\Gamma^{21}_{ij} - \Gamma^{12}_{ij} + 
		i \Gamma^{11}_{ij} - i \Gamma^{22}_{ij}).
\end{aligned}
\end{equation}


The relation of correlation in each order can be neatly captured by the Grassmannian Gaussian form:
\begin{equation}
\begin{aligned}
	\omega(\hat \rho, \theta) 
	= \frac{1}{2^N} \exp \left(\frac{i}{2} \sum_{i,j=1}^{2N}\Gamma_{ij}\theta_i \theta_j \right)
	=\frac{1}{2^N} + \sum_{n=1}^{N}\frac{i^n}{2^N}\sum_{1\le i_{1}<\cdots<i_{2n} \le 2N}\Gamma_{i_{1}\cdots i_{2n}} \theta_{i_1} \cdots \theta_{i_{2n}}.
\end{aligned}
\end{equation}
When the covariance matrix is obtained, we can use the same routine to canonicalize the skew-symmetric matrix $\Gamma$:
\begin{equation*}
	\Gamma = O \cdot \Sigma(\bm \lambda) \cdot O^T, \quad
	\tilde\theta_n = \sum_i O_{in} \theta_i,
\end{equation*}
and the density matrix in the Grassmann representation is
\begin{equation}
	\omega(\hat \rho, \theta) 
	= \prod_{n=1}^N \left(\frac{1}{2} e^{i \lambda_n \tilde\theta_n \tilde\theta_{n+N}} \right)
	= \prod_{n=1}^N \left(\frac{1+i\lambda_n \tilde\theta_n\tilde\theta_{n+N}}{2}  \right).
\end{equation}
This state correspond to a product state $\rho = \otimes_n \rho_n$ where
\begin{equation}
	\rho_n = \frac{1}{2} \left[\begin{array}{cc}
		1 + \lambda_n & 0 \\
		0 & 1 - \lambda_n
	\end{array} \right].
\end{equation}
The entanglement entropy is then
\begin{equation}
	S=\sum_n S_n = -\sum_n \left[
	\left(\frac{1+\lambda_n}{2}\right)\ln\left(\frac{1+\lambda_n}{2}\right)
	+ \left(\frac{1-\lambda_n}{2}\right)\ln\left(\frac{1-\lambda_n}{2}\right)\right].
\end{equation}




\subsubsection{Evolution of Covariance Matrix}
For Lindblad equation
\begin{equation}
	\frac{d}{dt} \hat\rho = -i[\hat H, \hat \rho] + \sum_{\mu=1}^{m} \hat L_\mu \hat\rho \hat L_\mu^\dagger -\frac{1}{2} \sum_{\mu=1}^{m} \{\hat L_\mu^\dagger \hat L_\mu, \hat \rho\}
\end{equation}
When the \textit{jump operator} $\hat L_\mu$ contains only the linear Majorana operator, the Lindblad equation preserve Gaussianity. 
For \textit{jump operator} contains up to quadratic Majorana terms, the evolution will break the Gaussian form, however, the $2n$-point correlation is still solvable for free fermion system.

We assume that the jump operator has up to quadratic Majorana terms. 
In particular, we denote the linear terms and the Hermitian quadratic terms as
\begin{equation}
	\hat L_r = \sum_{j=1}^{2N} L^r_{j} \omega_j, \quad
	\hat L_s = \sum_{j,k=1}^{2N} M^s_{jk} \omega_j \omega_k.
\end{equation}
Now consider the dynamics of the expectation value $\langle\hat O\rangle$:
\begin{equation}
\begin{aligned}
	\frac{d}{dt}\langle \hat O\rangle
	&= -i \mathrm{Tr} [\hat O (\hat H \hat\rho-\hat\rho \hat H)] 
		+ \sum_\mu \mathrm{Tr}[\hat O \hat L_\mu \hat\rho \hat L_\mu^\dagger]
		- \frac{1}{2}\sum_\mu \mathrm{Tr}[\hat O \hat L_\mu^\dagger \hat L_\mu \hat\rho
		+ \hat O \hat\rho \hat L_\mu^\dagger \hat L_\mu] \\
	&= \left\langle
		i[\hat H, \hat O] + \sum_\mu \hat L_\mu^\dagger \hat O\hat L_\mu - \frac{1}{2} \sum_\mu\{\hat L_\mu^\dagger \hat L_\mu, \hat O \}
		\right\rangle.
\end{aligned}
\end{equation}
We can express the dynamics of operator as in the Heisenberg picture:
\begin{equation}
	\frac{d\hat O}{dt} = i[\hat H, \hat O] + \mathcal D_r[\hat O] + \mathcal D_s[\hat O],
\end{equation}
where
\begin{equation}
\begin{aligned}
	\mathcal D_r[\hat O] 
	&= \sum_r \hat L_r^\dagger \hat O\hat L_r - \frac{1}{2} \sum_r\{\hat L_r^\dagger \hat L_r, \hat O \}
	= \frac{1}{2}\sum_r [\hat L_r^\dagger L_r, \hat O] - \sum_r \hat L_r^\dagger[\hat L_r,\hat O],  \\
	\mathcal D_s[\hat O] 
	&= \sum_s \hat L_s \hat O\hat L_s - \frac{1}{2} \sum_r\{\hat L_s^2, \hat O \}
	= -\frac{1}{2} \sum_s [\hat L_s,[\hat L_s,\hat O]].
\end{aligned}
\end{equation}
The equation of motion can be further simplified to:
\begin{equation}
	\frac{d\hat O}{dt} 
	= i[\hat H_{\mathrm{eff}}, \hat O] - \sum_r \hat L_r^\dagger[\hat L_r,\hat O] -\frac{1}{2} \sum_s [\hat L_s,[\hat L_s,\hat O]],
\end{equation}
where the effective Hamiltonian is
\begin{equation}
	\hat H_{\mathrm{eff}} = \sum_{ij} \left(-\frac{i}{4}H_{ij}-\frac{1}{2} B^I_{ij}\right)\omega_i\omega_j,
\end{equation}
where we have defined $B_{ij} = \sum_r L^r_i L^{r*}_j$.
Using the commutation relation $\{\omega_i, \omega_j\} = 2\delta_{ij}$, we have the following relation
\begin{equation}
\begin{aligned}[]
	[\omega_k,\omega_i \omega_j] &= 2(\delta_{ki}\omega_j-\delta_{kj}\omega_i), \\
	[\omega_k \omega_l, \omega_i \omega_j] 
	&= 2(\delta_{ki}\omega_j \omega_l-\delta_{kj} \omega_i \omega_l + \delta_{li}\omega_k \omega_j - \delta_{lj}\omega_k\omega_i),
\end{aligned}
\end{equation}
and let $\hat O_{ij} = \omega_i\omega_j - \delta_{ij}\mathbb I$.
The first term of EOM is:
\begin{equation*}
\begin{aligned}[]
	i\langle[\hat H_{\mathrm{eff}}, \hat O_{ij}]\rangle_t
	&= \sum_{kl}\left(\frac{1}{4}H-\frac{i}{2}B^I \right)_{kl} \langle[\omega_k \omega_l, \omega_i \omega_j]\rangle_t \\
	&= \sum_{kl} \left(\frac{1}{2}H-i B^I\right)_{kl} \langle 
		\delta_{ki}\omega_j \omega_l-\delta_{kj} \omega_i \omega_l + 
		\delta_{li}\omega_k \omega_j - \delta_{lj}\omega_k\omega_i
	\rangle_t \\
	&= \left[
		(H-2iB^I)^T \cdot \langle\hat O\rangle_t + 
		\langle\hat O\rangle_t \cdot (H-2iB^I)
	\right]_{ij}.
\end{aligned}
\end{equation*}
The second term is
\begin{equation*}
\begin{aligned}
	-\sum_r \langle L_r^\dagger[L_r, \hat O_{ij}] \rangle_t
	&= -\sum_{kl} B_{kl}^* \langle \omega_k [\omega_l, \omega_i \omega_j] \rangle_t 
	= -2\sum_{kl} B_{kl}^* \langle 
		\delta_{li} \omega_k \omega_j - 
		\delta_{lj} \omega_k \omega_i
	\rangle_t \\
	&= -\left[2B\cdot \langle\hat O\rangle_t + 2\langle\hat O\rangle_t\cdot B^* + 4i B^I \right]_{ij}.
\end{aligned}
\end{equation*}
And the third term is
\begin{equation*}
\begin{aligned}
	-\frac{1}{2}\sum_s \langle[\hat L_s,[\hat L_s, \hat O_{ij}]]\rangle_t
	&= -\frac{1}{2} \sum_s \sum_{kl} M^s_{kl}\langle[\hat L_s,[\omega_k \omega_l, \omega_i \omega_j]]\rangle_t \\
	&= 2\sum_s \sum_{k} \left\langle M^s_{ik}[\hat L_s,\omega_k \omega_j]-[\hat L_s,\omega_i \omega_k]M^s_{kj} \right\rangle_t \\
	&= 8\sum_{s,kl} \left\langle M^s_{ik}[-M^s_{kl}\omega_l\omega_j+\omega_k\omega_l M^s_{lj}]+[M^s_{il}\omega_l\omega_k-\omega_i\omega_l M^s_{lk}]M^s_{kj} \right\rangle_t \\
	&= 8\sum_s \left[2 M^s \cdot \langle\hat O\rangle_t\cdot M^s-(M^s)^2 \cdot \langle\hat O\rangle_t - \langle\hat O\rangle_t\cdot(M^s)^2 \right]_{ij}.
\end{aligned}
\end{equation*}
In together, we get the EOM of variance matrix $\Gamma_{ij}(t)=i\langle\hat O_{ij}\rangle_t$, the result is:
\begin{equation}
	\partial_t \Gamma = X^T\cdot\Gamma + \Gamma \cdot X + \sum_s (Z^s)^T \cdot \Gamma\cdot Z^s + Y,
\end{equation}
where
\begin{equation}
	X = H - 2B^R + 8 \sum_s (\mathrm{Im} M^s)^2, \quad
	Y = 4B^I, \quad 
	Z = 4 \mathrm{Im} M^s.
\end{equation}


\subsubsection{Measurement}
We consider the evolution of free fermion Gaussian pure state under Eq.~(\ref{eq:OS-SSE-Poisson}).
For particle number conserving systems, the Gaussian state can be represented as a matrix: $$|B\rangle \equiv \prod_{j=1}^N \sum_i B_{ij} c_{i}^\dagger |0\rangle = \bigotimes_{j=1}^N |B_j\rangle.$$
Two types of Lindblad operators are classically simulatable: the linear terms: $L_m = d \ \text{or}\ d^\dagger$, where $d^\dagger = \sum_i a_{i} c_i^\dagger$ is a quasi-mode, and the quadratic terms:
\begin{equation}
	L_m = d^\dagger d \ \text{or}\ d d^\dagger.
\end{equation}

For the linear term $d = \sum_i a_i^* c_i$, if the quantum jump happen, $p_m = \langle B|d^\dagger d |B\rangle = \| d|B\rangle \|^2.$
Note that
\begin{equation}
	d|B\rangle = \sum_k a_k^* c_k \prod_{j} \sum_i c_i^\dagger B_{ij} |0\rangle
	=\sum_j \langle a|B_j\rangle \bigotimes_{l\ne j}|B_l\rangle.
\end{equation}
So we have $p_m = \sum_j |\langle a|B_j\rangle|^2$.
For each $m$, there is a probability $p_m \gamma \Delta t$ that the jump process $|B\rangle \rightarrow L_m|B\rangle$ happens.
To see the result of the jump, we first note that the matrix $B$ representing the gaussian state has the unitary degree of freedom
\begin{equation}
	|B\rangle = |B'\rangle, \quad B'_{ij} = \sum_k B_{ik}U_{kj},
\end{equation}
where $U_{kj}$ is an $N\times N$ unitary matrix.
It means that the gaussian state is determined by the linear subspace that columns of $B$ span.
To better see how $L_m$ act on $|B\rangle$, we use such freedom to restrict $\langle a| B'_{j}\rangle = 0$ for $j>1.$
Such matrix $B'$ always exists since we can always find a column $j$ that $\langle a| B_{j}\rangle \ne 0$ (otherwise $p_m=0$ and the jump is impossible).
We then move the column to the first and define the column as
\begin{equation}
	|B'_{j}\rangle = |B_{j}\rangle - \frac{\langle a|B_{j}\rangle}{\langle a|B_{1}\rangle} |B_{1}\rangle, \quad j>1.
\end{equation}
Such column transformations do not alter the linear space $B$ spans, while the orthogonality and the normalization might be affected. 
We will discuss the orthonormalization procedure later.
The result of the quantum jump is $|B\rangle \rightarrow \bigotimes_{j> 1} |B'_{j}\rangle,$
which is represented by an $L\times (N-1)$ matrix.
On the other hand, if no jump process happens, the system undergoes a non-Hermitian evolution $|B\rangle \rightarrow |B'\rangle,$ where $B' = e^{-iH_{\mathrm{eff}}\Delta t} B.$
Since the Hamiltonian is non-Hermitian, the resulting matrix $B'$ is also not orthonormal.
The orthonormalization can be easily implemented by the QR decomposition: $B_{L\times N} = Q_{L\times N} R_{N\times N},$ where the $Q$ matrix is orthonormal and we can set $B' = Q$.

The procedure for $L_m = d^\dagger$ is similar.
Note that $dd^\dagger + d^\dagger d = 1$, it means the jump probability is $p_m = 1-\sum_j |\langle a|B_j\rangle|^2.$
The result of the quantum jump is $|B\rangle = |a\rangle \bigotimes_{j}|B'_j\rangle,$ where $|B'_{j}\rangle = |B_{j}\rangle - \langle a|B_{j}\rangle |a\rangle.$



\section{Keldysh Formalism}

In the Keldysh loop presentation, the evolution of an operator is
\begin{equation}
	\langle O\rangle_t = \Tr[\mathcal{U}_{t_0,+\infty}\mathcal{U}_{+\infty,t} O \mathcal{U}_{t,t_0}\rho(t_0)] \simeq \int D[\phi] e^{i\int_{\mathcal C} dt \mathcal L[\phi]} O(t).
\end{equation}
The partition function is
\begin{equation}
	Z = \int D[\phi] \exp\left\{i\int_{\mathcal C} dt \mathcal L[\phi]\right\}
	=  \int D[\phi^+,\phi^-] \exp\left\{i\int_{t_0}^\infty dt \mathcal L[\phi^+,\phi^-]\right\}.
\end{equation}





\end{document}


